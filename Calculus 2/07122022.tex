\documentclass{article}
\usepackage{amsmath}
\usepackage{amsthm}
\usepackage{amssymb}
\usepackage{pgfplots}
\usepackage{amsfonts}
\usepackage[margin=2.5cm]{geometry}
\usepackage{graphicx}
\usepackage[export]{adjustbox}
\usepackage{fancyhdr}
\usepackage[portuguese]{babel}
\usepackage{hyperref}
\usepackage{lastpage}

\pagestyle{fancy}
\fancyhf{}

\hypersetup{
    colorlinks,
    citecolor=black,
    filecolor=black,
    linkcolor=black,
    urlcolor=black
}

\newtheorem*{tm*}{\underline{Teorema:}}
\newtheorem*{sol*}{\underline{Solu\c c\~ao:}}
\newtheorem*{proof*}{\underline{Prova:}}
\renewcommand\qedsymbol{$\blacksquare$}

\rfoot{P\'agina \thepage \hspace{1pt} de \pageref{LastPage}}
\title{C\'ALCULO 2}
\author{Renan Wenzel}
\date{\today}

\begin{document}
\maketitle

\section{Teorema de Stokes}

 \begin{tm*}
  Seja S uma superf\'icie orientada, suave por partes, cuja fronteira \'e formada por uma curva
C fechada, simples, suave por partes e orientada positivamente. Seja F um campo vetorial de classe $C^1$ em uma
regi\~ao aberta de $\mathbb{R}^2$ contendo S, ent\~ao:
  $$\iint_{S}(\text{rot}\vec{F}\cdot{}\vec{n})dS = \oint_{C}\vec{F}\cdot{}dr $$ 
 \end{tm*}
\end{document}d
