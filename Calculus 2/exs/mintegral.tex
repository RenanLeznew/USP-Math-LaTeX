\documentclass{article}
\usepackage{amsmath}
\usepackage{amsthm}
\usepackage{amssymb}
\usepackage{pgfplots}
\usepackage{amsfonts}
\usepackage[margin=2.5cm]{geometry}
\usepackage{graphicx}
\usepackage[export]{adjustbox}
\usepackage{fancyhdr}
\usepackage[portuguese]{babel}
\usepackage{hyperref}
\usepackage{lastpage}
\usepackage{mathtools}

\pagestyle{fancy}
\fancyhf{}

\pgfplotsset{compat = 1.18}

\hypersetup{
    colorlinks,
    citecolor=black,
    filecolor=black,
    linkcolor=black,
    urlcolor=black
}
\newtheorem*{def*}{\underline{Definition}}
\newtheorem*{theorem*}{\underline{Theorem:}}
\newtheorem{example}{\underline{Example:}}[section]
\newtheorem*{proof*}{\underline{Proof:}}
\renewcommand\qedsymbol{$\blacksquare$}
\newcommand{\Lin}[1]{Lin_{\mathbb{K}}({#1})}

\rfoot{P\'agina \thepage \hspace{1pt} de \pageref{LastPage}}

\title{EXERC\'ICIOS DE C\'ALCULO}
\author{Renan Wenzel}
\date{\today}

\begin{document}
\maketitle

\section*{Campos e Potenciais}
\subsection{Exerc\'icio 12}
Dada a fun\c c\~ao $\vec{F}(x, y) = (e^{x}y ^{3} + y, 3e^{x}y^2 + x)$, buscamos uma fun\c c\~ao f tal que 
$\vec{\nabla}{f} = \vec{F}$. Assim, basta integrar a primeira entrada com rela\c c\~ao a x e a segunda com rela\c c
\~ao a y:
  $$
  f(x, y) = (e^{x}y ^{3} + yx, e^{x}y^3 + yx)
  $$

  Por outro lado, se $\vec{F}(x, y, z) = (12x^2, \cos(y)\cos(z), 1 - \sin(y)\sin(z),$ ser\'a preciso integrar f
nas 3 coordenadas. Com efeito, obtemos
  $$
  f(x, y, z) = (4x ^{3}, \sin(y)\cos(z), z + \sin(y)\cos(z)).
  $$
\qedsymbol

\subsection{Exerc\'icio 13}
Primeiramente, sem utilizar o TFIL, a resolu\c c\~ao se torna uma aplica\c c\~ao mec\^anica do que foi visto ao longo
do curso, ou seja, basta calcular
\begin{align*}
  \int_{C}\vec{F}\dot{}dr & = \int_{1}^{e} F(\gamma(t))\gamma'(t)dt = \int_{1}^{e}(2t^2\ln{t}, \frac{t ^{4}}{t} + t ^{2}, 2t.t)\cdot{}
  (2t, 1, 1)dt \\ 
  & =\int_{1}^{e}4t^3\ln{t} + t^3 + t^2 + 2t^2dt = \int_{1}^{e}t^3(4\ln{t^2} + 1) + 3t^2dt\\
  & =4\int_{1}^{e}t^3\ln{t}dt + \int_{1}^{e}t^3dt + 3\int_{1}^{e}t^2dt = (t^4\ln{t} - \frac{t^4}{4} + \frac{t ^{4}}{4} + t^3)\bigg|_{1}^{e} \\
  & =(t^4\ln{t} + t^3)\bigg|_{1}^{e} = e^{4} + e^{3} - 1 
\end{align*}

O segundo m\'etodo, ou seja, utilizando o TFIL, \'e mais interessante. Buscaremos uma fun\c c\~ao f tal que $\vec{\nabla}=\vec{F}$
\end{document}
