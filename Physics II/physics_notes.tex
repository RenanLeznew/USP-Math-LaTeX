\documentclass{article}
\usepackage{amsmath}
\usepackage{amsthm}
\usepackage{amssymb}
\usepackage{pgfplots}
\usepackage[utf8]{inputenc}
\usepackage{amsfonts}
\usepackage[margin=2.5cm]{geometry}
\usepackage{graphicx}
\usepackage[export]{adjustbox}
\usepackage{fancyhdr}
\usepackage[portuguese]{babel}
\usepackage{hyperref}
\usepackage{lastpage}
\usepackage{mathtools}
\setcounter{section}{-1}

\pagestyle{fancy}
\fancyhf{}

\pgfplotsset{compat = 1.18}

\hypersetup{
   colorlinks,
   citecolor=black,
   filecolor=black,
   linkcolor=black,
   urlcolor=black
}
\newtheorem*{def*}{\underline{Defini\c c\~ao}}
\newtheorem*{theorem*}{\underline{Teorema}}
\newtheorem*{lemma*}{\underline{Lema}}
\newtheorem*{prop*}{\underline{Proposi\c c\~ao}}
\newtheorem{example}{\underline{Exemplo}}
\newtheorem*{proof*}{\underline{Prova}}
\renewcommand\qedsymbol{$\blacksquare$}
\newcommand{\Lin}[1]{Lin_{\mathbb{K}}({#1})}

\rfoot{P\'agina \thepage \hspace{1pt} de \pageref{LastPage}}

\begin{document}
 \begin{figure}[ht]
  \minipage{0.76\textwidth}
    \includegraphics[width=4cm]{../icmc.png}
    \hspace{7cm}
    \includegraphics[height=4.9cm,width=4cm]{../brasao_usp_cor.jpg}
  \endminipage  
\end{figure}

\begin{center}
  \vspace{1cm}
  \LARGE
  UNIVERSIDADE DE S\~AO PAULO

  \vspace{1.3cm}
  \LARGE
  INSTITUTO DE CI\^ENCIAS MATEM\'ATICAS E COMPUTACIONAIS - ICMC

  \vspace{1.7cm}
  \Large
  \textbf{Notas de Física II}

  \vspace{1.3cm}
  \large
  \textbf{Renan Wenzel - 11169472}

  \vspace{1.3cm}
  \large
  \textbf{Professor - Luis Gustavo Marcassa}

  \textbf{E-mail: marcassa@ifsc.usp.br}

  \vspace{1.3cm}
  \today
\end{center}

 \newpage
 \textbf{{\Huge Disclaimer}}
 \vspace{5cm}

  {\huge Essas notas não possuem relação com professor algum. 

  Qualquer erro é responsabilidade solene do autor.

Caso julgue necessário, contatar: renan.wenzel.rw@gmail.com}
 \newpage

 \tableofcontents

 \newpage

 \section{Aula 00 - 07/08/2023}
   Avisos sobre o curso (Ler e baixar o pdf no e-disciplinas!!!!); 

\section{Aula 01 - 09/08/2023}
\subsection{Motivações}
\begin{itemize}
  \item Ângulos, velocidade angular e aceleração angular;
  \item Energia em sistemas com rotação.
\end{itemize}
\subsection{Rotação}
  Antes de qualquer coisa, convencionamos o sentido antihorário como aquele em que \(\Delta \theta >0\) e
o sentido horário como o que \(\Delta \theta <0.\) Uma volta completa em torno do círculo é dada pela versão com \(2\pi\) da fórmula do arco de círculo
 \(\Delta S_{i} = r_{i}\Delta \theta = 2\pi r_{i}\) e, com isso, a variação do ângulo em uma volta completa é dada por 
   \[
     \Delta \theta = \frac{S_{i}}{r_{i}} = \frac{2\pi r_{i}}{r_{i}} = 2\pi rad.
   \]
  Um dos assuntos de importância para nós é o estudo da variação temporal do ângulo. Definimos, nessa lógica, a 
velocidade angular média por 
  \[
    \omega _{med} = \frac{\Delta \theta }{\Delta t}.
  \]
  De modo análogo ao que vimos com cinemática, existe também a velocidade angular instantânea, obtida tomando o limite: 
    \[
      \omega = \lim_{\Delta t\to 0}\frac{\Delta \theta }{\Delta t} = \frac{d\theta }{dt}.
    \]
    Observa-se de cara que, se \(\omega >0, \theta \) aumenta e, se \(\omega <0, \theta \) diminui. Assim como antes,
  precisamos ver, também, a unidade. Em cinemática, a unidade de velocidade era metro por segundo. Dessa vez, já que
  o ângulo move-se em radianos, mas há outras unidades, como a revolução e o grau. Logo, as unidades de \(\omega \) podem ser \([\omega ] = \frac{radianos}{tempo} = \frac{rad}{s}, \frac{graus}{s}, \frac{\text{revolução}}{s},\)
  em que \(1\text{revolução} = 2\pi rad = 360\deg\)

    Por exemplo, se um CD roda a 3000rpm, pode-se expressar essa velocidade de rotação como 
      \[
        \omega = 3000rpm = \frac{3000 \cdot 2\pi}{60} = \frac{600}{6}\pi = 100\pi \frac{rad}{s}.
      \]

  Analogamente, é possível analisar a variação da própria velocidade angular com o tempo, resultando na chamada
  acelerações angulares média e instantânea: 
    \[
      \alpha_{med} = \frac{\Delta \omega }{\Delta t}\quad \alpha  = \frac{d\omega }{dt} = \frac{d}{dt}\biggl(\frac{d\theta }{dt}\biggr) = \frac{d^{2}\theta }{dt^{2}}.
    \]
    A unidade dessa grandeza, novamente, similar à versão linear dela, será dada em \([\alpha ]= \frac{radiano}{s^{2}}\), ou \([\alpha ]=\frac{grau}{s^{2}}\), etc. Nessas
  situações todas, se \(\alpha >0, \omega \) aumenta e, se \(\alpha <0, \omega \) diminui. 

    Agora, suponha que \(\alpha \) é constante. Todos os processos de movimento uniformemente acelerado são válidos aqui também:
    \begin{table}[h!]
    \centering
    \begin{tabular}{|c|c|c|}
        \hline
        & \textbf{Variáveis angulares} & \textbf{Variáveis escalares} \\
        \hline
        \textbf{Posição} & $\theta(t) = \theta_{0} + \omega_{0}t + \frac{\alpha t^{2}}{2}$ & $s(t) = R\theta(t)$ \\
        \hline
        \textbf{Velocidade} & $\omega(t) = \omega_{0} + \alpha t = \frac{d\theta }{dt}$ & $v(t) = v_{0} + at = \frac{dx}{dt}$ \\
        \hline
        \textbf{Aceleração} & $\alpha(t) = \frac{d\omega(t)}{dt} = \frac{d^2\theta(t)}{dt^2}$ & $|\vec{a}(t)| = \frac{dv(t)}{dt} = R\alpha(t),\quad |\vec{a}_{cp}| = \frac{v^2}{R}$ \\
        \hline
        \textbf{Torricelli} & $\omega^{2}(t) = \omega_{0}^{2} + 2\alpha \Delta \theta $ & $v^{2} = v_{0}^{2} + 2a\Delta s.$ \\
        \hline
    \end{tabular}
    \caption{Resumo movimento circular.}
    \label{tab:my_label}
  \end{table}

\begin{example}
  Suponha que há um CD que começa no repouso. Ele começa a girar, indo de 0 a 500rpm em 5.5s. Pergunta-se:
 \begin{itemize}
   \item[a)] Quanto vale \(\alpha \)?
     \item[b)] Quantas voltas o CD dá em 5.5s?
       \item[c)] Qual é a distância percorrida por uma ponta a 6cm do eixo de rotação?
 \end{itemize}

 \textbf{Soluções:}
 \begin{itemize}
   \item[a)] Temos \(\omega (0) = 0, \omega (5.5) = 500rpm.\) Segue que 
     \[
       \omega(t) = \omega_{0} + \alpha t \Rightarrow \alpha  = \frac{\omega (t)}{t} = \frac{500 \cdot 2\pi}{5.5 \cdot 60}\approx 9.52 \frac{rad}{s^{2}}
     \]

    \item[b)] Aplicamos o Torricelli angular com os dados que temos: 
      \[
        \omega^{2} = 2\alpha \Delta \theta \Rightarrow \delta \theta = \frac{\omega^{2}}{2\alpha }\approx 144 rad \Rightarrow \frac{144}{2\pi}rad\approx 23\text{rotações}.
      \]

    \item[c)] Por fim, multiplicando a variação do ângulo pelo raio, obtemos 
      \[
        \Delta S_{i} = r\Delta \theta = 6 \cdot 10^{-2}\cdot 144\approx 8.65m.
      \]
 \end{itemize}
\end{example}

  Olhando de forma cautelosa a fórmula de arco de circulo, podemos derivá-la com respeito ao tempo utilizando o que vimos até agora: 
    \[
      \frac{dS_{i}}{dt} = V_{t} = r_{i}\frac{d\theta }{dt} = r_{i}\omega.
    \]
  Essa derivação resulta em uma velocidade linear, que também pode ser derivada a fim de obter uma aceleração linear 
    \[
      \frac{dV_{t}}{dt} = r_{i}\frac{d\omega }{dt} \Rightarrow a_{t} = r_{i}\alpha.
    \]
    Note a relação entre as duas acelerações que obtivemos, \(a_{c} = \frac{V_{t}^{2}}{r_{i}}= \frac{r_{i}^{2}\omega^{2}}{r_{i}} = r_{i}\omega^{2}.\)

\subsection{Energia Cinética de Rotação}
  A energia cinética, como vista previamente, é dada por 
    \[
      \mathcal{K} = \frac{1}{2}m_{i}v_{i}^{2}.
    \]
  Agora, imagine um corpo discreto (formado por vários pontos). Somemos as energias deles, tal que a energia cinética total é 
    \[
      \mathcal{K}_{T} = \sum\limits_{}^{}\frac{1}{2}m_{i}v_{i}^{2}.
    \]
  Mas, sabemos que \(v_{i} = r_{i}\omega, \) tal que 
    \[
      \mathcal{K} = \frac{1}{2}\sum\limits_{}^{}m_{i}r_{i}^{2}\omega^{2} = \frac{1}{2}\biggl[\sum\limits_{}^{}m_{i}r_{i}^{2}\biggr]\omega^{2}
    \]
  Chamemos o termo em colchete de momento de inércia, denotado por \(I:= \sum\limits_{}^{}m_{i}r_{i}^{2} = \sum\limits_{}^{}I_{i}\). Logo, 
    \[
      \boxed{\hypertarget{kin_en}{\mathcal{K}_{T} = \frac{1}{2}I\omega^{2}.}}
    \]
\newpage
\section{Aula 02 - 10/08/2023}
\subsection{Motivações}
 \begin{itemize}
   \item Momento de Inércia
 \end{itemize}
\subsection{Distribuição Contínua de Massa}
  No caso de distribuições discretas de massa, vimos que o momento de inércia é dado por 
    \[
      I=\sum\limits_{i}^{}m_{i}r_{i}^{2}.
    \]
No entanto, muitas situações do mundo precisam que tratemos a distribuição de massa como algo único, uma
quantidade contínua. Para isso, passamos de somar cada massa para uma integral com respeito a ela: 
  \[
    \hypertarget{momentum_of_inertia_continuous}{\boxed{I = \int_{}^{}r^{2}dm.}}
  \]
  Para o caso de uma barra, por exemplo, na qual a distribuição de massa é dada por 
    \[
      \lambda = \frac{M}{L},
    \]
  segue que \(dm = \lambda dx \Rightarrow dI = x^{2}dm = x^{2}\lambda dx\). Portanto, 
    \[
      I = \lambda \int_{}^{}x^{2}dx = \lambda \frac{x^{3}}{3}.
    \]
  Por exemplo, se o tamanho da barra é 1 e o eixo de rotação está em uma extremidade, o momento de inércia será 
    \[
      I = \lambda \int_{0}^{1}x^{2}dx =\frac{1}{3}ML^{2}.
    \]
    Há outros casos importantes que devem ser tratados. O primeiro deles é o eixo central,
  no qual o eixo de rotação é posicionado na metade do tamanho da barra. Assim, 
    \[
      I = \lambda \int_{-\frac{1}{2}}^{\frac{1}{2}}x^{2}dx = \lambda \frac{x^{3}}{3}\biggl|_{-\frac{1}{2}}^{\frac{1}{2}}\biggr. = \lambda \frac{L^{3}}{12} = \frac{ML^{2}}{12}.
    \]
  O outro engloba a situação em que toda a massa na mesma distância. Neste caso, \(\lambda = \frac{M}{2\pi R}\)
    \[
      I = R^{2} \int_{}^{}dm = MR^{2}, 
    \]
  que também pode ser obtido fazendo uma integral com respeito ao ângulo \(\theta \): 
    \[
      I = R^{2}\lambda \int_{0}^{2\pi } R d\theta = R^{2}\lambda R\times2\pi = MR^{2}.
    \]
  Por fim, é importante olhar o caso dos discos. Discos consistem de dois círculos, um maior e outro menor dentro dele.
Chamaremos de R o raio do maior e de r o do menor. Para eles, há uma distribuição de massa
 \(\sigma = \frac{M}{\pi R^{2}}\), de maneira que o diferencial de massa será 
   \[
     dm = 2\pi r dr\sigma.
   \]
   Com isso, conseguimos encontrar que o momento de inércia é 
     \[
       I = \int_{0}^{R}2\pi \sigma r^{3}dr = 2\pi \sigma \frac{\pi^{4}}{4}\biggl|_{0}^{R}\biggr. = \frac{1}{2}MR^{2}
     \]

\newpage

\section{Aula 03 - 16/08/2023}
\subsection{Motivações}
\begin{itemize}
  \item Disco com buraco;
  \item Rodando disco e cilindro em torno do plano.
\end{itemize}
\subsection{Momento de Inércia em um Disco}
  Vamos considerar um disco de raio \(R_{2}\) que contém dentro de si um buraco de
raio \(R_{1}\). Nisso, consideramos o momento de inércia do disco inteiro como 
  \[
    I = I^{+} + I^{-}.
  \]
  Aqui, \(I^{+}\) desconsidera a existência do buraco, ou seja, tem valor 
    \[
      I^{+} = \frac{\pi R_{2}^{2}\sigma R_{2}^{2}}{2} = \frac{1}{2}M^{+}R_{2}^{2}
    \]
  e o valor de \(I^{-}\) vale 
    \[
      I^{-} = \frac{1}{2}M^{-}R_{1}^{2} = \frac{\pi R_{1}^{2}}{2}\sigma R_{1}^{2}.
    \]
    Assim, considerando o valor total, obtivemos o mesmo resultado que o de antes: 
      \[
        I = \frac{\pi \sigma }{2}(R_{2}^{4} - R_{1}^{4}).
      \]
    Em particular, a densidade de massa após o buraco ser feito, \(\sigma^{*} \), valerá
      \[
        \sigma ^{*} = \frac{M}{\pi(R_{2}^{2} - R_{1}^{2})},
      \]
    de forma que, através de \(I = \frac{\pi \sigma^{*}}{2}(R_{2}^{4} - R_{1}^{4}\), obtemos 
      \[
        I = \frac{\pi M}{2} \frac{(R_{2}^{2}-R_{1}^{2})(R_{2}^{2}+R_{1}^{2})}{\pi (R_{2}^{2}-R_{1}^{2})} = \frac{M}{2}(R_{2}^{2}+R_{1}^{2}).
      \]

    Agora, suponha que deixamos um disco girar em torno de um eixo com velocidade \(\omega \). Como podemos descrever esse sistema e seu momento de inércia? 
Faremos uso do Teorema dos Eixos Paralelos. Apesar de não conhecermos o momento de inércia, sabemos que em algum ponto, encontra-se o centro
de massa do objeto, estando a uma distância h do eixo. Este centro de massa move-se com velocidade \(\vec{v}_{cm}\). Como a energia cinética total
tem valor \(\mathbb{K}_{T} = \frac{1}{2}Mv_{cm}^{2} + \mathbb{K}_{relcm}\), utilizamos que \(\mathbb{K} = \frac{1}{2}I\omega^{2}\) e que \(\mathbb{K}_{relcm}=\frac{1}{2}I_{cm}\omega ^{2}\).
Logo, como \(v_{cm} = h\omega ,\)
 \begin{align*}
   \frac{1}{2}I\omega^{2} &= \frac{1}{2} Mv_{cm}^{2} + \frac{1}{2}I_{cm}\omega^{2}\\
                          &= Mh^{2}\omega^{2} + I_{cm}\omega ^{2}\\
                          &\Rightarrow I = Mh^{2} + I_{cm}.
 \end{align*}
 \begin{example}
   Considerando uma barra em a uma distância de \(\frac{L}{2}\) do eixo de rotação e com momento de inércia 
  \(I = \frac{1}{3}ML^{2},\) podemos utilizar a fórmula para obter 
 \begin{align*}
   &I = Mh^{2} + I_{cm}\\
   &\frac{1}{3}ML^{2} = M \frac{L^{2}}{4} + I_{cm}\\
   &I_{cm} = \frac{1}{12}ML^{2}.
 \end{align*}
 \end{example}
    Mas, o que aconteceria se o disco rodasse em eixos x, y contidos no plano do disco? O que sabemos é que 
      \[
        I_{z} = \sum\limits_{}^{}m_{i}r_{i}^{2}.
      \]
  Além disso, \(r_{i}^{2} = (x_{i}^{2} + y_{i}^{2})\), ou seja, 
 \begin{align*}
   I_{z} &= \sum\limits_{}^{}m_{i}x_{i}^{2} + \sum\limits_{}^{}m_{i}y_{i}^{2}\\
         &= I_{x} + I_{y}.
 \end{align*}
  Este resultado é conhecido como teorema dos eixos perpendiculares, mas vale apenas para corpos bidimensionais.
Em particular, no caso do cilindro, em que \(I_{x} = I_{y},\) 
  \[
    I_{x} = I_{y} \Rightarrow 2I_{x} = I_{z} \Rightarrow I_{x} = \frac{1}{4}MR^{2}.
  \]
  Além disso, considerando que \(dI_{x} = \frac{1}{4}dm R^{2} + dm z^{2}\), obtemos 
    \[
      I_{x} = \frac{1}{4}R^{2} \int_{}^{}dm + \int_{}^{}dm z^{2}.
    \]
    Como \(dm = \lambda dz = \frac{M}{L}dz\), em que L é o comprimento, segue o seguinte resultado 
      \[
        I_{x} = \frac{1}{4}R^{2}\int_{-\frac{L}{2}}^{\frac{L}{2}}\frac{M}{L}dz + \int_{-\frac{L}{2}}^{\frac{L}{2}}\frac{M}{L}z^{2}dz
      \]
    Assim, fazendo as contas, 
      \[
        I_{x} = \frac{1}{4}MR^{2} + \frac{ML^{2}}{12}
      \]

\begin{example}
  Considere uma barra de tamanho L e massa M e deixe-a descer em um pivô. Qual é a força que ele terá que fazer?

  Sabe-se que há uma força peso com módulo Mg agindo e que \(E_{mec_{i}} = E_{mec_{f}},\) tal que \(\mathbb{K}_{i} + U_{i} = \mathbb{K}_{f} + U_{f}\).
Mas, \(\mathbb{K}_{i} = U_{i} = 0\) e \(\mathbb{K}_{f} = \frac{1}{2}I\omega_{f}^{2}, U_{f} = Mg(\frac{-L}{2}).\) Assim, 
  \[
    \frac{1}{2}I\omega_{f}^{2} - Mg \frac{L}{2} = 0 \Rightarrow \omega_{f}^{2} = \frac{MgL}{I} = \frac{MgL}{\frac{1}{3}ML^{2}} = \frac{3g}{L}.
  \]
  Assim, usando que \(a_{cm} = r\omega_{f}^{2},\)
 \begin{align*}
   F- Mg &= Ma_{cm} \Rightarrow F = Mg + M \frac{L}{2}\omega_{f}^{2}\\
         &= Mg + \frac{ML}{2}\frac{3g}{L}\\
         &=Mg + \frac{3}{2}Mg = \frac{5}{2}Mg.
 \end{align*}
\end{example}
\begin{example}
  Considere uma roldana de raio R e massa \(m_{r}\). Atrele a ela, com uma corda de massa \(m_{c}\) e tamanho L, um balde de massa \(m_{b}\). 
Em seguida, solte-o para cair uma distância d. Qual é a velocidade do sistema?

  Sabemos que \(E_{mec_i} = E_{mec_f}\), ou seja, 
    \[
      \mathbb{K}_{i} + U_{i} = \mathbb{K}_{f} + U_{f}.
    \]
  Suponha que \(\mathbb{K}_{i} = U_{i} = 0.\) Quando o balde descer, sendo \(m_{c}^{*} = \frac{d}{L}m_{c}\) a massa da fração de corda que desceu, a potencial final passará a valer
 \(U_{f} = m_{b}(-d)g + m_{c}^{*}(\frac{-d}{2})g = -m_{b}gd - \frac{1}{2}m_{c}^{*}gd.\) Com relação à cinética, 
   \[
     \mathbb{K}_{f} = \frac{1}{2}m_{r}v^{2} + \frac{1}{2}m_{c}v^{2} + \frac{1}{2}m_{b}v^{2}.
   \]
   Utilizando as relações de energia que vimos, segue que 
  \begin{align*}
    &\mathbb{K}_{f} + U_{f} = 0\\
    &\Rightarrow \frac{1}{2}(m_{c}+m_{r}+m_{b})v^{2} = m_{b}gd + \frac{1}{2}m_{c}^{*}gd\\
    &\Rightarrow (m_{r}+m_{c}+m_{b})v^{2} = 2m_{b}gd + m_{c}g \frac{d^{2}}{L}\\
    &\Rightarrow v^{2} = \frac{(2m_{b}L + m_{c}d)}{m_{r}+m_{c}+m_{b}}\frac{gd}{L}\\
    &\Rightarrow v = \sqrt[]{\frac{(2m_{b}L + m_{c}d)}{m_{r}+m_{c}+m_{b}}\frac{gd}{L}}.
  \end{align*}
\end{example}
\newpage

\section{Aula 04 - 17/08/2023}
\subsection{Motivações }
\begin{itemize}
  \item Segunda lei de Newton do Movimento Circular;
  \item Torque da Gravidade.
\end{itemize}
\subsection{Segunda Lei de Newton}
Ao considerarmos uma força aplicada a um objeto em torno de um círculo de raio r, essa força faz um ângulo
 \(\theta \) com a paralela ao raio. Além disso, há uma componente dessa força que será tangente 
 à trajetória do objeto ao longo do círculo. Denotando essa segunda por \(F_{t},\) há duas formas de expressá-la: 
   \[
     F\sin{(\theta )} = F_{t},\quad F_{t} = ma_{t}.
   \]
   Além disso, a aceleração tangencial \(a_{t}\) satisfaz \(a_{t} = r\alpha \). Assim, obtemos a relação 
     \[
       F sin(\theta ) = ma_{t} \Rightarrow F\sin{\theta } = mr\alpha \Longleftrightarrow rF\sin{(\theta )} = mr^{2}\alpha.
     \]
     Esse termo à esquerda é conhecido como \textbf{torque} 
       \[
         \hypertarget{torque}{\boxed{\tau = mr^{2}\alpha = rF\sin{(\theta )}}}
       \]
  Em particular, sendo o torque total a soma de todos os torques, obtemos 
    \[
      \tau = \sum\limits_{}^{}\tau_{i} = \sum\limits_{}^{}m_{i}r_{i}^{2}\alpha = I\alpha 
    \]
    Uma propriedade é que a soma dos torques das forças internas vale zero.

    Olhando um caso mais específico, ao considerarmos um círculo de raio r e uma força que
faz um ângulo \(\theta \) com a paralela ao raio e outro círculo menor de raio r' com a mesma força aplicada,
mas ângulo \(\theta ',\) então \(l=r'\sin{(\theta ')}\) é a componente perpendicular à linha na qual a força está atuando.
  A vantagem disso é que o torque pode ser, então, expresso através de \(\tau = Fl = F_{t}r'\)
\subsection{O Torque da Força da Gravidade}
  Se considerarmos um corpo sofrendo a ação da força peso, o torque desse corpo pode ser descrito por
  \(\tau_{i} = m_{i}gx_{i}\) e, o torque total, será a soma desses torques: 
    \[
      \tau_{r} = \sum\limits_{}^{}\tau_{i} = \sum\limits_{}^{}[m_{i}x_{i}]g
    \]
  Mas, esse é exatamente o torque do centro de massa do objeto \(\tau_{r} = Mx_{cm}g\). Outro assunto que é
importante ressaltar é que, durante os estudos de dinâmica, a forma de estudar as forças em um sistema é através
dos chamados diagramas de força, o que traz à tona a questão do que funcionaria pro estudo do torque.
\begin{example}
  Considere uma roda de bicicleta e a catraca, que sofre uma força F de 18N. Suponha que o raio r
da catraca é de 7cm e o da roda, R, vale 35cm. Além disso, a massa vale 2.4kg. Qual é a velocidade angular para t=5,5s?

Começamos afirmando que o torque é \(\tau = I\alpha = Fr_{c}\). Assim, 
\[
  \alpha = \frac{Fr_{c}}{I} = \frac{Fr_{c}}{MR^{2}} = \frac{18 \cdot (0,07)}{2,4(0,35)^{2}}\frac{rad}{s^{2}}.
\]
Com isso, 
  \[
    \omega = \omega_{0} + \alpha t = \alpha t = \frac{18 \cdot (0,07)}{2,4(0,35)^{2}} \cdot 5,5 = 21,4 \frac{rad}{s}
  \]
\end{example}
\begin{example}
  Considere uma barra de massa m e comprimento l está presa por um pivô, o qual realiza uma força F. Após soltá-la, qual é a força que o pivô realiza?

  Sabemos que \(\tau = mg \frac{l}{2} = I\alpha = \frac{1}{3}ml^{2}.\) Logo, 
 \begin{align*}
   &mg \frac{l}{2} = \frac{1}{3}ml^{2}\alpha \\
   &\alpha = \frac{3}{2}\frac{g}{l}.
 \end{align*}
 Olhando no eixo y, sabemos que \(a_{cm_{y}} = r\alpha  = \frac{l}{2}\frac{3}{2}\frac{g}{l} = \frac{3}{4}g\), tal que 
   \[
     F - mg = -ma_{cm_y} \Rightarrow F = mg - \frac{3}{4}mg = \frac{1}{4}mg
   \]
\end{example}
\begin{example}
  Suponha que temos uma roldana de raio R e momento de inércia I. Pendura-se um corpo de massa m na roldana. Qual é a aceleração de queda do corpo?

\textbf{Roldana:}
As forças que atuam na roldana são o Peso dela, \(P_{r}\), a tensão T
e a força resultante ao peso \(F_{r}\). Assim, 
\begin{align*}
  &F_{r} = P_{r} + T\\
  &TR = I\alpha,\quad a = \alpha R. 
\end{align*}

\textbf{Corpo:}
No corpo, por outro lado, tem-se apenas a tensão T e o peso mg, de forma que 
  \[
    mg-T = ma.\quad a = \alpha R
  \]
\end{example}
  Continua na próxima aula...
  \newpage

\section{Aula 05 - 21/08/2023}
\subsection{Motivações}
\begin{itemize}
  \item Continuação do exemplo e outros;
  \item Potência;
  \item Corpos que rolam sem deslizar.
\end{itemize}
\subsection{Continuando o Exemplo}
\begin{example}[continuando...]
  Segue a relação de tração 
    \[
      TR = I \frac{a}{R} \Rightarrow T = \frac{I}{R^{2}}a.
    \] 
  Com isso, 
    \[
      mg = ma + T = ma + \frac{I}{R^{2}}a = a[1 + \frac{I}{mR^{2}} \Rightarrow a = \frac{g}{1 + \frac{I}{mR^{2}}}.
    \]
  Descobrimos, assim, os valores de T e de \(F_{s}\)
 \begin{align*}
   &T = \frac{I}{R^{2}}\frac{g}{1+ \frac{I}{mR^{2}}}\\
   &F_{s} = Mg + \frac{I}{R^{2}}\frac{g}{1+\frac{I}{mR^{2}}}.
 \end{align*}
 \end{example} 
\begin{example}
  Considere a máquina de Atroos - dois blocos presos a uma roldana, um de massa \(m_{1}\) e outro de massa \(m_{2}\) tais que \(m_{1} > m_{2}\).
A roldana tem massa M, momento de inércia I e raio R. Vejamos as forças

  \textbf{Bloco 1:}
    No primeiro bloco, agem forças de tração \(T_{1}\) e peso \(m_{1}g\). Escrevendo as equações,
   \begin{align*}
     m_{1}g - T_{1} = m_{1}a
   \end{align*}

  \textbf{Bloco 2:}
    No bloco dois, agem a tração \(T_{2}\) e o peso \(m_{2}g\)
   \begin{align*}
     T_{2}-m_{2}g = m_{2}a
   \end{align*}

  \textbf{Roldana:}
    Tem-se a equação 
   \begin{align*}
     (T_{1} - T_{2})R = I\alpha. \Longleftrightarrow T_{1} - T_{2} = \frac{I}{R}\frac{a}{R} = \frac{Ia}{R^{2}}
   \end{align*}

  Como a roldana está rodando, tem-se a relação \(m_{1g} > T_{1} > T_{2} > m_{2}g\)
A seguir, soma-se a equação do bloco 2 com a da Roldana, tal que 
  \[
    m_{1}g - m_{2}g - (T_{1}-T_{2}) = (m_{1}+m_{2})a \Longleftrightarrow (m_{1}-m_{2})g - \frac{Ia}{R^{2}} = (m_{1}+m_{2})a
  \]
  Isola-se a equação no a: 
    \[
      a = \frac{(m_{1}-m_{2})g}{(m_{1}+m_{2})+\frac{I}{R^{2}}}
    \]
\end{example}
\begin{example}
  Considere uma roldana com massa M, momento de inércia I e raio R presa à quina uma mesa. Atrela-se a ela dois corpos, um com massa \(m_{1}\) e que está em cima da mesa
e outro, de massa \(m_{2}\), suspenso pela corda. 
  \textbf{Corpo 1:}
    As forças atuando no bloco 1 são a normal \(F_{N_{1}}\), a peso \(m_{1}g\) e a tração \(T_{1}\), de forma que 
      \[
        T_{1}=m_{1}a
      \]

  \textbf{Corpo 2:}
    Para o bloco 2, podemos descrever o sistema considerando a tração \(T_{2}\) e o peso \(m_{2}g\), tal que 
      \[
        m_{2}g - T_{2} = m_{2}a.
      \]

  \textbf{Roldana:}
    As forças que atuam na roldana são a tração na direção do bloco 1, \(T_{1}\), a outra na do bloco 2, \(T_{2}\), o peso
  \(Mg\) e uma força da quinta nela \(\vec{F_{s}}\). Além disso, \(a=R\alpha \). A equação do sistema será
 \begin{align*}
   (T_{2}-T_{1})R = I\alpha \Rightarrow  T_{2} - T_{1} = \frac{Ia}{R^{2}}.
 \end{align*}

  Somando a equação do bloco 1 e a do bloco 2, chega-se em 
    \[
      m_{2}g - (T_{2}-T_{1}) = (m_{1}+m_{2})a
    \]
  Assim, 
 \begin{align*}
   &T_{2} - T_{1} = \frac{Ia}{R^{2}}\\
   &m_{2}g - \frac{Ia}{R^{2}} = (m_{1}+m_{2})a\\
   &a = \frac{m_{2}g}{(m_{1}+m_2) + \frac{I}{R^{2}}}
 \end{align*}
 Além disso, 
   \[
     T_{1} = \frac{m_{1}m_{2}g}{(m_{1}+m_{2})+\frac{I}{R^{2}}}.
   \]
\end{example}
\subsection{Potência}
  Previamente, a potência era dada pela relação \(dW = F ds.\) Considerando o caso de uma força agindo
emu ma situação circular, isso torna-se \(dW = FRd\theta \). No entanto, esse termo à direita lembra muito um torque. De fato,
a relação que obtemos é que \(dW = \tau d\theta \). Portanto, 
  \[
    \hypertarget{power_torque}{\boxed{P = \frac{dW}{dt} = \tau \frac{d\theta }{dt} = \tau \omega }}
  \]
 \begin{example}
   Um motor de combustão de um carro fornece um torque de \(\tau  = 678Nm\) e está rodando a \(\omega = 4500rpm \approx 471 \frac{rad}{s}\).
Com isso, a potência será 
  \[
    P\approx 315kW.
  \]
 \end{example}
\begin{example}
  Tome uma roda gigante \textbf{(em Londres).} Ela tem um diâmetro de \(135m\), pesa 1600 toneladas e dá 2 revoluções por hora.
Qual é o torque necessário para parar a roda em 10m?
  
  Para começar, observe que \(W = \tau \Delta \theta \) e que \(S = R\Delta \theta = 10m.\) Em particular, temos o valor de R, tal que 
    \[
      S = R\Delta \theta \Longleftrightarrow 10 = 67.5\Delta \theta \Rightarrow \Delta \theta \approx 0,148rad.
    \]
  Note que 
    \[
      W = -(\mathbb{K}_{f} - \mathbb{K}_{i}) \Longleftrightarrow \tau \Delta \theta = -\biggl[0 - \frac{I\omega^{2}}{2}\biggr].
    \]
  Logo, convertendo \(\omega \) para radianos por segundo (\(\omega = 3,5 \cdot 10^{-3}\frac{rad}{s}\),
    \[
      \tau = \frac{I\omega^{2}}{2\Delta \theta } = \frac{MR^{2}\omega^{2}}{2\Delta \theta } \approx 3 \cdot 10^{5}Nm.
    \]
    Em particular, 
      \[
        F = \frac{\tau }{R}\approx 4,4 \cdot 10^{3}N.
      \]
\end{example}
\subsection{Corpos que Rolam sem Deslizar}
  Imagine um sistema em que um disco de raio R está a rolar com velocidade do centro de massa \(\vec{v}_{cm}.\) Considerando o ponto que tangencia o chão, em que a velocidade é nula,
ele se mexe com velocidade angular \(\omega \) em um raio \(\vec{r}\). Assim, a energia cinética desse sistema será 
  \[
    \mathbb{K}_{T} = \frac{1}{2} Mv_{cm}^{2} + \mathbb{K}_{rel} = \frac{1}{2}Mv_{cm}^{2} + \frac{1}{2}I_{cm}\omega ^{2},
  \]
  em que considera-se que \(v_{cm} = R\omega.\) 

  Agora, considere um plano inclinado e uma bola de massa m, momento de inércia I e raio R que irá subir este plano inclinado até parar. Como podemos achar a altura que ela para, 
fornecida velocidade inicial do centro de massa \(v_{cm}\). Utilizando a conservação da energia mecânica, 
  \[
    E_{mec_{i}} = E_{mec_{f}}.
  \]
  Sabemos que 
    \[
      E_{mec_{i}} = \frac{1}{2}mv_{cm}^{2} + \frac{I_{cm}}{2}\omega ^{2}\quad \& E_{mec_{f}} = mgh.
    \]
  Assim, 
  \begin{align*}
    &mgh = \frac{1}{2} mv_{cm}^{2} + \frac{I_{cm}}{2}\omega^{2}\\
    &\Rightarrow h = \frac{1}{2}\biggl[v_{cm}^{2} + \frac{I_{cm}}{2m}\frac{v_{cm}^{2}}{R^{2}}\biggr]\\
    &\Rightarrow h = \frac{v_{cm}^{2}}{2g}\biggl[1 + \frac{I_{cm}}{mR^{2}}\biggr].
  \end{align*}
 \begin{example}
   Para o caso da esfera, em que \(I = \frac{2}{5}mR^{2},\) 
     \[
       h = \frac{v_{cm}^{2}}{2g}\biggl[1 + \frac{2}{5}\biggr] = \frac{7}{10}\frac{v_{cm}^{2}}{g}
     \]
 \end{example}
\begin{example}
  Considere um cenário de sinuca em que um taco aplica uma força F. Se ela é aplicada acima do eixo de rotação, a bola roda para frente. Caso seja exatamente no eixo de rotação,
ela apenas deslizará. Por fim, se for atingida abaixo do eixo de rotação, ela rodará ao contrário. Como fazer ela não rodar?

  Em qualquer um desses pontos, a força é \(F=ma.\) No caso em que ela roda para frente, ou seja, é atingida a uma distância d acima do eixo de rotação,
temos \(\tau = Fd = I\alpha.\) Segue que, para que ela não rode, 
\begin{align*}
  & Fd = I\alpha = \frac{Ia}{R}\\
  &\Rightarrow d = \frac{I}{mR} = \frac{2}{5}\frac{mR^{2}}{mR} = \frac{2}{5}R.
\end{align*}
\end{example}
\newpage

\section{Aula 06 - 23/08/2023}
\subsection{Motivações}
\begin{itemize}
  \item Planos inclinados;
  \item Refrigerante VS Cilindro;
  \item Bola deslizando.
\end{itemize}
\subsection{Objeto Rolante em Plano Inclinado}
  Coloque uma bola de massa M, momento de inércia I e raio R em um plano inclinado a uma ângulo \(\theta \).
Analisando as forças presentes, estão inclusas a da gravidade, a normal e uma força de atrito. Considerando x a direção em que a bola
está indo para. Nessa direção, as relações das forças serão 
  \[
    Mg\sin{(\theta )} - f_{at} = Ma_{cm}.
  \]
  Quanto à rotação, temos 
    \[
      f_{at}R = I_{cm}\alpha.
    \]
  Utilizando \(a_{cm} = R\alpha \), relacionamos as duas como segue:
 \begin{align*}
   &f_{at} = Mg\sin{(\theta )} - Ma_{cm}\\
   &Mg\sin{(\theta )} - Ma_{cm} = \frac{I_{cm}}{R}\frac{a_{cm}}{R}\\
   &Mg\sin{(\theta )} = a_{cm}\biggl[M + \frac{I_{cm}}{R^{2}}\biggr]\\
   &Mg\sin{(\theta )} = Ma_{cm}\biggl[1 + \frac{I_{cm}}{MR^{2}}\biggr]\\
   &a_{cm} = \frac{g\sin{(\theta )}}{1 + \frac{I_{cm}}{MR^{2}}}.
 \end{align*}
  Com isso, conseguimos descrever a força de atrito utilizando que, na esfera, \(I = \frac{2}{5}MR^{2}\). Logo, 
 \begin{align*}
   &a_{cm} = \frac{5}{7}g\sin{(\theta )}\\
   &f_{at} = \frac{2}{5}\frac{MR^{2}}{R^{2}}\frac{5}{7}g\sin{(\theta )} = \frac{2}{7}Mg\sin{(\theta )}.
 \end{align*}
\subsubsection{Simetrias}
  Se um corpo tem simetria esférica ou cilíndrica, então \(I = \beta mR^{2}\). Em particular, podemos usar isso para
generalizar o raciocínio realizado acima. De fato, tanto \(a_{cm}\) quanto \(f_{at}\) podem ser reescritos como segue:
\begin{align*}
  &a_{cm} = \frac{g\sin{(\theta )}}{1+\beta }\\
  f_{at} = \frac{\beta Mg\sin{(\theta )}}{1+\beta }& = \frac{Mg\sin{(\theta )}}{\frac{1}{\beta }(1+\beta )} = \frac{Mg\sin{(\theta )}}{\beta^{-1}+1}.
\end{align*}
  Observa-se, assim, que entre uma esfera, um cilindro e um aro num plano inclinado (de mesmo raio), a esfera chegará primeiro ao chão,
visto que ela tem um \(\beta \) menor.
  
  Agora, considere um cilindro de massa M e raio R iguais aos de uma lata de refrigerante e coloque os dois juntos para descer um plano inclinado. Quem chegará primeiro?
A resposta (por incrível que pareça) é o refrigerante, pois o liquido dentro não possui momento de inércia, ou seja, soltar ela com líquido ou vazia resultará no mesmo! Analisando a energia
desse sistema, temos 
  \[
    E_{mec_{i}} = mgh\quad \& E_{mec_{f}} = \frac{1}{2}mv_{cm}^{2} + \frac{1}{2}I_{cm}\omega^{2},
  \]
  em que \(\omega = \frac{v_{cm}}{R}\). Assim, 
    \[
      E_{mec_{f}}=\frac{1}{2}mv_{cm}^{2} + \frac{1}{2}\beta mR^{2}\frac{v_{cm}^{2}}{R^{2}}.
    \]
  Juntando ambas, 
    \[
      \frac{1}{2}mv_{cm}^{2}[1+\beta ] = mgh \Rightarrow v_{cm}^{2} = \frac{2gh}{1+\beta }
    \]
  Como a latinha é considerada um aro, ela ganha!

  Vamos voltar nossa atenção à força de atrito. Vimos antes que 
    \[
      f_{at} = \frac{mg\sin{(\theta )}}{1 + \beta^{-1}}.
    \]
  Sabe-se que a força de atrito tem um valor máximo, dado por \(f_{at_{max}} = \mu_{e}N\), ou seja,
 \(f_{at}\leq f_{at_{max}}\). Em outras palavras, 
   \[
     \frac{mg\sin{(\theta )}}{1+\beta^{-1}}\leq \mu_{e}mg\cos{(\theta )} \Longleftrightarrow \tan{(\theta )}\leq \mu_{e}(1+\beta^{-1}).
   \]
   Com isso, obtemos um ângulo máximo no qual o plano inclinado pode estar sem que a bola comece a rolar deslizando.
  \begin{example}
    Para uma esfera, em que \(\beta =\frac{2}{5}\), o ângulo máximo é 
      \[
        \tan{(\theta )}\leq 3.5\mu_{e}.
      \]
    Para um cilindro, com \(\beta = \frac{1}{2}\), tem-se 
      \[
        \tan{(\theta )}\leq 3\mu_{e}.
      \]
    Por fim, para um aro, no qual \(\beta =1\), 
      \[
        \tan{(\theta )}\leq 2\mu_{e}.
      \]
  \end{example}

  \subsection{Bola Deslizando}
    Agora, suponha que colocamos uma bola de massa m deslizando (apenas movimento translacional) com velocidade v em uma superfície com atrito. Note que as forças agindo são
  a peso, a normal e a força de atrito. Em termos de translação, então, temos 
    \[
      -f_{at} = ma_{cm} \Rightarrow a_{cm} = -\frac{f_{at}}{m} = -\frac{\mu_{c}N}{m} = -\mu_{c}g.
    \]
  Por outro lado, quanto à rotação, começamos notando que o torque vale \(\tau = I_{cm}\alpha \). Assim, 
  \[
    f_{at}R = I_{cm}\alpha \Longleftrightarrow a_{c}mgR = \frac{2}{5}mR^{2}\alpha
  \]
  Portanto, 
    \[
      \alpha = \frac{5}{2}\frac{\mu_{c}g}{R}.
    \]
  A velocidade do centro de massa, então, pode ser obtida com \(v_{cm} = v - a_{cm}t = v - \mu_{c}gt\). Nota-se, então,
que há uma redução na velocidade, até que, eventualmente, ela para de deslizar e passa a rotacionar, fazendo com que \(v_{cm} = \omega R\). 
Nesse instante, tem-se 
  \[
    v-\mu_{c}gt = R\frac{5}{2}\frac{\mu_{c}g}{R}t \Rightarrow v = \mu_{c}gt\biggl[1+\frac{5}{2}\biggr] = \mu_{c}gt \frac{7}{2}.
  \]
Portanto, o tempo parar ela parar de deslizar é 
  \[
    \boxed{t^{*} = \frac{2v}{7\mu_{c}g}.}
  \]
  A partir disso, podemos também encontrar a distância percorrida:
 \begin{align*}
   x(t^{*}) &= vt^{*} - \frac{\mu_{c}gt^{*^{2}}}{2}\\
            &= \frac{v2v}{7\mu_{c}g} - \frac{\mu_{c}g}{2}\frac{4v^{2}}{49\mu_{c}^{2}g^{2}}\\
            &= \frac{2v^{2}}{7\mu_{c}g} - \frac{2}{49}\frac{v^{2}}{\mu_{c}g}\\
            &= \frac{v^{2}}{\mu_{c}g}\biggl[\frac{14-2}{49}\biggr] = \boxed{\frac{12}{49}\frac{v^{2}}{\mu_{c}g}.}
 \end{align*}
 \newpage

\section{Aula 07 - 24/08/2023}
\subsection{Motivações} 
\begin{itemize}
  \item A natureza vetorial do torque. 
\end{itemize}
\subsection{Vetores}
  Vamos começar com um exemplo de como a nossa construção atual falha em algumas descrições.
Quando um peão está rodando, ou um giroscópio, o torque não é apenas em um dimensão. Assim, não podemos usar nosso modelo atual.

  A descrição do torque como grandeza vetorial é simplesmente \(\vec{\tau } = I \vec{\alpha } = I \frac{d \vec{\omega }}{dt}\). Disto, observa-se 
que a direção dele é a mesma da velocidade angular \(\vec{\omega }\). Considerando uma situação tridimensional e colocarmos 
 \(\vec{r}\) e a força \(\vec{F}\), fazendo ângulo \(\varphi \) com a reta de \(\vec{r}\), no plano xy, o torque sairá na direção z, visto que ele é o produto vetorial da força com 
 \(\vec{r}\): \(\vec{\tau } = \vec{r}\times \vec{F}\). Em particular, segue que \(|\vec{\tau }| = rF\sin{(\varphi )}\)
Para operacionalizar melhor, vamos padronizar \(\hat{i}, \hat{j}, \hat{k}\) como os versores nas direções x, y e z respectivamente. Segue que 
\begin{itemize}
  \item[a)] \(\vec{i}\times \vec{j} = \vec{k}\);
  \item[b)] \(\vec{j}\times \vec{k} = \vec{i}\);
  \item[c)] \(\vec{k}\times \vec{i} = \vec{j}\);
\end{itemize}
  Além disso, vetores em três dimensões serão da forma \(\vec{A} = A_{x}\hat{i} + A_{y}\hat{j} + A_{z}\hat{k}\) e \(\vec{B} = B_{x}\hat{i} + B_{y}\hat{j} + B_{z}\hat{k}\). 
Lembre-se que \(\vec{C} = \vec{A} \times \vec{B} = -\vec{B}\times \vec{A}\) e \(\vec{A} \times \vec{A} = 0.\) Assim, 
\begin{align*}
  \vec{A} \times \vec{B} &= (A_{x}\hat{i} + A_{y}\hat{j} + A_{z}\hat{k})\times(B_{x}\hat{i} + B_{y}\hat{j} + B_{z}\hat{k})\\
                         &= \hat{i}\biggl[A_{y}B_{z} - A_{z}B_{y}\biggr] + \hat{j}\biggl[-A_{x}B_{z} + A_{z}B_{x}\biggr] + \hat{k}\biggl[A_{x}B_{y}-A_{y}B_{x}\biggr]\\
                         &= \biggl[A_{y}B_{z} - A_{z}B_{y}\biggr]\hat{i} + \biggl[A_{z}B_{x} - A_{x}B_{z}\biggr] + \biggl[A_{x}B_{y}-A_{y}B_{x}\biggr]\hat{k}.
\end{align*}
  Outra propriedade é com relação ao quadrado do produto vetorial:
 \begin{align*}
   &|\vec{A}\times \vec{B}|^{2} = |\vec{A}|^{2}|\vec{B}|^{2}\sin^{2}{(\varphi )}\\
   &|\vec{A}\cdot \vec{B}|^{2} = |\vec{A}|^{2}|\vec{B}|^{2}\cos^{2}{(\varphi )}\\
   &\Rightarrow \frac{|\vec{A}\cdot \vec{B}|^{2}}{|\vec{A}|^{2}|\vec{B}|^{2}} + \frac{|\vec{A}\times \vec{B}|^{2}}{|\vec{A}|^{2}|\vec{B}|^{2}} = 1\\
   &\Rightarrow  |\vec{A}\cdot \vec{B}|^{2} + |\vec{A}\times \vec{B}|^{2} = |\vec{A}|^{2}|\vec{B}|^{2}.
 \end{align*}
 \newpage

\section{Aula 08 - 28/08/2023}
\subsection{Motivações}
\begin{itemize}
  \item Momento Angular;
  \item Torque em termos do produto vetorial.
\end{itemize}
\subsection{Um pouco mais de produto vetorial.}
 \begin{example}
  Considere que \(A = 2 \hat{j}\) e \(B = 5\hat{i} + 2 \hat{j}\). Então, 
     \[
       \vec{A} \times{\vec{B}} = 2\hat{j} \times (5\hat{i} + 2\hat{j}) = 2\hat{i}\times 5\hat{j} = -10\hat{k}.
     \]
  Além disso, \(|\vec{A}\times \vec{B}|^{2} = |\vec{A}|^{2}|\vec{B}|^{2} - (\vec{A}\cdot \vec{B})^{2} = 100.\) Note que 
 \(|\vec{A}|^{2}|\vec{B}|^{2} = 4 (25+4) = 29\times 4\) e que \(\vec{A}\cdot \vec{B} = 4, \) ou seja, 
   \[
     |\vec{A}|^{2}|\vec{B}|^{2} - (\vec{A}\cdot \vec{B})^{2} = 4\times 25 + 16 - 16 = 100.
   \]
 \end{example}
 \begin{example}
   Com os mesmos A e B de antes, coloque \(\vec{C} = 3\hat{j} + 2 \hat{k}.\) Vamos calcular \(\vec{A}\times(\vec{B} + \vec{C}):\)
  \begin{align*}
    \vec{a}\times(\vec{B}+\vec{C}) &= 2\hat{j} \times [5\hat{i} + 5\hat{j} + 2\hat{k}]\\
                                   &= - 10\hat{k} + 4\hat{i} = 4\hat{i} - 10\hat{k}.
  \end{align*}
  Por outro lado, 
 \begin{align*}
   \vec{A} \times \vec{B} + \vec{A} \times \vec{C} &= 2\hat{j}\times(5\hat{i} + 2\hat{j}) + 2\hat{j}\times(3\hat{j}+2\hat{k})\\
                                                   &= 4\hat{i} - 10\hat{k}.
 \end{align*}
 Ou seja, os dois de fato coincidem.
 \end{example}
 \begin{example}
  Agora, olhemos para \(\vec{A}\times(\vec{B}\times \vec{C}) = (\vec{A}\cdot \vec{C})\vec{B} - (\vec{A}\cdot \vec{B})\vec{C}\): 
 \begin{align*}
   \vec{A}\times(\vec{B}\times \vec{C}) &= 2\hat{j}\times \biggl[(5\hat{i} + 2\hat{j})\times(3\hat{j}+2\hat{k})\biggr] \\
                                        &= 2\hat{j}\times [15\hat{k} - 10\hat{j} + 4\hat{i}] = 30\hat{i} - 8\hat{k}.
 \end{align*}
 Quanto ao lado direito da igualdade,
\begin{align*}
  &(\vec{A}\cdot \vec{C})\vec{B} = (2\hat{j}\cdot (3\hat{j}+2\hat{k}))(5\hat{i}+2\hat{j}) = 30\hat{i} + 12\hat{j}\\
  &(\vec{A}\cdot \vec{B})\vec{C} = (2\hat{j}\cdot (5\hat{i} + 2\hat{j}))(3\hat{j} + 2\hat{k}) = 12\hat{j} + 8\hat{k}\\
  (\vec{A}\cdot \vec{C})\vec{B} - &(\vec{A}\cdot \vec{B})\vec{C} = 30\hat{i} + 12\hat{j} - 12\hat{j} - 8\hat{k} = 30\hat{i} - 8\hat{k}.
\end{align*}
 \end{example}
 A seguir, vamos considerar um exemplo mais aplicado. 
\begin{example}
  Considere que o movimento de uma partícula em duas dimensões é descrito por \(\vec{r}(t) = v_{0}t\hat{i} + y_{0}\hat{j}\), sendo \(y_{0}\) o tanto que ela se moveu no eixo y.
Temos \(\frac{d \vec{r}}{dt} = v_{0}\hat{i} = \vec{v}\) e, considerando \(\vec{B} = B_{0}t \hat{j}\), segue que 
\begin{align*}
  \vec{r}\times \vec{B} &= (v_{0}t\hat{i} + y_{0}\hat{j})\times(B_{0}t\hat{j})\\
                        &= v_{0}t^{2}B_{0}\hat{k}.  
\end{align*}
  Em particular, 
    \[
      \frac{d(\vec{r}\times \vec{B})}{dt} = 2v_{0}tB_{0}\hat{k}.
    \]
  De fato, em geral, temos uma versão da regra do produto para produtos escalares:
    \[
      \hypertarget{vector_product_rule}{\boxed{\frac{d(\vec{r}\times \vec{B})}{dt} = \frac{d \vec{r}}{dt}\times \vec{B} + \vec{r}\times \frac{d \vec{B}}{dt}.}}
    \]
\end{example}
  Essas formas que lidamos com produtos escalares até agora funcionam bem para vetores mais simples. No entanto, quando mais termos começam a surgir, pode tornar-se algo explicado
muito rapidamente. Para isso, é útil ter em mente a forma que o produto escalar realmente toma - a de um determinante de uma matriz. 
  \[
    \hypertarget{vector_product}{    \vec{A}\times \vec{B} = \det \begin{pmatrix}
          \hat{i} & \hat{j} & \hat{k}\\
          A_{x} & A_{y} & A_{z}\\
          B_{x} & B_{y} & B_{z}
      \end{pmatrix} = \boxed{\hat{i}(A_{y}B_{z} - A_{z}B_{y}) + \hat{j}(A_{z}B_{x} - A_{x}B_{z}) + \hat{k} (A_{x}B_{y}-A_{y}B_{x}).}}
  \]
  Estamos, agora, habilitados para aplicar esses conceitos na física.
\subsection{Momento Angular}
  Considere um sistema xyz e um vetor \(\vec{r}\) contido no plano xy. Aplica-se uma força \(\vec{F}\), também contida no plano xy. 
Definimos, sendo \(\vec{p} = m \vec{v}\) o momento linear, o momento angular da partícula como 
  \[
    \hypertarget{angular_momentum}{\boxed{\vec{L} = \vec{r}\times \vec{p}}.}
  \]
  Equivalentemente, \(\vec{L} = \vec{r}\times m \vec{v} = m(\vec{r}\times \vec{v}) = mrv\hat{k}.\) Fazendo
 \(v = r\omega,\) segue que \(\vec{L} = mr^{2}\omega \hat{k} = I\omega \hat{k} = I \vec{\omega }\). 

  Com relação ao torque, sabe-se que \(\vec{\tau} = \vec{r}\times \vec{F}\). Além disso, 
    \[
      \frac{d \vec{L}}{dt} = \frac{d \vec{r}}{dt}\times \vec{p} + \vec{r} \times \frac{d \vec{p}}{dt} = \vec{v}\times m \vec{v} + \vec{r}\times \frac{d \vec{p}}{dt}
    \]
  Mas, \(\frac{d \vec{p}}{dt} = \vec{F}\), tal que 
    \[
      \vec{r} \times \frac{d \vec{p}}{dt} = \vec{r}\times \vec{F} = \tau.
    \]
  Em outras palavras, 
    \[
      \vec{\tau } = \frac{d \vec{L}}{dt} = I \frac{d \vec{\omega }}{dt} = I \vec{\alpha }.
    \]
  Quando lidamos com movimento linear, havíamos definido a quantidade ``Impulso'' como a variação do momento linear. Fazemos o análogo aqui: 
    \[
      \Delta \vec{L} = \int_{t_{1}}^{t_{2}}\vec{\tau }dt.
    \]
  Com a partícula do início da seção, segue que 
    \[
      \vec{L} = m(v_{0}t\hat{i} + y_{0}\hat{j})\times v_{0}\hat{i} = -my_{0}v_{0}\hat{k}.
    \]
  Para uma partícula com movimento descrito por \(\vec{r}(t)\) no plano xyz, decompomos esse vetor como sendo 
    \[
      \vec{r} = \vec{r}_{rad} + \vec{r}_{z}.
    \]
  Aplicando uma força \(\vec{F}\), na mesma lógica, faremos 
    \[
      \vec{F} = \vec{F}_{xy} + \vec{F}_{z}
    \]
  Com isso, o torque é dado por 
  \[
    \vec{\tau } = \vec{r}\times \vec{F} = (\vec{r}_{rad} + \vec{r}_{z})\times(\vec{F}_{xy} + \vec{F}_{z})
  \]
  A componente na direção z do torque, \(\vec{\tau }_{z}\), pode ser encontrada no termo 
    \[
      \vec{\tau }_{z} = \vec{r}_{rad} + \vec{F}_{xy}.
    \]
  No entanto, observe que \(\vec{L} = \vec{r}\times (\vec{p}_{xy} + \vec{p}_{z}) = (\vec{r}_{rad} + \vec{r}_{z}) + (\vec{p}_{xy}+\vec{p}_{z})\), tal que,
chamando de \(\vec{L}_{z} = \vec{r}_{rad}\times \vec{p}_{xy}\) a componente z do momento angular, chegamos em 
  \[
    \vec{\tau }_{z} = \frac{d \vec{L}_{z}}{dt}.
  \]
  \begin{example}
    Considere a máquina de Atwood com dois corpos tais que \(m_{1} > m_{2}\). Sendo I o momento de inércia da polia, M sua massa e R seu raio, o momento angular total na direção z desse sistema será 
      \[
        L_{total_z} = m_{1}vR + m_{2}vR + I\omega.
      \]
    O torque em z desse sistema é 
      \[
        \tau_{z} = m_{1}gR - m_{2}gR = (m_{1}-m_{2})gR = \frac{dL_{z}}{dt}.
      \]
    Com isso, sendo \(I = \frac{1}{2}mR^{2}\) e \(a  = R\alpha \),
   \begin{align*}
     (m_{1}-m_{2})gR &= m_{1}aR + m_{2}aR + I\alpha \\
                     &= (m_{1}+m_{2}+\frac{1}{2}MR^{2})aR.\\
   \end{align*}
   Portanto, 
  \begin{align*}
    &a = \frac{(m_{1}-m_{2})g}{m_{1}+m_{2}+\frac{M}{2}}\\
    &\alpha = \frac{1}{R}\frac{(m_{1}-m_{2})g}{m_{1}+m_{2}+\frac{M}{2}}.\
  \end{align*}
  \end{example}
\newpage

\section{Aula 09 - 30/08/2023}
\subsection{Motivações}
\begin{itemize}
  \item Giroscópio;
  \item Conservação de Momento Angular.
\end{itemize}

\subsection{O Giroscópio}
  Considere uma barra apoiado a uma haste e passando por um disco de grande massa. Uma das forças atuando nele é a normal, atuando na barra, o peso, com valor M \(\vec{g}\) no centro de massa do disco
e considere \(\vec{r}_{cm}\) o raio da barra até o centro de massa. Considere que o disco tem um momento de inércia \(I_{s}\) e roda com velocidade angular \(\omega_{s}\). Por ele estar rodando,
há um momento angular \(\vec{L}\)
Calculando o torque nesse sistema, 
  \[
    \vec{\tau } = \vec{r}_{cm}\times M \vec{g},\quad |\vec{r}_{cm}| = D,
  \]
observa-se que ele faz um ângulo \(\theta\) que vale 90 graus com o momento angular. Passado um tempo \(\Delta t\), o torque estará
em \(\vec{\tau }\Delta t\) e L irá para \(L'\), mas, como o torque não pode mudá-lo em valor absoluto, \(|\vec{L}|=|\vec{L}'|\). Com isso, \(\vec{L}' = \vec{L} + \vec{dL}\),
em que \(\vec{dL} = \vec{\tau} dt\). No entanto, sabemos, nesse caso, quanto vale o torque, tal que 
  \[
    dL = MgDdt.
  \]
  O ângulo \(d\varphi \) entre L e L', também, será dado por \(d\varphi = \frac{dL}{L} = \frac{MgD}{L}dt = \frac{MgD}{I_{s}\omega_{s}}.\) A partir disso,
chamamos de velocidade de precessão o valor 
  \[
    \frac{d\varphi }{dt}= \omega_{p} = \frac{MgD}{I_{s}\omega_{s}}.
  \]
  ``Velocidade de precessão'' é o nome dado ao fenômeno responsável por fazer o momento angular ``seguir'' o torque quando o giroscópio está rodando - o movimento circular do eixo de rotação. Com isso,
vimos que o torque resultante externo é dado por 
  \[
    \vec{\tau }_{res_{ext}} = \frac{d \vec{L}}{dt},
  \]
  ou seja, quando não há torque externo, o momento angular é constante!

  Agora, assuma dois corpos de massas \(m_{1}, m_{2}\) que exercem forças \(\vec{F}_{12}\) e \(\vec{F}_{21}\) uma na outra. Além disso, coloque-as
a distâncias \(\vec{r}_{1}\) e \(\vec{r}_{2}\) de um referencial 0 (O desenho forma um triângulo). O torque total dessas forças será 
  \[
    \vec{\tau}_{T} = \vec{r}_{1}\times \vec{F}_{12} + \vec{r}_{2}\times \vec{F}_{21} = \vec{r}_{1}\times \vec{F}_{12}- \vec{r}_{2}\times \vec{F}_{12} = (\vec{r}_{1}-\vec{r}_{2})\times \vec{F}_{12},
  \]
  mas \(\vec{r}_{1}-\vec{r}_{2}\) é exatamente a distância entre os dois corpos, ou seja, a força é paralela a essa distância: \(\vec{r}_{1}-\vec{r}_{2}\parallel \vec{F}_{12}.\) 
  
 \begin{example}
   Suponha uma colisão inelástica entre corpos e imagine que o primeiro corpo tem momento de inércia \(I_{1}\), velocidade angular inicial \(\omega_i = \omega_{0}\),
   enquanto o segundo tem \(I_{2}, \omega _{i} = 0\). Após a colisão, quais são o momento de inércia e a velocidade angular finais?

 Aqui, o momento angular deve ser conservado - \(L_{i} = L_{f}\). Assim,
   \[
     I_{1}\omega_{0} = (I_{1}+I_{2})\omega_{f} \Rightarrow \omega_{f} = \frac{I_{1}}{I_{1}+I_{2}}\omega_{0}
   \]
 Quanto à energia cinética, 
   \[
     \mathbb{K} = \frac{1}{2}mv^{2} = \frac{m^{2}v^{2}}{2m} = \frac{p^{2}}{2m}
   \]
  Analogamente,
  \[
   \mathbb{K} = \frac{1}{2}I\omega^{2} = \frac{I^{2}\omega^{2}}{2I} = \frac{L^{2}}{2I}\\
  \]
  Aplicando isso aos corpos do exercício, 
 \begin{align*}
   &\mathbb{K}_{i} = \frac{1}{2}\frac{L^{2}}{I_{1}}\\
   &\mathbb{K}_{f} = \frac{1}{2}\frac{L^{2}}{I_{1}+I_{2}}\\
   \Rightarrow& \frac{\mathbb{K}_{i}}{\mathbb{K}_{f}} = \frac{I_{1}+I_{2}}{I_{1}} = 1 + \frac{I_{2}}{I_{1}}
 \end{align*}
\end{example}
\begin{example}
  Dado um disco preso a um ponto em seu centro, faça uma colisão inelástica dele com um objeto de massa m e velocidade v. O momento de inércia do disco é I, sua massa é M e seu raio é R.
Após a colisão, a massa gruda bem na borda do disco, passando a rodar com velocidade \(\omega \). Quanto vale essa velocidade e qual é a razão entre as energias cinéticas?

  Novamente, usando a conservação de momento angular, \(L_{i} = L_{f}\). Aqui, 
 \begin{align*}
   &L_{i} = mvR\\
   &L_{f} = (I+mR^{2})\omega\\
   \Rightarrow& mvR = (I+mR^{2})\omega\\
   \Rightarrow& \omega = \frac{mvR}{I+mR^{2}} = \frac{mRv}{mR^{2}(1+\frac{I}{mR^{2}})} = \frac{v}{R(1+\frac{I}{mR^{2}})}
 \end{align*}
  Sabemos de antes que 
 \begin{align*}
   &\mathbb{K}_{f}=\frac{L^{2}}{2I_{f}}\\
   &\mathbb{K}_{i} = \frac{1}{2}mv^{2}\\
   \Rightarrow& \frac{\mathbb{K}_{i}}{\mathbb{K}_{f}} = \frac{1}{2}\frac{mv^{2}}{\frac{1}{2}\frac{m^{2}v^{2}R^{2}}{I+mR^{2}}} = \frac{1}{mR^{2}}(I+mR^{2}) = 1 + \frac{I}{mR^{2}}
 \end{align*}
\end{example}

  Suponha que temos um eixo xyz, um objeto em movimento circular no plano xy com velocidade v e massa m, a uma distância \(\vec{R}\) do centro.
Escolhendo o centro das coordenadas, o momento angular \(\vec{L}\) será na direção z, já que 
  \[
    \vec{L} = \vec{R}\times m \vec{v} = \vec{R}\times \vec{p}.
  \]
  Considerando, por outro lado, um ponto abaixo do plano xy, formando um comprimento \(\vec{R}'\) até o círculo de antes, seguirá que \(\vec{L}'\), perpendicular a 
 \(\vec{R}'\), não será na direção z. Além disso, conforme a partícula precessiona, ele também moverá-se, formando um cone. Num sistema em que há uma outra partícula diametralmente
 oposta fazendo a mesma coisa, a soma dos dois momentos angulares apontaria, sim, para o eixo z.
\end{document}

