\documentclass{article}
 \usepackage{amsmath}
 \usepackage{amsthm}
 \usepackage{amssymb}
 \usepackage{pgfplots}
 \usepackage[utf8]{inputenc}
 \usepackage{amsfonts}
 \usepackage[margin=2.5cm]{geometry}
 \usepackage{graphicx}
 \usepackage[export]{adjustbox}
 \usepackage{fancyhdr}
 \usepackage[portuguese]{babel}
 \usepackage{hyperref}
 \usepackage{lastpage}
 \usepackage{mathtools}
 \setcounter{section}{-1}

 \pagestyle{fancy}
 \fancyhf{}

 \pgfplotsset{compat = 1.18}

 \hypersetup{
     colorlinks,
     citecolor=black,
     filecolor=black,
     linkcolor=black,
     urlcolor=black
 }
 \newtheorem*{def*}{\underline{Defini\c c\~ao}}
 \newtheorem*{theorem*}{\underline{Teorema}}
 \newtheorem*{lemma*}{\underline{Lema}}
 \newtheorem*{prop*}{\underline{Proposi\c c\~ao}}
 \newtheorem{example}{\underline{Exemplo}}
 \newtheorem*{proof*}{\underline{Prova}}
 \renewcommand\qedsymbol{$\blacksquare$}
 \newcommand{\Lin}[1]{Lin_{\mathbb{K}}({#1})}

 \rfoot{P\'agina \thepage \hspace{1pt} de \pageref{LastPage}}

 \begin{document}
 \begin{figure}[ht]
  \minipage{0.76\textwidth}
    \includegraphics[width=4cm]{../icmc.png}
    \hspace{7cm}
    \includegraphics[height=4.9cm,width=4cm]{../brasao_usp_cor.jpg}
  \endminipage  
\end{figure}

\begin{center}
  \vspace{1cm}
  \LARGE
  UNIVERSIDADE DE S\~AO PAULO

  \vspace{1.3cm}
  \LARGE
  INSTITUTO DE CI\^ENCIAS MATEM\'ATICAS E COMPUTACIONAIS - ICMC

  \vspace{1.7cm}
  \Large
  \textbf{Notas de Física II}

  \vspace{1.3cm}
  \large
  \textbf{Renan Wenzel - 11169472}

  \vspace{1.3cm}
  \large
  \textbf{Professor - Luis Gustavo Marcassa}

  \textbf{E-mail: marcassa@ifsc.usp.br}

  \vspace{1.3cm}
  \today
\end{center}

 \newpage
 \textbf{{\Huge Disclaimer}}
 \vspace{5cm}

  {\huge Essas notas não possuem relação com professor algum. 

  Qualquer erro é responsabilidade solene do autor.

Caso julgue necessário, contatar: renan.wenzel.rw@gmail.com}
 \newpage

 \tableofcontents

 \newpage

 \section{Aula 00 - 07/08/2023}
   Avisos sobre o curso (Ler e baixar o pdf no e-disciplinas!!!!); 

\section{Aula 01 - 09/08/2023}
\subsection{Motivações}
\begin{itemize}
  \item Ângulos, velocidade angular e aceleração angular;
  \item Energia em sistemas com rotação.
\end{itemize}
\subsection{Rotação}
  Antes de qualquer coisa, convencionamos o sentido antihorário como aquele em que \(\Delta \theta >0\) e
o sentido horário como o que \(\Delta \theta <0.\) Uma volta completa em torno do círculo é dada pela versão com \(2\pi\) da fórmula do arco de círculo
 \(\Delta S_{i} = r_{i}\Delta \theta = 2\pi r_{i}\) e, com isso, a variação do ângulo em uma volta completa é dada por 
   \[
     \Delta \theta = \frac{S_{i}}{r_{i}} = \frac{2\pi r_{i}}{r_{i}} = 2\pi rad.
   \]
  Um dos assuntos de importância para nós é o estudo da variação temporal do ângulo. Definimos, nessa lógica, a 
velocidade angular média por 
  \[
    \omega _{med} = \frac{\Delta \theta }{\Delta t}.
  \]
  De modo análogo ao que vimos com cinemática, existe também a velocidade angular instantânea, obtida tomando o limite: 
    \[
      \omega = \lim_{\Delta t\to 0}\frac{\Delta \theta }{\Delta t} = \frac{d\theta }{dt}.
    \]
    Observa-se de cara que, se \(\omega >0, \theta \) aumenta e, se \(\omega <0, \theta \) diminui. Assim como antes,
  precisamos ver, também, a unidade. Em cinemática, a unidade de velocidade era metro por segundo. Dessa vez, já que
  o ângulo move-se em radianos, mas há outras unidades, como a revolução e o grau. Logo, as unidades de \(\omega \) podem ser \([\omega ] = \frac{radianos}{tempo} = \frac{rad}{s}, \frac{graus}{s}, \frac{\text{revolução}}{s},\)
  em que \(1\text{revolução} = 2\pi rad = 360\deg\)

    Por exemplo, se um CD roda a 3000rpm, pode-se expressar essa velocidade de rotação como 
      \[
        \omega = 3000rpm = \frac{3000 \cdot 2\pi}{60} = \frac{600}{6}\pi = 100\pi \frac{rad}{s}.
      \]

  Analogamente, é possível analisar a variação da própria velocidade angular com o tempo, resultando na chamada
  acelerações angulares média e instantânea: 
    \[
      \alpha_{med} = \frac{\Delta \omega }{\Delta t}\quad \alpha  = \frac{d\omega }{dt} = \frac{d}{dt}\biggl(\frac{d\theta }{dt}\biggr) = \frac{d^{2}\theta }{dt^{2}}.
    \]
    A unidade dessa grandeza, novamente, similar à versão linear dela, será dada em \([\alpha ]= \frac{radiano}{s^{2}}\), ou \([\alpha ]=\frac{grau}{s^{2}}\), etc. Nessas
  situações todas, se \(\alpha >0, \omega \) aumenta e, se \(\alpha <0, \omega \) diminui. 

    Agora, suponha que \(\alpha \) é constante. Todos os processos de movimento uniformemente acelerado são válidos aqui também:
    \begin{table}[h!]
    \centering
    \begin{tabular}{|c|c|c|}
        \hline
        & \textbf{Variáveis angulares} & \textbf{Variáveis escalares} \\
        \hline
        \textbf{Posição} & $\theta(t) = \theta_{0} + \omega_{0}t + \frac{\alpha t^{2}}{2}$ & $s(t) = R\theta(t)$ \\
        \hline
        \textbf{Velocidade} & $\omega(t) = \omega_{0} + \alpha t = \frac{d\theta }{dt}$ & $v(t) = v_{0} + at = \frac{dx}{dt}$ \\
        \hline
        \textbf{Aceleração} & $\alpha(t) = \frac{d\omega(t)}{dt} = \frac{d^2\theta(t)}{dt^2}$ & $|\vec{a}(t)| = \frac{dv(t)}{dt} = R\alpha(t),\quad |\vec{a}_{cp}| = \frac{v^2}{R}$ \\
        \hline
        \textbf{Torricelli} & $\omega^{2}(t) = \omega_{0}^{2} + 2\alpha \Delta \theta $ & $v^{2} = v_{0}^{2} + 2a\Delta s.$ \\
        \hline
    \end{tabular}
    \caption{Resumo movimento circular.}
    \label{tab:my_label}
  \end{table}

\begin{example}
  Suponha que há um CD que começa no repouso. Ele começa a girar, indo de 0 a 500rpm em 5.5s. Pergunta-se:
 \begin{itemize}
   \item[a)] Quanto vale \(\alpha \)?
     \item[b)] Quantas voltas o CD dá em 5.5s?
       \item[c)] Qual é a distância percorrida por uma ponta a 6cm do eixo de rotação?
 \end{itemize}

 \textbf{Soluções:}
 \begin{itemize}
   \item[a)] Temos \(\omega (0) = 0, \omega (5.5) = 500rpm.\) Segue que 
     \[
       \omega(t) = \omega_{0} + \alpha t \Rightarrow \alpha  = \frac{\omega (t)}{t} = \frac{500 \cdot 2\pi}{5.5 \cdot 60}\approx 9.52 \frac{rad}{s^{2}}
     \]

    \item[b)] Aplicamos o Torricelli angular com os dados que temos: 
      \[
        \omega^{2} = 2\alpha \Delta \theta \Rightarrow \delta \theta = \frac{\omega^{2}}{2\alpha }\approx 144 rad \Rightarrow \frac{144}{2\pi}rad\approx 23\text{rotações}.
      \]

    \item[c)] Por fim, multiplicando a variação do ângulo pelo raio, obtemos 
      \[
        \Delta S_{i} = r\Delta \theta = 6 \cdot 10^{-2}\cdot 144\approx 8.65m.
      \]
 \end{itemize}
\end{example}

  Olhando de forma cautelosa a fórmula de arco de circulo, podemos derivá-la com respeito ao tempo utilizando o que vimos até agora: 
    \[
      \frac{dS_{i}}{dt} = V_{t} = r_{i}\frac{d\theta }{dt} = r_{i}\omega.
    \]
  Essa derivação resulta em uma velocidade linear, que também pode ser derivada a fim de obter uma aceleração linear 
    \[
      \frac{dV_{t}}{dt} = r_{i}\frac{d\omega }{dt} \Rightarrow a_{t} = r_{i}\alpha.
    \]
    Note a relação entre as duas acelerações que obtivemos, \(a_{c} = \frac{V_{t}^{2}}{r_{i}}= \frac{r_{i}^{2}\omega^{2}}{r_{i}} = r_{i}\omega^{2}.\)

\subsection{Energia Cinética de Rotação}
  A energia cinética, como vista previamente, é dada por 
    \[
      \mathcal{K} = \frac{1}{2}m_{i}v_{i}^{2}.
    \]
  Agora, imagine um corpo discreto (formado por vários pontos). Somemos as energias deles, tal que a energia cinética total é 
    \[
      \mathcal{K}_{T} = \sum\limits_{}^{}\frac{1}{2}m_{i}v_{i}^{2}.
    \]
  Mas, sabemos que \(v_{i} = r_{i}\omega, \) tal que 
    \[
      \mathcal{K} = \frac{1}{2}\sum\limits_{}^{}m_{i}r_{i}^{2}\omega^{2} = \frac{1}{2}\biggl[\sum\limits_{}^{}m_{i}r_{i}^{2}\biggr]\omega^{2}
    \]
  Chamemos o termo em colchete de momento de inércia, denotado por \(I:= \sum\limits_{}^{}m_{i}r_{i}^{2} = \sum\limits_{}^{}I_{i}\). Logo, 
    \[
      \boxed{\hypertarget{kin_en}{\mathcal{K}_{T} = \frac{1}{2}I\omega^{2}.}}
    \]
\newpage
\section{Aula 02 - 10/08/2023}
\subsection{Motivações}
 \begin{itemize}
   \item Momento de Inércia
 \end{itemize}
\subsection{Distribuição Contínua de Massa}
  No caso de distribuições discretas de massa, vimos que o momento de inércia é dado por 
    \[
      I=\sum\limits_{i}^{}m_{i}r_{i}^{2}.
    \]
No entanto, muitas situações do mundo precisam que tratemos a distribuição de massa como algo único, uma
quantidade contínua. Para isso, passamos de somar cada massa para uma integral com respeito a ela: 
  \[
    \hypertarget{momentum_of_inertia_continuous}{\boxed{I = \int_{}^{}r^{2}dm.}}
  \]
  Para o caso de uma barra, por exemplo, na qual a distribuição de massa é dada por 
    \[
      \lambda = \frac{M}{L},
    \]
  segue que \(dm = \lambda dx \Rightarrow dI = x^{2}dm = x^{2}\lambda dx\). Portanto, 
    \[
      I = \lambda \int_{}^{}x^{2}dx = \lambda \frac{x^{3}}{3}.
    \]
  Por exemplo, se o tamanho da barra é 1 e o eixo de rotação está em uma extremidade, o momento de inércia será 
    \[
      I = \lambda \int_{0}^{1}x^{2}dx =\frac{1}{3}ML^{2}.
    \]
    Há outros casos importantes que devem ser tratados. O primeiro deles é o eixo central,
  no qual o eixo de rotação é posicionado na metade do tamanho da barra. Assim, 
    \[
      I = \lambda \int_{-\frac{1}{2}}^{\frac{1}{2}}x^{2}dx = \lambda \frac{x^{3}}{3}\biggl|_{-\frac{1}{2}}^{\frac{1}{2}}\biggr. = \lambda \frac{L^{3}}{12} = \frac{ML^{2}}{12}.
    \]
  O outro engloba a situação em que toda a massa na mesma distância. Neste caso, \(\lambda = \frac{M}{2\pi R}\)
    \[
      I = R^{2} \int_{}^{}dm = MR^{2}, 
    \]
  que também pode ser obtido fazendo uma integral com respeito ao ângulo \(\theta \): 
    \[
      I = R^{2}\lambda \int_{0}^{2\pi } R d\theta = R^{2}\lambda R\times2\pi = MR^{2}.
    \]
  Por fim, é importante olhar o caso dos discos. Discos consistem de dois círculos, um maior e outro menor dentro dele.
Chamaremos de R o raio do maior e de r o do menor. Para eles, há uma distribuição de massa
 \(\sigma = \frac{M}{\pi R^{2}}\), de maneira que o diferencial de massa será 
   \[
     dm = 2\pi r dr\sigma.
   \]
   Com isso, conseguimos encontrar que o momento de inércia é 
     \[
       I = \int_{0}^{R}2\pi \sigma r^{3}dr = 2\pi \sigma \frac{\pi^{4}}{4}\biggl|_{0}^{R}\biggr. = \frac{1}{2}MR^{2}
     \]

\newpage

\section{Aula 03 - 16/08/2023}
\subsection{Motivações}
\begin{itemize}
  \item Disco com buraco;
  \item Rodando disco e cilindro em torno do plano.
\end{itemize}
\subsection{Momento de Inércia em um Disco}
  Vamos considerar um disco de raio \(R_{2}\) que contém dentro de si um buraco de
raio \(R_{1}\). Nisso, consideramos o momento de inércia do disco inteiro como 
  \[
    I = I^{+} + I^{-}.
  \]
  Aqui, \(I^{+}\) desconsidera a existência do buraco, ou seja, tem valor 
    \[
      I^{+} = \frac{\pi R_{2}^{2}\sigma R_{2}^{2}}{2} = \frac{1}{2}M^{+}R_{2}^{2}
    \]
  e o valor de \(I^{-}\) vale 
    \[
      I^{-} = \frac{1}{2}M^{-}R_{1}^{2} = \frac{\pi R_{1}^{2}}{2}\sigma R_{1}^{2}.
    \]
    Assim, considerando o valor total, obtivemos o mesmo resultado que o de antes: 
      \[
        I = \frac{\pi \sigma }{2}(R_{2}^{4} - R_{1}^{4}).
      \]
    Em particular, a densidade de massa após o buraco ser feito, \(\sigma^{*} \), valerá
      \[
        \sigma ^{*} = \frac{M}{\pi(R_{2}^{2} - R_{1}^{2})},
      \]
    de forma que, através de \(I = \frac{\pi \sigma^{*}}{2}(R_{2}^{4} - R_{1}^{4}\), obtemos 
      \[
        I = \frac{\pi M}{2} \frac{(R_{2}^{2}-R_{1}^{2})(R_{2}^{2}+R_{1}^{2})}{\pi (R_{2}^{2}-R_{1}^{2})} = \frac{M}{2}(R_{2}^{2}+R_{1}^{2}).
      \]

    Agora, suponha que deixamos um disco girar em torno de um eixo com velocidade \(\omega \). Como podemos descrever esse sistema e seu momento de inércia? 
Faremos uso do Teorema dos Eixos Paralelos. Apesar de não conhecermos o momento de inércia, sabemos que em algum ponto, encontra-se o centro
de massa do objeto, estando a uma distância h do eixo. Este centro de massa move-se com velocidade \(\vec{v}_{cm}\). Como a energia cinética total
tem valor \(\mathbb{K}_{T} = \frac{1}{2}Mv_{cm}^{2} + \mathbb{K}_{relcm}\), utilizamos que \(\mathbb{K} = \frac{1}{2}I\omega^{2}\) e que \(\mathbb{K}_{relcm}=\frac{1}{2}I_{cm}\omega ^{2}\).
Logo, como \(v_{cm} = h\omega ,\)
 \begin{align*}
   \frac{1}{2}I\omega^{2} &= \frac{1}{2} Mv_{cm}^{2} + \frac{1}{2}I_{cm}\omega^{2}\\
                          &= Mh^{2}\omega^{2} + I_{cm}\omega ^{2}\\
                          &\Rightarrow I = Mh^{2} + I_{cm}.
 \end{align*}
 \begin{example}
   Considerando uma barra em a uma distância de \(\frac{L}{2}\) do eixo de rotação e com momento de inércia 
  \(I = \frac{1}{3}ML^{2},\) podemos utilizar a fórmula para obter 
 \begin{align*}
   &I = Mh^{2} + I_{cm}\\
   &\frac{1}{3}ML^{2} = M \frac{L^{2}}{4} + I_{cm}\\
   &I_{cm} = \frac{1}{12}ML^{2}.
 \end{align*}
 \end{example}
    Mas, o que aconteceria se o disco rodasse em eixos x, y contidos no plano do disco? O que sabemos é que 
      \[
        I_{z} = \sum\limits_{}^{}m_{i}r_{i}^{2}.
      \]
  Além disso, \(r_{i}^{2} = (x_{i}^{2} + y_{i}^{2})\), ou seja, 
 \begin{align*}
   I_{z} &= \sum\limits_{}^{}m_{i}x_{i}^{2} + \sum\limits_{}^{}m_{i}y_{i}^{2}\\
         &= I_{x} + I_{y}.
 \end{align*}
  Este resultado é conhecido como teorema dos eixos perpendiculares, mas vale apenas para corpos bidimensionais.
Em particular, no caso do cilindro, em que \(I_{x} = I_{y},\) 
  \[
    I_{x} = I_{y} \Rightarrow 2I_{x} = I_{z} \Rightarrow I_{x} = \frac{1}{4}MR^{2}.
  \]
  Além disso, considerando que \(dI_{x} = \frac{1}{4}dm R^{2} + dm z^{2}\), obtemos 
    \[
      I_{x} = \frac{1}{4}R^{2} \int_{}^{}dm + \int_{}^{}dm z^{2}.
    \]
    Como \(dm = \lambda dz = \frac{M}{L}dz\), em que L é o comprimento, segue o seguinte resultado 
      \[
        I_{x} = \frac{1}{4}R^{2}\int_{-\frac{L}{2}}^{\frac{L}{2}}\frac{M}{L}dz + \int_{-\frac{L}{2}}^{\frac{L}{2}}\frac{M}{L}z^{2}dz
      \]
    Assim, fazendo as contas, 
      \[
        I_{x} = \frac{1}{4}MR^{2} + \frac{ML^{2}}{12}
      \]

\begin{example}
  Considere uma barra de tamanho L e massa M e deixe-a descer em um pivô. Qual é a força que ele terá que fazer?

  Sabe-se que há uma força peso com módulo Mg agindo e que \(E_{mec_{i}} = E_{mec_{f}},\) tal que \(\mathbb{K}_{i} + U_{i} = \mathbb{K}_{f} + U_{f}\).
Mas, \(\mathbb{K}_{i} = U_{i} = 0\) e \(\mathbb{K}_{f} = \frac{1}{2}I\omega_{f}^{2}, U_{f} = Mg(\frac{-L}{2}).\) Assim, 
  \[
    \frac{1}{2}I\omega_{f}^{2} - Mg \frac{L}{2} = 0 \Rightarrow \omega_{f}^{2} = \frac{MgL}{I} = \frac{MgL}{\frac{1}{3}ML^{2}} = \frac{3g}{L}.
  \]
  Assim, usando que \(a_{cm} = r\omega_{f}^{2},\)
 \begin{align*}
   F- Mg &= Ma_{cm} \Rightarrow F = Mg + M \frac{L}{2}\omega_{f}^{2}\\
         &= Mg + \frac{ML}{2}\frac{3g}{L}\\
         &=Mg + \frac{3}{2}Mg = \frac{5}{2}Mg.
 \end{align*}
\end{example}
\begin{example}
  Considere uma roldana de raio R e massa \(m_{r}\). Atrele a ela, com uma corda de massa \(m_{c}\) e tamanho L, um balde de massa \(m_{b}\). 
Em seguida, solte-o para cair uma distância d. Qual é a velocidade do sistema?

  Sabemos que \(E_{mec_i} = E_{mec_f}\), ou seja, 
    \[
      \mathbb{K}_{i} + U_{i} = \mathbb{K}_{f} + U_{f}.
    \]
  Suponha que \(\mathbb{K}_{i} = U_{i} = 0.\) Quando o balde descer, sendo \(m_{c}^{*} = \frac{d}{L}m_{c}\) a massa da fração de corda que desceu, a potencial final passará a valer
 \(U_{f} = m_{b}(-d)g + m_{c}^{*}(\frac{-d}{2})g = -m_{b}gd - \frac{1}{2}m_{c}^{*}gd.\) Com relação à cinética, 
   \[
     \mathbb{K}_{f} = \frac{1}{2}m_{r}v^{2} + \frac{1}{2}m_{c}v^{2} + \frac{1}{2}m_{b}v^{2}.
   \]
   Utilizando as relações de energia que vimos, segue que 
  \begin{align*}
    &\mathbb{K}_{f} + U_{f} = 0\\
    &\Rightarrow \frac{1}{2}(m_{c}+m_{r}+m_{b})v^{2} = m_{b}gd + \frac{1}{2}m_{c}^{*}gd\\
    &\Rightarrow (m_{r}+m_{c}+m_{b})v^{2} = 2m_{b}gd + m_{c}g \frac{d^{2}}{L}\\
    &\Rightarrow v^{2} = \frac{(2m_{b}L + m_{c}d)}{m_{r}+m_{c}+m_{b}}\frac{gd}{L}\\
    &\Rightarrow v = \sqrt[]{\frac{(2m_{b}L + m_{c}d)}{m_{r}+m_{c}+m_{b}}\frac{gd}{L}}.
  \end{align*}
\end{example}
\newpage
\section{Aula 05 - 17/08/2023}
\subsection{Motivações }
\begin{itemize}
  \item Segunda lei de Newton do Movimento Circular;
  \item Torque da Gravidade.
\end{itemize}
\subsection{Segunda Lei de Newton}
Ao considerarmos uma força aplicada a um objeto em torno de um círculo de raio r, essa força faz um ângulo
 \(\theta \) com a paralela ao raio. Além disso, há uma componente dessa força que será tangente 
 à trajetória do objeto ao longo do círculo. Denotando essa segunda por \(F_{t},\) há duas formas de expressá-la: 
   \[
     F\sin{(\theta )} = F_{t},\quad F_{t} = ma_{t}.
   \]
   Além disso, a aceleração tangencial \(a_{t}\) satisfaz \(a_{t} = r\alpha \). Assim, obtemos a relação 
     \[
       F sin(\theta ) = ma_{t} \Rightarrow F\sin{\theta } = mr\alpha \Longleftrightarrow rF\sin{(\theta )} = mr^{2}\alpha.
     \]
     Esse termo à esquerda é conhecido como \textbf{torque} 
       \[
         \hypertarget{torque}{\boxed{\tau = mr^{2}\alpha = rF\sin{(\theta )}}}
       \]
  Em particular, sendo o torque total a soma de todos os torques, obtemos 
    \[
      \tau = \sum\limits_{}^{}\tau_{i} = \sum\limits_{}^{}m_{i}r_{i}^{2}\alpha = I\alpha 
    \]
    Uma propriedade é que a soma dos torques das forças internas vale zero.

    Olhando um caso mais específico, ao considerarmos um círculo de raio r e uma força que
faz um ângulo \(\theta \) com a paralela ao raio e outro círculo menor de raio r' com a mesma força aplicada,
mas ângulo \(\theta ',\) então \(l=r'\sin{(\theta ')}\) é a componente perpendicular à linha na qual a força está atuando.
  A vantagem disso é que o torque pode ser, então, expresso através de \(\tau = Fl = F_{t}r'\)
\subsection{O Torque da Força da Gravidade}
  Se considerarmos um corpo sofrendo a ação da força peso, o torque desse corpo pode ser descrito por
  \(\tau_{i} = m_{i}gx_{i}\) e, o torque total, será a soma desses torques: 
    \[
      \tau_{r} = \sum\limits_{}^{}\tau_{i} = \sum\limits_{}^{}[m_{i}x_{i}]g
    \]
  Mas, esse é exatamente o torque do centro de massa do objeto \(\tau_{r} = Mx_{cm}g\). Outro assunto que é
importante ressaltar é que, durante os estudos de dinâmica, a forma de estudar as forças em um sistema é através
dos chamados diagramas de força, o que traz à tona a questão do que funcionaria pro estudo do torque.
\begin{example}
  Considere uma roda de bicicleta e a catraca, que sofre uma força F de 18N. Suponha que o raio r
da catraca é de 7cm e o da roda, R, vale 35cm. Além disso, a massa vale 2.4kg. Qual é a velocidade angular para t=5,5s?

Começamos afirmando que o torque é \(\tau = I\alpha = Fr_{c}\). Assim, 
\[
  \alpha = \frac{Fr_{c}}{I} = \frac{Fr_{c}}{MR^{2}} = \frac{18 \cdot (0,07)}{2,4(0,35)^{2}}\frac{rad}{s^{2}}.
\]
Com isso, 
  \[
    \omega = \omega_{0} + \alpha t = \alpha t = \frac{18 \cdot (0,07)}{2,4(0,35)^{2}} \cdot 5,5 = 21,4 \frac{rad}{s}
  \]
\end{example}
\begin{example}
  Considere uma barra de massa m e comprimento l está presa por um pivô, o qual realiza uma força F. Após soltá-la, qual é a força que o pivô realiza?

  Sabemos que \(\tau = mg \frac{l}{2} = I\alpha = \frac{1}{3}ml^{2}.\) Logo, 
 \begin{align*}
   &mg \frac{l}{2} = \frac{1}{3}ml^{2}\alpha \\
   &\alpha = \frac{3}{2}\frac{g}{l}.
 \end{align*}
 Olhando no eixo y, sabemos que \(a_{cm_{y}} = r\alpha  = \frac{l}{2}\frac{3}{2}\frac{g}{l} = \frac{3}{4}g\), tal que 
   \[
     F - mg = -ma_{cm_y} \Rightarrow F = mg - \frac{3}{4}mg = \frac{1}{4}mg
   \]
\end{example}
\begin{example}
  Suponha que temos uma roldana de raio R e momento de inércia I. Pendura-se um corpo de massa m na roldana. Qual é a aceleração de queda do corpo?

\textbf{Roldana:}
As forças que atuam na roldana são o Peso dela, \(P_{r}\), a tensão T
e a força resultante ao peso \(F_{r}\). Assim, 
\begin{align*}
  &F_{r} = P_{r} + T\\
  &TR = I\alpha,\quad a = \alpha R. 
\end{align*}

\textbf{Corpo:}
No corpo, por outro lado, tem-se apenas a tensão T e o peso mg, de forma que 
  \[
    mg-T = ma.\quad a = \alpha R
  \]
\end{example}
  Continua na próxima aula...
  \newpage

\section{Aula 06 - 21/08/2023}
\subsection{Motivações}
\begin{itemize}
  \item Continuação do exemplo e outros;
  \item Potência;
  \item Corpos que rolam sem deslizar.
\end{itemize}
\subsection{Continuando o Exemplo}
\begin{example}[continuando...]
  Segue a relação de tração 
    \[
      TR = I \frac{a}{R} \Rightarrow T = \frac{I}{R^{2}}a.
    \] 
  Com isso, 
    \[
      mg = ma + T = ma + \frac{I}{R^{2}}a = a[1 + \frac{I}{mR^{2}} \Rightarrow a = \frac{g}{1 + \frac{I}{mR^{2}}}.
    \]
  Descobrimos, assim, os valores de T e de \(F_{s}\)
 \begin{align*}
   &T = \frac{I}{R^{2}}\frac{g}{1+ \frac{I}{mR^{2}}}\\
   &F_{s} = Mg + \frac{I}{R^{2}}\frac{g}{1+\frac{I}{mR^{2}}}.
 \end{align*}
 \end{example} 
\begin{example}
  Considere a máquina de Atroos - dois blocos presos a uma roldana, um de massa \(m_{1}\) e outro de massa \(m_{2}\) tais que \(m_{1} > m_{2}\).
A roldana tem massa M, momento de inércia I e raio R. Vejamos as forças

  \textbf{Bloco 1:}
    No primeiro bloco, agem forças de tração \(T_{1}\) e peso \(m_{1}g\). Escrevendo as equações,
   \begin{align*}
     m_{1}g - T_{1} = m_{1}a
   \end{align*}

  \textbf{Bloco 2:}
    No bloco dois, agem a tração \(T_{2}\) e o peso \(m_{2}g\)
   \begin{align*}
     T_{2}-m_{2}g = m_{2}a
   \end{align*}

  \textbf{Roldana:}
    Tem-se a equação 
   \begin{align*}
     (T_{1} - T_{2})R = I\alpha. \Longleftrightarrow T_{1} - T_{2} = \frac{I}{R}\frac{a}{R} = \frac{Ia}{R^{2}}
   \end{align*}

  Como a roldana está rodando, tem-se a relação \(m_{1g} > T_{1} > T_{2} > m_{2}g\)
A seguir, soma-se a equação do bloco 2 com a da Roldana, tal que 
  \[
    m_{1}g - m_{2}g - (T_{1}-T_{2}) = (m_{1}+m_{2})a \Longleftrightarrow (m_{1}-m_{2})g - \frac{Ia}{R^{2}} = (m_{1}+m_{2})a
  \]
  Isola-se a equação no a: 
    \[
      a = \frac{(m_{1}-m_{2})g}{(m_{1}+m_{2})+\frac{I}{R^{2}}}
    \]
\end{example}
\begin{example}
  Considere uma roldana com massa M, momento de inércia I e raio R presa à quina uma mesa. Atrela-se a ela dois corpos, um com massa \(m_{1}\) e que está em cima da mesa
e outro, de massa \(m_{2}\), suspenso pela corda. 
  \textbf{Corpo 1:}
    As forças atuando no bloco 1 são a normal \(F_{N_{1}}\), a peso \(m_{1}g\) e a tração \(T_{1}\), de forma que 
      \[
        T_{1}=m_{1}a
      \]

  \textbf{Corpo 2:}
    Para o bloco 2, podemos descrever o sistema considerando a tração \(T_{2}\) e o peso \(m_{2}g\), tal que 
      \[
        m_{2}g - T_{2} = m_{2}a.
      \]

  \textbf{Roldana:}
    As forças que atuam na roldana são a tração na direção do bloco 1, \(T_{1}\), a outra na do bloco 2, \(T_{2}\), o peso
  \(Mg\) e uma força da quinta nela \(\vec{F_{s}}\). Além disso, \(a=R\alpha \). A equação do sistema será
 \begin{align*}
   (T_{2}-T_{1})R = I\alpha \Rightarrow  T_{2} - T_{1} = \frac{Ia}{R^{2}}.
 \end{align*}

  Somando a equação do bloco 1 e a do bloco 2, chega-se em 
    \[
      m_{2}g - (T_{2}-T_{1}) = (m_{1}+m_{2})a
    \]
  Assim, 
 \begin{align*}
   &T_{2} - T_{1} = \frac{Ia}{R^{2}}\\
   &m_{2}g - \frac{Ia}{R^{2}} = (m_{1}+m_{2})a\\
   &a = \frac{m_{2}g}{(m_{1}+m_2) + \frac{I}{R^{2}}}
 \end{align*}
 Além disso, 
   \[
     T_{1} = \frac{m_{1}m_{2}g}{(m_{1}+m_{2})+\frac{I}{R^{2}}}.
   \]
\end{example}
\subsection{Potência}
  Previamente, a potência era dada pela relação \(dW = F ds.\) Considerando o caso de uma força agindo
emu ma situação circular, isso torna-se \(dW = FRd\theta \). No entanto, esse termo à direita lembra muito um torque. De fato,
a relação que obtemos é que \(dW = \tau d\theta \). Portanto, 
  \[
    \hypertarget{power_torque}{\boxed{P = \frac{dW}{dt} = \tau \frac{d\theta }{dt} = \tau \omega }}
  \]
 \begin{example}
   Um motor de combustão de um carro fornece um torque de \(\tau  = 678Nm\) e está rodando a \(\omega = 4500rpm \approx 471 \frac{rad}{s}\).
Com isso, a potência será 
  \[
    P\approx 315kW.
  \]
 \end{example}
\begin{example}
  Tome uma roda gigante \textbf{(em Londres).} Ela tem um diâmetro de \(135m\), pesa 1600 toneladas e dá 2 revoluções por hora.
Qual é o torque necessário para parar a roda em 10m?
  
  Para começar, observe que \(W = \tau \Delta \theta \) e que \(S = R\Delta \theta = 10m.\) Em particular, temos o valor de R, tal que 
    \[
      S = R\Delta \theta \Longleftrightarrow 10 = 67.5\Delta \theta \Rightarrow \Delta \theta \approx 0,148rad.
    \]
  Note que 
    \[
      W = -(\mathbb{K}_{f} - \mathbb{K}_{i}) \Longleftrightarrow \tau \Delta \theta = -\biggl[0 - \frac{I\omega^{2}}{2}\biggr].
    \]
  Logo, convertendo \(\omega \) para radianos por segundo (\(\omega = 3,5 \cdot 10^{-3}\frac{rad}{s}\),
    \[
      \tau = \frac{I\omega^{2}}{2\Delta \theta } = \frac{MR^{2}\omega^{2}}{2\Delta \theta } \approx 3 \cdot 10^{5}Nm.
    \]
    Em particular, 
      \[
        F = \frac{\tau }{R}\approx 4,4 \cdot 10^{3}N.
      \]
\end{example}
\subsection{Corpos que Rolam sem Deslizar}
  Imagine um sistema em que um disco de raio R está a rolar com velocidade do centro de massa \(\vec{v}_{cm}.\) Considerando o ponto que tangencia o chão, em que a velocidade é nula,
ele se mexe com velocidade angular \(\omega \) em um raio \(\vec{r}\). Assim, a energia cinética desse sistema será 
  \[
    \mathbb{K}_{T} = \frac{1}{2} Mv_{cm}^{2} + \mathbb{K}_{rel} = \frac{1}{2}Mv_{cm}^{2} + \frac{1}{2}I_{cm}\omega ^{2},
  \]
  em que considera-se que \(v_{cm} = R\omega.\) 

  Agora, considere um plano inclinado e uma bola de massa m, momento de inércia I e raio R que irá subir este plano inclinado até parar. Como podemos achar a altura que ela para, 
fornecida velocidade inicial do centro de massa \(v_{cm}\). Utilizando a conservação da energia mecânica, 
  \[
    E_{mec_{i}} = E_{mec_{f}}.
  \]
  Sabemos que 
    \[
      E_{mec_{i}} = \frac{1}{2}mv_{cm}^{2} + \frac{I_{cm}}{2}\omega ^{2}\quad \& E_{mec_{f}} = mgh.
    \]
  Assim, 
  \begin{align*}
    &mgh = \frac{1}{2} mv_{cm}^{2} + \frac{I_{cm}}{2}\omega^{2}\\
    &\Rightarrow h = \frac{1}{2}\biggl[v_{cm}^{2} + \frac{I_{cm}}{2m}\frac{v_{cm}^{2}}{R^{2}}\biggr]\\
    &\Rightarrow h = \frac{v_{cm}^{2}}{2g}\biggl[1 + \frac{I_{cm}}{mR^{2}}\biggr].
  \end{align*}
 \begin{example}
   Para o caso da esfera, em que \(I = \frac{2}{5}mR^{2},\) 
     \[
       h = \frac{v_{cm}^{2}}{2g}\biggl[1 + \frac{2}{5}\biggr] = \frac{7}{10}\frac{v_{cm}^{2}}{g}
     \]
 \end{example}
\begin{example}
  Considere um cenário de sinuca em que um taco aplica uma força F. Se ela é aplicada acima do eixo de rotação, a bola roda para frente. Caso seja exatamente no eixo de rotação,
ela apenas deslizará. Por fim, se for atingida abaixo do eixo de rotação, ela rodará ao contrário. Como fazer ela não rodar?

  Em qualquer um desses pontos, a força é \(F=ma.\) No caso em que ela roda para frente, ou seja, é atingida a uma distância d acima do eixo de rotação,
temos \(\tau = Fd = I\alpha.\) Segue que, para que ela não rode, 
\begin{align*}
  & Fd = I\alpha = \frac{Ia}{R}\\
  &\Rightarrow d = \frac{I}{mR} = \frac{2}{5}\frac{mR^{2}}{mR} = \frac{2}{5}R.
\end{align*}
\end{example}
\end{document}

