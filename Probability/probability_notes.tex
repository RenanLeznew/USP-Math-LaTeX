\documentclass{article}
 \usepackage{amsmath}
 \usepackage{amsthm}
 \usepackage{amssymb}
 \usepackage{pgfplots}
 \usepackage[utf8]{inputenc}
 \usepackage{amsfonts}
 \usepackage[margin=2.5cm]{geometry}
 \usepackage{graphicx}
 \usepackage[export]{adjustbox}
 \usepackage{fancyhdr}
 \usepackage[portuguese]{babel}
 \usepackage{hyperref}
 \usepackage{lastpage}
 \usepackage{mathtools}
 \setcounter{section}{-1}

 \pagestyle{fancy}
 \fancyhf{}

 \pgfplotsset{compat = 1.18}

 \hypersetup{
     colorlinks,
     citecolor=black,
     filecolor=black,
     linkcolor=black,
     urlcolor=black
 }
 \newtheorem*{def*}{\underline{Defini\c c\~ao}}
 \newtheorem*{theorem*}{\underline{Teorema}}
 \newtheorem*{lemma*}{\underline{Lema}}
 \newtheorem*{prop*}{\underline{Proposi\c c\~ao}}
 \newtheorem{example}{\underline{Exemplo}}
 \newtheorem*{proof*}{\underline{Prova}}
 \renewcommand\qedsymbol{$\blacksquare$}
 \newcommand{\Lin}[1]{Lin_{\mathbb{K}}({#1})}

 \rfoot{P\'agina \thepage \hspace{1pt} de \pageref{LastPage}}

 \begin{document}
 \begin{figure}[ht]
  \minipage{0.76\textwidth}
    \includegraphics[width=4cm]{../icmc.png}
    \hspace{7cm}
    \includegraphics[height=4.9cm,width=4cm]{../brasao_usp_cor.jpg}
  \endminipage  
\end{figure}

\begin{center}
  \vspace{1cm}
  \LARGE
  UNIVERSIDADE DE S\~AO PAULO

  \vspace{1.3cm}
  \LARGE
  INSTITUTO DE CI\^ENCIAS MATEM\'ATICAS E COMPUTACIONAIS - ICMC

  \vspace{1.7cm}
  \Large
  \textbf{Introdução à Probabilidade}

  \vspace{1.3cm}
  \large
  \textbf{Renan Wenzel - 11169472}

  \vspace{1.3cm}
  \large
  \textbf{Professor: Oilson Alberto Gonzatto Junior}

  \textbf{E-mail: oilson.agjr@icmc.usp.br}

  \vspace{5cm}
  \today
\end{center}

 \newpage

 \tableofcontents

 \newpage

\section{Informações (Possivelmente) Úteis}
\subsection*{Monitoria}
\begin{itemize}
  \item[Monitora:] Patrícia Stülp
  \item[E-mail:] \textit{patriciastulp2@gmail.com}
\end{itemize}
\subsection*{Datas das Provas}
  \subsubsection*{Mini provas}
 \begin{itemize}
   \item[i)] 29/08/2023;
   \item[ii)] 12/09/2023; 26/09/2023;
   \item[iii)] 10/10/2023; 24/10/2023;
   \item[iv)] 07/11/2023; 21/11/2023;
   \item[v)] 05/12/2023
 \end{itemize}
 \subsubsection*{(Talvez) P3}
  Pode não ocorrer, mas, a depender dos resultados das mini-provas, será dia 12/12/2023.
\subsection*{Bibliografia}
\begin{itemize}
  \item[Principal:] MEYER, P. L. ``Probabilidade: Aplicações à Estatística'', 2a edição, LTC, Rio de Janeiro, 2009.
  \item[Complementar:] ROSS, S. A. ``First Course in Probability", 8th edition, Pearson, 2010.
\end{itemize}

 \newpage
\section{Aula 01 - 22/08/2023}
\subsection{Motivações}
\begin{itemize}
  \item O que é aleatoriedade e probabilidade?
  \item Conceitos Fundamentais;
\end{itemize}
\subsection{Aleatoriedade, probabilidade e conceitos fundamentais}
  A probabilidade está nos conceitos bases da ciência atual, sendo resultado de uma revolução na ciência há um século atrás.
A noção de aleatoriedade, no entanto, é muito mais difícil de obter uma resposta.

  Começamos com um experimento E - um mecanismo gerado. Dele, obtemos um resultado \(\omega \). A estes resultados,
associamos um conjunto, chamado de evento \(A = \{\omega_{1}, \cdots, \omega_{k}\}\). Com estas ideias, associa-se um valor
a um evento, chamado probabilidade \(\mathbb{P}(A).\) Vamos compreender estas ideias mais a fundo.

\subsubsection{Experimentos}
  Experimentos podem ser distinguidos em alguns tipos. O primeiro deles é o determinístico. Nele,
o resultado obtido é determinado pelas condições sob as quais o experimento foi executado. A outra forma 
é conhecida como experimento aleatório, que engloba experimentos cujos resultados não sabemos \textit{a priori}, isto é,
 ainda que as condições iniciais sejam fixas, os resultados não podem ser previstos. Eles possuem as seguintes características:
 \begin{itemize}
   \item[a)] Mesmo repetindo várias vezes com as mesmas condições iniciais, o resultado pode mudar;
   \item[b)] Apesar da falta de exatidão, é possível descrever o conjunto de todos os resultados possíveis;
   \item[c)] Há uma regularidade nos resultados após ele ser repetido muitas vezes, permitindo uma modelagem matemática dele.
\subsubsection{Espaço Amostral}
  O espaço amostral denota todos os resultados que podem ocorrer ao realizar um experimento aleatório.
  
  \textbf{Observação:} O mecanismo gerador está limitado a um determinado conjunto de possibilidades de saídas.
 \end{itemize}
\begin{example}
  Se considerarmos características sócio demográficas de um grupo de pessoas, poderíamos ter
 \begin{itemize}
   \item[Sexo:)] \{Masculino, Feminino, Intersexo\}
   \item[Idade:)] \{0, 1, 2, ...\}
   \item[Estado civil:)] \{Solteiro, Casado, Viúvo, outros.\}
   \item[Renda familiar:)] \(\{x: x\in \mathbb{R}^{+}\}\)
 \end{itemize}
\end{example}
\begin{example}
  Dados os experimentos aleatórios, quais são os espaços amostrais?
 \begin{itemize}
   \item[\(E_{1}\))] Lançar uma moeda 2 vezes e observar as faces obtidas; \(\Omega =\{(C, C), (C, K), (K, C), (K, K)\}\)
   \item[\(E_{2}\))] Retirar uma carta de um barulho comum e observar o naipe; \(\Omega\)=\{``ouros'', ``copas'', ``paus'', ``espadas''\}
   \item[\(E_{3}\))] Duração de lâmpadas, deixando-as acesas até que queimem; \(\Omega = \{t: t\geq 0\}\)
   \item[\(E_{4}\))] Número de mensagens por dia entre uma empresa e um determinado cliente. \(\Omega = \{t: t\geq 0\}\)
 \end{itemize}
\end{example}
\subsubsection{Evento}
  Um evento representa qualquer dúvida que possa surgir sobre o resultado de um experimento. Em particular, podem ser visualizados como subconjuntos do espaço amostral. Com isso,
o próprio amostral é um evento, chamado evento certo, assim como o vazio é um evento, dito evento impossível.
\begin{example}
  Considere o resultado obtido com o lançamento de um dado de seis faces, equilibrado. O espaço amostral é \(\{1, 2, 3, 4, 5, 6\}\). Descrevamos os seguintes eventos:
 \begin{itemize}
   \item[1)] A = ``ocorrência do número 3'' = \{3\};
   \item[2)] B = ``sair a face de número 7'' = \(\emptyset\);
   \item[3)] C = ``sair um número menor ou igual a 6'' = \(\Omega \).
 \end{itemize}
\end{example}
\begin{example}
  Podemos considerar experimentos cujos resultados podem ser vetores. A exemplo, o preço de fechamento de determinadas
ações em uma data específica, como \(\omega = \)(PETR4, ITUB4, ..., AZUL4). Nesse contexto, um evento possível poderia ser 
a situação em que a média desse vetor de preços seja maior do que a média do dia anterior.
\end{example}
\subsubsection{Operações com Eventos}
  Veja que um experimento aleatório pode ser tão complicado quanto necessário. Buscamos, porém, buscar simplificações confiáveis para descrever com probabilidades
estes comportamentos, ou seja, estabelecer as propriedades básicas da função de probabilidade. Antes de mais nada, porém, torna-se necessário
saber como operar os eventos. 

  Como eventos são subconjuntos do espaço amostral, a operação entre eles é dada por meio das operações entre conjuntos - 
união, interseção, complemento, etc. Nesta lógica, compreendemos como
\begin{itemize}
  \item[] União de eventos - a capacidade do evento A OU do evento B ocorram;
  \item[] Interseção de eventos - a capacidade do evento A E do evento B ocorrerem simultaneamente.
  \item[] Complementar - a capacidade do evento A não acontecer
  \item[] Eventos mutuamente exclusivos - A e B não ocorrem simultaneamente, isto é, \(A\cap B = \emptyset\)
\end{itemize}
\begin{example}
  Um número entre 1 e 10 é selecionado ao acado. Considere A como o evento em que o número selecionado é múltiplo de 3 e B o conjunto em que o número selecionado é par. Então, 
 \(\Omega = \{1, 2, \cdots, 10\}, A = \{3, 6, 9\}, B = \{2, 4, 6, 8, 10\}\). O evento que ocorre nos dois é \(A\cap B=\{6\}\). O evento representando que o número seja um múltiplo de 3 ou
 um número par é \(A\cup B = \{2, 3, 4, 6, 8, 9, 10\}\). Alguns outros são: \(A\cap B^{c} = \{3, 9\}, A^{c}\cap B = \{2, 4, 8, 10\}.\)
\end{example}
\newpage
\section{Aula 02 - 24/08/2023}
\subsection{Motivações}
\begin{itemize}
  \item As definições de probabilidade;
  \item A probabilidade segundo Kolmogorov.
\end{itemize}
\subsection{Conceitos}
\paragraph{}As ideias desenvolvidas até aqui são as que Laplace desenvolveu, contendo toda a parte do cálculo
de probabilidade por meio da contagem de casos favoráveis e dos possíveis. No entanto, foram desenvolvidas outras
abordagens para lidar com as limitações da dependência na uniformidade das saídas, no número finito de resultados possíveis, etc.
A definição clássica de probabilidade é a razão entre o número de casos favoráveis e o de possíveis, ou seja, 
\begin{def*}
  Seja o evento A, associado a um espaço amostral finito e equiprovável, \(\Omega \). Definimos a probabilidade de ocorrência do evento A por: 
    \[
      \mathbb{P}(A) = \frac{\#(A)}{\#(\Omega )}.\quad \square
    \]
\end{def*}
  Richard von Mises buscou enxergar a probabilidade como algo que pode ser definido apenas para um gerador aleatório capaz de produzir uma sequência infinita de resultados.
A probabilidade, aqui, será a frequência limite do resultado nessa sequência. Ele baseou isso na ideia de que a aleatoriedade é um processo que produz resultados imprevisíveis
e não claramente determinados.
\begin{def*}
  Seja \(n_{A}\) o número de vezes que o evento A ocorre em n repetições independente de um mesmo experimento. Então, 
    \[
      \mathbb{P}(A) = \lim_{n\to \infty}\frac{n_{A}}{n},
    \]
    desde que o limite exista. \(\square\)
\end{def*}
 Essa definição mostrou-se importante até durante a contemporaneidade. No entanto, não é muito prático para fazer contas e encontrar probabilidades, 
sua força está em exibir a noção que a aplicabilidade traz. Além disso, ela funciona como uma ponte entre a primeira definição e a próxima que será vista.
Outro problema com essa é que, quando quantidades enormes de saídas surgem, torna-se impossível de usá-lo, pois não dá para saber o total de possibilidades.
  
  Assim, Kolmogorov estende a definição de probabilidade para espaços mais gerais, conhecido como a pessoa que formalizou a teoria da probabilidade. A ideia dele permite que
propostas mais flexíveis de probabilidades sejam usadas, saindo do limite de contabilidade finita e equiprovável da ideia de Laplace.
\begin{def*}
  Seja E um experimento aleatório e \(\Omega \) o espaço amostral associado. A cada evento A associamos um número real
 \(\mathbb{P}(A)\), denominado probabilidade de A, que satisfaz 
\begin{itemize}
  \item[P1)] \(\mathbb{P}(\Omega )=1;\)
    \item[P2)] \(0\leq \mathbb{P}(A)\leq 1,\) para todo A decorrente de \(\Omega \);
      \item[P3)] Se A e B forem eventos mutuamente exclusivos, então 
        \[
          \mathbb{P}(A\cup B) = \mathbb{P}(A) + \mathbb{P}(B);
        \]
      \item[P4)] Para qualquer sequência de eventos disjuntos dois-a-dois, \(A_{1}, A_{2}, \cdots, A_{n}\), tem-se 
        \[
          \mathbb{P}\biggl(\bigcup_{i=1}^{n}A_{i}\biggr) = \sum\limits_{i=1}^{n}\mathbb{P}(A_{i}).\quad \square
        \]
\end{itemize}
\end{def*}
  Seguem algumas propriedades:
 \begin{theorem*}
   Se \(\emptyset\) é um evento impossível, então \(\mathbb{P}(\emptyset) = 0.\)
 \end{theorem*}
 \begin{theorem*}
  Se \(A^{c}\) for o complementar de A, então \(\mathbb{P}(A^{c}) = 1 - \mathbb{P}(A).\)
 \end{theorem*}
\begin{theorem*}
  Se A e B são dois eventos quaisquer de \(\Omega \), então 
    \[
      \mathbb{P}(A\cup B) = \mathbb{P}(A) + \mathbb{P}(B) - \mathbb{P}(A\cap B).
    \]
\end{theorem*}
\begin{theorem*}
  Se \(A_{1}, A_{2}, \cdots, A_{i}\) são eventos dois-a-dois disjuntos, então 
  \begin{align*}
    \mathbb{P}(A_{1}\cup \cdots\cup A_{i}) &= \sum\limits_{i=1}^{n}\mathbb{P}(A_{i}) - \sum\limits_{i < j}^{n} \mathbb{P}(A_{i}\cap A_{j}) + \sum\limits_{i < j < r}^{n} \mathbb{P}(A_{i}\cap A_{j}\cap A_{r}) + \cdots\\
                                           &\cdots + (-1)^{n-1}\mathbb{P}(A_{1}\cap \cdots A_{n})
  \end{align*}
\end{theorem*}
\begin{example}
  Um lote é formado por 10 peças boas, 4 com defeitos menores e 2 com defeitos graves. Uma peça é escolhida ao acaso. Calcule a probabilidade de que 
 \begin{itemize}
   \item[a)] A peça não tenha defeito grave;
   \item[b)] A peça não tenha defeito;
   \item[c)] A peça seja boa ou tenha defeito grave.
 \end{itemize}
  a) Considerando A o conjunto de peças boas, B o de peças levemente defeituosas e C o de peças gravemente defeituosas, esses conjuntos são dois-a-dois disjuntos. Assim, 
    \[
      \mathbb{P}(C^{c}) = 1 - \mathbb{P}(C) = 1 - \frac{2}{16} = \frac{7}{8}.
    \]
    
  b) Com a notação anterior, 
    \[
      \mathbb{P}((B\cup C)^{c}) = \mathbb{P}(B^{c}\cap C^{c}) = \mathbb{P}(B^{c}) + \mathbb{P}(C^{c}) - \mathbb{P}(C^{c}\cup B^{c}) = \mathbb{P}(A) = \frac{10}{16} = \frac{5}{8}.
    \]

  c) Novamente, mantendo a anotação, 
    \[
      \mathbb{P}(A\cup C) = \mathbb{P}(A) + \mathbb{P}(C) = \frac{10}{16} + \frac{2}{16} = \frac{12}{16} = \frac{3}{4}.
    \]
\end{example}
\newpage

\section{Aula 03 - 31/08/2023}
\subsection{Motivações}
\begin{itemize}
  \item Probabilidade Condicional;
  \item Dependência de Eventos.
\end{itemize}
\subsection{A Probabilidade Condicional}
\begin{def*}
  Sejam dois eventos A e B associados ao mesmo espaço amostral \(\Omega \). A probabilidade condicional de A 
dado que ocorreu B é representada por \(\mathbb{P}(A|B\) e dada por 
  \[
    \mathbb{P}(A|B) = \frac{\mathbb{P}(A\cap B)}{\mathbb{P}(B)},\quad \mathbb{P}(B) > 0\quad \square.
  \]
\end{def*}
  Em particular, seguem duas representações para a probabilidade de dois eventos ocorrerem simultaneamente, sendo elas
  \begin{align*}
    &\mathbb{P}(A|B) = \frac{\mathbb{P}(A\cap B)}{\mathbb{P}(B)} \Rightarrow \mathbb{P}(A\cap B) = \mathbb{P}(A|B)\mathbb{P}(B)\\
    &\mathbb{P}(B|A) = \frac{\mathbb{P}(A\cap B)}{\mathbb{P}(A)} \Rightarrow \mathbb{P}(A\cap B) = \mathbb{P}(B|A)\mathbb{P}(A)\\
  \end{align*}
  \begin{example}
    Um dado de seis faces, equilibrado, é lançado e o número voltado para cima é obsecrado.
    \begin{itemize}
      \item[(a)] Se o resultado obtido for par, qual a probabilidade dele ser maior ou igual a 5?
      
    \item[(b)] Se o resultado obtido for maior ou igual a 5, qual a probabilidade dele ser par?
\end{itemize}

    Para o item a, o espaço amostral é \(\Omega = \{1, 2, 3, 4, 5, 6\}\). Considere os eventos
    A como o resultado de ser par e B o resultado obtido sendo maior ou igual a 5. Então, 
      \[
        \mathbb{P}(B|A) = \frac{\mathbb{P}(A\cap B)}{\mathbb{P}(A)} = \frac{\mathbb{P}(\{6\})}{\mathbb{P}(\{2, 4, 6\})} = \frac{1}{3};
      \]

      Para o item b, temos 
        \[
          \mathbb{P}(A|B) = \frac{\mathbb{P}(A\cap B)}{\mathbb{P}(B)} = \frac{\mathbb{P}(\{6\})}{\mathbb{P}(\{5, 6\})} = \frac{\frac{1}{6}}{\frac{2}{6}} = \frac{1}{2}.
        \]
  \end{example}
 \begin{def*}
   Os eventos A e B são independentes se a informação da ocorrência de B não altera a probabilidade atribuída ao evento A, isto é, 
     \[
       \mathbb{P}(A|B) = \mathbb{P}(A),
     \]
    ou, equivalentemente, 
    \[
      \mathbb{P}(A\cap B) = \mathbb{P}(A)\mathbb{P}(B).\quad\square
    \]
 \end{def*}
\begin{example}
  Uma moeda é viciada, de modo que a chance de sair cara é o dobro da de sair coroa.
  \begin{itemize}
    \item[(a)] Dê o espaço amostral

  \item[(b)] Calcule a probabilidade de ocorrer cara no lançamento desta moeda.
\end{itemize}

  (a) O espaço amostral desse evento é \(\Omega = \{\text{Cara, Coroa}\}\). Seja A o evento que cai cara e B o que cai coroa.

  (b) Sabemos que \(\mathbb{P}(A) + \mathbb{P}(B) = 1\) e que \(\mathbb{P}(A) = 2 \mathbb{P}(B)\), ou seja, 
    \[
      \mathbb{P}(A) + \frac{\mathbb{P}(A)}{2} = 1 \Rightarrow \mathbb{P}(A) = \frac{2}{3}.
    \]
\end{example}
\begin{example}
  Duas lâmpadas queimadas foram acidentalmente misturadas com seis boas. Se vamos testar as lâmpadas, uma por uma, até
encontrar duas defeituosas, qual é a probabilidade de que a última defeituosa seja encontrada no quarto teste?

  Estamos interessados em calcular a probabilidade do seguinte evento:
 \begin{align*}
   (\overline{D_{1}}\cap \overline{D_{2}}\cap D_{3}\cap D_{4})\cup(\overline{D_{1}}\cap D_{2}\cap \overline{D_{3}}\cap D_{4})\cup(D_{1}\cap \overline{D_{2}}\cap \overline{D_{3}}\cap D_{4})\\
   &\Rightarrow \mathbb{P}(\overline{D_{1}}\cap \overline{D_{2}}\cap D_{3}\cap D_{4})\cup(\overline{D_{1}}\cap D_{2}\cap \overline{D_{3}}\cap D_{4})\cup(D_{1}\cap \overline{D_{2}}\cap \overline{D_{3}}\cap D_{4})\\
   &= \mathbb{P}(\overline{D_{1}}\cap \overline{D_{2}}\cap D_{3}\cap D_{4}) + \mathbb{P}(\overline{D_{1}}\cap D_{2}\cap \overline{D_{3}}\cap D_{4}) + \mathbb{P}(D_{1}\cap \overline{D_{2}}\cap \overline{D_{3}}\cap D_{4}).
 \end{align*}
  Após manipulações algébricas e contas, chegamos em 
    \[
      \frac{1}{28} + \frac{1}{28} + \frac{1}{28} = \frac{3}{28}\approx 0,1071.
    \]
\end{example}
\begin{def*}
  Dizemos que \(A_{1}, A_{2}, \cdots, A_{n}\) formam uma partição de \(\Omega \) se eles são dois-a-dois disjuntos e a sua união é \(\Omega.\quad\square \)
\end{def*}
\begin{theorem*}
  Suponha que os eventos \(A_{1}, A_{2}, \cdots, A_{n}\) formam uma partição de \(\Omega \) e que todos têm probabilidade positiva. Então, para qualquer evento B, 
    \[
      \mathbb{P}(B) = \sum\limits_{i=1}^{n}\mathbb{P}(B|A_{i})\mathbb{P}(A_{i})
    \]
\end{theorem*}
\begin{example}
  Uma companhia produz circuitos integrados em três fábricas, sendo elas A, B e C. A fábrica A produz 40\% deles, e as outras produzem 30\% cada. As probabilidades de que um circuito integrado produzido por essas fábricas não funcione são 
 \(0,01; 0,04; 0,03\) respectivamente. Escolhido um circuito da produção conjunta das três fábricas, qual a probabilidade dele não funcionar? 
  
  Sendo D o evento em que o circuito é defeituoso e A, B, C os eventos de cada fábrica, sabemos que 
    \[
      \mathbb{P}(A) = 0,40,\quad \mathbb{P}(B) = 0,30,\quad \mathbb{P}(C) = 0,30.
    \]
    Além disso, sabemos que 
    \[
      \mathbb{P}(D|A) = 0,02,\quad \mathbb{P}(D|B) = 0,04,\quad \& \mathbb{P}(D|C) = 0,03
    \]
    Segue do teorema que 
      \[
        \mathbb{P}(D) = \sum\limits_{i=1}^{n}\mathbb{P}(D|A_{i})\mathbb{P}(A_{i}) = 0,025.
      \]

    Determine a probabilidade do defeituoso ter sido produzido pela empresa A. 
      \[
        \mathbb{P}(A|D) = \frac{\mathbb{P}(A\cap D)}{\mathbb{P}(D)} = \frac{\mathbb{P}(D|A)\mathbb{P}(A)}{\mathbb{P}(D)} = \frac{0,01\times 0,4}{0,025} = \frac{0,004}{0,025} = 0,16.
      \]
\end{example}
\end{document}
