\documentclass{article}
\usepackage{amsmath}
\usepackage{amsthm}
\usepackage{amssymb}
\usepackage{pgfplots}
\usepackage{amsfonts}
\usepackage[margin=2.5cm]{geometry}
\usepackage{graphicx}
\usepackage[export]{adjustbox}
\usepackage{fancyhdr}
\usepackage{hyperref}
\usepackage{lastpage}
\usepackage{mathtools}

\pagestyle{fancy}
\fancyhf{}

\pgfplotsset{compat = 1.18}

\hypersetup{
    colorlinks,
    citecolor=black,
    filecolor=black,
    linkcolor=black,
    urlcolor=black
}

\rfoot{Page \thepage \hspace{1pt} of \pageref{LastPage}}

\newtheorem*{def*}{Definition}

\title{CANTOR SETS AND SEQUENCE SPACES}
\author{Renan Wenzel}
\date{\today}

\begin{document}
\maketitle

\section*{Day 12/12/2022}
\subsection{Reviewing}
Topics from last class:
Showed that cylinders form open sets and are equivalent to open balls with the metric:
  $$
  d_{\lambda}(\omega, \omega') = \sum_{n=-\infty}^{\infty}\frac{|\omega _{n} - \omega' _{n}|}{\lambda ^{|n|}}
  $$

Found a connection between product topology and metric topology

Showed that cylinders are clopen sets

Showed that the metric above has less contributions from higher order terms.
\subsection{Connectedness}
On today's seminar, we are going to see a few topological concepts needed to better grasp the ideas behind
the relationship between Cantor Sets and Sequence Spaces. Starting with the notion of connectedness of a space,
the intuition behind is that of a "space that is a single piece." Formally, we define it as
\begin{def*}
  If X is a set such that it cannot be divided into two disjoint non-empty open sets, then X is connected.
\end{def*}
For our purposes, even though this definition given is the most intuitive, we are going to use an equivalent 
definition, namely 
 \begin{def*}
   A space X is connected when the only two sets that are both open and closed are X and $\emptyset.$
 \end{def*}
 The usefulness comes from the fact that since our Cylinder sets are clopen, just like it was proved on the last
seminar, there is no connected subspace of X other than one-point sets, which are trivially connected. Thus the 
space is totally disconnected. Hence, we've cooked one ingredient in our Cantor Set homeomorphism soup.

\subsection{Perfect Set}
To cut it short, 
\begin{def*}
  A set X is said to be perfect if it is closed and has no isolated point.
\end{def*}
However, we must clarify what is an isolated point. We define an element x of X to be an isolated point provided
that there exists a neighborhood U of x that contains no other points besides x itself. In other words,
a point is not an isolated points if every neighborhood U of x contains at least one more point p other than x.

Now, recall from calculus and analysis the definition of a limit point:
\begin{quote}
  ``An element x of a set S is said to be a limit point if every neighorhood U of x contains at least one point
  p such that $p\neq{x}.$"  
\end{quote}
We conclude that a set X is perfect provided that it is closed AND every point of X is a limit point! 

Let's show that $\Omega _{2} ^{R}$ is a perfect set, i.e., we are going to prove that given a point $\omega\in\Omega _{2}^{R}$,
then $\omega\in{C _{\alpha_i}^{n_i}}.$ However, notice that whichever other $\omega'\in\Omega$ whose i-th entry is also
$\alpha_i$ also belongs to that neighborhood of $\omega$. Hence, the same reasoning shows that any neighborhood of
$\omega$ contains other points different from it. Thus, since our point was arbitrary, this is true for any element
of $\Omega _{2}^{R}$, showing that any point is a limit point. We conclude from this that our space is perfect.
\end{document}

