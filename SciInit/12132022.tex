\documentclass{article}
\usepackage{amsmath}
\usepackage{tikz}
\usepackage{minted}
\usepackage{amsthm}
\usepackage{amssymb}
\usepackage{pgfplots}
\usepackage{amsfonts}
\usepackage[margin=2.5cm]{geometry}
\usepackage{graphicx}
\usepackage[export]{adjustbox}
\usepackage{fancyhdr}
\usepackage{hyperref}
\usepackage{lastpage}
\usepackage{mathtools}

\pagestyle{fancy}
\fancyhf{}

\pgfplotsset{compat = 1.18}

\hypersetup{
    colorlinks,
    citecolor=black,
    filecolor=black,
    linkcolor=black,
    urlcolor=black
}

\rfoot{Page \thepage \hspace{1pt} of \pageref{LastPage}}

\newcommand\SLash{\char`\\}
\newtheorem*{def*}{Definition}
\newtheorem*{theo*}{Theorem}

\title{CANTOR SETS AND SEQUENCE SPACES}
\author{Renan Wenzel}
\date{\today}

\begin{document}
\maketitle
\tableofcontents
\newpage
\section{Preliminaries(Done during lectures)}

\paragraph{} This first part is to recall and introduce a few ideas needed to grasp the content within these notes.

\begin{def*}
  A topology $\tau$ on X is a collection of sets such that
\end{def*}
\begin{itemize}
  \item[i)] $X, \emptyset\in{\tau}$
  \item[ii)] $\text{ If } U _{1}, \cdots, U _{\alpha} \text{ are an arbitrary family of sets } U _{\alpha}\in{\tau}, 
  \alpha\in{J}, \text{ then } \bigcup _{\alpha\in{J}}U _{\alpha}\in{\tau}$
  \item[iii)] $\text{ If } U _{1}, \cdots, U _{n} \text{ belong to } \tau, \text{ then } \bigcap _{i=1}^{n}U _{i}\in{\tau}.$
\end{itemize}

\begin{def*}
  The discrete topology on X is the one in which open sets are subsets of X.
\end{def*}

\begin{def*}
  A metric on a set X is a function $d:X\times{X}\rightarrow \mathbb{R}$ which satisfies the following
\end{def*}
\begin{itemize}
  \item[i)] $d(x, y)\geq{0}, d(x, y) = 0\Longleftrightarrow x = y;$
  \item[ii)] $d(x, y) = d(y, x);$ 
  \item[iii)] $d(x, z) \leq d(x, y) + d(y, z).$
\end{itemize}

\begin{def*}
  A basis $\mathcal{B}$ for a topology is a collection of subsets of X such that
 \begin{itemize}
   \item[i)] For each x in X, there is at least one B in $\mathcal{B}$ containing x;
   \item[ii)] If x belongs to the intersection of two basis elements, there is a basis element containing x and inside their intersection.
 \end{itemize}
\end{def*}

Another characterization of a basis is as follows:
\begin{theo*}
  Let X be a topological space. Suppose that $\mathcal{C}$ is a collection of open sets of X such that for each open set
U and each x in U, there is an element C of $\mathcal{C}$ such that $x\in{C}\subseteq{U}$. Then, C is a basis for the topology 
of X.
\end{theo*}

For a product of topological spaces, we can define a topology on it based on the topologies of each individual space as follows:
\begin{def*}
  Let X = $\prod\limits _{\alpha\in{J}} X _{\alpha}$, where each $X _{\alpha}$ is a topological space. We define the product
topology on X as the one given by the basis $\mathcal{B}$ such that each basis element is defined as
  $$
  B = \prod _{\alpha\in{J}} U _{\alpha}, \quad U _{\alpha}\neq{X _{\alpha}} \text{ for all but finitely many }\alpha
  $$
\end{def*}

Each metric defines a topology themselves, known as the metric topology, in which basis elements are open balls and
the topology is induced by this basis.
\begin{def*}
  Given a metric space (X, d), an open ball of radius $\epsilon$ centered at a is the set
  $$
  B_d(a, \epsilon) := \{x\in{X}: d(x, a) < \epsilon\}.
  $$
\end{def*}
\begin{def*}
  Given a metric space (X, d), the metric topology $\tau_d$ induced by d on X is the one generated by the basis
 $\mathcal{B}$ whose basis elements are open balls.
\end{def*}

\section{Day 24/11/2022}
\subsection{Introduction and Main Set}
The main goal of these seminar (/notes) is to prepare whoever is using them to understand sequence spaces and their
construction, with the final goal of reaching the Cantor Set construction. We start off defining our main set,
known as the Sequence Space of N-1 digits, i.e., 
 \begin{align*}
   &\Omega_{N} := \{\omega = (\cdots, \omega _{-1}, \omega _{0}, \omega _{1}, \cdots): \omega _{i}\in \{0, 1, \cdots, N-1\}\} = \{0, 1, \cdots, N-1\}^{\mathbb{Z}} \\
   &\Omega _{N}^{R} := \{\omega = (\omega _{0}, \cdots): \omega _{i}\in \{0, 1, \cdots, N-1\}\} = \{0, 1, \cdots, N-1\} ^{\mathbb{N}}
 \end{align*}
One way to think of this family of spaces is that it records the results from an infinite game of throwing N-sided dices.
For instance, if we stay simple and assume a game of heads-and-tails, with only two possibilities, the space would be
 \begin{align*}
 \Omega _{2}^{R} = \{0, 1\} ^{\mathbb{N}} = \{\omega = (\omega _{0}, \cdots): \omega _{i}\in \{0, 1\}\}
 \end{align*}
This space specifically is the one we are going to work the most with, since it is going to be
shown that it is equivalent to a Cantor Set, and it also makes for an easy example regarding whatever
new construction we add to the space.

\subsection{Structuring the Space}
  The next few steps are related to the structure of this space. We're going to define a topology on it, followed 
by a metric. Start by endowing $\{0, 1, \cdots, N-1\}$ with the discrete topology, in a way that
since the space is finite, it is also compact. From here, consider the sets $X _{i} = \{0, 1, \cdots, 
N-1\}, i\in \mathbb{N}$, and define
  $$
    B = \prod\limits_{i=1}^{n-1}U _{i} \times{} \prod\limits_{i=n}^{\infty}X _{i},
  $$ 
where each $U _{i}$ is a proper subset of $X _{i}$. Hence, the collection $\mathcal{B}$ of all the sets
B form a basis for the product topology on $\Omega _{N}^{R}  $
  
For example, consider N = 2, i.e., the topology on $\{0, 1\}$, in which the discrete topology is $\tau = \{\emptyset, \{0, 1\}, \{0\}, \{1\}\}$. 
Then, the basis for the product topology is given by 
  $$
  B = \prod\limits_{i=1}^{n-1}U _{i} \times{} \prod\limits_{i=n}^{\infty}X _{i} = \prod\limits_{i=1}^{n}\{b_i\}\times{}\prod\limits_{i=n}^{\infty}U_i.
  $$ 
em que $b _{i}\in\{0, 1\}$ e $U _{i} = \{0, 1\} $. Another way of defining the topology is via what is known as cylinder sets, defined by fixing integers $n _{1} < \cdots < n _{k}$ and $\alpha _{1}, \cdots, 
\alpha _{k}\in \{0, 1, \cdots, N-1\} $, and putting
  $$
  C _{\alpha _{1}, \cdots, \alpha _{k}}^{n _{1}, \cdots, n _{k}}:= \{\omega\in \Omega _{N}^{R}: \omega _{n_i} = \alpha _{i}\}
  $$
where i varies from 1 to k. Specifically, this is known as a cylinder of rank k, and the topology to be defined is
such that all open sets are cylinders, hence the basis is given by them. However, to show that it is indeed a basis,
take $x\in{C _{\alpha _{1}, \cdots, \alpha _{k}}^{n _{1}, \cdots, n _{k}}}$. Hence, for some $1\leq{j}\leq{k}$,
  $$
  x\in C _{\alpha _{1}}^{n _{1}}\cap C _{\alpha _{2}}^{n _{2}} \cap \cdots \cap C _{\alpha _{k}}^{n _{k}}
  $$
so that it satisfies the theorem characterizing a basis, i.e., $\mathcal{C}$ is a basis of cylinders. Moreover,
notice that given a cylinder, 
  $$
  \Omega _{N}^{R}\SLash\{C_{\alpha_{1}, \cdots, \alpha_{k}}^{n_{1}, \cdots, n_{k}}\} = \{\omega\in \Omega_{N}^{R}: \omega_{n_{i}}\neq \alpha_{i}, i = 1, \cdots, k\} = C^{n_{1}, \cdots, n_{k}}_{\sigma(\alpha_{1}), \cdots, \sigma(\alpha_{k}),}
  $$
where $\sigma$ is a permutation that excludes the j-th term $(\sigma(\alpha_{i})\neq \alpha_{k}\forall i=1, \cdots, k).$ As a consequence,
we see that every cylinder is both open and closed in $\Omega_{N}^{R}.$

  Our next step will be defining a metric for our space and prove some of its properties.
\newpage

\section{Day 05/12/2022}
\subsection{Motivations}
\begin{itemize}
  \item Cylinders are clopen sets;
  \item Sequence space metric;
  \item Cylinders are open balls in that metric;
  \item $\Omega_{N}$ is a perfect set (First part of showing the homeomorphism with a Cantor Set).
\end{itemize}
\subsection{Reviewing}
  Topics from last class:
 \begin{itemize}
   \item Introduced the main space we're working on;
   \item Defined the bases for the topology on $\Omega_{N}^{R}$;
   \item We showed that each cylinder set form open sets.
 \end{itemize}
 \subsection{$\Omega_{N}^{R}$ as a Metric Space}
  Our goal with this is to define a metric such that the space is perfect and can be related to the cylinder topology.
For a fixed $\lambda > 1,$ define $d:\Omega_{N}^{R}\times{\Omega_{N}^{R}}\rightarrow \mathbb{R}$ by
  $$
    d_{\lambda}(\omega, \omega') = \sum\limits_{n=-\infty}^{\infty} \frac{|\omega_{n} - \omega_{n}'|}{\lambda^{|n|}}.
  $$
  Checking the metric axioms is not the point of this seminar, thus it's left undone on the notes. However, it hurts no one to 
comment that it all follows from the properties granted from the usual metric of $\mathbb{R}^{n}$ (i.e. the absolute value).

\subsection{Open Balls in $\Omega_{N}^{R}$}
  Using the definition of open balls (c.f. Section 1)k, let's analyze the case of an open ball centered at 0 with raidus r
with the metric we just defined. For the sake of simplicity, take N to be 2, so that for large $\lambda$, specifically $\lambda=10N=20$, fix
the zero sequence $\omega' = (\cdots, 0, 0, 0, \cdots)$. We want to find $r > 0$ such that $B_{r}(\omega') = C_{0}^{0} = \{\nu = (\nu_{n})_{n\in\mathbb{Z}}: \nu_{0} = 0\}.$
Equivalently, we're going to prove that
  $$
  \{\omega\in  \Omega_{2}: d_{\lambda}(\omega, \omega') < r\} = \{\omega\in \Omega_{2}: \omega_{0} = 0\}.
  $$
  Analyzing the terms of the series under these condition, one gets 
  $$
  d_{\lambda}(\omega, \omega') = \sum\limits_{n=-\infty}^{\infty} \frac{|\omega_{n} - \omega_{n}'|}{\lambda^{|n|}} = \sum\limits_{n=-\infty}^{\infty} \frac{|\omega_{n}|}{20^{|n|}}
  $$
It helps to visualize a few cases of n to understand the behavior of the series. In fact, it follows that
  \begin{align*}
  &n=0 \rightarrow \frac{|\omega_{0} - \omega_{0}'|}{20^{n}} = \left\{\begin{array}{ll}
      0, \quad \text{if } \omega_{0} = \omega_{0}' = 0 \\
      \frac{1}{1} = 1, \quad \text{if } \omega_{0}\neq \omega_{0}'
    \end{array}\right.\\
  &n=1 \rightarrow \frac{|\omega_{1} - \omega_{1}'|}{20} = \left\{\begin{array}{ll}
        0, \quad \text{if } \omega_{1} = \omega_{1}' = 0 \\
        \frac{1}{20}, \quad \text{if } \omega_{1}\neq \omega_{1}'
      \end{array}\right.\\
  &  \vdots\\
  &n=k \rightarrow \frac{|\omega_{\pm k} - \omega_{\pm k}'|}{20^{|k|}} = \left\{\begin{array}{ll}
      0, \quad \text{if } \omega_{\pm k} = \omega_{\pm k}' \\
      \frac{1}{20^{|k|}}, \quad \text{if } \omega_{\pm k}\neq \omega_{\pm k}
    \end{array}\right.
  \end{align*}
  In other words, we are looking for an r such that
  $$
    d_{\lambda}(\omega, \omega') = \left\{\begin{array}{ll}
        \sum\limits_{n=-\infty}^{\infty}\frac{1}{20^{|n|}}, \quad \omega_{n}\neq \omega_{n} \\
        0, \quad \omega_{n} = \omega_{n}'
      \end{array}\right.
  $$
  Because $\displaystyle \frac{1}{20} < 1,$ this series converges, so that it works for any $r\in\biggl(\sum\limits_{|n|\geq{1}}^{}20^{-n}, 1\biggr]$ works.
In fact, the cases in which $\omega_{n} = \omega_{n}'$ for some $n\in \mathbb{Z}$ yield
  $$
    d(\omega, \omega')\leq \sum\limits_{|n|\geq{1}}^{}20^{-n} < r
  $$
Notice that the case r = 1 must be included if $\omega_{0}\neq \omega_{0}'$ but $\omega_{n} = \omega_{n}'$ for $n\in \mathbb{Z}\SLash \{ 0\} $. 
Moreover, notice that $20^{-n}$ is just $\lambda^{-n}$, such that we can calculate the radius of the ball for a given large $\lambda.$

  One final comment should be added. On any metric space, a topology given by the metric itself can be created by setting
the basis to be open balls and open sets being their union. That way, the topology induced by the metric just studied, in which
open balls have been shown to be cylinders, is one generated by them (cylinders). Hence, it is equivalent to the first topology we
defined on $\Omega_{N}^{R}!$
\newpage

\section{Day 12/12/2022}
\subsection{Reviewing}
Topics from last class:
  \begin{itemize}
    \item Showed that cylinders form open sets and are equivalent to open balls with the metric:
  $$
  d_{\lambda}(\omega, \omega') = \sum_{n=-\infty}^{\infty}\frac{|\omega _{n} - \omega' _{n}|}{\lambda ^{|n|}};
  $$
  \item Found a connection between product topology and metric topology;
  \item Showed that cylinders are clopen sets;
  \item Showed that the metric above has less contributions from higher order terms.
  \end{itemize}
\subsection{Connectedness}
On today's seminar, we are going to see a few topological concepts needed to better grasp the ideas behind
the relationship between Cantor Sets and Sequence Spaces. Starting with the notion of connectedness of a space,
the intuition behind is that of a "space that is a single piece." Formally, we define it as
\begin{def*}
  If X is a set such that it cannot be divided into two disjoint non-empty open sets, then X is connected.
\end{def*}
For our purposes, even though this definition given is the most intuitive, we are going to use an equivalent 
definition, namely 
 \begin{def*}
   A space X is connected when the only two sets that are both open and closed are X and $\emptyset.$
 \end{def*}
 The usefulness comes from the fact that since our Cylinder sets are clopen, just like it was proved on the last
seminar, there is no connected subspace of X other than one-point sets, which are trivially connected. Thus the 
space is totally disconnected. Hence, we've cooked one ingredient in our Cantor Set homeomorphism soup.

\subsection{Perfect Set}
To cut it short, 
\begin{def*}
  A set X is said to be perfect if it is closed and has no isolated point.
\end{def*}
However, we must clarify what is an isolated point. We define an element x of X to be an isolated point provided
that there exists a neighborhood U of x that contains no other points besides x itself. In other words,
a point is not an isolated points if every neighborhood U of x contains at least one more point p other than x.

Now, recall from calculus and analysis the definition of a limit point:
\begin{quote}
  ``An element x of a set S is said to be a limit point if every neighorhood U of x contains at least one point
  p such that $p\neq{x}.$"  
\end{quote}
We conclude that a set X is perfect provided that it is closed AND every point of X is a limit point! 

Let's show that $\Omega _{2} ^{R}$ is a perfect set, i.e., we are going to prove that given a point $\omega\in\Omega _{2}^{R}$,
then $\omega\in{C _{\alpha_i}^{n_i}}.$ However, notice that whichever other $\omega'\in\Omega$ whose i-th entry is also
$\alpha_i$ also belongs to that neighborhood of $\omega$. Hence, the same reasoning shows that any neighborhood of
$\omega$ contains other points different from it. Thus, since our point was arbitrary, this is true for any element
of $\Omega _{2}^{R}$, showing that any point is a limit point. We conclude from this that our space is perfect.
\newpage

\section{Simulations}
\begin{minted}{python}
  int x = 0
  
\end{minted}

\end{document}

