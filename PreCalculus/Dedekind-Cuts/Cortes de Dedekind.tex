\documentclass{article}
\usepackage{amsmath}
\usepackage{amsthm}
\usepackage{amssymb}
\usepackage{amsfonts}
\usepackage[margin=2.5cm]{geometry}
\usepackage{graphicx}
\usepackage[export]{adjustbox}
\usepackage{fancyhdr}
\usepackage[portuguese]{babel}
\usepackage{hyperref}
\usepackage{lastpage}
\usepackage{biblatex}

\pagestyle{fancy}
\fancyhf{}

\hypersetup{
    colorlinks,
    citecolor=black,
    filecolor=black,
    linkcolor=black,
    urlcolor=black
}

\newtheorem*{proposition*}{\underline{Proposi\c c\~ao:}}
\newtheorem{definition}{\underline{Defini\c c\~ao}}
\newtheorem*{theorem*}{\underline{Teorema:}}
\newtheorem*{sol*}{\underline{Solu\c c\~ao:}}
\newtheorem*{proof*}{\underline{Prova:}}
\renewcommand\qedsymbol{$\blacksquare$}

\rfoot{P\'agina \thepage \hspace{1pt} de \pageref{LastPage}}

\title{EXERC\'ICIOS DE C\'ALCULO}
\author{Renan Wenzel}
\date{\today}

\begin{document}

\begin{figure}[ht]
	\minipage{0.76\textwidth}
		\includegraphics[width=4cm]{../icmc.png}
		\hspace{5cm}
		\includegraphics[height=4.9cm,width=4cm]{../brasao_usp_cor.jpg}
	\endminipage	
\end{figure}

\begin{center}
	\vspace{1cm}
	\LARGE
	UNIVERSIDADE DE S\~AO PAULO

	\vspace{1.3cm}
	\LARGE
	INSTITUTO DE CI\^ENCIAS MATEM\'ATICAS E COMPUTACIONAIS - ICMC

	\vspace{1.7cm}
	\Large
	\textbf{CONSTRUINDO OS REAIS $\mathbb{R}$}
	\vspace{0.5cm}

	\small
	\textbf{UMA INTRODU\c C\~AO AOS CORTES DE DEDEKIND}

	\vspace{1.3cm}
	\large
	\textbf{Renan Wenzel - 11169472}

	\vspace{1.3cm}
	\large
	\textbf{Pedro Magalh\~aes Rios - prios@icmc.usp.br}

	\vspace{1.3cm}
	\today
\end{center}

\newpage
\section{Breve Introdu\c c\~ao}

\paragraph{} O uso dos cortes de Dedekind tem como preceito a constru\c c\~ao dos n\'umeros reais a partir dos racionais, e se baseia em subconjuntos espec\'ificos que satisfazem o axioma do supremo. A forma que isso resulta nos n\'umeros reais segue de um teorema (que assumiremos ser verdade aqui) que diz que o conjunto dos reais \'e o \'unico corpo ordenado que satisfaz o axioma do Supremos, ou seja, mostrando que os cortes satisfazem ele, seu conjunto deve ser R tamb\'em. 

Sem mais delongas, vx`amos defin\'i-los a seguir e provar algumas propriedades antes de prosseguir com a constru\c c\~ao dos reais em seguida.

\section{Defini\c c\~ao e Resultados b\'asicos}
\begin{definition}[Cortes] 
Um corte \'e um conjunto $\alpha\subset\mathbb{Q}$ tal que:
\begin{itemize}
\item [i)] $\alpha\neq\emptyset$ e $\alpha\neq\mathbb{Q}$;
\item [ii)] Se $p\in\alpha, q\in\mathbb{Q}$ e $q<p$ , ent\~ao $q\in\alpha$;
\item [iii)] Se $p\in\alpha$, ent\~ao existe $r\in\alpha$ tal que $r < p$.
\end{itemize}
\end{definition}

\paragraph{} Vale adicionar alguns coment\'arios para esclarecer e entender essa defini\c c\~ao. Em primeiro lugar, garantimos que um corte n\~ao ser\'a vazio e nem o conjunto todo, at\'e porque iremos de alguma forma construir os reais a partir disso, ent\~ao um corte n\~ao pode totalizar os racionais. A segunda propriedade garante que todo elemento racoinal \`a esquerda de um corte tamb\'em pertencer\'a a ele. Por fim, a \'ultima propriedade afirma que um corte n\~ao possui um elemento maximal, visto que, dado qualquer n\'umero pertencente ao corte, \'e poss\'ivel encontrar um maior que ele.

\paragraph{} Antes de provarmos que h\'a um axioma do supremo, \'e necess\'ario definir uma rela\c c\~ao de ordem entre os cortes. Utilizaremos como base as rela\c c\~oes de pertin\^encias entre conjuntos.

\begin{definition}
Uma rela\c c\~ao de ordem $<$ entre cortes \'e dada por: $\alpha < \beta \Rightarrow \alpha\subsetneq\beta$
\end{definition}

Verifiquemos, por fim, que essa rela\c c\~ao \'e realmente uma ordem:
\begin{proof*} No que segue, sejam $\alpha, \beta, \gamma$ cortes de Dedekind

a) Transitividade: Suponha que $\alpha < \beta, \beta < \gamma.$ Explicitamente falando, isso significa que $\alpha\subsetneq\beta, \beta\subsetneq\gamma.$ A partir disto, segue das propriedades de pertin\^encia de conjuntos que:
$$
	\alpha\subsetneq\beta\subsetneq\gamma \Rightarrow \alpha\subsetneq\gamma.
$$
Logo, por defini\c c\~ao, $\alpha < \gamma.$

b)Tricotomia: Em primeiro lugar, suponha que $\alpha < \beta.$ Ent\~ao, $\alpha\subsetneq\beta$, ou seja, $\alpha$ est\'a exclusivamente contido em $\beta$, do que segue que $\beta$ n\~ao pode estar exclusivamente contido em $\alpha$, isto \'e, $\beta \nless \alpha.$ A prova de que se $\beta < \alpha$, ent\~ao $\alpha \nless \beta$ \'e an\'aloga. Por fim, caso $\alpha = \beta$, ent\~ao $\alpha\subseteq\beta$ e $\beta\subseteq\alpha$, tal que a conten\c c\~ao n\~ao \'e restrita em nenhum dos casos. Destarte, apenas a igualdade pode ocorrer.

Portanto, $<$ define uma ordem entre os cortes.
\qedsymbol
\end{proof*}

\section{O Axioma do Supremo}
Nesta se\c c\~ao, provaremos que o conjunto dos cortes munido da ordem definida satisfaz o axioma do supremo.
\begin{theorem*}
O conjunto $\mathbb{R}$ munido da ordem de cortes satisfaz o axioma do supremo.
\end{theorem*}
\begin{proof*}
Seja A um conjunto n\~ao-vazio de $\mathbb{R}$ e $\beta\in\mathbb{R}$ um limitante superior de A. Seja $\gamma=\bigcup_{\alpha\in{A}}\alpha.$ Como A \'e n\~ao-vazio, existe um elemento $\alpha_0\in{A}$ n\~ao-vazio (pois \'e um corte). Como $\alpha_0\subseteq\gamma, \gamma\neq\emptyset$ e $\gamma\subset\beta$, j\'a que $\beta > \alpha$ para todo corte $\alpha.$

Precisamos, tamb\'em, explicar que $\gamma$ n\~ao \'e o conjunto todo dos racionais. Para tal, note que, como $\beta > \alpha$ para todo $\alpha, \beta > \bigcup_{\alpha\in{A}} = \gamma.$ Por $\beta$ ser um corte, $\beta\subsetneq\mathbb{Q}$ e, como a rela\c c\~ao de ordem definida \'e conten\c c\~ao de conjuntos restrita, temos 
$$ 
\gamma \subsetneq \beta \subsetneq \mathbb{Q} \Rightarrow \gamma\subsetneq\mathbb{Q}.
$$
Desta forma, $\gamma\neq\mathbb{Q}.$ 

O pr\'oximo passo \'e mostrar que $\gamma$ n\~ao possui m\'aximo. Para isso, tome $r\in\gamma.$ Ent\~ao, $r\in\alpha_0$ para algum $\alpha_0,$ tal que existe um $s\in\alpha_0$ para o qual vale $s > r.$ Assim, n\~ao importa qual elemento de $\gamma$ seja selecionado, sempre \'e poss\'ivel encontrar um outro maior que ele.

Por fim, afirmamos que $\gamma = \sup{A}.$ J\'a sabemos que $\gamma$ \'e cota superior de todo $\alpha,$ visto que $\gamma > \alpha.$ Seja $\gamma'\in\mathbb{R}$ tal que $\gamma' < \gamma$, ou seja, $\gamma' \subsetneq \gamma,$ de modo que existe um $q\in\gamma'$ e que $q\notin\gamma.$ Assim, existe um $\alpha\in{A}$ para o qual $q\in\alpha$ e, consequentemente, $\gamma' < \alpha \leq \gamma.$ Portanto, $\gamma=\sup{A},$ mostrando que o conjunto $\mathbb{R}$ satisfaz o axioma do supremo.
  
\qedsymbol
\end{proof*}

\section{Axiomas de Corpo}

Juntando o que obtivemos nas \'ultimas duas se\c c\~oes, chegamos em um conjunto $\mathbb{R}$ formado pelos cortes e que satisfaz o axioma do supremo. A seguir, vamos acrescentar duas opera\c c\~oes, + e ., a fim de tornar este conjunto um corpo ordenado.

\begin{definition}[Soma]
Em $\mathbb{R}$, defina a seguinte opera\c c\~ao entre os cortes:
$$
\alpha + \beta:= \{r + s: r\in\alpha, s\in\beta\}
$$
\end{definition}
\begin{proposition*}
$(\mathbb{R}, +)$ \'e um grupo abeliano (satisfaz as propriedades de soma de corpo) cujo elemento nulo \'e $\theta:=\{r\in\mathbb{Q}: r < 0\}$
\end{proposition*}
\begin{proof*}
Para poupar tempo, mostraremos apenas a propriedade do elemento nulo. Deixando ela expl\'icita, queremos mostrar que $\alpha + \theta = \alpha$ e, se tratando de conjuntos, isso \'e o mesmo que mostrar que $\alpha + \theta \subseteq \alpha \text{ e } \alpha \subseteq \alpha + \theta.$ Com efeito, tome $\alpha\in\mathbb{R}$ e seja $\theta$ como no enunciado acima. Segue que:
$$
\alpha + \theta = \{r + q: r\in\mathbb{Q}, q < 0\}
$$
Mas, note que $r + q < r$, ou seja, $r + q\in\alpha$ pela propriedade 2 dos cortes. Com isso, $\alpha + \theta \subseteq \alpha$

Por outro lado, sejam $s, r\in\alpha$ tais que $s > r$. Ent\~ao, $r - s < 0$, tal que $r - s\in\theta$ e $r = s + (r - s) \in\alpha + \theta.$ Destarte, $\alpha\subseteq\alpha + \theta,$ donde segue que $\alpha + \theta = \alpha$. 
\end{proof*}

Antes de dar a defini\c c\~ao do produto em si, definimos a opera\c c\~ao, para $\alpha > \theta, \beta > \theta$, $\alpha\circ\beta:=\{pq: p\in\alpha, q\in\beta\}$ e afirmamos (sem provar que ela) \'e um corte. Com isso, o produto entre cortes \'e dado por:

\begin{definition}[Produto]
Em $\mathbb{R}$, defina a seguinte opera\c c\~ao entre os cortes:
$$
\alpha\cdot\beta = \left\{\begin{array}{ll}
		\alpha\circ\beta, & \quad \text{se } \alpha > \theta, \beta > \theta \\
		(-\alpha)\circ\beta, & \quad \text{se } \alpha < \theta, \beta > \theta\\
		\alpha\circ(-\beta), & \quad \text{se } \alpha > \theta, \beta < \theta \\
		(-\alpha)\circ(-\beta), & \quad \text{se } \alpha < \theta, \beta < \theta
\end{array}\right.
$$
\end{definition}

O elemento neutro da multiplica\c c\~ao \'e o conjunto $\mathfrak{1}:=\{q\in\mathbb{Q}: q < 1\}$

\begin{proposition*}
$(\mathbb{R}, .)$ satisfaz as propriedades de multiplica\c c\~ao de corpo
\end{proposition*}
\begin{proof*}
Para poupar tempo, mostraremos apenas a propriedade do elemento neutro. Com efeito, seja $\alpha > \theta$. Ent\~ao, 
$$
\alpha\cdot\mathfrak{1}=\{pq: p\in\alpha, q < 1\}
$$
Assim, os elementos de $\alpha\cdot\mathfrak{1}$ s\~ao tais que $pq < p\cdot{1} = p\in\alpha$, ou seja, $\alpha\cdot\mathfrak{1}\subseteq\alpha.$

Por outro lado, tome um elemento de $p, q\in\alpha, p < q$ Ent\~ao, $p = (p \cdot q^{-1}) \cdot q.$ Como $p < q, p\cdot{q^{1}} < q$, de forma que $p\in\alpha\mathfrak{1}$, mostrando como $\alpha\subseteq\alpha\cdot\mathfrak{1}$. Portanto, $\alpha = \alpha\cdot\mathfrak{1}.$
\qedsymbol
\end{proof*}

\section{A Conclus\~ao}

\paragraph{}Com essas duas coisas, obtivemos um corpo odernado que satisfaz a propriedade do supremo, donde segue de um resultado famoso na matem\'atica que ele deve ser os reais $\mathbb{R}.$ Aqui, muitos passos foram pulados para economizar tempo, mas para quem se interessar, posso mandar coisas extras e exerc\'icios sobre cortes e a constru\c c\~ao.

Enfim conclui-se a constru\c c\~ao de Dedekind dos n\'umeros reais.
\begin{thebibliography}{9}
\bibitem{Rudin}
Rudin, w.: "Principles of Mathematical Analysis", 3rd ed., McGraw-Hill Book Company, New York, 1976
\end{thebibliography}
\end{document}