\section{Reta Tangente, Derivada, Derivada de Algumas fun\c c\~oes, Regra da Cadeia}
\subsection[Derivadas de Ordem Superior]{Derivadas de Ordem Superior}
\subsubsection{Exerc\'icio 1.} Encontre uma forma de expressar as derivadas de ordem qualquer 
das fun\c c\~oes $\cos(x)$ e 1/x.

\begin{proof*}
Nosso objetivo \'e obter f\'ormulas gerais para a n-\'esima derivada de $\cos(x) \text{ e } \frac{1}{x}$. Para isso, vamos levar em conta 
o processo realizado com rela\c c\~ao ao seno. A natureza c\'iclica de ambas as fun\c c\~oes sinaliza
a liga\c c\~ao entre as derivadas. Com efeito, 

$$
\begin{array}{ll}
	\frac{d \cos(x)}{d x} = -\sin(x) \\
	\frac{d^2 \cos(x)}{d x^2} = -\frac{d}{dx}\sin(x) = -\cos(x)\\
	\frac{d^3 \cos(x)}{d x^3}\cos(x) = -\frac{d^2}{dx^2}\sin(x) = -\frac{d}{dx}\cos(x) = \sin(x)\\
	\frac{d^4 \cos(x)}{d x^4}\cos(x) = -\frac{d^3}{dx^3}\sin(x) = -\frac{d^2}{dx^2}\cos(x) = \frac{d}{dx}\sin(x) = \cos(x)
\end{array}
$$  
Com base nisso, \'e poss\'ivel concluir que as derivadas de cosseno tamb\'em formam um ciclo e, portanto, podemos escrever uma forma geral como:
$$
\frac{d^n \cos(x)}{d x^n} = \left\{
\begin{array}{ll}
	\cos(x), &\quad x\equiv 0\text{ mod}4 \\
	-\sin(x), &\quad x\equiv 1\text{ mod}4\\
	-\cos(x), &\quad x\equiv 2\text{ mod}4\\
	\sin(x), &\quad x\equiv 3\text{ mod}4 
\end{array}
\right.
$$
Agora, lidemos com a quest\~ao da inversa de x. Note que $\frac{1}{x} = x^{-1}$, e provemos por indu\c c\~ao que a 
n-\'esima derivada de x \'e $\frac{d^n}{d x^n}\frac{1}{x} = (-1)^n\frac{1}{x^{n+1}}$. Come\c cando 
com o caso base n = 1:
$$
\frac{d}{dx}\frac{1}{x} = \frac{d}{dx}x^{-1} = -x^{-2} = -\frac{1}{x^2}.
$$
Agora, suponha que seja verdade para o caso n-1, isto \'e, 
$$
\frac{d^{n-1}}{d x^{n-1}}\frac{1}{x} = (-1)^{n-1}\frac{1}{x^n}. 
$$
Ent\~ao, temos:
$$
\frac{d^{n}}{d x^{n}}\frac{1}{x} = \frac{d}{dx}\biggl(\frac{d^{n-1}}{d x^{n-1}}\frac{1}{x}\biggr) = 
$$
$$
=\frac{d}{dx}\biggl((-1)^{n-1}\frac{1}{x^n}\biggr) 
= (-1)^{n-1}\frac{d}{dx}\frac{1}{x^n} = (-1)^{n-1}(-1)\frac{1}{x^{n+1}}
= \frac{(-1)^n}{x^{n+1}}.
$$
Portanto, segue que
$$
\frac{d^{n}}{d x^{n}}\frac{1}{x} = \frac{(-1)^n}{x^{n+1}}.
$$
\end{proof*}

\subsubsection{Exerc\'icio 2.}Determine a f\'ormula da derivada de ordem superior de duas das fun\c c\~oes abaixo:
\begin{align*}
	a)f(x) = e^{ax}, a\neq{0} \\
	b)h(x) = \ln(ax), a\geq1 \\
	c)i(x) = \sin(ax)
\end{align*}

\begin{proof*}
Considere uma fun\c c\~ao f(ax) qualquer, com $a\in{X}$, $X\subseteq{\mathbb{R}}$. Quando derivamos
a fun\c c\~ao com rela\c c\~ao a x, obtemos, pela regra da cadeia:
$$
\frac{d}{dx}f(ax) = f^{'}(ax)\frac{d}{dx}ax = f^{'}(ax)a.
$$
Por indu\c c\~ao, supondo que o caso n-1 seja verdade, isto \'e, $\frac{d^{n-1}}{d x^{n-1}}f(ax) = a^{n-1}f^{(n-1)}(ax)$,
temos o caso para um n geral:
$$
\frac{d^n}{d x^n}f(ax) = \frac{d}{dx}\biggl(\frac{d^{n-1}}{d x^{n-1}}f(ax)\biggr) =
$$
$$
= \frac{d}{dx}a^{n-1}f^{(n-1)}(ax) = a\cdot{a^{n-1}}f^{(n)}(ax) = a^{n}f^{(n)}(ax)
$$

Para $a\in{\mathbb{R}\backslash\{0\}}$, defina $f(x) = e^x$. Pelo processo visto acima, segue que:
$$
f^{(n)}(ax) = a^nf^{n}(ax) = a^ne^{ax}
$$

Por outro lado, para $a\in{\mathbb{R}_{\geq1}}$, coloque $f(x) = \ln(ax)$ e note que 
$$
\frac{d^n}{dx^n}\ln(x) = \frac{d^{n-1}}{dx^{n-1}}\frac{1}{x},
$$
tal que 
$$
\frac{d^n}{dx^n}\ln(ax) = a^{n}\frac{1}{a}\frac{1}{x^{n+1}}
$$
\end{proof*}

\subsubsection{Exerc\'icio 3.} Dada $f(x) = ax^2 + bx + c, a\neq{0}$, se $a > 0$, mostre que 
a concavidade da par\'abola est\'a virada para cima e que, caso contr\'ario, para baixo.

\begin{proof*}
Segundo o que foi visto nas aulas de c\'alculo, a concavidade de uma fun\c c\~ao \'e determinada
por sua segunda derivada. Explicitamente falando, isso quer dizer que a concavidade ser\'a para
cima se $f^{''}(x) > 0$ e para baixo caso contr\'ario.

Com base nisso, analisemos a quest\~ao com rela\c c\~ao a f(x) do exerc\'icio. De fato, derivando 
duas vezes com respeito a x por meio da regra do tombo, chegamos em:
$$
\frac{d^2}{dx^2}f(x) = 2a.
$$
Como 2 \'e uma constante, o sinal da segunda derivada \'e puramente determinado pelo valor de a,
tal que, se $a > 0$, ent\~ao $2a > 0$ e, logo, $f^{''}(x) > 0$, o que significa que a concavidade
\'e para cima. Analogamente, caso $a < 0$, segue que $2a < 0$, de modo tal que $f^{''}(x) < 0$ e
a concavidade \'e para baixo. Portanto, segue o que quer\'iamos.
\qedsymbol 
\end{proof*}

\subsubsection{Exerc\'icio 4.} 
Com base nos itens anteriores, estude a fun\c c\~ao 
$$
f(x) = \ln(|x|), \quad x\neq{0}
$$
por completo, incluindo os limites em infinito e infinito negativo.

\begin{proof*}
O primeiro passo \'e separar a fun\c c\~ao com base no m\'odulo de x. Neste prisma,
$$
f(x) = \left\{\begin{array}{ll}
	\ln(x), \quad& x > 0\\
	\ln(-x), \quad& x < 0
\end{array}\right.
$$
Comecemos pelo caso em que $x > 0$, ou seja, quando
$$
f(x) = \ln(x).
$$
Sabe-se que o logaritmo \'e cont'inua nesse intervalo $(0, \infty)$, 
ou seja, dado $x_0 > 0$, temos
$$
\lim_{x\to{x_0}}\ln(x) = \ln(x_0).
$$
Al\'em disso, com rela\c c\~ao \`as derivadas da fun\c c\~ao, podemos determinar que a concavidade do 
gr\'afico da fun\c c\~ao \'e sempre para baixo, pois
$$
\frac{d^2}{dx^2}\ln(x) = -\frac{1}{x^2} < 0, \quad \forall x\neq0
$$
e que, com rela\c c\~ao ao seu crescimento,
$$
\frac{d}{dx}\ln(x) = \frac{1}{x} > 0, \quad \forall x > 0.
$$
Em outras palavras, a fun\c c\~ao \'e estritamente crescente no intervalo de 0 a infinito.

Agora, precisamos estudar o caso em que $x < 0$. Neste caso, a fun\c c\~ao se torna
$$
f(x) = \ln(-x), \quad x < 0.
$$
J\'a que $-x\in(0, \infty)$, a fun\c c\~ao continua sendo con\t'inua no intervalo, tal que
$$
\lim_{x\to{x_0}}\ln(-x) = \ln(-x_0), \quad x_0 < 0.
$$
Por outro lado, diferente do caso anterior, ao derivarmos a nova fun\c c\~ao, chegamos em:
$$
\frac{d}{dx}\ln(-x) = -\frac{1}{x}\cdot\frac{d}{dx}(-x) = \frac{1}{x} < 0, \quad \forall x < 0
$$
Com isto, conclu\'imos que a fun\c c\~ao \'e decrescente para x negativo. Agora, analisemos a 
segunda derivada com respeito a x para que possamos concluir sobre a concavidade:
$$
\frac{d^2}{d x^2}ln(-x) = \frac{d}{dx}\frac{1}{x} = -\frac{1}{x^2} < 0, \quad \forall x < 0. 
$$
Portanto, a fun\c c\~ao tem concavidade para baixo quando x \'e negativo.
\end{proof*}

\subsubsection{Exerc\'icio 5.}
Estenda a constru\c c\~ao do polin\^omio de Taylor para os graus 3, 4 e 5. Al\'em disto, escolha 
duas fun\c c\~oes do Exerc\'icio 2 para aplicar o conceito do polin\^omio de Taylor e esboce seus
gr\'aficos.
\begin{proof*}
	Considere uma fun\c c\~ao qualquer f(x) e seja 
$$
h(x) = f(x_0) + f^{(1)}(x-x_0) + \frac{f^{(2)}}{2}(x-x_0)^2 + m_3(x-x_0)^3.
$$
Note que, quando $x = x_0, h(x_0) = f(x_0)$, de onde obtemos $h^{(3)}(x_0) = f^{(3)}(x_0)$. Por outro
lado, 
$$
\frac{d^3}{dx^3}h(x) = 6m_3
$$
\end{proof*}

\subsection{Taxas de Varia\c c\~ao.}
\subsubsection{Exerc\'icio 1.} Com a aplica\c c\~ao da segunda lei de Newton, diga se 
o movimento da part\'icula dada por 
$$
s(t) = (t^2 + e)ln(t), \quad t>0, \quad s(0) = 0
$$
\'e el\'astico.
\begin{proof*}
Para determinar se um movimento el\'astico, \'e necess\'ario analisar sua segunda derivada. De 
forma expl\'icita, uma fun\c c\~ao de tempo x(t) descrevendo uma part\'icula ser\'a el\'astica se 
$$
\ddot{x}(t) = -kx(t), \quad k\in\mathbb{R},
$$
em que $\ddot{x}(t)$ denota a segunda derivada de x com respeito a t, ou seja, sua acelera\c c\~ao com o tempo. Vamos aplicar esta ideia 
\`a fun\c c\~ao fornecida:
$$
a(t) = \frac{d^2}{dx^2} s(t) = 2\ln (t) -1 - \frac{e}{t^2} + 4 \neq -kx(t)
$$
Portanto, o movimento n\~ao \'e el\'astico.
\end{proof*}