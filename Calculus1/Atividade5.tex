\section{Reta Tangente, Derivada, Derivada de Algumas fun\c c\~oes, Regra da Cadeia}
\subsection[Derivadas de Ordem Superior]{Derivadas de Ordem Superior}
\subsubsection{Exerc\'icio 1.}
\paragraph{} Nosso objetivo \'e obter f\'ormulas gerais para a n-\'esima derivada de $\cos(x) \text{ e } \frac{1}{x}$. Para isso, vamos levar em conta 
o processo realizado com rela\c c\~ao ao seno. A natureza c\'iclica de ambas as fun\c c\~oes sinaliza
a liga\c c\~ao entre as derivadas. Com efeito, 

$$
\begin{array}{ll}
	\frac{d \cos(x)}{d x} = -\sin(x) \\
	\frac{d^2 \cos(x)}{d x^2} = -\frac{d}{dx}\sin(x) = -\cos(x)\\
	\frac{d^3 \cos(x)}{d x^3}\cos(x) = -\frac{d^2}{dx^2}\sin(x) = -\frac{d}{dx}\cos(x) = \sin(x)\\
	\frac{d^4 \cos(x)}{d x^4}\cos(x) = -\frac{d^3}{dx^3}\sin(x) = -\frac{d^2}{dx^2}\cos(x) = \frac{d}{dx}\sin(x) = \cos(x)
\end{array}
$$  
Com base nisso, \'e poss\'ivel concluir que as derivadas de cosseno tamb\'em formam um ciclo e, portanto, podemos escrever uma forma geral como:
$$
\frac{d^n \cos(x)}{d x^n} = \left\{
\begin{array}{ll}
	\cos(x), &\quad x\equiv 0\text{ mod}4 \\
	-\sin(x), &\quad x\equiv 1\text{ mod}4\\
	-\cos(x), &\quad x\equiv 2\text{ mod}4\\
	\sin(x), &\quad x\equiv 3\text{ mod}4 \\
\end{array}
\right.
$$