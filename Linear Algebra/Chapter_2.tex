\section{Transforma\c c\~oes Lineares}
Quando tratamos de conjuntos, qual \'e a forma usual de relacionar dois conjuntos A e B? Neste ponto, j\'a sabemos que a resposta para
isso s\~ao as fun\c c\~oes - ``Mapas'' que relacionam um elemento de A a um elemento de B. Mas, \`as vezes, n\~ao basta apenas relacion\'a-los de forma qualquer, no caso, buscamos propriedades bonitas para as fun\c c\~oes, e disso surgem os conceitos de injetividade,
sobrejetividade e bije\c c\~oes.

Queremos fazer a mesma coisa com espa\c cos vetoriais, mas prestando aten\c c\~ao em um detalhe extra: Precisamos 
preservar a estrutura de espa\c co. Em outras palavras, que condi\c c\~oes queremos na fun\c c\~ao $T:U\rightarrow{V}$ 
entre os espa\c cos U e V que respeite a estrutura deles, e \'e para isso que surge a ideia de Transforma\c c\~ao 
Linear.

\subsection{Defini\c c\~oes}

\begin{def*}
    Dados dois $\mathbb{K}-$espa\c cos U e V, uma fun\c c\~ao $T:U\rightarrow{V}$ \'e uma transforma\c c\~ao (ou aplica\c c\~ao) linear
se
    \begin{itemize}
        \item [TL1)] $T(u_1 + u_2) = T(u_1) + T(u_2), \quad\forall u_1, u_2\in{U}$ \label{TL1}
        \item [TL2)] $T(\lambda u) = \lambda{T(u)}, \quad\forall\lambda\in{\mathbb{K}}, u\in{U}.$ \label{TL2}
    \end{itemize}
\end{def*}

Em particular, $T(\sum_{i=1}^{n}\lambda_iu_i) = \sum_{i=1}^{n}\lambda_iT(u_i).$
\begin{example}[Inj]
    Seja $U\leq{V}$ um subespa\c co de V e considere a aplica\c c\~ao de inclus\~ao $i:U\hookrightarrow V$ dada por 
i(u) = u, $u\in{U}$. Tome $u, u'\in{U}$ e $k\in\mathbb{K}$. Temos:
$$
    i(u + u') = u + u' = i(u) + i(u')
$$
e
$$
    i(ku) = ku = ki(u).
$$
Portanto, i \'e uma aplica\c c\~ao linear. N\~ao apenas isso, observe que, se $u\neq{u'},$ ent\~ao 
$$
    i(u) = u \neq u' = i(u'),
$$
em outras palavras, i \'e injetora! 
\qedsymbol
\end{example}
\begin{def*}
    Sejam U, V dois $\mathbb{K}$-espa\c cos vetoriais e $T: U\rightarrow V$ uma transforma\c c\~ao linear. Diremos
que:
\begin{itemize}
    \item [i)] T \'e monomorfismo se T for injetora; \label{MONO}
    \item [ii)] T \'e epimorfismo se T for sobrejetora; \label{EPIM}
    \item [iii)] T \'e isomorfismo se T for bijetora. \label{ISO}
\end{itemize}
\end{def*}

\begin{proposition*}
    Suponha que U, V, W s\~ao $\mathbb{K}$-espa\c cos vetoriais e que $T: U\rightarrow V$, $S: V\rightarrow W$ s\~ao
transforma\c c\~oes lineares. Ent\~ao, $S\circ{T}:U\rightarrow{W}$ \'e uma transforma\c c\~ao linear.
\end{proposition*}
\begin{proof*}
    Primeiramente, sejam $u_1, u_2, u\in{U}$ vetores e $k\in\mathbb{K}$ um escalar. Vamos provar as propriedades
de transforma\c c\~ao linear de $S\circ{T}$, come\c cando pela soma. Por S e T serem transforma\c c\~oes lineares,
temos:
$$
    S\circ{T}(u_1 + u_2) = S(T(u_1 + u_2)) = S(T(u_1) + T(u_2)) = S(T(u_1)) + S(T(u_2)) = S\circ{T}(u_1) + S\circ{T}(u_2).
$$
    Resta provar a propriedade de "passar o escalar pra fora" e, novamente, utilizando o fato de S e T serem transforma\c c\~oes
lineares, obtemos o seguinte:
$$
    S\circ{T}(ku) = S(T(ku)) = S(kT(u)) = kS(T(u)) = k(S\circ{T})(u).
$$
    Portanto, $S\circ{T}$ \'e uma transforma\c c\~ao linear.
\qedsymbol
\end{proof*}
    Denotamos o espa\c co de todas as transforma\c c\~oes lineares entre dois espa\c cos U e V por 
$Lin_{\mathbb{K}}(U, V)$. Em outras palavras, o conjunto \'e dado por 
$$
    Lin_{\mathbb{K}}(U, V):=\{T:U\rightarrow V: T \text{ \'e transforma\c c\~ao linear.}\}
$$