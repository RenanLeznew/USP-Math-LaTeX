\section{Transforma\c c\~oes Lineares}
Quando tratamos de conjuntos, qual \'e a forma usual de relacionar dois conjuntos A e B? Neste ponto, j\'a sabemos que a resposta para
isso s\~ao as fun\c c\~oes - ``Mapas`` que relacionam um elemento de A a um elemento de B. Mas, \`as vezes, n\~ao basta apenas relacion\'a-los de forma qualquer, no caso, buscamos propriedades bonitas para as fun\c c\~oes, e disso surgem os conceitos de injetividade,
sobrejetividade e bije\c c\~oes.

Queremos fazer a mesma coisa com espa\c cos vetoriais, mas prestando aten\c c\~ao em um detalhe extra: Precisamos 
preservar a estrutura de espa\c co. Em outras palavras, que condi\c c\~oes queremos na fun\c c\~ao $T:U\rightarrow{V}$ 
entre os espa\c cos U e V que respeite a estrutura deles, e \'e para isso que surge a ideia de Transforma\c c\~ao 
Linear.

\subsection{Defini\c c\~oes}

\begin{def*}
    Dados dois $\mathbb{K}-$espa\c cos U e V, uma fun\c c\~ao $T:U\rightarrow{V}$ \'e uma transforma\c c\~ao (ou aplica\c c\~ao) linear
se
    \begin{itemize}
        \item [TL1)] $T(u_1 + u_2) = T(u_1) + T(u_2), \quad\forall u_1, u_2\in{U}$ \label{TL1}
        \item [TL2)] $T(\lambda u) = \lambda{T(u)}, \quad\forall\lambda\in{\mathbb{K}}, u\in{U}.$ \label{TL2}
    \end{itemize}
\end{def*}

Em particular, $T(\sum_{i=1}^{n}\lambda_iu_i) = \sum_{i=1}^{n}\lambda_iT(u_i).$
\begin{example}[Inj]
    Seja $U\leq{V}$ um subespa\c co de V e considere a aplica\c c\~ao de inclus\~ao $i:U\hookrightarrow V$ dada por 
i(u) = u, $u\in{U}$. Tome $u, u'\in{U}$ e $k\in\mathbb{K}$. Temos:
$$
    i(u + u') = u + u' = i(u) + i(u')
$$
e
$$
    i(ku) = ku = ki(u).
$$
Portanto, i \'e uma aplica\c c\~ao linear. N\~ao apenas isso, observe que, se $u\neq{u'},$ ent\~ao 
$$
    i(u) = u \neq u' = i(u'),
$$
em outras palavras, i \'e injetora! 
\qedsymbol
\end{example}
\begin{def*}
    Sejam U, V dois $\mathbb{K}$-espa\c cos vetoriais e $T: U\rightarrow V$ uma transforma\c c\~ao linear. Diremos
que:
\begin{itemize}
    \item [i)] T \'e monomorfismo se T for injetora; \label{MONO}
    \item [ii)] T \'e epimorfismo se T for sobrejetora; \label{EPIM}
    \item [iii)] T \'e isomorfismo se T for bijetora. \label{ISO}
\end{itemize}
\end{def*}

\begin{proposition*}
    Suponha que U, V, W s\~ao $\mathbb{K}$-espa\c cos vetoriais e que $T: U\rightarrow V$, $S: V\rightarrow W$ s\~ao
transforma\c c\~oes lineares. Ent\~ao, $S\circ{T}:U\rightarrow{W}$ \'e uma transforma\c c\~ao linear.
\end{proposition*}
\begin{proof*}
    Primeiramente, sejam $u_1, u_2, u\in{U}$ vetores e $k\in\mathbb{K}$ um escalar. Vamos provar as propriedades
de transforma\c c\~ao linear de $S\circ{T}$, come\c cando pela soma. Por S e T serem transforma\c c\~oes lineares,
temos:
$$
    S\circ{T}(u_1 + u_2) = S(T(u_1 + u_2)) = S(T(u_1) + T(u_2)) = S(T(u_1)) + S(T(u_2)) = S\circ{T}(u_1) + S\circ{T}(u_2).
$$
    Resta provar a propriedade de ``passar o escalar pra fora" e, novamente, utilizando o fato de S e T serem transforma\c c\~oes
lineares, obtemos o seguinte:
$$
    S\circ{T}(ku) = S(T(ku)) = S(kT(u)) = kS(T(u)) = k(S\circ{T})(u).
$$
    Portanto, $S\circ{T}$ \'e uma transforma\c c\~ao linear.
\qedsymbol
\end{proof*}
    Denotamos o espa\c co de todas as transforma\c c\~oes lineares entre dois espa\c cos U e V por 
$Lin_{\mathbb{K}}(U, V)$. Em outras palavras, o conjunto \'e dado por 
$$
    Lin_{\mathbb{K}}(U, V):=\{T:U\rightarrow V: T \text{ \'e transforma\c c\~ao linear.}\}.
$$
Para transformar isso em um espa\c co propriamente dito, vamos definir as opera\c c\~oes nele: 
\begin{itemize}
    \item $(T_1 + T_2)u := T_1u + T_2u$
    \item $(kT)u := kTu$.
\end{itemize}
O elemento neutro aqui \'e a transforma\c c\~ao $0: u\mapsto{0}$ (0(u) = 0 para todo $u\in{U}$), basta verificar
que (T+0)u = Tu para todo u de U. 

Uma coisa importante que acontece \'e com rela\c c\~ao \`as bases de um espa\c co vetorial sob uma transforma\c c\~ao
linear - elas s\~ao preservadas. Especificamente,

\begin{proposition*}
    Sejam U e V $\mathbb{K}-$espa\c cos vetoriais e considere bases $b_1, \cdots, b_n$ de U e $v_1, \cdots, v_n$ de V.
Ent\~ao, existe uma aplica\c c\~ao linear $A:U \rightarrow V$ \'unica e que satisfaz $Ab_j = v_j$ para todo
$j = 1, \cdots, n$.
\end{proposition*}
\begin{proof*}
    Sabe-se que todo elemento $u\in{U}$ satisfaz a igualdade 
    $$
        u = \sum_{i=1}^{n}\lambda_ib_i, \lambda_i\in\mathbb{K}.
    $$
    A partir disso, aplicando-se A, obtemos
    $$
        Au = A\biggl(\sum_{i=1}^{n}\lambda_ib_i = \sum_{i=1}^{n}\lambda_iAb_i,  \lambda_i\in\mathbb{K}\biggr).
    $$
    Com isso, vamos definir a aplica\c c\~ao desejada por $Au:= \sum_{i=1}^{n}\lambda_iv_i.$ Assim, segue que 
    $Ab_i = v_i$ e segue a unicidade, pois temos:
    $$
        0 = Au - Au = \sum_{i=1}^{n}\lambda_iAb_i - \sum_{i=1}^{n}\lambda_iv_i = \sum_{i=1}^{n}\lambda_i(Ab_i - v_i).
    $$
    Resta checarmos que essa A \'e de fato uma transforma\c c\~ao linear. Para isso, tome outro elemento $u' = 
    \sum_{i=1}^{n}\lambda_i'b_i.$ Vamos primeiro calcular A em u e, em seguida, em u', para assim compararmos a soma 
    dos dois elementos. Com efeito,
    $$
        Au = \sum_{i=1}^{n}\lambda_iv_i \quad Au' = \sum_{i=1}^{n}\lambda_i'v_i 
    $$
    de maneira que 
    $$
        A(u+u') = \sum_{i=1}^{n}(\lambda_i + \lambda_i')v_i = \sum_{i}\lambda_iv_i + \sum_{i}^{n}\lambda_i'v_i
        = Au + Au'.
    $$
    Assim, resta provar apenas que A comuta com escalares. Nesta linha de racioc\'inio, dado um escalar $k\in\mathbb{K}$,
    percebemos que
    $$
        A(ku) = \sum_{i=1}^{n}k\lambda_iv_i = k\sum_{i=1}^{n}\lambda_iv_i = kAu.
    $$
    Portanto, A \'e uma aplica\c c\~ao linear \'unica que satisfaz a propriedade desejada.
    \qedsymbol
\end{proof*}
\begin{corol*}
    Se duas aplica\c c\~oes lineares coincidem em uma base linear, elas s\~ao iguais.
\end{corol*}
\begin{proof*}
    Sejam $A_1, A_2$ duas aplica\c c\~oes lineares coincidindo em uma base linear. Em outras palavras, dado
    um elemento da base $b\in{\mathcal{B}}$, vale que $A_1b = A_2b$. Assim, dado um elemento u de V, sabemos que 
    $$
        u = \sum_{i=1}^{n}\lambda_ib_i,
    $$
    em que $\lambda_i\mathbb{K}$ s\~ao escalares e $b_i\in\mathcal{B}.$ Agora, nessas condi\c c\~oes, 
    $$
        A_1u = \sum_{i=1}^{n}\lambda_iA_1b_i = \sum_{i=1}^{n}\lambda_iA_2b_i = A_2u.
    $$
    Portanto, as duas aplica\c c\~oes coincidem em todos os elementos, ou seja, s\~ao iguais.
    \qedsymbol
\end{proof*}
\begin{def*}
    Seja $A:U\rightarrow{V}$ uma aplica\c c\~ao entre $\mathbb{K}$-espa\c cos vetoriais. Chamamos $A^{-1}0 =
    \{u\in{U}: Au = 0\}$ de n\'ucleo (ou kernel) de A.
\end{def*}
    Uma pergunta que surge \'e - j\'a que aplica\c c\~oes lineares preservam as opera\c c\~oes dos espa\c cos vetoriais,
ser\'a que elas preservam outras estruturas, como os subespa\c cos? A resposta, afinal, \'e sim!
\begin{proposition*}
    Seja $A:U\rightarrow{V}$ uma aplica\c c\~ao entre $\mathbb{K}-$espa\c cos vetoriais e sejam $U'\leq{U'}$ e
    $V'\leq{V}$ subespa\c cos. Ent\~ao, a imagem AU' e a imagem inversa $A^{-1}V'$ s\~ao subespa\c cos, 
    $AU'\leq{V} \text{ e } A^{-1}V'\leq{U}.$ Em particular, o n\'ucleo de A \'e um subespa\c co de U.
\end{proposition*}
\begin{proof*}
    Nas hip\'oteses da proposi\c c\~ao, tome dois elementos $v_1, v_2\in{AU'}$ e $k\in\mathbb{K}$. Primeiramente,
    note que 0 = 0.Au = A(u.0) = A0, ou seja, $0\in{AU'}$. Al\'em disso, 
    $$
        v_1 + v_2 = Au_1 + Au_2 = A(u_1 + u_2)
    $$    
    e, como U' \'e um subespa\c co, $u_1 + u_2\in{U'}$. Assim, $v_1 + v_2\in{AU'}$. Al\'em disso, 
    $$
        kv_1 =  kAu_1 = A(ku_1) 
    $$
    novamente, como U' \'e um subespa\c co, $ku_1\in{U'}$ e $kv_1\in{AU'}$, mostrando que \'e um subespa\c co. Para
    $A^{-1}V'$, o processo \'e an\'alogo. Para o n\'ucleo, \'e claro que $0\in{A0}$, mas agora considere
    $w_1, w_2\in{A0}.$ Ent\~ao, 
    $$
        A(w_1 + w_2) = A(w_1) + A(w_2) = 0 + 0 = 0,
    $$
    tal que $w_1 + w_2\in{A0}.$ Ademais, fornecido um escalar $k\in\mathbb{K}$, 
    $$
        A(kw_1) = kAw_1 = k0 = 0,
    $$
    que tamb\'em partence a A0. Portanto, A0 \'e um subespa\c co vetorial.
    \qedsymbol
\end{proof*}
    Uma vantajem de definirmos o n\'ucleo de uma aplica\c c\~ao linear \'e que ele nos fornece um crit\'erio para 
    a aplica\c c\~ao ser injetora:
\begin{proposition*}
    Seja $A:U\rightarrow{V}$ uma aplica\c c\~ao entre $\mathbb{K}-$espa\c cos vetoriais. Ent\~ao, A \'e um 
    monomorfismo se, e s\'o se, o n\'ucleo de A \'e nulo.
\end{proposition*}
\begin{proof*}
    Por um lado, suponha que A \'e um monomorfismo (injetora). Ent\~ao, dado $u\in{U}$, Au = 0 implica que u = 0, 
    isto \'e, $A0 = \{0\},$ provando o que quer\'iamos. 

    Por outro lado, suponha que o n\'ucleo de A \'e nulo. Suponha que $u_1, u_2\in{U}$ s\~ao tais que $Au_1 = Au_2$,
    de forma que
    $$
        0 = Au_1 - Au_2 = A(u_1 - u_2).
    $$
    Em outras palavras, $u_1 - u_2\in{A0}$. Como o n\'ucleo \'e nulo, $u_1 - u_2 = 0$. Portanto, $u_1 = u_2$, e A
    \'e injetora.
\end{proof*}
\begin{proposition*}
    Seja $A:U\rightarrow{V}$ uma aplica\c c\~ao entre $\mathbb{K}$-espa\c cos vetoriais com U finitamente gerado e 
    seja W um subespa\c co complementar ao n\'ucleo $N:= A^{-1}0$ de A, isto \'e, $U = N \oplus W$. Ent\~ao, 
    $A|W: W \rightarrow AU$ \'e um isomorfismo.
\end{proposition*}
\begin{proof*}
    Considere um elemento $u\in{U}$ e seu respectivo $Au\in{AU}$. Tome tamb\'em um representante $v\in{N}$. Nessas
    condi\c c\~oes, observe que u pode ser escrito como u = w + v, para algum w de W, de modo que 
    $$
        Au = A(w + v) = Aw + Av = Aw + 0 = Aw.
    $$
    Em outras palavras, $A|_W: W\rightarrow AU$ \'e sobrejetora, pois todo elemento de AU pode ser escrito como
    um esta aplica\c c\~ao agindo em um elemento de w. Al\'em disso, considere o n\'ucleo $N|_W:=\{w\in{W}: Aw = 0\}.$
    Ent\~ao, se $w\in{N|_W}$, temos Aw = 0, ou seja, $w\in{N}.$ Como $U = N\oplus{W}, N\cap{W} = \{0\}$ e, como 
    $w\in{N\cap{W}}, w = 0.$ Logo, $N|_W = \{0\}$, donde segue que $A|_W$ \'e injetora. Portanto, $A|_W$ \'e um
    isomorfismo.
    \qedsymbol 
\end{proof*}
    Com este resultado em m\~aos, obtivemos uma caracteriza\c c\~ao importante: A imagem da Aplica\c c\~ao agindo
sobre um espa\c co vetorial U independe dos elementos do n\'ucleo, podendo ser vista como o resultado da a\c c\~ao
dela a uma restri\c c\~ao de seu dom\'inio. Outra consequ\^encia imediata de $U = N \oplus{W}$ \'e que 
$\dim_{\mathbb{K}}A^{-1}0 + \dim_{\mathbb{K}} AU = \dim_{\mathbb{K}}$.
\begin{def*}
    A dimens\~ao da imagem de A \'e chamada posto de A, e ser\'a denotada por $rkA:=\dim_{\mathbb{K}}AU.$
\end{def*}

\subsection{Aplica\c c\~oes Lineares e Matrizes}

    O pr\'oximo passo na jornada das aplica\c c\~oes lineares \'e uma liga\c c\~ao essencial entre elas e as matrizes.
Como ser\'a visto, toda matriz pode ser vista como uma aplica\c c\~ao linear e vice-versa. Essa rela\c c\~ao \'e,
de fato, muito mais intr\'inseca do que aparenta ser, n\~ao dependendo nem mesmo da base do espa\c co (por isso, diz
-se que \'e uma tradu\c c\~ao ``natural'' entre os dois conceitos). Vamos ver como construir isso.

    Sejam U e V $\mathbb{K}$-espa\c cos vetoriais, seja $\beta: b_1, b_2, \cdots, b_n$ uma base linear de U e seja 
$\gamma: c_1, c_2, \cdots, c_m$ uma base linear de V. Tomemos uma aplica\c c\~ao linear qualquer $A: U \rightarrow{V}$.
Ent\~ao, para todo $j = 1, 2, \cdots, n$, vale que $Ab_j = \sum\limits_{i=1}^{m}a_{ij}c_i$ para \'unicos $a_{ij}\in\mathbb{K}$.
Essa identifica\c c\~ao \'e respons\'avel por dar para cada elemento $b_j$ da base $\beta$ uma imagem escrita em termos
da base $\gamma$ com coordenada $(a_{1j}, a_{2j}, \cdots, a_{mj})$. Associamos \`a aplica\c c\~ao linear A a 
($m \times n$)-matriz $[A]_{\gamma}^{\beta}:=[a_{ij}]$ (leia da seguinte forma: A representa a transfoma\c c\~ao dos elementos
da base $\beta$ \`a sua respectiva combina\c c\~ao linear em $\gamma$.) com coeficientes em $\mathbb{K}$. Com este processo, 
associamos a cada transforma\c c\~ao linear uma matriz correspondente; agora, faremos a rec\'iproca disso.

    Considere $[a_{ij}]$ uma ($m\times{n}$)-matriz qualquer com coeficientes em $\mathbb{K}$. Vimos que existe uma
\'unica aplica\c c\~ao linear $A:U\rightarrow{V}$ tal que $Ab_j = \sum\limits_{i=1}^{m}a_{ij}c_i$ para todo $j = 1,
2, \cdots, n$. Com isso, quando bases s\~ao dadas em ambos os espa\c cos, temos um dicion\'ario entre as aplica\c c\~oes
de $\Lin{U, V}$ e as ($m\times{n}$)-matrizes escalares. Para todo $j = 1, \cdots, n$, temos 
$Ab_j = \sum\limits_{i=1}^{m}a_{ij}c_{i}$ e $A'b_j = \sum\limits_{i=1}^{m}a'_{ij}c_{i}$, em que 
$a_{ij}, a'_{ij}\in\mathbb{K}.$ Ent\~ao, 
$$
    (A + A')b_j = Ab_j + A'b_j = \sum_{i=1}^{m}(a_{ij} + a'_{ij})c_{i}, \quad \text{\&} \quad
    (kA)b_j = kAb_j = \sum\limits_{i=1}^{m}(ka_{ij})c_{i}.
$$  
Em outras palavras, $[A]^\beta_\gamma = [a_{ij}], [A']_\gamma^\beta, [A + A']^\beta_\gamma = [a_{ij} + a'_{ij}],
\text{ e } [kA]_\gamma^\beta = k[A]_\gamma^\beta.$
    Al\'em disso, escrevendo $[u]_\beta = \begin{bmatrix} k_{1} \\ k_{2} \\ \vdots \\ k_{n} \end{bmatrix} $,
ent\~ao 
$$
    Au = \sum_{j=1}^{n}k_jAb_j = \sum_{j=1}^{n}k_j\biggl(\sum_{i=1}^{m}a_{ij}c_{i}\biggr)
    = \sum_{i=1}^{m}\biggl(\sum_{j=1}^{n}a_{ij}k_j\biggr)c_{i},
$$
de modo que, notando como $\sum\limits_{j=1}^{n}a_{ij}k_j$ \'e o i-\'esimo coeficiente da coluna da matriz
produto $[a_{ij}]\begin{bmatrix} k_{1} \\ k_{2} \\ \vdots \\ k_{n} \end{bmatrix}$, ent\~ao
$[Au]_\gamma = [A]_\gamma^\beta[u]_\beta$ para todos $u\in{U}, A\in{\Lin{U, V}}$. Traduzindo isso para
um portugu\^es leg\'ivel, acabamos de mostrar que a a\c c\~ao de uma transforma\c c\~ao linear num elemento
qualquer $u\in{U}$ \'e descrita pelo produto dos coeficientes dessa transforma\c c\~ao com a matriz dos coeficientes
de u!
    Essa quest\~ao levanta um inqu\'erito importante - o que acontece com a composi\c c\~ao de duas transforma\c c\~oes?
Se considerarmos agora uma base $\delta: d_1, d_2, \cdots, d_l$ uma base linear de W, chamando a outra matriz de 
$[B]_\delta^\gamma = [b_{si}]$, s\~ao dadas pelas igualdades $Ab_j = \sum\limits_{i=1}^{m}a_{ij}c_{i}$ para todo 
$j = 1, \cdots, n$ e $Bc_i = \sum\limits_{s=1}^{l}b_{si}d_{s}$ para todo $i = 1, \cdots, m$, ent\~ao
$$
    (B\circ{A})b_j = B(Ab_j) = B(\sum_{i=1}^{m}a_{ij}c_{i}) = \sum_{i=1}^{m}a_{ij}Bc_i = \sum_{i=1}^{m}a_{ij}
    \biggl(\sum_{s=1}^{l}b_{si}d_s\biggr) = \sum_{s=1}^{l}\biggl(\sum_{i=1}^{m}b_{si}a_{ij}\biggr)d_s.
$$
Assim, o sj-\'esimo coeficiente da matriz $[B\circ{A}]_{\delta}^{\beta}$ \'e igual a $\sum\limits_{i=1}^{m}
b_{si}a_{ij}$. Note que este coeficiente \'e o mesmo que o produto de duas matrizes, sendo elas 
$[B\circ{A}]_{\delta}^{\beta} = [B]_{\delta}^{\beta}\cdot[A]_{\gamma}^{\beta}.$ Conclu\'imos disso que
a composta de aplica\c c\~oes lineares \'e o produto das matrizes de cada uma!

    Com tudo isso, obtemos um exemplo particularmente importante de matriz, a chamada mudan\c ca de base:
\begin{def*}
    Seja V um $\mathbb{K}$-espa\c co vetorial e sejam $\beta, \gamma$ duas bases lineares de V. A matriz
$M_\gamma^{\beta} := [1_V]_{\gamma}^{\beta}$ \'e chamada matriz de mudan\c ca de base $\beta$ para $\gamma$.
\end{def*}

\subsection{Funcionais Lineares e Espa\c co Dual}
Um tipo particular de aplica\c c\~oes lineares s\~ao os funcionais lineares, definidos a seguir
\begin{def*}
    Seja V um $\mathbb{K}$-espa\c co vetorial. Um funcional linear \'e um mapeamento $f: V\rightarrow{\mathbb{K}}$ 
que satisfaz, para $u, v\in{V}, k\in\mathbb{K}$, as seguintes propriedades
    $$
        f[u + v] = f[u] + f[v]
        f[ku] = kf[u].
    $$
\end{def*}
    Essencialmente, a diferen\c ca entre funcionais lineares e aplica\c c\~oes lineares \'e que, quando falando de
aplica\c c\~oes lineares, damos entrada de vetores para receber outros vetores no outro espa\c co, e no caso dos 
funcionais, damos entrada de vetores para a fun\c c\~ao e obtemos escalares na imagem.
    Em particular, o conjunto dos funcionais lineares de um espa\c co vetorial V forma um espa\c co sobre $\mathbb{K}$,
o chamado espa\c co dual $V^*$.
\begin{example}
    Considere V = $\mathbb{K}^n$. Podemos exibir um funcional linear 
    $$
        f: \mathbb{K}^n\rightarrow{\mathbb{K}},
    $$ 
$f(x_1, \cdots, x_n) = \sum\limits_{i=1}^{n}\alpha_ix_i, \alpha_1, \cdots, \alpha_n\in{\mathbb{K}}.$ H\'a um outro
funcional $\pi: \mathbb{K}^{n}\rightarrow\mathbb{K}$ nesse espa\c co dado por $\pi(x_1, \cdots, x_n) = x_i$, a chamada
i-\'esima proje\c c\~ao de $\mathbb{K}^{n}$ em $\mathbb{K}$.
\qedsymbol
\end{example}
    Um exemplo particularmente interessante j\'a \'e um velho amigo conhecido do c\'alculo diferencial e integral
\begin{example}
    Considere o espa\c co das fun\c c\~oes cont\'inuas de um intervalo $[a, b] \text{ em } \mathbb{C}, 
\mathcal{C}([a, b], \mathbb{C}).$ Definimos um funcional linear $I:\mathcal{C}([a, b], \mathbb{C})\rightarrow
\mathbb{C}$ dado por 
    $$
        I(x) = \int_a^b x(t)dt.
    $$
Em geral, note que integrais e derivadas podem ser vistas como aplica\c c\~oes lineares devido \`as suas propriedades
de soma e multiplica\c c\~ao por escalar, enquanto a integral definida \'e um funcional, pois ela devolve um valor
num\'erico como resultado.
\end{example}
    Nosso pr\'oximo passo \'e construir uma base para o espa\c co dual tendo como refer\^encia a base do espa\c co
vetorial.
    Seja $\beta: b_1, b_2, \cdots, b_n$ uma base de V. Para todo $j = 1, 2, \cdots, n$, podemos encontrar um 
\'unico funcional linear $b_j^*: V\rightarrow\mathbb{K}$ dado por $b_j^*b_j = 1, b_j^*b_i = 0$ para $i\neq{j}$.
Com isso, $\beta^*: b_1^*, b_2^*, \cdots, b_n^*$ forma uma base para $V^*$, chamada base dual. A independ\^encia linear 
n\~ao \'e complicada de ser mostrada, ent\~ao vamos mostrar que esse candidato a base gera o espa\c co. Com efeito,
seja $f\in{V^*}$. Temos 
$$
    \biggl(\sum_{j=1}^{n}fb_jb_j^*\biggr) = \sum_{j=1}^{n}(fb_j)(b_j^*b_i) = fb_i (\text{ todas as outras parcelas s\~ao 0}).
$$
Logo, todo funcional pode ser escrito como uma combina\c c\~ao linear de elementos de $\beta^*$ de maneira \'unica,
tornando $\beta^*$ uma base.
    Em suma, constru\'imos manualmente uma base para o espa\c co dual $V^*$ usando a base de V. Esta nomenclatura,
\textit{dual}, se origina do fato de que podemos escrever elementos de um espa\c co com base nos elementos do outro,
num processo formalizado no seguinte Teorema
\begin{theorem*}
    Seja V um $\mathbb{K}$-espa\c co de dimens\~ao finita com base $\beta: b_1, b_2, \cdots, b_n$. Ent\~ao, existe
uma \'unica base $\beta^*: b_1^*, b_2^*, \cdots, b_n^*$ tal que $b_j^*b_i = 0, i\neq{j}, \& b_j^*b_j = 1$. Al\'em
disto, para cada $v\in{V}$, temos 
    $$
        v = \sum_{i=1}^{n}b_i^*(v)b_i
    $$
e, para cada $f\in{V*}$, 
    $$
        f = \sum_{i=1}^{n}f(b_i)b_i^*
    $$
\end{theorem*}
    Emerge uma quest\~ao - \'e poss\'ivel fazer o caminho oposto? Dada uma base de $V^*$, conseguimos construir um
espa\c co dual a este, resultando em uma base $\beta$ de V? A resposta disso leva ao surgimento de um espa\c co
chamado bidual, definido a seguir
\begin{def*}
    Seja V um $\mathbb{K}$-espa\c co vetorial. Chamamos o espa\c co $(V^*)^*$ de espa\c co bidual a V e usamos
a nota\c c\~ao $V^{**}$.
\end{def*}
    Temos uma aplica\c c\~ao natural $I_V: V\rightarrow{V^{**}}$ agindo em $v\in{V}, f\in{V*}$ que \'e dada por 
$(I_Vv)f:= f[v]\in{\mathbb{K}}$, ou seja, entendemos qualquer $v\in{V}$ como um funcional linear sobre $V^*$,
enviando $f\in{V^*}$ para $f[v]\in\mathbb{K}.$ Vamos conferir que o representante $I_Vv$ \'e linear:
    $$
        (I_Vv)(f_1 + f_2) = (f_1 + f_2)[v] = f_1[v] + f_2[v] = (I_Vv)(f_1) + (I_Vv)(f_2), \quad 
        (I_Vv)(kf_1) = (kf_1)[v] = k(I_Vv)(f_1)
    $$
para $f_1, f_2\in{V*}, k\in\mathbb{K}, v\in{V}$. Precisamos agora mostrar que a aplica\c c\~ao $I_V$ \'e linear.
Para isso, tome $v_1, v_2\in{V}, f\in{V^*}, k\in\mathbb{K}$. Assim,
    $$
        (I_V(v_1 + v_2))f = f[v_1 + v_2] = f[v_1] + f[v_2] = (I_Vv_1)f + (I_Vv_2)f, \quad
        (I_V(kv_1))f = f[kv_1] = kf[v_1] = k(I_Vv_1)f.
    $$
Portanto, esta aplica\c c\~ao \'e linear entre os espa\c cos. No entanto, qual \'e o sentido de fazermos essa 
constru\c c\~ao? A resposta \'e que ela prov\'em um isomorfismo entre $V^{**}$ e V, ou seja, o espa\c co bidual
\'e igual ao espa\c co original!! (N\~ao ser\'a provado, acredito que saia do escopo das notas, essa \'ultima 
constru\c c\~ao \'e mais a t\'itulo de curiosidade.)
