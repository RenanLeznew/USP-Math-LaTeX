\documentclass{article}
\usepackage{amsmath}
\usepackage{amsthm}
\usepackage{amssymb}
\usepackage{pgfplots}
\usepackage{amsfonts}
\usepackage[margin=2.5cm]{geometry}
\usepackage{graphicx}
\usepackage[export]{adjustbox}
\usepackage{fancyhdr}
\usepackage[portuguese]{babel}
\usepackage{hyperref}
\usepackage{lastpage}

\pagestyle{fancy}
\fancyhf{}

\hypersetup{
    colorlinks,
    citecolor=black,
    filecolor=black,
    linkcolor=black,
    urlcolor=black
}

\newtheorem*{tm*}{\underline{Teorema:}}
\newtheorem*{sol*}{\underline{Solu\c c\~ao:}}
\newtheorem*{proof*}{\underline{Prova:}}
\renewcommand\qedsymbol{$\blacksquare$}

\rfoot{P\'agina \thepage \hspace{1pt} de \pageref{LastPage}}
\title{LINEAR ALGEBRA}
\author{Renan Wenzel}
\date{\today}
 \begin{document}
  \section*{Exerc\'icio 1} 
  a.) 
  \[ 
    \begin{bmatrix}
      -1 \\
      2
    \end{bmatrix}
    \cdot 
    \begin{bmatrix}
        -1 \\
        2
    \end{bmatrix} 
    = -1.(-1) + 2.2 = 1 + 4 = 5. 
  \] 
  \qedsymbol 

  \[ 
  \begin{bmatrix}
   -1 \\
   2
  \end{bmatrix}
  \cdot
  \begin{bmatrix}
    4 \\
    6
  \end{bmatrix}
  = -1.4 + 2.6 = -4 + 12 = 8. 
  \]
  \qedsymbol

  Por fim, basta dividirmos um resultado pelo outro:
  $$
    \frac{u.v}{u.u} = \frac{8}{5}
  $$

  b.)
  $$
    \begin{bmatrix}
      3 \\
      -1 \\ 
      -5
    \end{bmatrix}
    \cdot
    \begin{bmatrix}
      3 \\
      -1 \\
      -5
    \end{bmatrix}
  = 3.3 + 1.1 + 5.5 = 9 + 1 + 25 = 35
  $$ \qedsymbol

  $$
  \begin{bmatrix}
    -5 
  \end{bmatrix}
  \cdot
  \begin{bmatrix}
    6 \\
    -2 \\
    3
  \end{bmatrix}
  = 3.6 + (-1)(-2) + (-5)(3) = 18 + 2 - 15 = 5
  $$\qedsymbol
 
  Como antes, resta dividir os dois resultados:
  $$
    \frac{x.w}{w.w} = \frac{5}{35} = \frac{1}{7}
  $$ \qedsymbol

  c.) Agora, \'e simplesmente dividir cada valor da matriz por 35 para obtermos o valor em quest\~ao:
  $$
    \frac{w}{w.w} = \begin{bmatrix}
      \frac{3}{35} \\
      \frac{-1}{35} \\
      \frac{-5}{35} \\
    \end{bmatrix} = \begin{bmatrix}
      \frac{3}{35} \\
      \frac{-1}{35} \\
      \frac{-1}{7}
    \end{bmatrix}
  $$

  d.) Finalmente, primeiro calculamos a divis\~ao e, em seguida, dividimos o vetor x pelo valor obtido.
  Segue que
  $$
  \frac{1}{343} = \frac{x.w}{x.x}
  $$
  tal que 
  $$
  \frac{1}{343}.x = \begin{bmatrix}
    \frac{6}{343} \\
    \frac{-2}{343} \\
    \frac{3}{343}
  \end{bmatrix}
  $$ \qedsymbol
  
  \section*{Exerc\'icio 2}

  a.) \'E preciso normalizar o vetor para encontrar o seu unit\'ario, ou seja, dividimos ele por "seu tamanho". Com
  efeito, temos $v.v = -30.-30 + 40.40 = 2500$ e $\sqrt{v.v} = 50$. Assim, segue que o versor de v \'e:
  $$
  \begin{bmatrix}
    \frac{-30}{50} \\
    \frac{40}{50}
  \end{bmatrix} = 
  \begin{bmatrix}
    \frac{-3}{5}\\
    \frac{4}{5} 
  \end{bmatrix}
  $$

  b.) Agora, calculando a norma desse vetor, obt\'em-se $v.v = -6.(-6) + 4.4 + (-3)(-3) = 36 + 16 + 9 = 61$, ou seja,
o vetor unit\'ario \'e dado por:
  $$
  \begin{bmatrix}
    \frac{-6}{\sqrt{61}} \\
    \frac{4}{\sqrt{61}} \\
    \frac{-3}{\sqrt{61}}
  \end{bmatrix}
  $$

  \section*{Exerc\'icio 3}

  Calculamos a ra\'iz da diferen\c ca para obter a dist\^ancia, i.e., 
  $$
    d(x, u) = \sqrt{(u_1 - x_1)^2 + (u_2 - x_2)^2} = \sqrt{121 + 25} = \sqrt{146}
  $$
  Assim, conclu\'imos que a dist\^ancia entre os vetores \'e $\sqrt{146}$.

  \section*{Exerc\'icio Qualquer}
  
  \textbf{Mostre que se $<u, v> = 0$ para todo v, ent\~ao u = 0}

  \textit{Prova:} Suponha, primeiramente, que $\left<u, v\right> = 0$ para todo v no espa\c co. Em particular, como \'e
v\'alido para todo v, tome v = u. Por hip\'otese, $\left< u, u \right> = 0$. Portanto, u=0. \qedsymbol

  \section*{Exerc\'icio 8}

  Suponha que y \'e ortogonal a u e a v, ou seja, 
  $$
  \left< y, u \right> = \left< y, v \right> = 0.
  $$
  Logo, o produto
  $$
  \left< y, u+v \right> = \left< y, u \right> + \left< y, v \right> = 0 + 0 = 0.
  $$
  Implica que y \'e ortogonal a u + v. \qedsymbol

  \section*{Exerc\'icio 10}
c.) Utilizando o item b, temos, para quaisquer ind\'ices i, j:
  $$
  \left< f(e_i), f(e_j) \right> = \left< e_i, e_j \right> = 0, i\neq{j} \quad\& \left< f(e_i), f(e_i) \right>
  = \left< e_i, e_i \right> = 1.
  $$
  Para ver que \'e realmente uma base, considere o conjunto $\mathcal{B} = \{f(e_1), \cdots, f(e_n)\}$ e suponha que 
  $$
  0 = \alpha_1f(e_1) + \cdots + \alpha_nf(e_n).
  $$
  Ent\~ao, segue que para algum i, 
  $$
    0 = \left< f(e_i), \alpha_1f(e_1) + \cdots + \alpha_nf(e_n) \right> = \left< f(e_i), \alpha_1f(e_1) \right> + \cdots
+ \left< f(e_i), \alpha_if(e_i) \right> + \cdots + \left< f(e_i), \alpha_nf(e_n) \right> = 
  $$
  $$
  \left< f(e_i), \alpha_if(e_i) \right> = \alpha_i\left< f(e_i), f(e_i) \right> = \alpha_i.
  $$
  Como i era arbitr\'ario, vale que $\alpha_i = 0, \forall 1\leq{i}\leq{n}$, donde conclu\'imos que $\mathcal{B}$ \'e
uma independ\^encia linear. Portanto, forma uma base. \qedsymbol

d.) Utilzando o item c, obtemos
  $$
    f(v) = x_1f(e_1) + \cdots + x_nf(e_n).
  $$
  tal que 
  $$
    \left< f(v), u_i \right> = \left< f(v), f(e_i) \right> = \left< x_1f(e_1), f(e_i) \right> + \cdots + \left< x_n, f(e_n) \right> =
  $$
  $$
    \left< x_if(e_i), f(e_i) \right> = x_i
  $$

e.) N\~ao entendi :(.

\section*{Exerc\'icio 11}
Suponha que $A:E\rightarrow E$ e $B:E\rightarrow E$ s\~ao transforma\c c\~oes lineares auto-adjuntas, ou seja,
  $\left< Au, v \right> = \left< u, Av \right>$ e $\left< Bu, v \right> = \left< u, Bv \right>$. Al\'em disso,
por hip\'otese, exigimos que $\left< Au, u \right> = \left< Bu, u \right>.$ Com isso, temos:
  $$
  \left< Au, u \right> = \left< Bu, u \right> \Rightarrow \left< Au - Bu, u \right> = 0 \Rightarrow \left< (A-B)u, u \right> = 0.
  $$
  Como A e B s\~ao autoadjuntos, segue da \'ultima igualdade que  A - B = 0. Portanto, A = B. \qedsymbol

  \section*{Exerc\'icio 12}
Seja $A:E\rightarrow E$ auto-adjunto e que $A^kv = 0$, tal que $\left< A^kv, v \right> = 0.$ Sendo A auto-adjunto,
obtemos
  $$
  0 = \left< A^kv, v \right> = \left< A(A^{k-1}v), v \right> \Rightarrow A^{k-1} = 0
  $$
Agora, temos
  $$
  0 = \left< A^{k-1}v, v \right> = \left< A(A^{k-2}v), v \right> \Rightarrow A^{k-2} = 0
  $$
  $$
  \vdots
  $$
  $$
  0 = \left< A^2v, v \right> = \left< A(Av), v \right> \Rightarrow A = 0
  $$
Portanto, Av = 0. \qedsymbol

\section*{Exerc\'icio 2.6}
Come\c cando pela implica\c c\~ao de que se A \'e normal, ent\~ao BC = CB, vamos computar $\left< Au, Av \right>$, $\left< A*u, A*v \right>$
e comparar os resultados. Com efeito,
  $$
  \left< Au, Av \right> = \left< (B+C)u, (B+C)v \right> = \left< Bu, Bv \right> + \left< Bu, Cv \right> + \left< Cu, Bv \right> +
  \left< Cu, Cv \right> =
  $$
  $$
  \left< B^{2}u, v \right> + \left< (BC - CB)u, v \right> - \left< C  ^{2}, v \right> 
  $$
Repetindo o mesmo para o outro produto,
  $$
  \left< A ^{*}u, A ^{*}v \right> = \left< (B ^{*} + C ^{*})u, (B ^{*} + C ^{*})v \right> = \left< B^2u, v \right> + \left< (CB - BC)u, v \right>
  - \left< C ^{2}, v \right>
  $$
Agora, observe que $\left< Au, Av \right> - \left< A ^{*}u, A ^{*}v \right> = 0.$ Logo,
  $$
\left< B^2u, v \right> + \left< (BC - CB)u, v \right> - \left< C ^{2}, v \right> - \left< B^{2}u, v \right> - \left< (BC - CB)u, v \right> + \left< C  ^{2}, v \right> =
  $$
  $$
  2\left< (BC - CB)u, v \right> = 0 \Rightarrow \left< (BC - CB)u, v \right> = 0.
  $$
Como u e v s\~ao arbitr\'arios, a \'unica alternativa \'e que BC - CB = 0. Portanto, BC = CB.

  Por outro lado, caso BC = CB, utilizando as expans\~oes obtidas acima, temos:
  $$
  \left< A ^{*}u, A ^{*}v \right> = \left< B ^{2}u, v \right> + \left< (CB - BC)u, v \right> - \left< C ^{2}u, v \right> =
  \left< B ^{2}u, v \right> + \left< (BC - CB)u, v \right> - \left< C ^{2}u, v \right> = \left< Au, Av \right>
  $$
  Portanto, A \'e normal.
\qedsymbol

\section*{Exerc\'icio 2.7}
Supondo que A \'e ortogonal, segue que $A ^{T} = A ^{-1}$. Al\'em disso, suponha que A \'e triangular superior, ou seja,
  $$
  A = \begin{bmatrix}
    a _{11} & 0 & 0 & \cdots & 0 \\
    a_{21} & a _{22} & 0 & \cdots & 0 \\
    \vdots & & \ddots & & \vdots\\
    a _{n1} & a _{n2} & a _{n3} & \cdots & a _{nn}
  \end{bmatrix}
  $$
Nessas condi\c c\~oes, as contas mostram que a transposta dessa matriz ser\'a a vers\~ao triangular superior dela mesma,
ou seja, $Id = AA^{-1}$ \'e o produto de duas matrizes triangulares, uma superior e outra inferior. Assim, um elemento
da matriz produto ser\'a igual a
  $$
    c_{ij} = \sum_{l=1}^{n} a _{il}a _{lj}.
  $$
  Com isso, note que, se $i\neq{j}$, ent\~ao ou $a_{ij} = 0$, ou $a _{ji} = 0$. Logo, os \'unicos elementos que 
  s\~ao n\~ao-nulos t\^em a forma $c _{ii}$. Por\'em, $c_{ii} = 1$, pois $Id = AA ^{-1}$, de modo que 
  $$
    1 = \sum_{l=1}^{n} a_{il}a_{li} = \sum_{l=1}^{n} a _{ll}a _{ll} = \sum _{l=1}^{n} a _{ll}^{2}
  $$
Em outras palavras, mostramos duas coisas: A primeira \'e que todos os elementos que n\~ao possuem ind\'ices iguais
s\~ao nulos (Caso contr\'ario, a matriz n\~ao seria ortogonal, pois seu produto pela transposta n\~ao daria a identidade).
A segunda \'e que, como os \'unicos elementos restantes est\~ao na diagonal, o produto dela pela transposta \'e formado
por apenas coeficientes de A elevados ao quadrado e todos eles valem 1. Portanto, a matriz A \'e diagonal e seu
quadrado \'e a identidade.

\section*{Ap\^endice: Base ortonormal}
\textbf{Encontre uma base ortonormal a partir de } $\mathcal{B} = \{(1, 1), (0, 1)\}$ \textbf{de} $\mathbb{R} ^{2}$.

\textit{Solu\c c\~ao}: Obtemos o primeiro vetor fazendo 
  $$
  c _{1} = \frac{b _{1}}{||b _{1}||} = \frac{(1, 1)}{\sqrt{\left< b_1, b_1 \right>}} = \biggl(\frac{1}{\sqrt{2}}, \frac{1}{\sqrt{2}}\biggr)
  $$
temos nosso primeiro vetor da base ortonormal. Com isso, obtemos o segundo pela f\'ormula 
  $$
  c_2' = b_2 - \sum _{i=1}^{1} \frac{\left< b _{2}, c _{i} \right>}{\left< c _{i}, c _{i} \right>}c _{i} =
  (0, 1) - \frac{\frac{1}{\sqrt{2}}}{1}\biggl(\frac{1}{\sqrt{2}}, \frac{1}{\sqrt{2}}\biggr) = (-\frac{1}{2}, \frac{1}{2})
  $$
  Agora, normalizamos este vetor, notando que sua norma \'e $\sqrt{\frac{1}{4} + \frac{1}{4}} = \frac{1}{\sqrt{2}}$, tal que
o segundo vetor da nossa base ortonormal ser\'a
  $$
  c _{2} = \frac{c _{2}'}{||c _{2}'||} = \biggl(\frac{-\sqrt{2}}{2}, \frac{\sqrt{2}}{2}\biggr)
  $$
  Portanto, a base obtida \'e $\mathcal{O} = ((\frac{1}{\sqrt{2}}, \frac{1}{\sqrt{2}}), (-\frac{\sqrt{2}}{\sqrt{2}}, \frac{\sqrt{2}}{\sqrt{2}})$.Resta conferir se \'e de fato ortonormal. Com efeito,
  $$
  \left< c _{1}, c _{2} \right> = -1 + 1 = 0,
  $$
  $$
  \left< c _{1}, c _{1} \right> = \sqrt{\frac{1}{2} + \frac{1}{2}} = 1,
  $$
  $$
  \left< c _{2}, c _{2} \right> = \sqrt{\frac{2}{4} + \frac{2}{4}} = 1.
  $$
  \qedsymbol
\end{document}


