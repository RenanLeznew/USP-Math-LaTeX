\section{Espa\c cos Vetoriais}
Para falar sobre espa\c cos vetoriais, \'e preciso antes falar sobre corpos, pois espa\c cos
vetoriais s\~ao definidos com base nessas estruturas, que ser\~ao ``onde os vetores ir\~ao morar".
Para exemplificar, um espa\c co vetorial sobre $\mathbb{R}$ ter\'a como vetores n\'umeros reais,
e um espa\c co vetorial sobre $\mathbb{C}$ ter\'a como vetores n\'umeros complexos. O que 
faremos a seguir \'e definir mais precisamente cada um destes conceitos, desde corpos a vetores.
\subsection{Corpos e Espa\c cos Vetoriais}
\begin{def*}
    \text{Diremos que um conjunto} $\mathbb{K}$ \text{\'e  um corpo se ele satisfaz as seguintes
    propriedades: }
    \begin{itemize}
        \item [A1)] x + y = y + x, $\forall x, y\in\mathbb{K}$
        \item [A2)] (x + y) + z = x + (y + z), $\forall x, y, z\in\mathbb{K}$
        \item [A3)] \text{Existe um elemento 0 tal que } x + 0 = 0 + x = x, $\forall x \in\mathbb{K}$
        \item [A4)] \text{Para todo x, existe um elemento -x tal que } x + (-x) = -x + x = 0, $\forall x \in\mathbb{K}$. 
        \item [M1)] $x\cdot{}y = y\cdot{}x, \forall x, y\in\mathbb{K}$
        \item [M2)] ($x\cdot{y})\cdot{z} = x(y\cdot{z}), \forall x, y, z \in\mathbb{K}$
        \item [M3)] \text{Existe um elemento 1 tal que } $x\cdot{1} = 1\cdot{x} = x, \forall x \in\mathbb{K}$
        \item [M4)] \text{Para todo x, existe um elemento } $x^{-1}$ \text{ tal que } $x^{-1}\cdot{}x = x\cdot{x^{-1}} = 1, \forall x \in\mathbb{K}$
        \item [D1)] $x\cdot{}(y + z) = x\cdot{y} + x\cdot{z}$
        \item [D2)] $(x + y)\cdot{}z = x\cdot{z} + y\cdot{z}$
    \end{itemize}
\end{def*}
Faremos algumas conven\c c\~oes com rela\c c\~ao a essa defini\c c\~ao. Ao inv\'es de escrevermos x + (-x), $x\cdot{y}, x^{-1}$, faremos:
$$
    x - x := x + (-x), \quad x\cdot{y} := xy, \quad x^{-1} := \frac{1}{x}.
$$
Exemplos b\'asicos de corpos s\~ao o corpo dos n\'umeros reais, dos n\'umeros racoinais e dos n\'umeros complexos. 
Um bom exerc\'icio para se acostumar com esses conceitos \'e provar que eles s\~ao realmente corpos e mostrar que
os n\'umeros inteiros n\~ao formam um corpo. 

Agora que temos uma ideia sobre corpos, podemos definir um espa\c co vetorial:
\begin{def*}
    Dizemos que um conjunto V \'e um espa\c co vetorial se seus elementos, chamados vetores, satisfazem os axiomas abaixo:
    \begin{itemize}
        \item [V1)] u + v = v + u, $v, u\in{V}$; \label{V1}
        \item [V2)] u + (v + w) = (u + v) + w, $v, u, w\in{V}$; \label{V2}
        \item [V3)] Existe $0\in{V}$ tal que v + 0 = v, $v\in{V}$; \label{V3}
        \item [V4)] Para todo $v\in{V}, $ existe $-v\in{V}$ tal que v - v = 0; \label{V4}
        \item [E1)] Dado $v\in{V}, 1v = v$; \label{E1}
        \item [E2)] Dados $\alpha, \beta\in\mathbb{K}, (\alpha\beta)v = \alpha(\beta{v}), v\in{V}$; \label{E2}
        \item [DV1)] $\alpha(u + v) = \alpha{u} + \alpha{v}$; \label{DV1}
        \item [DV2)] $(\alpha + \beta)v = \alpha{v} + \beta{v}$. \label{DV2}
    \end{itemize}
\end{def*}

Vamos ver um exemplo de espa\c co vetorial:

\begin{example}
\label{Vspace}
    Considere o corpo $\mathbb{R}^2$. Vamos mostrar que $\mathbb{R}$ \'e um espa\c co vetorial sobre $\mathbb{R}^2$.
    Esse tipo de demonstra\c c\~ao \'e, em geral, sempre igual. Tome dois elementos $v, u\in\mathbb{R}$ e dois escalares
    $\mathbf{k_1}, \mathbf{k_2}\in\mathbb{R}^2$. 
    
    Segue das propriedades dos n\'umeros reais que existe um elemento neutro da adi\c c\~ao, o 0 usual, um inverso
    aditivo (dado $x\in\mathbb{R}, -x\in\mathbb{R}$) e as propriedades usuais de soma, isto \'e, comutatividade e
    associatividade. Agora, coloque $\mathbf{k_1} = (\alpha_1, \beta_1), \mathbf{k_2 = (\alpha_2, \beta_2)}$ e 
    1 = (1, 0).
    Com isto, temos:
    $$
        1.x = (1, 0).x = (1.x, 0.x) = (x, 0) = x, 
    $$
    $$
        (\mathbf{k_1}\mathbf{k_2})x = ((\alpha_1\alpha_2, \beta_1\beta_2))(x, 0) = ((\alpha_1\alpha_2){x}, \beta_1\beta_2{0}) =
        = (\alpha_1(\alpha_2{x}), 0) = (\alpha_1, \beta_1)((\alpha_2, \beta_2)(x, 0)) = k_1(k_2x).
    $$
    Resta mostrar a distributiva. Note que 
    $$
        (\mathbf{k_1} + \mathbf{k_2})x = (\alpha_1 + \alpha_2, \beta_1 + \beta_2)x = (\alpha_1x + \alpha_2x, 0) = (\alpha_1, \beta_1)x + (\alpha_2, \beta_2)x.
    $$
    e
    $$
        (x + y)\mathbf{k_1} = ((x, 0) + (y, 0))(\alpha_1, \beta_1) = (x + y, 0)(\alpha_1, \beta_1) = (x\alpha_1 + y\alpha_1, 0) = x\mathbf{k_1} + y\mathbf{k_1}. 
    $$
\qedsymbol
\end{example}
Agora que temos uma no\c c\~ao b\'asica de espa\c cos vetoriais, aprofundaremos na teoria.

\subsection{Bases}
Sabe como todo n\'umero real pode ser escrito como 1.x? Vamos buscar uma forma an\'aloga para um espa\c co vetorial
qualquer. Para isso, ser\'a introduzido o conceito da combina\c c\~ao linear e independ\^encia linear de vetores quaisquer.

\begin{def*}
    Dado um  espa\c co vetorial V sobre $\mathbb{K}$, diremos que um conjunto de vetores $\mathcal{B}: v_1, \cdots{}, v_n$ gera V
    se qualquer elemento $v\in{V}$ puder ser escrito como: 
    $$
        v = \sum_{i=1}^{n}\alpha_iv_i,
    $$
    com $\alpha_i\in\mathbb{K}$ para cada i. Dado um conjunto $\mathcal{B}$, dizemos que ele \'e um conjunto 
    gerador de V se todo elemento de V pode ser escrito como combina\c c\~ao linear finita de elementos de 
    $\mathcal{B}$.
\end{def*}
H\'a, por\'em, um problema com isso. Vamos ilustrar isso no exemplo a seguir: 
\begin{example}[NB]
\label{Not basis}
    Considerando $\mathbb{C}^2$ como um espa\c co vetorial sobre $\mathbb{C}$, o conjunto 
    $\mathcal{B}: (1, 0), (0, i), (i, 0), (0, 1)$ gera $\mathbb{C}^2$.
    
    De fato, visto que um elemento $(a, b) = (x + iy, z + iw)$, em que $a, b\in\mathbb{C}, x, y, z, w\in\mathbb{R}$, pode ser escrito como:
    $$
        (a, b) = x(1, 0) + y(i, 0) + z(0, 1) + w(0, i) 
    $$
    No entanto, observe que se $(a, b) = (0, 0)$, ent\~ao 
    $$
        (0, 0) = 1(1, 0) + i(i, 0) + 0(0, 1) + 0(0, i).
    $$
    Assim, o elemento $(0, 0)$ pode ser escrito de duas formas diferentes! Sendo a outra:
    $$
        (0, 0) = 0(1, 0) + 0(i, 0) + 0(0, 1) + 0(0, i).
    $$
\end{example}
Explicitamente, queremos representar \textbf{de maneira \'unica} cada elemento de V. \'E para isso que surge
a no\c c\~ao de independ\^encia linear e de base, isto \'e, 
\begin{def*}
    Dado um  espa\c co vetorial V sobre $\mathbb{K}$, diremos que um conjunto de vetores $\mathcal{B}: v_1, \cdots{}, v_n$ gera V
    se: 
    $$
        \sum_{i=1}^{n}\alpha_iv_i = 0 \Leftrightarrow \alpha_1 = \alpha_2 = \cdots = \alpha_n = 0,
    $$
    com $\alpha_i\in\mathbb{K}$ para cada i. Caso isso n\~ao ocorra, ou seja, existe ao menos um $\alpha_i\in\mathbb{K}\neq{0}$ para cada i.
    tal que 0 pode ser escrito como combina\c c\~ao linear que inclua $\alpha_iv_i$, diremos que o conjunto \'e
    linearmente dependente.
\end{def*} 

\begin{def*}
    Dado um  espa\c co vetorial V sobre $\mathbb{K}$, diremos que um conjunto de vetores $\mathcal{B}: b_1, b_2, \cdots{}, b_n$ \'e uma base de V
    $\mathcal{B}$ gera V e \'e linearmente independente.
\end{def*}

Conclui-se que o conjunto apresentado no exemplo \ref{Not basis} n\~ao \'e uma base! (por que?). Com base nisso, 
voc\^e consegue encontrar uma base para o espa\c co vetorial do exemplo? E para o exemplo \ref{Vspace}?

A seguir, seguem algumas propriedades dos conceitos vistos acima que s\~ao \'uteis para treinar demonstra\c c\~ao,
ent\~ao recomendo que voc\^es tentem:

\begin{proposition*}
    \begin{itemize}
        \item[a)] Seja $\mathcal{B}$ um conjunto gerador de V. Ent\~ao, todo subconjunto W de V que cont\'em $\mathcal{B}$ \'e um 
        conjunto gerador de V.
        \item[b)] Todo conjunto contendo o vetor nulo \'e linearmente dependente.
        \item[c)] Todo subconjunto de um conjunto linearmente independente \'e linearmente independente. 
    \end{itemize}
\end{proposition*}

Seguem abaixo alguns resultados um pouco mais complicados para serem deixados como exerc\'icio, junto de uma nova defini\c c\~ao.

\begin{def*}
    Dizemos que um espa\c co vetorial V sobre $\mathbb{K}$ \'e finitamente gerado se possuir um conjunto gerador finito.
\end{def*}

\begin{proposition*}
    Seja V um $\mathbb{K}$-espa\c co vetorial finitamente gerado n\~ao nulo e suponha que $\{v_1, \cdots, v_m\}$ seja
    um conjunto gerador de V. Ent\~ao todo conjunto linearmente independente de vetores em V tem no m\'aximo m elementos.
\end{proposition*}
\begin{proof*}
    Vamos mostrar algo equivalente a isso, ou seja, se um conunto tem mais que m elementos, ele \'e linearmente dependente.
Com efeito, considere $\mathcal{A} = \{v_1, \cdots, v_n\} \subset{V}, n > m.$ Como $\{v_1, \cdots, v_m\}$ gera V, associamos
para cada \'indice $u_j, j = 1, \cdots, n$ uma combina\c c\~ao linear da forma:
$$
    u_{j} = \alpha_{1j}v_1 + \cdots + \alpha_{mj}v_m = \sum_{i=1}^{m} \alpha_{ij}v_i.
$$
Agora, vamos supor que o conjunto dos $u_{j}'s$ gera 0, ou seja,
$$
    0 = \sum_{j=1}^{n} \lambda_{j}u_{j}.
$$
Reescrevendo cada $u_{j}$, temos
$$
  0 = \sum_{j=1}^{n} \lambda_{j}u_{j} = \sum_{j=1}^{n}\lambda_{j}\underbrace{\sum_{i=1}^{m}{\alpha_{ij}v_i}}_{\mathclap{\substack{\text{essa soma vem da}, \\ \text{expans\~ao dos } u_j, \\ \text{sei que pode parecer intimidador,} \\ \text{mas juro que n\~ao morde!}.}}}
$$
$$
    = \underbrace{\sum_{j=1}^{n}\sum_{i=1}^{m}}_{\mathclap{\substack{\text{reorganizei a soma} \\ \text{pra ficar mais limpo visualmente}}}} \lambda_{j}\alpha_{ij}v_{i} \overbrace{=}^{\text{inverte a ordem}} \sum_{i=1}^{m}\biggl(\sum_{j=1}^{n}\lambda_{j}\alpha_{ij}\biggr)v_{i}
$$

Agora que temos essa soma, pense em cada termo de $\sum_{j=1}^{n}\lambda_{j}\alpha_{ij}$ como um escalar (a soma de escalares \'e um escalar) e suponha que 
$\sum_{j=1}^{n}\lambda_{j}\alpha_{ij} = 0$ para analisarmos o tipo de depend\^encia desse termo. Para cada $i = 1, 2, \cdots, m$, temos 
uma equa\c c\~ao do tipo 
$$
\lambda_{1}\alpha_{i1} + \lambda_{2}\alpha_{i2} + \cdots + \lambda_{n}\alpha_{in} = 0,
$$
o que sugere um sistema linear para cada valor de i na inc\'ognita $\lambda_{j}$:
$$
\left\{\begin{array}{ll}
    \lambda_{1}\alpha_{11} + \lambda_{2}\alpha_{12} + \cdots + \lambda_{n}\alpha_{1n} = 0, \\
    \lambda_{1}\alpha_{21} + \lambda_{2}\alpha_{22} + \cdots + \lambda_{n}\alpha_{2n} = 0, \\
    \vdots \\
    \lambda_{1}\alpha_{m1} + \lambda_{2}\alpha_{m2} + \cdots + \lambda_{n}\alpha_{mn} = 0,
\end{array}\right.
$$
Agora, como n \'e maior que m, h\'a mais inc\'ognitas que equa\c c\~oes! Em outras palavras, o sistema possui ao menos
uma solu\c c\~ao n\~ao nula, ou seja, existem $\gamma_1, \cdots, \gamma_n\in\mathbb{K}$ tais que nem todos s\~ao nulos e
$$
    \sum_{j=1}^{n} \gamma_{i}\alpha_{ij} = 0.
$$
Disto segue que existem $\gamma_{i}'s$ nem todos nulos tais que
$$
    0 = \sum_{i=1}^{m}\biggl(\sum_{j=1}^{n}\gamma_{j}\alpha_{ij}\biggr)v_{i} = \sum_{j=1}^{n}\sum_{i=1}^{m}\lambda_{j}\alpha_{ij}v_{i} =
$$
$$
    = \sum_{j=1}^{n}\sum_{i=1}^{m}\lambda_{j}u_{j}.
$$
Portanto, $\{u_1, \cdots, u_n\}$ \'e linearmente dependente, consequentemente $\mathcal{A}$ \'e tamb\'em.
\end{proof*}
\begin{corol*}
    Seja V um $\mathbb{K}$-espa\c co vetorial finitamente gerado n\~ao nulo. Ent\~ao, duas bases quaisquer de V t\^em
    o mesmo n\'umero de elementos.
\end{corol*}

Em vista disso, podemos definir o tamanho de um espa\c co vetorial unicamente de acordo com o n\'umero de elementos
da base, o que chamamos de dimens\~ao:

\begin{def*}
    Seja V um $\mathbb{K}$-espa\c co vetorial. Se V admite uma base finita, chamamos de dimens\~ao de V o n\'umero
    de elementos da base. Em outras palavras, se a base tem n elementos, ent\~ao 
    $$
        \dim_{\mathbb{K}}V = n.
    $$
    Caso n\~ao exista base finita, dizemos que V possui dimens\~ao infinita.
\end{def*}

Alguns resultados que eu n\~ao vou provar por um tempo seguem:
\begin{proposition*}
    Seja V um $\mathbb{K}$-espa\c co vetorial. Se V possui dimens\~ao $dim_{\mathbb{K}}V = n \geq 1$ e seja 
$\mathcal{B}$ um subconjunto de V com n elementos. As seguintes afirma\c c\~oes s\~ao equivalentes:
\begin{itemize}
    \item [a)] $\mathcal{B}$ \'e uma base.
    \item [b)] $\mathcal{B}$ \'e linearmente independente.
    \item [c)] $\mathcal{B}$ \'e um conjunto gerador.
\end{itemize}
\end{proposition*}
\begin{proposition*}
\label{Base_completion}
    Seja V um $\mathbb{K}$-espa\c co vetorial e considere $\mathcal{B} = \{v_1, \cdots, v_m\}$ um conjunto linearmente
independente em V. Se existir $v\in{V}$ que n\~ao seja combina\c c\~ao linear dos elementos de $\mathcal{B}$, ent\~ao
$\{v_1, \cdots, v_m, v\}$ \'e  linearmente independente. (Isso permite o m\'etodo de completamento de base!).
\end{proposition*}
\begin{theorem*}
    Todo espa\c co vetorial finitamente gerado n\~ao nulo possui uma base. 
\end{theorem*}
\begin{proposition*}
    Seja V um $\mathbb{K}$-espa\c co vetorial de dimens\~ao $n\geq{1}$ e seja $\mathcal{B}\subseteq{V}$. As seguintes
afirma\c c\~oes s\~ao equivalentes:
\begin{enumerate}
    \item [a)] $\mathcal{B}$ \'e uma base de V.
    \item [b)] Cada elemento de V se escreve unicamente como combina\c c\~ao linear de elementos de $\mathcal{B}.$
\end{enumerate}
\end{proposition*}

Como conseque\^encia dessa \'ultima proposi\c c\~ao, podemos introduzir a noc\c c\~ao de coordenada. Seja V
um $\mathbb{K}$-espa\c co vetorial de dimens\~ao $n \geq{1}$ e seja $\mathcal{B} = \{v_1, \cdots, v_n\}$ uma base.
Vamos fixar a ordem dos elementos de $\mathcal{B}$, formando o que chamamos de base ordenada de V, e agora aplicamos
a proposi\c c\~ao acima. 

Dado $v\in{V}$, existem \'unicos $\alpha_1, \cdots, \alpha_n\in\mathbb{K}$ tais que $v = \sum_{i=1}^{n}\alpha_iv_i$. 
Por conta dessa unicidade, \'e comum determinarmos v puramente pelos coeficientes, e escrevemos 
$$
    [v] = (\alpha_1, \alpha_2, \cdots, \alpha_n)_{\mathcal{B}}
$$
e dizemos que $\alpha_1, \alpha_2, \cdots, \alpha_n$ s\~ao as coordenadas de v com rela\c c\~ao \`a base $\mathcal{B}.$

\subsection{Subespa\c cos Vetoriais e Somas de Espa\c cos}
Tal como um conjunto possui um subconjunto, sob quais condi\c c\~oes um espa\c co vetorial possui
outro espa\c co vetorial ``dentro de si"? Esta no\c c\~ao, conhecida como subespa\c co vetorial,
inclui conceitos como $\mathbb{R}^2$ ser basicamente ``duas retas $\mathbb{R}$", sobre todo n\'umero
racional ser tamb\'em um n\'umero real e sobre todo n\'umero real ser um n\'umero complexo.

Com base nessa motiva\c c\~ao, damos continuidade para formalizar esses conceitos.

\begin{def*}
    Seja V um espa\c co vetorial sobre $\mathbb{K}$ e W um subconjunto \textbf{do conjunto V}. 
    Diremos que W \'e um subespa\c co vetorial de V se:
    \begin{itemize}
        \item [SS1)] $0\in{W}$ \label{SS1}
        \item [SS2)] Se $w_1, w_2\in{W}$, ent\~ao $w_1 + w_2 \in{W}$ \label{SS2}
        \item [SS3)] Dados $w_1 \in{W}, k\in\mathbb{K}$, ent\~ao $kw_1 \in{W}.$ \label{SS3}
    \end{itemize}
\end{def*}

Vamos checar que W com essas condi\c c\~oes \'e realmente um espa\c co vetorial:

Sejam $w_1, w_2, w_3\in{W}$ e $k_1, k_2\in\mathbb{K}$. Como $0\in{W}$, segue da propriedade \ref{SS2} 
que $w_1 + 0 = w_1 \in{W}$, satisfazendo a propriedade \ref{V4} de espa\c co vetorial. Al\'em disso, sendo
em particular $w_1, w_2, w_3\in{V}$ (pois $W\subseteq{V}$), eles satisfazem \ref{V1}, \ref{V2} e \ref{V3}.
Juntando isso com a propriedade \ref{SS2}, segue que $w_1 + w_2 = w_2 + w_1, w_1 + (w_2 + w_3) = (w_1 + w_2) + w_3$ e
existe $-w_1\in{W}$ tal que $w_1 - w_1 = 0$ e todas essas opera\c c\~oes est\~ao em W por conta de \ref{SS2}.
\paragraph*{} Agora, vamos verificar as opera\c c\~oes com escalares. Como 1 \'e um escalar, segue
da propriedade \ref{E1} que $1w_1 = w_1$ e, por conta de \ref{SS3}, $1.w_1\in{W}.$ Al\'em disso,
de \ref{E2} e \ref{SS3}. \'e v\'alido em W que $(k_1k_2)w_1 = k_1(k_2w_1)\in{W}.$ Logo, W
satisfaz as opera\c c\~oes dos escalares.
\paragraph*{} Por fim, a checagem da distributiva \'e como feito acima, usando as opera\c c\~oes 
de V como espa\c co vetorial e \ref{SS2}, \ref{SS3}.

Portanto, W \'e um espa\c co vetorial dentro de V! Um exerc\'icio bom para treinar a ideia \'e
tentar provar que $\mathbb{Q}$ \'e subespa\c co de $\mathbb{R}$.

Se tivermos dois espa\c cos vetoriais $V_1, V_2$, uma constru\c c\~ao importante \'e a soma 
deles como espa\c cos. Assim, obtemos um espa\c co $V_1 + V_2$ sobre $\mathbb{K}$ cujos elementos
s\~ao da forma $v_1 + v_2, v_1\in{V_1}, v_2\in{V_2}$. (Fica de exerc\'icio conferir provar
que \'e de fato um espa\c co vetorial). Um tipo espec\'ifico de soma de espa\c cos importante
\'e aquele em que cada elemento pode ser escrito de maneira \'unica como soma de elementos de cada
um, formando o que se chama Soma Direta:
\begin{def*}
    Dados dois espa\c cos vetoriais $V_1, V_2$ tais que $V_1\cap{V_2} = \{0\}$, escrevemos 
    $$  
        V_1 \oplus{V_2}
    $$
    para denotar a soma desses espa\c cos, e chamamos ela de Soma Direta.
\end{def*}

\begin{example}
    Considere a reta real $\mathbb{R}$ e o conjunto dos n\'umeros imagin\'arios $\mathbb{I}.$ Note
    que o \'unico n\'umero que \'e tanto um real quanto um imagin\'ario \'e o 0, pois $0.i = 0$.
    Em outras palavras, $\mathbb{R}\cap\mathbb{I} = \{0\}.$ 
    
    Agora, \'e poss\'ivel definir um espa\c co vetorial que seja a soma direta de ambos e que cada elemento ter\'a a forma
    $a + b, a\in\mathbb{R}, b\in\mathbb{I}$. No entanto, como $b\in{\mathbb{I}}, b = ix, x\in\mathbb{R}
    , i = \sqrt{-1}.$ Logo, temos um elemento da forma $a + ix \in\mathbb{R}\oplus\mathbb{I}.$
    Isso te lembra alguma coisa? Caso n\~ao, note que esse n\'umer obtido \'e um n\'umero complexo,
    ou seja, a soma direta de $\mathbb{R}$ com $\mathbb{I}$ \'e:
    $$
        \mathbb{R}\oplus\mathbb{I} := \mathbb{C}.
    $$
    
    \textbf{Observa\c c\~ao:} o conjunto dos imagin\'arios $\mathbb{I}$ \'e o dos n\'umeros 
    que possuem a forma $\mathbb{I}=\{ix: x\in\mathbb{R}, i = \sqrt{-1}\}.$
\end{example}

\begin{proposition*}
    Sejam V um espa\c co vetorial e $W_1, W_2$ dois subespa\c cos vetoriais de V, ambos de dimens\~ao finita. Ent\~ao,
$$
    \dim_{\mathbb{K}} (W_1 + W_2) = \dim_{\mathbb{K}} W_1 + \dim_{\mathbb{K}} W_2 + \dim_{\mathbb{K}}(W_1\cap W_2).
$$
\end{proposition*}
\begin{proof*}
    Essa demonstra\c c\~ao serve como um grande exemplo de manipula\c c\~ao de bases, isto \'e, completamento e remo\c c\~ao
de elementos da base. Vamos supor, a priori, que a intersec\c c\~ao dos dois espa\c cos \'e n\~ao-trivial, ou seja, 
$W_1\cap{W_2} \neq \{0\}$ e seja $\mathcal{B} = \{w_1, \cdots, w_n\}$ uma base de $W_1\cap{W_2}$. Como $W_1\cap{W_2}$
\'e subespa\c co tanto de $W_1$ quanto de $W_2$, \'e poss\'ivel estender a base at\'e ambos.

Considere $\mathcal{B'} = \{w_1, \dots, w_n, v_1, \cdots, v_r\}, \mathcal{B``} = \{w_1, \cdots, w_n, u_1, \cdots, u_s\}$ 
bases de $W_1$ e de $W_2$ respectivamente. A constru\c c\~ao dessas duas bases tem como base ``aumentar o tamanho de 
$\mathcal{B}$ adicionando novos elementos". O objetivo disso \'e mostrar que o conjunto $\mathcal{C} = \{w_1, \cdots, 
w_n, v_1, \cdots, v_r, u_1, \cdots, u_s\}$ \'e uma base de $W_1 + W_2.$

H\'a muitos \'indices aqui, ent\~ao recapitulando: come\c camos com uma base com n elementos da interse\c c\~ao $W_1\cap{W_2}$,
representando a dimens\~ao de $W_1\cap{W_2}$, $\dim_{\mathbb{K}}(W_1\cap{W_2}) = n.$ A partir disso, adicionamos novos r
elementos, representando a dimens\~ao de $W_1$, $\quad\dim_{\mathbb{K}}(w_1) = n + r$ e outros s elementos, representado
a dimens\~ao de $W_2$, $\dim_{\mathbb{K}}(W_2) = n + s.$ O resultado que queremos provar se resume a encontrar uma base de
$W_1 + W_2$ que possua 
$$
    \dim_{\mathbb{K}}(W_1) + \dim_{\mathbb{K}}(W_2) - \dim_{\mathbb{K}}(W_1\cap{W_2}) = n + r + n + s - n = n + r + s
$$
elementos, e por isso o nosso objetivo \'e aquele!

Dando continuidade, come\c camos mostrando que $\mathcal{C}$ gera $W_1 + W_2$. Seja $v\in{W_1+W_2}$, ou seja, 
$v = w_1 + w_2, w_1\in{W_1}, w_2\in{W_2}$. Usando as bases de cada respectivo espa\c co e da interse\c c\~ao, temos 
$$
    w_1 = \sum_{i=1}^{n}\lambda_{i}w_{i} + \sum_{j=1}^{r}\gamma_{j}v_{j} 
$$
e
$$
    w_2 = \sum_{i=1}^{n}\alpha_{i}w_{i} + \sum_{l=1}^{s}\beta_{l}u_{l}.
$$
Assim, v pode ser escrito como 
$$
    v = w_1 + w_2 = \sum_{i=1}^{n}\lambda_{i}w_{i} + \sum_{j=1}^{r}\gamma_{j}v_{j} + 
    \sum_{i=1}^{n}\alpha_{i}w_{i} + \sum_{l=1}^{s}\beta_{l}u_{l} =
$$
$$
    = \biggl(\sum_{i=1}^{n}\lambda_{i}w_{i} + \sum_{i=1}^{n}\alpha_{i}w_{i}\biggr) + \sum_{j=1}^{r}\gamma_{j}v_{j} + \sum_{l=1}^{s}\beta_{l}u_{l} 
$$
$$
    = \sum_{i=1}^{n}(\lambda_{i} + \alpha_{i})w_{i} + \sum_{j=1}^{r}\gamma_{j}v_{j} + \sum_{l=1}^{s}\beta_{l}u_{l}.
$$
Portanto, como a soma $\lambda_{i} + \alpha_{i}$ \'e tamb\'em um escalar, v \'e combina\c c\~ao linear de elementos de $\mathcal{C}$ e,
por termos escolhido v sendo qualquer, $\mathcal{C}$ gera $W_1 + W_2.$ 

O pr\'oximo e \'ultimo passo \'e mostrar que $\mathcal{C}$ \'e linearmente independente. Considere a soma 
$$
    (2.3)\quad\sum_{i=1}^{n}\alpha_{i}w_{i} + \sum_{j=1}^{r}\gamma_{j}v_{j} + \sum_{l=1}^{s}\beta_{l}u_{l} = 0, \label{EQ1}
$$
em que $\alpha_{i}, \beta_{j}, \gamma_{l}\in\mathbb{K}$. Assim, reescrevendo a equa\c c\~ao, obtemos:
$$
    \sum_{l=1}^{s}\beta_{l}u_{l} = \sum_{i=1}^{n}(-\alpha_{i})w_{i} + \sum_{j=1}^{r}(-\gamma_{j})v_{j}, \in{W_1\cap{W_2}}
$$
sendo, ao mesmo tempo, combina\c c\~ao linear de elementos de $\mathcal{B'}$ e de elementos de $\mathcal{B``}.$ 
Logo, existem $\lambda_1, \cdots, \lambda_n\in\mathbb{K}$ tais que 
$$
    \sum_{l=1}^{s}\beta_{l}u_{l} = \sum_{i=1}^{n}\lambda_{i}u_{i}, 
$$
isto \'e, 
$$
    \sum_{l=1}^{s}\beta_{l}u_{l} + \sum_{i=1}^{n}(-\lambda_{i})u_{i} = 0.
$$
Como $\{u_1, \cdots, u_s, w_1, \cdots, w_n\}$ \'e linearmente independente, temos $\beta_l = 0$ para todo 
$l = 1, \cdots, s$ e $\lambda_i = 0$, para todo $i = 1, \cdots, n.$ Em particular, a equa\c c\~ao \ref{EQ1} se reduz
a 
$$
    \sum_{i=1}^{n}\alpha_{i}w_{i} + \sum_{j=1}^{r}\gamma_{j}v_{j} = 0.
$$
Sendo $\{w_1, \cdots, w_n, v_1, \cdots, v_r\}$ \'e linearmente independente, teremos $\alpha_{i} = 0$ para todo
$i = 1, \cdots, n$ e $\gamma_{j} = 0$ para todo $j = 1, \cdots, r$. Portanto, $\{w_1, \cdots, w_n, v_1, \cdots, v_r, u_1,
\cdots, u_s\}$ \'e linearmente independente e uma base de $W_1 + W_2.$

No caso em que $W_1\cap{W_2} = \{0\}$, sejam $\mathcal{B_1}, \mathcal{B_2}$ bases respectivas de $W_1, W_2$. Por um
processo an\'alogo ao que foi feito, mostra-se que $\mathcal{B_1}\cup\mathcal{B_2}$ \'e uma base de $W_1 + W_2$. 
Portanto, a demonstra\c c\~ao est\'a completa. 

\end{proof*}
Finalizamos esta parte com duas proposi\c c\~oes de exerc\'icios, j\'a que \'e uma aplica\c c\~ao do que foi visto
at\'e agora:
\begin{proposition*}
    Sejam V um $\mathbb{K}$-espa\c co vetorial e $W_1, W_2$ dois subespa\c cos de V. Ent\~ao, $V = W_1 \oplus W_2$ se e 
s\'o se cada elemento $v\in{V}$ se escreve de maneira \'unica como uma soma $x_1 + x_2$, com $x_1\in{W_1}, x_2\in{W_2}.$
\end{proposition*}
\begin{proposition*}
    Sejam V um $\mathbb{K}$-espa\c co vetorial finitamente gerado n\~ao nulo e $W_1$ um subespa\c co de V. Ent\~ao,
existe um subespa\c co $W_2$ de V tal que $V = W_1 \oplus W_2$.
\end{proposition*}


