\documentclass{article}
\usepackage{amsmath}
\usepackage{amsthm}
\usepackage{amssymb}
\usepackage{pgfplots}
\usepackage{amsfonts}
\usepackage[margin=2.5cm]{geometry}
\usepackage{graphicx}
\usepackage[export]{adjustbox}
\usepackage{fancyhdr}
\usepackage[portuguese]{babel}
\usepackage{hyperref}
\usepackage{lastpage}

\pgfplotsset{compat=1.18}

\pagestyle{fancy}
\fancyhf{}

\hypersetup{
    colorlinks,
    citecolor=black,
    filecolor=black,
    linkcolor=black,
    urlcolor=black
}

\newtheorem*{sol*}{\underline{Solu\c c\~ao:}}
\newtheorem*{proof*}{\underline{Prova:}}
\renewcommand\qedsymbol{$\blacksquare$}

\rfoot{P\'agina \thepage \hspace{1pt} de \pageref{LastPage}}

\begin{document}
Como exemplo de aplica\c c\~ao de tudo o que foi visto, veremos um tipo de espa\c co topol\'ogico
compacto que aparece na \'area de sistemas din\^amicos, composto de n\'umeros inteiros infinitos.
Definimos o espa\c co de sequ\^encias de N digitos inteiros como 
$$
    \Omega_N = \{\omega = (\cdots, \omega_{-1}, \omega_0, \omega_1, \dots): \omega_i\in\{0, 1, \cdots, N-1\}, i\in\mathbb{Z}\}
    = {0, 1, \cdots, N-1}^{\mathbb{Z}}
$$
Al\'em disso, \'e \'util separar o espa\c co no seu lado direito, colocando
$$
    \Omega_N^R = \{\omega = (\omega_0, \omega_1, \dots): \omega_i\in\{0, 1, \cdots, N-1\}, i\in\mathbb{Z}\}
    = {0, 1, \cdots, N-1}^{\mathbb{N}}
$$
Fixaremos $N = 2$ e a parte direita do espa\c co de sequ\^encias, ou seja, o conjunto
$$
    \Omega_2^R = \{\omega = (\omega_0, \omega_1, \dots): \omega_i\in\{0, 1\}, i\in\mathbb{Z}\},
$$
pois as vers\~oes mais gerais desses conjuntos requerem um rigor que vai al\'em do escopo dessa apresenta\c c\~ao.

A ideia por tr\'as desse conjunto \'e que seus elementos tomam a forma $\omega = (1, 1, 0, 0, 0, 0, 1, \cdots)$.
Assim, pode-se pensar nele como o conjunto de resultados de um jogo de duas op\c c\~oes, por exemplo
brincar de cara ou coroa ad infinitum, em que cada entrada de omega representa um resultado.

Nesse espa\c co, definimos uma topologia baseada no produto de $\{0, 1\}$ com ele mesmo, sendo cada
espa\c co $\{0, 1\}$ munido da topologia discreta. Nessas condi\c c\~oes, os abertos tomam a forma
de cilindros, estruturas que funcionam como proje\c c\~oes das coordenadas de $\omega$ - estamos
fixando um valor para uma entrada dele:
$$
    C_{\alpha_1, \cdots, \alpha_k}^{n_1, \cdots, n_k} := \{\omega\in\Omega_2^R: \omega_{n_i} = \alpha_i, 
\alpha_i\in\{0, 1\}\},
$$
em que $n_1 < \cdots < n_k$ e $\alpha_1, \cdots, \alpha_k\in\{0, 1\}$ s\~ao fixos. Intuitivamente,
estamos analisando as k primeiras jogadas da nossa partida infinita de cara ou coroa.

Al\'em de uma topologia, definimos uma m\'etrica neste espa\c co fixando $\lambda > 1$ e colocando
$$
    d_\lambda(\omega, \omega') = \sum_{n=-\infty}^{\infty}\frac{|\omega_n - \omega'_n|}{\lambda^{|n|}}.
$$
Com isso, o espa\c co adquire as propriedades de ser perfeito e totalmente disconexo, fazendo dele
homeomorfo ao conjunto de Cantor. 

Agora que vimos um pouco sobre o espa\c co e suas propriedades, \'e hora de entender em quais contextos
ele se prova necess\'ario. Essa caracter\'istica de ser definido por valores inteiros confere-o
a classifica\c c\~ao de discreto, ou seja, seus valores n\~ao s\~ao cont\'inuos. 

Isso faz do conjunto muito \'util para descrever estados, uma aplica\c c\~ao que pode at\'e mesmo ser 
vista na intui\c c\~ao aqui fornecida, com cada digito representando um estado da moeda no cara
ou coroa. Assim, num contexto mais geral, essa \'area de din\^amica simb\'olica \'e \'util descrevendo
a evolu\c c\~ao de sistemas, com cada valor representando um tempo diferente. Ele tamb\'em aparece
na descri\c c\~ao de estados qu\^anticos, realizando um papel importante na defini\c c\~ao das
integrais de caminho de Feynman formais por meio dos cilindros definidos.

Outras aplica\c c\~oes que n\~ao ser\~ao t\~ao detalhadas incluem a transmiss\~ao e recep\c c\~ao de 
dados e o armazenamento de informa\c c\~oes.

\end{document}