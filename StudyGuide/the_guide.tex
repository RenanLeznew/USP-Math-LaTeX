\documentclass{article}
\usepackage{amsmath}
\usepackage{amsthm}
\usepackage{amssymb}
\usepackage{pgfplots}
\usepackage[utf8]{inputenc}
\usepackage{amsfonts}
\usepackage[margin=2.5cm]{geometry}
\usepackage{graphicx}
\usepackage[export]{adjustbox}
\usepackage{fancyhdr}
\usepackage[portuguese]{babel}
\usepackage{hyperref}
\usepackage{lastpage}
\usepackage{mathtools}
\usepackage{csquotes}
\usepackage[style=abnt]{biblatex}
\addbibresource{ref}

 \pagestyle{fancy}
 \fancyhf{}

 \pgfplotsset{compat = 1.18}

 \hypersetup{
     colorlinks,
     citecolor=black,
     filecolor=black,
     linkcolor=black,
     urlcolor=black
 }
 \newtheorem*{def*}{\underline{Defini\c c\~ao}}
 \newtheorem*{theorem*}{\underline{Teorema}}
 \newtheorem*{lemma*}{\underline{Lema}}
 \newtheorem*{prop*}{\underline{Proposi\c c\~ao}}
 \newtheorem{example}{\underline{Exemplo}}
 \newtheorem*{proof*}{\underline{Prova}}
 \renewcommand\qedsymbol{$\blacksquare$}
 \newcommand{\Lin}[1]{Lin_{\mathbb{K}}({#1})}

 \rfoot{P\'agina \thepage \hspace{1pt} de \pageref{LastPage}}

 \begin{document}
 \begin{figure}[ht]
  \minipage{0.76\textwidth}
    \includegraphics[width=4cm]{../icmc.png}
    \hspace{7cm}
    \includegraphics[height=4.9cm,width=4cm]{../brasao_usp_cor.jpg}
  \endminipage  
\end{figure}

\begin{center}
  \vspace{1cm}
  \LARGE
  UNIVERSIDADE DE S\~AO PAULO

  \vspace{1.3cm}
  \LARGE
  INSTITUTO DE CI\^ENCIAS MATEM\'ATICAS E COMPUTACIONAIS - ICMC

  \vspace{1.7cm}
  \Large
  \textbf{UM GUIA DE BOAS PRÁTICAS NO ESTUDO}

  \vspace{1.3cm}
  \large
  \textbf{Renan Wenzel - 11169472}

  \vspace{6.3cm}
  \today
\end{center}

 \newpage
\textbf{{\Huge Disclaimer}}
 \vspace{5cm}

  {\huge Essas notas não possuem relação com professor algum. 

  Qualquer erro é responsabilidade solene do autor.

Caso julgue necessário, contatar: renan.wenzel.rw@gmail.com}
\newpage
\tableofcontents

\newpage
\section{Afinal, o que e para quem é esse PDF?}
\paragraph{} Todo e qualquer estudante que se preze já passou pela terrível fase de não saber como estudar,
organizar os estudos e qualquer coisa do tipo. Quando chega-se nesse ponto, alguns comportamentos podem acontecer, 
tais como procurar ajuda na internet, falar com um docente, organizar um grupo de estudos, etc. Tendo em mente 
essa necessidade, escrevo este PDF com base em artigos, conselhos e minha própria experiência como um caçador de conhecimento nesse
mundo complexo e estranho.

\paragraph{} A estrutura desse texto é relativamente simples, afinal, o propósito não é o de complicar o assunto (Já tem a internet pra isso!). Abordarei
ao longo dos capítulos alguns tópicos recorrentes na área da pesquisa da aprendizagem. Para as informações utilizadas, deixarei ao final as referências
que utilizei, até porque, como qualquer pessoa, não faz sentido confiar apenas na minha palavra, sempre posso estar errado. Sendo assim, para quem quiser 
conferir o que está sendo dito e aprofundar-se nessa bela área que é o estudo das conexões cerebrais, aprendizagem e memória, as referências podem ser 
um bom ponto de partida.

\paragraph{} Ao final do texto, acrescentei dois apêndices, um mais focado na minha área de estudo e outro sobre pesquisa em geral (no momento em que isso está sendo escrito). O primeiro contém algumas luzes quanto à escrita matemática,
a qual pode, muitas vezes, parecer desafiadora, especialmente no começo. Quando eu passei por essa situação, deparei-me com um ótimo texto do matemático J. M. Lee sobre
boas práticas de demonstração, então pretendo incluir algumas coisas sobre o assunto. 
O segundo apêndice é mais geral e expositivo, visando oferecer auxílio àqueles que buscam começar um hábito de leitura de artigos científicos, como procurar eles, a ordem de leitura de cada parte, entre outros 
aspectos que podem ajudar na compreensão. 

\paragraph{} Por fim, vale mencionar que incluirei algumas fontes extras para estudos e tópicos mais específicos, sendo estas fontes as que eu gosto de usar na maioria das vezes. Espero que tenha uma leitura agradável,
caro leitor, e que suas dúvidas e amarguras sejam minimamente aquietadas ao final da jornada.
\newpage
\section{Um Pouco Sobre nossa Memória}
\newpage
\section{As Pausas Durante Atividades}
\newpage
\section{Digitar vs Escrever à Mão}
\newpage
\section{Kinestesia}
\newpage
\section{A Questão do Sono}
\newpage
\section{Apêndices}
\subsection{Apêndice A: A Escrita Matemática}
\subsection{Apêndice B: A Prática da Leitura Acadêmica}
\newpage

\printbibliography
%\begin{thebibliography}{9}
%\bibitem{texbook}
%Donald E. Knuth (1986) \emph{The \TeX{} Book}, Addison-Wesley Professional.
%\end{thebibliography}
 

 \end{document}
