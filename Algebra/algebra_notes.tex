 \documentclass{article}
 \usepackage{amsmath}
 \usepackage{ stmaryrd }
 \usepackage{amsthm}
 \usepackage{amssymb}
 \usepackage{pgfplots}
 \usepackage{amsfonts}
 \usepackage[margin=2.5cm]{geometry}
 \usepackage{graphicx}
 \usepackage[export]{adjustbox}
 \usepackage{fancyhdr}
 \usepackage[portuguese]{babel}
 \usepackage{hyperref}
 \usepackage{lastpage}
 \usepackage{mathtools}

 \pagestyle{fancy}
 \fancyhf{}

 \pgfplotsset{compat = 1.18}

 \hypersetup{
     colorlinks,
     citecolor=black,
     filecolor=black,
     linkcolor=black,
     urlcolor=black
 }
 \newtheorem*{def*}{\underline{Defini\c c\~ao}}
 \newtheorem*{prop*}{\underline{Proposi\c c\~ao}}
 \newtheorem*{theorem*}{\underline{Teorema}}
 \newtheorem*{crl*}{\underline{Corol\'ario}}
 \newtheorem{example*}{\underline{Exemplo}}
 \newtheorem*{proof*}{\underline{Prova}}
 \newtheorem*{lemma*}{\underline{Lema}}
 \renewcommand\qedsymbol{$\blacksquare$}
 \newcommand{\Lin}[1]{Lin_{\mathbb{K}}({#1})}

 \rfoot{P\'agina \thepage \hspace{1pt} de \pageref{LastPage}}

 \title{Notas de }
 \author{Renan Wenzel}
 \date{\today}

 \begin{document}
 \begin{figure}[ht]
	\minipage{0.76\textwidth}
		\includegraphics[width=4cm]{../icmc.png}
		\hspace{7cm}
		\includegraphics[height=4.9cm,width=4cm]{../brasao_usp_cor.jpg}
	\endminipage	
\end{figure}

\begin{center}
	\vspace{1cm}
	\LARGE
	UNIVERSIDADE DE S\~AO PAULO

	\vspace{1.3cm}
	\LARGE
	INSTITUTO DE CI\^ENCIAS MATEM\'ATICAS E COMPUTACIONAIS - ICMC

	\vspace{1.7cm}
	\Large
	\textbf{Notas de Aula de \'Algebra}

	\vspace{1.3cm}
	\large
	\textbf{Renan Wenzel - 11169472}

	\vspace{1.3cm}
	\large
	\textbf{Roberto Carlos - alvarenga@icmc.usp.br}

	\vspace{1.3cm}
	\today
\end{center}

 \newpage

 \tableofcontents

 \newpage

\section{Aula 01 - 14/03/2023}
\subsection{Motiva\c c\~oes}
\begin{itemize}
  \item Compreender o que ser\'a estudado ao longo do curso;
\end{itemize}
\subsection{Introdu\c c\~ao ao Curso}
  Este curso \'e sobre teoria de grupos, a qual possui origem no estudo de simetrias, sejam elas de figuras ou de objetos alg\'ebricos.
Um exemplo de grupo seria o seguinte: 

  Considere um tri\^angulo equil\'atero. Existem algumas formas de olharmos para as simetrias
do tri\^angulo, como rotacionando-o, refletindo-o com rela\c c\~ao a um ponto m\'edio e um v\'ertice fixo. Contabilizando todas as poss\'iveis
formas delas acontecerem, h\'a seis simetrias deste ret\^angulo. Ademais, compondo simetrias resulta em outra, i.e., rotacionar e refletir
um certo v\'ertice continuar\'a sendo uma simetria do tri\^angulo. Al\'em disto, \'e um fato (futuramente visto) que essas seis simetrias 
totalizam todas as poss\'iveis simetrias de um tri\^angulo equil\'atero. De fato, dado um pol\'igono regular de n lados, ele possui
n! simetrias.

\subsection{Grupos e Opera\c c\~oes}
 \begin{def*}
   Seja S um conjunto n\~ao-vazio. Uma opera\c c\~ao em S \'e um mapa
   \begin{align*}
     \mu:&S \times S\rightarrow S\\
         &(a, b)\mapsto{\mu(a, b)}
   \end{align*}
 \end{def*}
\begin{example*}
  A opera\c c\~ao soma em $\mathbb{Z}, +:\mathbb{Z}\times{\mathbb{Z}}\rightarrow \mathbb{Z}, (a, b)\mapsto a + b$ \'e uma opera\c c\~ao.
\end{example*}
\begin{example*}
  Uma opera\c c\~ao em $\mathbb{R}$ \'e a multiplica\c c\~ao $.:\mathbb{R}\times{\mathbb{R}}\rightarrow \mathbb{R}, (a, b)\mapsto ab.$
\end{example*}
\begin{example*}
  Um exemplo do que n\~ao \'e opera\c c\~ao seria a subtra\c c\~ao dos naturais, $-:\mathbb{N}\times{\mathbb{N}}\rightarrow \mathbb{N}, (a, b)\mapsto a - b.$ (Consegue responder por que n\~ao \'e?)
\end{example*}
\begin{example*}
  Se S \'e o conjunto de simetrias de um tri\^angulo equil\'atero, ent\~ao a composi\c c\~ao
  \begin{align*}
    \circ: &S\times{S}\rightarrow S \\
           &(\sigma, \tau)\mapsto \sigma\circ \tau
  \end{align*}
  \'e uma opera\c c\~ao bin\'aria.
\end{example*}
  Faremos a conven\c c\~ao de denotar $\mu(a, b)$ por $a.b$ ou $a + b$, com base no contexto.
 \begin{def*}
   Uma opera\c c\~ao $\mu$ em S n\~ao-vazio, denotada pelo produto, \'e dita associativa se, para todos a, b, c em S,
   $$
      (a.b).c = a.(b.c), \quad \biggl(\mu(a, \mu(b, c)) = \mu(\mu(a, b), c)\biggr).
   $$
   Por outro lado, ser\'a dita comutativa se 
   $$
      a.b = b.a, \quad \biggl(\mu(a, b) = \mu(b, a)\biggr).
   $$
   Diremos, tamb\'em, que ela tem elemento neutro (ou identidade) se existe um elemento e em S tal que 
   $$
    a.e = e.a = a,\forall a\in{S}.
   $$
   Neste caso, diremos que e \'e o elemento neutro, ou a identidade, para $\mu$.
 \end{def*}
  Utilizaremos a nota\c c\~ao 1 para a identidade no caso em que $\mu$ \'e denotada por um produto e 0 pro caso em que \'e denotada por adi\c c\~ao.

 \begin{example*}
   A multiplica\c c\~ao de matrizes \'e associativa, n\~ao \'e comutativa e possui identidade.
 \end{example*}
\begin{example*}
  A soma de n\'umeros inteiros \'e associativa, comutativa e possui identidade.
\end{example*}
\begin{example*}
  A pot\^encia nos n\'umeros reais \'e n\~ao associativa, nem comutativa, mas possui identidade: $a^{(b^{c})} \neq (a^{b})^{c} = a^{bc}$
\end{example*}

\begin{prop*}
  Seja S um conjunto n\~ao-vazio e $\mu$ uma opera\c c\~ao em S denotada pelo produto. Ent\~ao, existe um \'unico jeito de definir o
produto (denotado temporariamente por $[a_{1}, \cdots, a_{n}]$) de n elementos em S tal que 
\begin{align*}
  &(i)\quad [a_{1}] = a_{1}; \\
  &(ii)\quad [a_{1}, a_{2}] = \mu(a_{1}, a_{2}) = a_{1}a_{2}; \\
  &(iii)\quad\forall 1\leq{i}<n, [a_{1}, \cdots, a_{n}] = [a_{1}, \cdots, a_{i}][a_{i+1}, \cdots, a_{n}].
\end{align*}
\end{prop*}
\begin{proof*}
  (iii)$\Rightarrow$ Para o caso $n\leq{2}$ \'e ok. Agora, suponha o produto bem-definido de r elementos em S, $r\leq{n=1}$. Ent\~ao,
  defina $[a_{1}, \cdots, a_{n}]\coloneqq [a_{1}, \cdots, a_{n-1}][a_{n}]. $ Como a defini\c c\~ao acima satisfaz a condi\c c\~ao (iii) para
i=n-1, se ela estiver bem-definida, ela ser\'a \'unica. Com efeito, seja $1\leq{i}<n-1$, tal que 
  \begin{align*}
    [a_{1}, \cdots, a_{n}] &= [a_{1}, \cdots, a_{n-1}][a_{n}] = [a_{1}, \cdots, a_{i}][a_{i+1}, \cdots, a_{n-1}][a_{n}]\\
                           &= \biggl([a_{1}, \cdots, a_{i}]\biggr)\biggl([a_{i+1}, \cdots, a_{n-1}][a_{n}]\biggr) \\
                           &= [a_{1}, \cdots, a_{i}][a_{i+1}, \cdots, a_{n}].\text{ \qedsymbol}
  \end{align*}
\end{proof*}
\begin{def*}
  Seja S n\~ao-vazio e $\mu$ uma opera\c c\~ao em S com identidade 1. Um elemento a de S \'e dito invers\'ivel se existe b em S tal que
 $ab = ba = 1.$ Neste caso, b \'e o inverso de a, denotado por $b\coloneqq a^{-1}$.
\end{def*}
  Note que tanto o elemento inverso quanto o elemento neutro, se existirem, s\~ao \'unicos (c.f. Lema abaixo). Al\'em disso, o inverso da adi\c c\~ao \'e
denotad por -a.
\begin{lemma*}
  Seja S n\~ao-vazio, $\mu$ uma opera\c c\~ao associativa denotada pelo produto. Ent\~ao, 
 \begin{itemize}
   \item[i)] Existe no m\'aximo um elemento neutro para S e $\mu$;
   \item[ii)] Se o elemento neutro existe, ent\~ao para cada elemento de S, existe no m\'aximo um inverso;
   \item[iii)] Se um elemento a de S tem inverso \`a esquerda l e \`a direita r, i.e. $l.a = 1 \text{ e } a.r = 1$, ent\~ao
a \'e invers\'ivel com inverso l = r.
   \item[iv)] Se a, b em S s\~ao invers\'iveis, ent\~ao o produto ab \'e invers\'ivel, com inverso $b^{-1}a^{-1}.$
 \end{itemize}
\end{lemma*}
  Antes de provar, observe que a exist\^encia de um elemento inverso \`a esquerda ou \`a direita n\~ao garante que um elemento
seja invers\'ivel (exerc\'icio), eles devem coincidir.
\begin{proof*}
 $(i)\Rightarrow)$ Suponha que existem 1, 1' em S como seus elementos neutros. Basta mostramos que eles coincidem. Com efeito,
 $$
    1 = 1.1' = 1'.1 = 1'.
 $$
 Portanto, o elemento neutro \'e \'unico.

 $(ii)\Rightarrow)$ Assuma a exist\^encia de dois elementos inversos em S para um elemento a, denotados por b, b'. Ent\~ao, como
ab = ba = 1, temos
  $$
    b = b1 = b(ab') = (ba)b' = 1b' = b'.
  $$
  Portanto, o elemento inverso \'e \'unico. Os itens (iii) e (iv) s\~ao exerc\'icios. \qedsymbol
\end{proof*}
  
 \begin{def*}
   Um monoide \'e um par (G, $\mu$), em que G \'e um conjunto n\~ao-vazio e $\mu$ uma opera\c c\~ao associativa e com elemento neutro em G.
  Se, ainda por cima, $\mu$ for comutativa, (G, $\mu$) \' e um monoide abeliano (ou comutativo).
 \end{def*}
 \begin{def*}
   Um grupo \'e um par (G, $\mu$) \'e um monoide (G, $\mu$) com a condi\c c\~ao extra que todo elemento de G possui inverso. Caso
 $\mu$ seja comutativa, chamamos G de grupo abeliano.
 \end{def*}
\begin{example*}
  Os inteiros com a soma, $(\mathbb{Z}, +)$, \'e um grupo comutativo, enquanto $(\mathbb{Z}, .)$ n\~ao \'e um grupo, mas sim um monoide.
\end{example*}
\begin{example*}
  O grupo das matrizes com entradas reais e sua multiplica\c c\~ao, $(\mathbb{M}_{n}(\mathbb{R}), .)$, \'e um grupo n\~ao-abeliano.
\end{example*}

\section{Aula 02 - 16/03/2023}
\subsection{Motiva\c c\~oes}
\begin{itemize}
  \item Outras estruturas alg\'ebricas e exemplos;
  \item Tamanho de um grupo;
  \item Subgrupos
\end{itemize}
\subsection{Usos de Grupos}
  Podemos usar grupos para definir outras constru\c c\~oes alg\'ebricas, como segue.
 \begin{def*}
   Um anel \'e uma terna $(A, \mu, \varphi)$, em que $(A, \mu)$ \'e um grupo abeliano e $(A, \varphi)$ \'e um monoide. Al\'em disso,
vale a distributiva.
   $$
    \varphi(a, \mu(c, d)) = \varphi(\mu(a, c), \mu(a, d)), \quad (a(b + c) = ab + ac).
   $$
   Usualmente, escrevemos $(A, \mu, \varphi) = (A, +, \cdot).\square$
 \end{def*}
\begin{def*}
  Um corpo \'e um anel $(A, +, \cdot)$ tal que $(A-\{0\}, .)$ \'e um grupo abeliano. $\square$
\end{def*}
  Deste ponto em diante, abandonaremos as letras gregas para usar apenas os s\'imbolos ``+'' ou ``.'' para um grupo com adi\c c\~ao
ou com multiplica\c c\~ao. Vejamos alguns exemplos.
\begin{example*}
  \begin{center}
  \begin{tabular}{||c c c c c||}
  \hline
  Conjunto & Monoide & Monoide Comutativo & Grupo & Grupo Comutativo \\ [1ex] 
 \hline\hline
 $(GL_{n}, \cdot)$ & Sim & N\~ao & Sim & N\~ao \\ 
 \hline
 $(SL_{n}, \cdot)$ & Sim & N\~ao & Sim & N\~ao \\
 \hline
 $(\mathbb{Z}, +)$ & Sim & Sim & Sim & Sim \\
 \hline
 $(\mathbb{Z}, \cdot)$ & Sim & Sim & N\~ao & N\~ao \\
 \hline
 $(\mathbb{Q}, +)$ & Sim & Sim & Sim & Sim \\ [1ex] 
 \hline 
 $(\mathbb{Q}, \cdot)$ & Sim & Sim & N\~ao & N\~ao \\ [1ex] 
 \hline 
 $(S = \{z\in \mathbb{C}: |z| = 1\}, \cdot)$ & Sim & Sim & Sim & Sim \\ [1ex] 
 \hline 
 $(\mathbb{M}_{n}(\mathbb{R}), +)$ & Sim & Sim & Sim & Sim \\ [1ex] 
 \hline 
 $(\mathbb{M}_{n}(\mathbb{R}), .)$ & Sim & N\~ao & N\~ao & N\~ao \\ [1ex] 
 \end{tabular}
\qedsymbol
\end{center}
\end{example*}
\begin{example*}
  Seja T um conjunto qualquer e 
  $$
    G = \{f:T\rightarrow T: f \text{ bijetora.}\}
  $$
  Ent\~ao, $(G, \circ)$ \'e um grupo, chamado grupo das permuta\c c\~oes ou simetrias de T. Se T \'e um conjunto finito, e.g.
  $T = \{1, \cdots, n\}, $ ent\~ao denotamos $(G, \circ)$ por $(S_{n}, \circ).$ \qedsymbol
\end{example*}
\begin{def*}
  A ordem de um grupo $(G, \cdot)$ \'e a cardinalidade de G: $|G|$. Caso $|G| < \infty,$ dizemos que $(G, \cdot)$ \'e um grupo finito. $\square$
\end{def*}
\begin{example*}
  A ordem de $|\mathbb{Z}| = \infty$ e $|S_{n}| = n!$ \qedsymbol
\end{example*}
\begin{prop*}
  Se $(G, \cdot)$ \'e um grupo e a, b, c s\~ao elementos de G tais que ab = ac ou ba = ca, ent\~ao b = c. Al\'em disso, se ab = a ou
ba = a, ent\~ao b = 1.
\end{prop*}
\begin{proof*}
  Seja $a^{-1}$ o inverso de a, ent\~ao $a^{-1}(ab) = a^{-1}(ac).$ Mais ainda, se ab = a, ent\~ao $b = (a^{-1}a)b = a^{-1}(ab) = a^{-1}a = 1.$
\end{proof*}
  Fica de exerc\'icio mostrar que s\'o existe um grupo de ordem 2 e que $S_{3}$ \'e um grupo n\~ao-comutativo.
\subsection{Subgrupos}
 \begin{def*}
   Um subgrupo H de um grupo $(G, \cdot)$ \'e um subconjunto H de contido em G tal que
  \begin{align*}
    &1) 1\in H; \\
    &2) a, b\in H\Rightarrow ab \in H; \\
    &3) a\in H\Rightarrow a^{-1}\in H.
  \end{align*}
  Denotaremos subrupos por $H \leq{G}$ ou $(H, \cdot) \leq{(G, \cdot)}. \square$
 \end{def*}
\begin{prop*}
 Com opera\c c\~ao induzida pela multiplica\c c\~ao de G restrita a H, $(H, \cdot)$ \'e um grupo. 
\end{prop*}
\begin{proof*}
  Como H est\'a contido em G, podemos restringir o produto de G a H para 
  $$
    ._H:H \times H\rightarrow H.
  $$
  Afirmamos que $(H, \cdot)$ \'e um grupo. Com efeito, a restri\c c\~ao de . a H est\'a bem-definida pelo segundo item da defini\c c\~ao
de subgrupo. Mais ainda, ela \'e associativa em H por ser em G e todo elemento em H tem inverso pela condi\c c\~ao 3. Por fim,
ela tem elemento neutro pela primeira requisi\c c\~ao ao definir subgrupo. Portanto, $(H, \cdot)$ \'e um grupo.
\end{proof*}
Observe que todo grupo tem ao menos dois subgrupos, chamados triviais, sendo eles $\{1\}$ e ele mesmo. Qualquer outro leva o nome de
subgrupo pr\'oprio.
\begin{example*}
  $(SL_{n}, \cdot) \leq{(GL_{n}, \cdot)}$ e $(\mathbb{Z}, +) \leq{(\mathbb{Q}, +)}.$ \qedsymbol
\end{example*}
\begin{example*}
  Seja n um inteiro, ent\~ao $n\mathbb{Z}\coloneqq \{nk: k\in \mathbb{Z}\}$ \'e um subgrupo dos inteiros. De fato, 0 = n0 pertence
  a $n\mathbb{Z}.$ Al\'em disso, se nk e nk' pertencem a $n\mathbb{Z},$ ent\~ao
  $$
    nk + nk' = n(k + k')\in n \mathbb{Z}.
  $$
  Por fim, se nk pertence a $ n\mathbb{Z}$, ent\~ao n(-k) tamb\'em pertence a $n \mathbb{Z}$ e nk + n(-k) = 0. \qedsymbol
\end{example*}
\begin{prop*}
  Todo subgrupo de $(\mathbb{Z}, +)$ \'e da forma $n \mathbb{Z}$ para algum n inteiro.
\end{prop*}
\begin{proof*}
  Caso n seja 1, $n \mathbb{Z} = \mathbb{Z}$ e, se n = 0, ent\~ao $n \mathbb{Z} = \{0\}. $ Agora, seja H um subgrupo pr\'oprio dos
inteiros e n o menor inteiro positivo em H. Afirmamos que $n \mathbb{Z} = H$. 

  De fato, $n \mathbb{Z} \leq{H}$, pois n \'e um elemento de H, ent\~ao $nk = \underbrace{n + \ldots + n}_{\text{k-vezes}} \in H$. Al\'em disso, $-n\in H,$ de forma
que $-nk = \underbrace{(-n) + \ldots + (-n)}_{\text{k-vezes}}\in H.$ Portanto, $n \mathbb{Z}\in H.$

  Por outro lado, seja m um inteiro de H e considere $m = nq + r, 0 \leq{r} < n, q\in \mathbb{Z}.$ Pelo algoritmo de divis\~ao de Euclides,
  $$
  m - nq = r \Rightarrow r\in H\Rightarrow r = 0.
  $$
  e, assim, m = nq pertence a $n\mathbb{Z}$. Portanto, $H = n \mathbb{Z}$. \qedsymbol
\end{proof*}
\begin{prop*}
 \begin{itemize}
   \item[1)] $n \mathbb{Z} + m \mathbb{Z}$ \'e subgrupo de $\mathbb{Z};$
   \item[2)] $n \mathbb{Z} + m \mathbb{Z} = d \mathbb{Z}$, em que d \'e tal que 
    \item[2.1)] $d | n$ e $d | m$;
      \item[2.2)] Se $l | n$ e $l | m$, ent\~ao $l | d;$
        \item[2.3)] Existem r, s inteiros tais que $rn + sm = d.$
 \end{itemize}
 Definimos $d = \gcd{(n, m)}$ como o m\'aximo divisor comum de m e n.
\end{prop*}
\begin{proof*}
  A prova do item 1 fica como exerc\'icio. 

  $2.1)\Rightarrow n \mathbb{Z} + m \mathbb{Z} = d \mathbb{Z}.$ Em particular, se n, m pertencem a $d \mathbb{Z}$, ent\~ao
$d | n \text{ e } d | m.$

  $2.2)\Rightarrow$ Suponha que $n = lq_{1}, m = lq_{2} $. Se x pertence a $n \mathbb{Z} + m \mathbb{Z},$ ent\~ao
  \begin{align*}
    &x = nk_{1} + mk_{2} = lq_{1}k_{1} + lk_{2}q_{2} \in l \mathbb{Z} \\
    &\Rightarrow n \mathbb{Z} + m \mathbb{Z} \subseteq{l \mathbb{Z}} \Rightarrow l | d.
  \end{align*}
  tamb\'em \'e poss\'ivel mostrar isso usando o item 3 da proposi\c c\~ao.

  $2.3)\Rightarrow$ imediato. \qedsymbol
\end{proof*}
\begin{prop*}
 \begin{itemize}
   \item[1)] $m \mathbb{Z}\cap n \mathbb{Z}$ \'e subgrupo dos inteiros;
   \item[2)] $m \mathbb{Z}\cap n \mathbb{Z} = l \mathbb{Z}$, em que l \'e tal que 
     \item[2.1)] $m | l, n | l;$
       \item[2.2)] Se $m | l'$ e $n | l'$, ent\~ao $l | l'.$
 \end{itemize}
 Definimos $l = mmc(m, n)$ o m\'inimo m\'ultiplo comum de m e n.
\end{prop*}
\begin{proof*}
  Fica como exerc\'icio.
\end{proof*}
\begin{crl*}
  Se m, n s\~ao inteiros, ent\~ao $mn = mmc(m, n)\gcd{(m, n)}$.
\end{crl*}
\newpage

\section{Aula 03 - 21/03/2023}
\subsection{Motiva\c c\~oes}
\begin{itemize}
  \item Outros exemplos de subgrupos;
  \item Subgrupos gerado por subconjuntos;
  \item Grupo c\'iclico e ordem de elementos.
\end{itemize}
\subsection{Subgrupos - Outras Propriedades}
  Quando o conjunto candidato a subgrupo \'e n\~ao-vazio, n\~ao \'e necessario exigir que a identidade seja parte dele. De fato,
\begin{prop*}
  Se G \'e um grupo e $H\subseteq{G}, H \neq\emptyset$, ent\~ao $H\leq{G}$ se, e somente se,
 \begin{align*}
   &1) ab\in H, \quad a, b\in H\\
   &2) a^{-1}\in H, \quad a\in H.
 \end{align*}
\end{prop*}
\begin{proof*}
  $\Rightarrow$ Segue da defini\c c\~ao de subgrupo (ab e $a^{-1}$ pertencem a H por defini\c c\~ao);

  $\Leftarrow$ Sendo H n\~ao-vazio, existe $a\in H$. Atrav\'es de (2), $a^{-1}\in H$ e, por (1), $aa^{-1} = 1\in H.$ Portanto,
H \'e subgrupo de G. \qedsymbol
\end{proof*}
  Outra formula\c c\~ao de subgrupo requer apenas uma condi\c c\~ao:
 \begin{prop*}
   Se G \'e um grupo e $H\subseteq{G}, H \neq\emptyset$, ent\~ao $H\leq{G}$ se, e s\'o se, $ab^{-1}\in H$ para todos $a, b\in H.$
 \end{prop*}
\begin{proof*}
  $\Rightarrow$ Suponha que H \'e um subgrupo de G e sejam $a, b\in H$. Ent\~ao, por defini\c c\~ao, $a^{-1}, b^{-1}\in H$. Assim,
segue da defini\c c\~ao de subgrupo que $ab^{-1}\in H$.

  $\Leftarrow$ Se $H \neq\emptyset$, existe ao menos um a em H. Por hip\'otese, $1 = aa^{-1}\in H$. Assim, $a^{-1} = 1a^{-1}\in H$.
Por fim, se a, b s\~ao membros de H, ent\~ao $b^{-1}\in H$, tal que $ab = a(b^{-1})^{-1}\in H$. Portanto, H \'e subgrupo de G. \qedsymbol
\end{proof*}

 \begin{def*}
   Se G \'e um grupo, ent\~ao $Z(G) = \{g\in G: ga = ag\forall a\in{G}\} $ \'e um subgrupo de G chamado centro de G.
 \end{def*}
  Provemos que Z(G) \'e de fato um subgrupo. De fato, $1\in Z(G)$ pela defini\c c\~ao de elemento neutro. Al\'em disso, se $g, h\in Z(G),$
ent\~ao gha = gah = agh, tal que $gh\in Z(G).$ Al\'em disso, $g^{-1}\in Z(G),$ pois $g^{-1}a = (a^{-1}g)^{-1} = (ga^{-1})^{-1} = ag^{-1}.$
  Uma propriedade interessante \'e que G ser\'a um grupo abeliano se, e somente se, $Z(G) = G.$ 
 \begin{example*}
 Exerc\'icio: Dados $G_{1}, G_{2}$ grupo, defina o grupo produto como $G_{1}\times G_{2} = \{(g_{1}, g_{2}): g_{i}\in G_{i}\}$. Encontre
uma opera\c c\~ao que torne este conjunto um grupo de fato.
 \end{example*}
 \begin{example*}
  \begin{itemize}
    \item[1)]  Se V \'e um subespa\c co vetorial de um corpo qualquer $\mathbb{K}$, ent\~ao $V<\leq{\mathbb{K}}$.  
    \item[2)] O conjunto 
      $$
      SU_{2}(\mathbb{C}) = \biggl\{ \begin{bmatrix}
          \alpha & -\overline{\beta} \\
          \beta & \overline{\alpha}
    \end{bmatrix}: \alpha, \beta \in \mathbb{C}, |\alpha|^{2} + |\beta|^{2} = 1\biggr\} 
    $$
    \'e um subgrupo de $GL_{2}(\mathbb{C})$.
    \item[3)] O conjunto
      $$
      SO_{2}(\mathbb{R}) = \biggl\{\begin{bmatrix}
          \cos{(\theta)} & \sin{(\theta)}\\
          \sin{(\theta)} & \cos{(\theta)}
      \end{bmatrix}: \theta\in \mathbb{R}\biggr\} \leq{GL_{2}(\mathbb{R})}
      $$
  \end{itemize}
 \end{example*}
\begin{prop*}
  Se G \'e um grupo abeliano, ent\~ao todo subgrupo de G \'e tamb\'em abeliano
\end{prop*}
\begin{proof*}
  Se $H\leq{G}, a, b\in H$, em particular a, b tamb\'em pertencem a G, tal que $ab = ba.$ \qedsymbol
\end{proof*}
  Observe que a rec\'iproca e falsa. Com efeito, o subgrupo 
  $$
  \biggl\{\begin{bmatrix}
      1 & a \\
      0 & 1
    \end{bmatrix}\in GL_{2}(\mathbb{R}): a\in \mathbb{R}\biggr\}
  $$
  \'e subgrupo abeliano de $GL_{2}(\mathbb{R}).$ Al\'em disso, a rec\'iproca n\~ao vale nem mesmo se todo subgrupo pr\'oprio de um
grupo for abeliano, visto que todo subgrupo de $S_{3}$ \'e abeliano, mas o pr\'oprio $S_{3}$ n\~ao \'e.

 \begin{def*}
   Seja G um grupo e $S\subseteq{G}$ um subconjunto n\~ao-vazio. Definimos o conjunto gerado por S como 
   $$
   <S>\coloneqq \biggl\{a_{1}\cdots a_{n}: a_{i}\in S\text{ ou }a_{i}^{-1}\in{S}\biggr\}, \quad n\in \mathbb{N}\square.
   $$
 \end{def*}
\begin{prop*}
  $<S>$ \'e um subgrupo de G.
\end{prop*}
\begin{proof*}
  \'E claro que $<S> \neq\emptyset$. Agora, se $a_{1}\cdots a_{n} (= x), b_{1}\cdots b_{m}(= y)\in <S>$, ent\~ao
  $$
  xy^{-1} = a_{1} \cdots a_{n}(b_{1}\cdots b_{m})^{-1} = a_{1}\cdots a_{n}b_{m}^{-1}\cdots b_{1}^{-1}\in <S>.
  $$
  Portanto, pela segunda defini\c c\~ao equivalente de subgrupo, $<S>\leq{G}.$ \qedsymbol
\end{proof*}
\begin{def*}
  Nas condi\c c\~oes da propsi\c c\~ao, $<S>$ \'e o subgrupo gerado por S. Caso S seja finito, digamos $S=\{g_{1},\cdots, g_{n}\}$, denotamos
  $<S>$ por $<g_{1}, \cdots, g_{n}>.\square$
\end{def*}
\begin{def*}
  Sejam G um grupo e g um elemento seu. Se $G=<g>$, diremos que G \'e um grupo c\'iclico. $\square$
\end{def*}
\begin{def*}
  Se G \'e um grupo e g seu elemento, definimos a ordem de g (nota\c c\~ao: $|g|$ ou ord(g)) como a ordem de $<g>$.
\end{def*}
\begin{example*}
  $\mathbb{Z} = (1)$ \'e um grupo c\'iclico infinito, $S_{2}$ \'e um grupo c\'iclico finito e $S_{3}$ n\~ao \'e c\'iclico. \qedsymbol
\end{example*}
Atente-se ao fato de que $<g>\coloneqq\{\cdots, g^{-2}, g^{-1}, g^{0}=1, g, g^{2}, \cdots\} = \{g^{\mathbb{Z}}\} $
\begin{example*}
  Exerc\'icio: Calcule as ordens dos elementos de $S_{2}, S_{3}.$
\end{example*}
  Note que todo subgrupo de $\mathbb{Z}$ \'e c\'iclico. Al\'em disso, se $|G|<\infty,$ segue que $|g|<\infty$. Em particular,
$|g|\leq{|G|}$. Vale mencionar tamb\'em que mesmo se o grupo tem ordem infinita, o grupo c\'iclico pode ter ordem finita. De fato,
se $(G, \cdot) = (\mathbb{R}^{\times}, \cdot)$, tome g = 1. Ent\~ao, $<g> = \{-1, 1\}$, que \'e finito de ordem 2.
 \begin{prop*}
   Sejam G um grupo e g um elemento seu. Denotemos por S o conjunto dos inteiros n tais que $g^{n} = 1.$ Ent\~ao,
  \begin{itemize}
    \item[i)]$S\leq{\mathbb{Z}}$;
    \item[ii)]As pot\^encias $g^{m}, g^{n}, m\geq{n}$ s\~ao iguais se, e somente se, $g^{m-n} = 1(i.e. m-n\in S);$
    \item[iii)] Se $S\neq0 \mathbb{Z},$ ent\~ao $S=n\mathbb{Z}$ e as pot\^encias $1, g, g^{2}, \cdots, g^{n-1}$ s\~ao distintas e 
  s\~ao todos os elementos em $<g>.$ Em particular, $|g|=n.$
  \end{itemize}
 \end{prop*}
\begin{proof*}
  $(i)\Rightarrow$ Se m, n pertence a S, ent\~ao $g^{m-n} = g^{m}(g^{n})^{-1} = 1$, logo m-n pertence a S. \'E claro que S \'e n\~ao-vazio, pois 0 sempre \'e um elemento seu.

  $(ii)\Rightarrow$ \'E a lei do cancelamento.

  $(iii)\Rightarrow$ Se $S = \{0\}$, \'e autom\'atico. Como $S\leq{\mathbb{Z}}$, pela classifica\c c\~ao dos subgrupos de $\mathbb{Z},$
existe n em $\mathbb{Z}$ tal que $S = n \mathbb{Z}$. Agora, seja k um inteiro qualquer. Segue da divis\~ao Euclidiana que
$k = nq + r, 0\leq{r}<n$. Assim, $g^{k} = g^{nq}g^{r} = 1g^{r},$ tal que $<g> \subseteq{\{g^{0}=1, \cdots, g^{n-1}\}}$. Finalmente, pelo
item (ii) e da minimalidade de n. \qedsymbol
\end{proof*}
\begin{crl*}
  $<g> = \{1, g, g^{2}, \cdots, g^{n-1}\} $ 
\end{crl*}
\begin{crl*}
  Se a ordem de g \'e diferente de zero, ent\~ao ela \'e o menor inteiro positivo n tal que $g^{n} = 1$.
\end{crl*}
\begin{crl*}
  Se a ordem de g \'e $n>{0}$, ent\~ao $g^{k} = 1,$ se, e somente se, $n|k.$
\end{crl*}
\begin{crl*}
  Se a ordem de g \'e $n>0, k\in \mathbb{Z}$, ent\~ao $|g^{k}|=\displaystyle \frac{n}{mdc(n, k)}$.
\end{crl*}
\newpage

\section{Aula 04 - 23/03/2023}
\begin{itemize}
  \item Ciclos e Grupo de Permuta\c c\~oes
  \item Morfismo de Grupos
  \item Classes laterais
\end{itemize}
\subsection{Ciclos e Grupos de Permuta\c c\~ao}
Introduzimos a seguir o grupo das permuta\c c\~oes, denotado $S_{n}.$
\begin{def*}
  Uma permuta\c c\~ao $\sigma\in S_{n}$ \'e um r-ciclo se existem $a_{1},\cdots, a_{r}\in\{1,\cdots, n\}$ tais que $\sigma(a_{1})=a_{2},
  \sigma(a_{2})=a_{3}, \cdots, \sigma(a_{r-1})=a_{r}, \sigma(a_{r}) = a_{1}$ e, al\'em disso, $\sigma(j) = j$ para todo j em $\{1,\cdots, n\}/\{a_{1},\cdots,a_{r}\}$. 
  Dizemos que r \'e o comprimento de r, e denotamos $\sigma$ por $\sigma=(a_{1}\cdots a_{r}).\square$
\end{def*}
\begin{def*}
  Um 2-ciclo \'e chamado transposi\c c\~ao. $\square$
\end{def*}
 \begin{example*}
   Seja $\sigma\in S_{5}.$ Um 5-ciclo \'e , por exemplo, $\sigma(1)=2, \sigma(2)=3, \sigma(3)=4, \sigma(4)=5, \sigma(5)=1,$ ou $\sigma=(12345)=(34512).$
Um 3-ciclo seria $\sigma(1)=4, \sigma(2)=2, \sigma(3)=1, \sigma(4)=3, \sigma(5)=5$ e uma transposi\c c\~ao seria $\sigma(1)=2, \sigma(2)=1, \sigma(3)=3, \sigma(4)=4, \sigma(5)=5.$
 \end{example*}
\begin{def*}
  Duas permuta\c c\~oes $\sigma, \tau\in S_{n}$ s\~ao disjuntas se para todo $j\in\{1, \cdots, n\}, \sigma(j) = j$ ou $\tau(j) = j.$
\end{def*}
\begin{example*}
  $\tau\in S_{5}, \tau=(34), \sigma(12) \Rightarrow \tau, \sigma$ s\~ao disjuntas.
\end{example*}
  Observe que nem toda permuta\c c\~ao \'e um r-ciclo. De fato, $\sigma\in S_{5}$ dada por $\sigma(1)=3, \sigma(2) = 4, \sigma(3) = 5, \sigma(4) = 2$
e $\sigma(5)=1$ n\~ao \'e um r-ciclo. O fato \'e que toda permuta\c c\~ao \'e o produto de ciclos disjuntos de comprimento maior ou igual a 2.
Assim, $\sigma = (135)(24)$ descreve a permuta\c c\~ao enviando 1 pra 3, 2 pra 4, 3 pra 5, 4 pra 2 e 5 pra 1. 
\begin{example*}
  Seja $\sigma\in S_{5}, \sigma=(12)(13)(15) = (1532)$. Note que l\^e-se o produto de permuta\c c\~oes como a composi\c c\~ao de fun\c c\~oes, isto \'e,
come\c ca-se pela direita e termina na esquerda (afinal, \'e a composi\c c\~ao de permuta\c c\~oes, que s\~ao, particularmente, fun\c c\~oes!). Deste produt\'orio, vimos que o n\'umero 1 \'e o \'unico que ser\'a alterado, i.e., uma permuta\c c\~ao ap\'os o 1 demarca o fim da a\c c\~ao. Assim, este
exemplo indica que 1 se torna 5 e permanece assim (a primeira a\c c\~ao torna 1 no elemento 5). 5 se torna 1, depois 3 e permanece assim ($1->5->3->3$), 2 se torna 1
no final ($2->2->2->1$), 3 se torna eventualmente 2 ($3->2->1->2$) e 4 permanece constante. (P.S. Se essa parte ficar confusa, me chamem no celular pra eu explicar melhor).
\end{example*}
\begin{prop*}
  Toda permuta\c c\~ao em $S_{n}$ \'e um produto de transposi\c c\~oes (2-ciclos). Isto \'e, $S_{n}=\langle\text{transposi\c c\~oes}\rangle$. 
Al\'em disso, se $\sigma\in S_{n}, \sigma=\tau_{1}\cdots\tau_{r} = \rho_{1}\cdots\rho_{s}$ fatora\c c\~oes em transposi\c c\~oes, ent\~ao $2|r-s.$
\end{prop*}
\begin{proof*}
  Observe que $Id = (12)(21)\in\langle\text{ transposi\c c\~oes }\rangle$. Se $\sigma\in S_{n}$ \'e uma permuta\c c\~ao qualquer, ent\~ao $\sigma$ \'e 
o produto de ciclos. Logo, basta verificar a proposi\c c\~ao para um r-ciclo $\sigma.$ Suponha, assim, que $\sigma = (a_{1}\cdots a_{r}$ \'e um r-ciclo.
Com isso, $r=(a_{1}a_{2})(a_{1}a_{3})\cdots(a_{1}a_{r}).$ \qedsymbol
\end{proof*}
\begin{prop*}
  Exerc\'icio: Mostre que qualquer fatora\c c\~ao de um r-ciclo em transposi\c c\~oes tem mesma paridade.
\end{prop*}
\begin{def*}
  Seja $\sigma\in S_{n}$. Ent\~ao, a matriz de permuta\c c\~ao $\sigma$ \'e
  $$
    U(\sigma)\coloneqq \begin{pmatrix}
      e_{\sigma(1)}\\
      \vdots\\
      e_{\sigma(n)}
    \end{pmatrix}
  $$
  em que $e_{i}$ \'e o i-\'esimo vetor can\^onico de $\mathbb{R}^{n}.\square$
\end{def*}
\begin{example*}
  Seja $\sigma = (135)(24)\in S_{5}.$ Para esta permuta\c c\~ao, a matriz \'e
  $$
    U(\sigma) =
    \begin{pmatrix}
      e_{3}\\
      e_{4}\\
      e_{5}\\
      e_{2}\\
      e_{1}
    \end{pmatrix} =
    \begin{pmatrix}
      0 & 0 & 1 & 0 & 0\\
      0 & 0 & 0 & 1 & 0\\
      0 & 0 & 0 & 0 & 1\\
      0 & 1 & 0 & 0 & 0\\
      1 & 0 & 0 & 0 & 0
    \end{pmatrix}
  $$
  Note que 
  $$
    U(\sigma) = \begin{pmatrix}
      1\\
      2\\
      3\\
      4\\
      5
    \end{pmatrix} =
    \begin{pmatrix}
      5\\
      4\\
      3\\
      2\\
      1
    \end{pmatrix}
  $$
\end{example*}
 \begin{prop*}
   Sejam $\sigma, \tau\in S_{n}$ e $U(\sigma), U(\tau)$ as matrizes associadas respectivas. Ent\~ao,
  \begin{itemize}
    \item[1)] $U(\sigma)(12 \cdots n)^{T} = a_{1}e_{1} + \cdots + a_{n}e_{n} \Longleftrightarrow \sigma(j) = a_{j},\quad j = 1, \cdots, n.$
    \item[2)] $U(\sigma)$ sempre tem um \'unico 1 em cada linha e em cada coluna. Reciprocamente, toda matriz desse tipo
    \'e uma matriz de alguma permuta\c c\~ao.
    \item[3)] $\det{U(\sigma)}\in\{-1, 1\}.$
    \item[4)] A matriz de permuta\c c\~ao de $\tau\sigma$ \'e $U(\tau)\cdot U(\sigma)$.
  \end{itemize}
 \end{prop*}
\begin{proof*}
  Exerc\'icio.
\end{proof*}
\begin{def*}
  Se $\sigma\in S_{n}$ e $U(\sigma)$ \'e a matriz associada, definimos o sinal de $\sigma (sgn(\sigma))$ como sendo o 
$\det{(U(\sigma)}.$Al\'em disso, diremos que $\sigma$ \'e uma permuta\c c\~ao par quando $sgn(\sigma) = 1$ e \'impar quando
 $sgn(\sigma) = -1.\quad\square$ 
\end{def*}
  Observe que \'e poss\'ivel demonstrar que $sgn(\sigma) = (-1)^{r},$ em que r \'e o n\'umero de transposi\c c\~oes que aparecem
na decomposi\c c\~ao de $\sigma.$

\subsection{Morfismos de Grupos}
  Morfismos de grupos funcionam como fun\c c\~oes entre conjuntos, mas que levam em conta a opera\c c\~ao existente nos grupos.
 \begin{def*}
   Sejam G, G' dois grupos. Um morfismo de grupos \'e um mapa $\varphi:G\rightarrow G'$ tal que $\varphi(gh)=\varphi(g)\varphi(h)$ para todo
  g, h em G. $\quad\square$
 \end{def*}
 \begin{example*}
   S\~ao morfismos:
  \begin{align*}
    &1) sgn:S_{n}\rightarrow \{+1, -1\}, \sigma\mapsto sgn(\sigma),&&sgn(\sigma\tau) = \det(U(\sigma)U(\tau)) = \det(U(\sigma))\det(U(\tau)) = sgn(\sigma)sgn(\tau)\\
    &2) \det:GL_{n}\rightarrow \mathbb{R}^{\times}, A\mapsto\det(A)\\
    &3) \exp:(\mathbb{R}, +)\rightarrow (\mathbb{R}, \cdot), x\mapsto e^{x}\\
    &3) \varphi:G\rightarrow G', g\mapsto 1', &&\text{ em que }1'\text{ \'e o elemento neutro de G'.}\\
    &3) \text{ Se }H\leq{G}, \text{ ent\~ao a inclus\~ao} i:H\rightarrow G, h\mapsto h\text{ \'e um morfismo.}\\
    &&3.1) \text{ Em particular, } U:S_{n}\rightarrow GL_{n}, \sigma\mapsto U(\sigma)\\
    &3) \mathbb{Z}\rightarrow G, n\mapsto g^{n}, g\in G\text{ fixo.}
  \end{align*}
 \end{example*}
 \begin{prop*}
   Seja $\varphi:G\rightarrow G'$ um morfismo. Ent\~ao,
  \begin{align*}
    &1)g_{1}\cdots g_{n}\in G, \varphi(g_{1}\cdots g_{n}) = \varphi(g_{1})\cdots\varphi(g_{n}).\\
    &2)\text{ Se 1 \'e o elemento neutro de G e 1' o elemento neutro de G', } \varphi(1)=1'.\\
    &3)\varphi(g^{-1}) = \varphi(g)^{-1}.
  \end{align*}
 \end{prop*}
\begin{proof*}
  1.) Os casos 1 e 2 s\~ao ok. Assim, vamos mostrar por indu\c c\~ao. Suponha que vale para
n-1. Ent\~ao, 
  $$
  \varphi(g_{1}\cdots\varphi_{n})=\varphi((g_{1}\cdots g_{n-1})g_{n}) = \varphi(g_{1}\cdots g_{n-1})\varphi(g_{n}) = \varphi(g_{1})\cdots\varphi(g_{n-1})\varphi(g_{n}).
  $$

  2.) $\varphi(1) = \varphi(1.1)\coloneqq \varphi(1)\varphi(1) \Rightarrow 1' = \varphi(1)\varphi(1)^{-1} = \varphi(1)$

  3.) $1' = \varphi(1) = \varphi(gg^{-1}) = \varphi(g)\varphi(g^{-1}) \Rightarrow \varphi(g)^{-1} = \varphi(g^{-1}).$ \qedsymbol
\end{proof*}
\begin{def*}
  Se $\varphi:G\rightarrow G'$ \'e um morfismo, defina a imagem de $\varphi$ por $Im \varphi\coloneqq\{u\in G': \exists x\in G, \varphi(x)=y\}$ 
  e o n\'ucleo (ou kernel) de $\varphi$ por $\ker{(\varphi)}\coloneqq\{x\in G: \varphi(x) = 1'\} $, em que 1' \'e o elemento neutro de G'.  
\end{def*}
\begin{prop*}
  A imagem de um morfismo $\varphi:G\rightarrow G'$ \'e um subgrupo de G' e o kernel de $\varphi$ \'e um de G.
\end{prop*}
\begin{proof*}
  Se y, y' pertencem a $Im\varphi$, ent\~ao existem x, x' em G tais que $\varphi(x) = y, \varphi(x')=y'.$ Assim, 
  $$
    yy' = \varphi(x)\varphi(x') = \varphi(xx') \Rightarrow yy'\in Im\varphi.
  $$
  Al\'em disso, \'e claro que $\varphi(1) = 1'\in Im\varphi.$ Finalmente, se y pertence a $Im \varphi,$ ent\~ao
 $\varphi(x^{-1}) = \varphi(x)^{-1} = y^{-1} \Rightarrow y^{-1}\in Im \varphi.$ A prova de que $\ker\varphi\leq{G}$ fica como
 exerc\'icio. \qedsymbol
\end{proof*}
\begin{def*}
  Seja $sgn:S_{n}\rightarrow \{+1, -1\}$. Definimos $A_{n} = \ker{(sgn)}$ como o grupo alternado. $\quad\square$
\end{def*}
\begin{def*}
  Se H \'e um subgrupo de G e g um elemento de G, defina a classe lateral \`a esquerda de G em H como
  $$
    gH\coloneqq \{gh: h\in H\}.\quad\square
  $$
\end{def*}
\begin{prop*}
  Seja $\varphi:G\rightarrow G'$ um morfismo e K $= \ker{(\varphi)}.$ Se a, b s\~ao elementos de G, s\~ao equivalentes:
 \begin{align*}
   &1) \varphi(a) = \varphi(b)\\
   &2) a^{-1}b\in K\\
   &3) b\in aH\\
   &4) aK = bK.
 \end{align*}
\end{prop*}
\begin{proof*}
  \begin{align*}
  &1)\Rightarrow2): \varphi(a) = \varphi(b) \Rightarrow \varphi(a^{-1}b) = 1' \Rightarrow a^{-1}b\in K;\\ 
  &2)\Rightarrow1): a^{-1}b\in K \Rightarrow \varphi(a^{-1}b) = 1' \Rightarrow \varphi(a)=\varphi(b);\\
  &1)\Rightarrow3): a^{-1}b\in K \text{ se } \exists h\in K \text{ tais que } a^{-1}h = bh \Rightarrow b\in aK;\\
  &3)\Rightarrow1): \text{ Suponha que }b\in aH, b = ah \Rightarrow \varphi(b) = \varphi(a)\varphi(h) = \varphi(a);\\
  &(1)\Longleftrightarrow(4): \text{Exerc\'icio. \qedsymbol}
  \end{align*}
\end{proof*}
\newpage

\section{Aula 05 - 30/03/2023}
\subsection{Motiva\c c\~oes}
 \begin{itemize}
   \item Subgrupos Normais;
   \item Isomorfismos e Automorfismos;
   \item Parti\c c\~oes e rela\c c\~oes de equival\^encia.
 \end{itemize}
\subsection*{Errata \'Ultima Aula}
Seja $P_{ij}\in \mathbb{M}_{n}(\mathbb{R})$ tal que se $P_{ij}=(a_{kl})$, ent\~ao $a_{kl} = 1$ se k = i, l = j e 0 caso contr\'ario.
Assim, se $\sigma\in S_{n},$ 
  $$
    U(\sigma) = (e_{\sigma(1)}\cdots e_{\sigma(n)}) = \sum\limits_{}^{}P_{\sigma(i), i}, 
  $$
  em que $e_{j}$ \'e o vetor em $\mathbb{R}^{n}$ com 1 na j-\'esima entrada e zero nos demais. De fato, $U(\sigma)$ \'e a matriz
da tansforma\c c\~ao linear $\mathbb{R}^{n}\rightarrow \mathbb{R}^{n}, e_{j}\mapsto e_{\sigma(j)}.$
\subsection{Subgrupos Normais}
  Come\c camos com um corol\'ario \`a \'ultima aula:
 \begin{crl*}
   Uma $\varphi$ \'e injetora se, e somente se, $\ker{(\varphi)} =\{0\}.$
 \end{crl*}
\begin{proof*}
  $ \Rightarrow)$ Seja a um elemento do kernel de $\varphi$. Ent\~ao, 
    $$
      \varphi(a) = 1' = \varphi(1).
    $$
    Mas, como $\varphi$ \'e injetora, segue que a = 1 \'e o \'unico elemento no kernel.

  $ \Leftarrow)$ Suponha que $\varphi$ tem kernel trivial, i.e., $\ker{(\varphi)} =\{0\}.$ Ent\~ao,
    $$
    \varphi(a)\varphi(b)^{-1} = 1' \Rightarrow 1 = \varphi(a)\varphi(b^{-1}) = \varphi(ab^{-1}) \Rightarrow ab^{-1}\in\ker{\varphi} = {1}.
    $$
    Portanto, $ab^{-1} = 1$ e, assim, $a = b.$ \qedsymbol
\end{proof*}
 \begin{def*}
   Se G \'e um gurpo e a, g seus elementos, dizemos que $gag^{-1}\in G$ \'e um conjugado de a com respeito a g. Dois elementos a, b
  de G s\~ao conjugados se existe um g no grupo tal que $a = g b g^{-1}.\square$
 \end{def*}
 \begin{def*}
   Sejam G um grupo e $H\leq{G}$. Dizemos que H \'e um subgrupo normal a G ($H\trianglelefteq G$) se para todos $h\in H$ e $g\in G$,
  $ghg^{-1}\in H$, i.e., H absorve os conjugados de seus elementos. $\square$
 \end{def*}
  Em outras palavras, um subgroup \'e normal se ele \'e fechado pela conjuga\c c\~ao, o que pode ser denotado por $gHg^{-1}\subseteq{H},$ para todo
g de G.
 \begin{prop*}
   Se $H\leq{g},$ s\~ao equivalente
  \begin{align*}
    &i)\quad H\trianglelefteq{G} \\
    &ii)\quad gHg^{-1} = H\\
    &iii)\quad gH = Hg.
  \end{align*}
 \end{prop*}
\begin{proof*}
  (1) $ \Rightarrow$ (2): Obviamente, $gHg^{-1}\leq{H}$ por defini\c c\~ao. Sejam h em H e g em G. Ent\~ao, $ghg^{-1}\in gHg^{-1}\subseteq{H},$
  tal que existe x em H que satisfaz $ghg^{-1} = x \Rightarrow h = \underbrace{(g^{-1})x(g^{-1})^{-1}}_{\in gHg^{-1}}$

  (2) $ \Rightarrow$ (1): Ok.

  (1) $ \Rightarrow$ (3): Se x pertence a gH, x = gh para algum h de H. Por hip\'otese, $gHg^{-1}\subseteq{H}$, de maneira que
 $ghg^{-1} = y\in H$, ou seja, $gh = yg.$ Como x = gh, $x = yg\in Hg$. Portanto, $gH\subseteq{Hg}.$ O outro lado da inclus\~ao fica como exerc\'icio.

  (3) $\Rightarrow$ (1): Se $x\in gHg^{-1}, x = ghg^{-1}, h\in H$, segue da hip\'otese que $gh = h'g$ para algum h' em H. Assim,
  $x = h'\in H.$ \qedsymbol
\end{proof*}
\begin{example*}
 \begin{itemize}
   \item[1)] Se G \'e um grupo, s\~ao subgrupos normais: G, $\{e\}$, Z(g).
   \item[2)] Se G \'e um grupo abeliano, todo subgrupo \'e normal, mas n\~ao vale a volta.
 \end{itemize}
\end{example*}
 \begin{example*}
   Exerc\'icio: Seja $Q = \{\pm1, \pm i, \pm j, \pm k: -1^{2} = 1, i^{2} = j^{2} = k^{2} = -1\}.$ Mostre que Q \'e um grupo
n\~ao abeliano, mas que todo subgrupo \'e normal.
 \end{example*}
\begin{example*}
  $<(12)> = <id, (12)>\not\trianglelefteq{S_{3}}$, visto que 
  $$
    (123)(12)(123)^{-1} = (32)\not\in <(12)>\\
  $$
  Portanto, $<(12)>$ n\~ao \'e um subgrupo normal de $S_{3}.$
\end{example*}
 \begin{prop*}
   Se $\varphi:G\rightarrow G'$ \'e um morfismo, ent\~ao $\ker{\varphi}\trianglelefteq{G}.$
 \end{prop*}
\begin{proof*}
  Sejam g um elemento de G e h um elemento de $\ker{\varphi}$. Ent\~ao, 
    $$
      \varphi(ghg^{-1}) = \varphi(g)\varphi(h)\varphi(g^{-1}) = \varphi(g)1'\varphi(g)^{-1} = \varphi(g)\varphi(g)^{-1} = 1'.
    $$
    Portanto, $ghg^{-1}\in\ker{\varphi}$ e, portanto, $\ker{\varphi}\trianglelefteq{G}.$ \qedsymbol
\end{proof*}
 \begin{example*}
  \begin{itemize}
    \item[1)] $SL_{n}\trianglelefteq{GL_{n}}, \quad SL_{n} = \ker{det}.$
    \item[2)] $A_{n} = \ker{sgn}\trianglelefteq{S_{n}}, \quad sgn:S_{n}\rightarrow\{\pm 1\}, \sigma\mapsto \det{U(\sigma)}.$
  \end{itemize}
 \end{example*}
 Lembre-se que, dado $\sigma\in S_{n}, \sigma = \tau_{1}\cdots\tau_{r}$ s\~ao 2-ciclos, ent\~ao $sgn(\sigma) = (-1)^{r}.$
\begin{def*}
  Sejam G, G' grupos. Um isomorfismo $\varphi$ \'e um morfismo $\sigma:G\rightarrow G'$ bijetor. Se G = G', $\varphi$ \'e chamado automorfismo.
Por fim, se existe um isomorfismo entre dois grupos, dizemos que eles s\~ao isomorfos, escrevendo $G\cong G'$
\end{def*}
\textbf{(Nota ao leitor)} Mas o que h\'a de \'util em isomorfismo? Por que nos importamos? 

  Em \'Algebra linear, estudamos os isomorfismos entre espa\c cos vetoriais, e como eles preservavam algumas propriedades. Essencialmente,
o mesmo ocorrer\'a aqui, ou seja, se h\'a um isomorfismo entre dois grupos, essencialmente estamos estudando o mesmo grupo, mas sob uma \'otica diferente.
Os elementos de um grupo podem ser escritos utilizando os do outro, eles ter\~ao os mesmos tamanhos, a propriedade abeliana ser\'a preservada, etc.
Com isso, caso encontre um grupo aparentemente muito dif\'icil de trabalhar, \'e poss\'ivel simplificar o problema encontrando um outro grupo
isomorfo e que facilitar\'a seu servi\c co. Veremos exemplos a seguir.
 \begin{example*}
  \begin{itemize}
    \item[1)] Todo subgrupo de ordem 2 \'e isomorfo a $S_{2}$.
    \item[2)] H\'a um isomorfismo entre o grupo aditivo dos reais e o multiplicativo positivo dado por $exp:(\mathbb{R}, +)\rightarrow (\mathbb{R}_{>0}, \cdot), x\mapsto e^{x}.$'
    \item[3)] Se g \'e um elemento de G de ordem infinita, ent\~ao $\mathbb{Z}\rightarrow <g>\leq{G}, n\mapsto g^{n}$ \'e um isomorfismo.
    \item[4)] Seja $P\leq{GL_{n}}$ o conjunto das matrizes com somente um 1 em cada linha e cada coluna, tendo entrada 0 nos demais. Ent\~ao,
  $P\leq{GL_{n}}$ e $S_{n}\rightarrow P, \sigma\mapsto U(\sigma)$ \'e isomorfismo.
    \item[5)] $id:G\rightarrow G'$ \'e um isomorfismo.
    \item[6)] Se g pertence a G, $\varphi_{g}:G\rightarrow G, x\mapsto gxg^{-1}$ \'e isomorfismo.
  \end{itemize}
 \end{example*}
 \begin{prop*}
   Se $\varphi$ \'e isomorfismo, ent\~ao $|g| = |\varphi(g)|$. Em part\'icular, $|g| = |aga^{-1}|$ para todo a de G.
 \end{prop*}
\begin{proof*}
    No caso em que $|g| = \infty,$ se $|\varphi(g)| = g < \infty$. Assim, 
      $$
      1' = \varphi(g)^{m} = \varphi(g^{m}) \Rightarrow g^{m}\in\ker{\varphi} = \{1\} \Rightarrow g^{m} = 1.
      $$

    Agora, se $|g| = n <\infty$. Seja $m = |\varphi(g)|$, ent\~ao 
      $$
        1' = \varphi(g)^{m} = \varphi(g^{m}) \Rightarrow g^{m} = 1 \Rightarrow n|m.
      $$ 
    Por outro lado, como $g^{n} = 1,$ 
      $$
        1' = \varphi(g^{n}) = \varphi(g)^{n} \Rightarrow m|n.
      $$
    Portanto, m = n. \qedsymbol
\end{proof*}
\begin{lemma*}
  Se $\varphi:G\rightarrow G'$ \'e um isomorfismo, ent\~ao $\varphi^{-1}:G'\rightarrow G$ tamb\'em \'e isomorfismo. 
\end{lemma*}
\begin{proof*}
  Segue que $\varphi^{-1}$ est\'a bem-definida e \'e uma bije\c c\~ao pois $\varphi$ \'e bije\c c\~ao. Sejam x, y elementos de G'.
Sendo $\varphi$ uma bije\c c\~ao, existem a, b em G tais que 
  $$
  x = \varphi(a)\quad\text{e}\quad y = \varphi(b).
  $$
  Desta forma, $\varphi^{-1}(xy) = \varphi^{-1}(\varphi(a)\varphi(b)) = \varphi^{-1}(\varphi(ab)) = ab = \varphi^{-1}(x)\varphi^{-1}(y).$ \qedsymbol
\end{proof*}
 \begin{def*}
   Dado S um conjunto, uma parti\c c\~ao para S \'e uma cobertura por subconjuntos n\~ao-vazios e disjuntos. Em outras palavras,
existe $U_{j}\subseteq{S}, U_{j}\neq\emptyset$ e $U_{j}\cap U_{i} = \emptyset$ se $i\neq j$ tal que 
  $$
    S = \bigsqcup_{j\in I}U_{j}.
  $$
 \end{def*}
\begin{def*}
  Uma rela\c c\~ao de equival\^encia em um conjunto S \'e um subconjunto R de $S\times S$ tal que 
 \begin{itemize}
   \item[-]$\overbrace{(a, a)}^{a\sim a}\in R, \forall a\in S$ (Reflexiva);
   \item[-]$(a, b)\in R \Rightarrow (b, a)\in R (a\sim b \Rightarrow b\sim a)$ (Sim\'etrica);
   \item[-]$(a, b)(b, c)\in R \Rightarrow (a, c)\in R (a\sim b, b\sim c \Rightarrow a\sim c)$ (Transitiva).
 \end{itemize}
 Denotamos $(a, b)\in R$ por $a~b$, e l\^e-se ``a est\'a relacionado com b''. $\square$
\end{def*}
 \begin{def*}
   Se R \'e uma rela\c c\~ao de equival\^encia em S, denotamos por $[a], a\in S$ o conjunto dos elementos de S que se relacionam com a. Em outras palavras, 
     $$
     [a]\coloneqq\{b\in S: a~b\} = \{b\in S: (a, b)\in R\}
     $$
     e chamamos $[a]$ de classe de equival\^encia de a. $\square$
 \end{def*}
\begin{example*}
 \begin{itemize}
   \item[1)] A ordem em um grupo define uma rela\c c\~ao de equival\^encia;
   \item[2)] A conjuga\c c\~ao define uma rela\c c\~ao de equival\^encia em um grupo G. $(a~b \Longleftrightarrow \exists g\in G: a = gbg^{-1})$
     Neste caso, denotamos $[a] = Cl(a)$ como a classe de conjuga\c c\~ao de a.
 \end{itemize}
\end{example*}
\begin{theorem*}
  Uma parti\c c\~ao em um conjunto S define uma rela\c c\~ao de equival\^encia em S. Reciprocamente, uma rela\c c\~ao de 
equival\^encia define uma parti\c c\~ao em S.
\end{theorem*}
\newpage

\section{Aula 06 - 11/04/2023}
\subsection{Motiva\c c\~oes}
\begin{itemize}
  \item Classes de equival\^encia;
  \item \'Indice de um grupo;
  \item Teorema de Lagrange.
\end{itemize}

\subsection{Classe de Equival\^encia de Parti\c c\~oes}
\begin{theorem*}
  Uma parti\c c\~ao em um conjunto S define uma rela\c c\~ao de equival\^encia em S. Reciprocamente, uma rela\c c\~ao de 
equival\^encia define uma parti\c c\~ao em S.
\end{theorem*}
\begin{proof*}
  Seja $S = \bigsqcup_{i\in I}S$ uma parti\c c\~ao, tal que
    $$
    R = \{(a,b)\in S\times S: \exists i\in I, a ,b\in S_{i}\}
    $$
  \'e uma defini\c c\~ao de equival\^encia em S.

  Reciprocamente, dada uma rela\c c\~ao de equival\^encia em S, temos $[a]$ como a classe de equival\^encia de um elemento a de S.
  Por reflexividade, $a\in[a],$ ou seja, $[a]$ \'e n\~ao-vazio e 
    $$
    S = \bigcup_{a\in S}[a].
    $$
    \textbf{Afirma\c c\~ao:} Se $[a]\cap[b]\neq\emptyset,$ ent\~ao $[a]=[b].$
    Com efeito, seja c um elemento na intersec\c c\~ao das classes [a] e [b]. Segue que $a\sim c$ e $b\sim c$. Por simetria,
    $c\sim b$, logo, por transitividade, $a\sim c\sim b$, ou seja, $c\sim b$. Tome x em [b], ou seja, $b\sim x.$ Novamente,
    por transitividade, $a\sim x,$ isto \'e, $x\in[a].$ Provamos, ent\~ao, que $[a]\subseteq{[b]}$ e $[b]\subseteq{[a]}$, ou seja,
    $[a]=[b].$ \qedsymbol
\end{proof*}
\begin{example*}
  A ordem de um elemento define uma rela\c c\~ao de equival\^encia em um grupo. Particularmente, no caso de $S_{3},$ as classes
de equival\^encia s\~ao 
 \begin{align*}
  &(i)\quad [(12)]  = \{(12), (13), (23)\}\\
  &(ii)\quad [(123)] = \{(123), (132)\}\\
  &(iii)\quad [id] = \{id\}.
 \end{align*} 
 Assim, $S_{3}=[(12)]\bigsqcup{}[(123)]\bigcup{[id]}.$ \qedsymbol
\end{example*}
\begin{def*}
  Se $S = \bigsqcup{S_{i}}_{i\in I}$ \'e uma parti\c c\~ao, denotamos $\overline{S}\coloneqq\{[S_{i}]\} $ o conjunto das classes
de equival\^encia dadas por $S_{i}.$
\end{def*}
\begin{example*}
Se $\mathbb{Z}=\{\text{n\'umeros pares}\}\bigsqcup{\{\text{n\'umeros \'impares}\}}$, ent\~ao $\overline{\mathbb{Z}} =\{[\text{pares}], [\text{\'impares}]\} 
 \{\overline{0}, \overline{1}\}.$
\end{example*}
  Observe que, se S tem uma rela\c c\~ao de equival\^encia, \'e poss\'ivel ``projetarmos'' um elemento de S em sua classe de equival\^encia atrav\'es de 
    $$
    \pi:S\rightarrow \overline{S},\quad a\mapsto[a]=\overline{a}.
    $$
 \begin{example*}
   Fixe um inteiro n. Dados a, b tamb\'em inteiros, dizemos que $a\sim b$ m\'odulo n (ou $a\equiv b\mod n$) se $n|a-b.$ Mostre que
   a congru\^encia mod n \'e de fato uma rela\c c\~ao de equival\^encia em $\mathbb{Z}.$ Al\'em disso, $\overline{\mathbb{Z}}^{n} = \mathbb{Z}/n\mathbb{Z} = \{\overline{0}, \cdots, \overline{n-1}\}.$
 \end{example*}
\begin{def*}
  Se $\varphi:S\rightarrow T$ \'e um mapa entre conjuntos, ent\~ao $\varphi$ define uma rela\c c\~ao de equival\^encia em S dada por 
    $$
      a\sim b\quad \Longleftrightarrow \varphi(a) = \varphi(b).
    $$
    Al\'em disso, [a] \'e definida como a fibra por $\varphi$ de $\varphi(a) = t,$ ou seja, 
      $$
      \varphi^{-1}(t) = \{s\in S: \varphi(s) = t\}.
      $$
      Em outras palavras, se $\varphi(a) = t$, ent\~ao $\overline{a} = [a] = \varphi^{-1}(t).$ Em part\'icular, 
        $$
          S = \bigsqcup_{t\in Im\varphi}{\varphi^{-1}(t)}.
        $$
\end{def*}

\subsection{Classes Laterias e \'Indices.}
\begin{prop*}
  Sejam $\varphi:G\rightarrow G'$ morfismos de grupos e K = $\ker{\varphi}$. Ent\~ao, a fibr de $\varphi$ que cont\'em o elemento a
a de G \'e a classe lateral $aK$. Al\'em disso, essas classes particionam G e correspondem aos elementos da imagem de $\varphi.$
\end{prop*}
\begin{proof*}
  Segue que, dado b em aK, $\varphi(a) = \varphi(b) = t$. Disto segue que a, b pertencem \`a pr\'e-imagem de $\varphi, \varphi^{-1}(t).$
\end{proof*}
\begin{prop*}
  As classes laterais \`a esquerda de H em G s\~ao as classes de equival\^encia da seguinte rela\c c\~ao 
    $$
      a\cong{b} \Longleftrightarrow b = ah, \quad h\in H.
    $$
  Denotamos $a\cong{b}$ por $a\equiv b.$
\end{prop*}
\begin{proof*}
  Observe que a = 1a, $1\in H.$ Logo, $a\cong{a}.$ Se $a\cong{b},$ existe h em H tal que b = ah, ou seja, $a = bh^{-1}, h^{-1}\in H$ 
de modo que $b\cong{a}.$ 

  Se $a\cong{b}, b\cong{c},$ existem $h_{1}, h_{2}\in H$ tais que $b = ah_{1}, c = bh_{2},$ de forma que $c = ah_{1}h_{2},$ i.e.,
 $a\cong{c}.$ 

  Por fim, 
  \begin{align*}
    [a] &= \{b\in G: a\cong{b}\}\\
        &= \{b\in G: \exists h\in H\text{ tal que }b = ah\}\\
        &= \{ah: h \in H\} = aH.\text{ \qedsymbol}
  \end{align*}
\end{proof*}
\begin{crl*}
  G = $\bigsqcup_{a\in G\text{distintos}}^{}{aH}.$
\end{crl*}
\begin{example*}
  Em $S_{3} = <(12), (123)>$, seja $H = <(12)>.$ Temos 
 \begin{align*}
   &(i)id H = H = \{id, (12)\} = (12)H\\
   &(ii)(123)H = \{(123), (123)(12)\} = (123)(12)H\\
   &(iii)(123)^{2}H = \{(123)^{2}, (123)^{2}(12)\} = (123)^{2}(12)H.
 \end{align*}
 De fato, $S_{3} = H\bigsqcup_{}^{}{(123)H}\bigsqcup_{}^{}{(123)^{2}H}$ \qedsymbol
\end{example*}
\begin{def*}
  O n\'umero de classes laterais (\`a esquerda) de H em G \'e chamado o \'indice de H em G, denotado por $[G:H].\square$
\end{def*}
\begin{example*}
  No exemplo anterior, $[S_{3} : H]=3, |H| = 2.$
\end{example*}
\begin{lemma*}
  Todas as classes laterais aH de H em G t\^em mesma ordem.
\end{lemma*}
\begin{proof*}
  O mapa $m_{a}:H\rightarrow H_{a}, h\mapsto ah$ \'e um mapa bijetor com inversa $m_{a^{-1}}.$ \qedsymbol
\end{proof*}
  Este lema ser\'a usado para demonstrar um resultado extremamente importante em Teoria de Grupos. Essencialmente falando,
ele afirma que a ordem de um grupo \'e um m\'ultiplo da ordem dos seus subgrupos. Mas por que isso importa? Al\'em de ser um
resultado intrigante por si s\'o, ele pode ser usado para mostrar que grupos n\~ao s\~ao isom\'orficos, quantidade de elementos de 
uma dada ordem dentro de um grupo, entre outras coisas. Segue seu enunciado.
\begin{theorem*}
  Se G \'e um grupo finito e H um subgrupo de G, ent\~ao 
    $$
      |G| = |H|[G:H].
    $$ 
  Em particular, $|H|$ divide $|G|.$
\end{theorem*}
\begin{proof*}
  Da proposi\c c\~ao anterior, sabemos que $G = \bigsqcup_{a\in{G}\text{diferentes}}^{}{aH}.$ Portanto,
    $$
    |G| = \sum\limits_{a\in{G}\text{diferentes}}^{}|aH| = |H|[G:H]\text{ \qedsymbol}
    $$
\end{proof*}
\begin{crl*}
  Se a pertence a G, ent\~ao $|a|\biggl|\biggr.|G|$
\end{crl*}
\begin{proof*}
  Segue do Teorema de Lagrange que  
    $$
    |a|=|<a>|\biggl|_{}^{}\biggr.|G|\text{ \qedsymbol}
    $$
\end{proof*}
\begin{crl*}
  Se $|G|=p$ n\'umero primo e a \'e um elemento de G diferente do elemento neutro, ent\~ao 
    $$
      G = <a>.
    $$
\end{crl*}
\begin{proof*}
  Se |a| divide p, ent\~ao $|a|=1$ ou p, mas, como $a\neq 1, |a| = p,$ segue que
    $$
    G = <a>\text{ \qedsymbol}
    $$
\end{proof*}
\begin{crl*}
  Se $\varphi:G\rightarrow G'$ \'e um morfismo de grupos finitos, segue que 
 \begin{align*}
   &1)\quad |G| = |\ker{\varphi}||im\varphi|\\
   &2)\quad |\ker{\varphi}| \biggl|_{}^{}\biggr. |G|\\
   &3)\quad |im\varphi|\text{ divide } |G|\text{ e }|G'|.
 \end{align*}
\end{crl*}
\begin{proof*}
  $1\Rightarrow)$ Temos 
 \begin{align*}
   G &= \bigsqcup_{}^{}{a\ker{\varphi}}\\
     & \Rightarrow |G| = |im\varphi||\ker{\varphi}|\\
     & \Rightarrow |G| = |\ker\varphi|[G:\ker{\varphi}] = |\ker{\varphi}||im\varphi|.
 \end{align*}
  Os outros itens ficam como exerc\'icio para os estudantes. \qedsymbol
\end{proof*}
\begin{example*}
  Considere $sgn:S_{n}\rightarrow \{\pm 1\}.$ Temos 
    $$
      |im(sgn)| = 2,
    $$
    de modo que 
    $$
    |A_{n}| = |\ker{sgn}| = \frac{|S_{n}|}{2} = \frac{n!}{2}.
    $$
\end{example*}
\begin{prop*}
  Se $K\leq{H}\leq{G},$ ent\~ao 
    $$
    [G:K] = [G:H][H:K] 
    $$ 
\end{prop*}
\begin{proof*}
  Suponha que $[G:H] = n$ e $[H:K]=m$. Ent\~ao, 
    $$
    G = g_{1}H\bigsqcup_{}^{}{\cdots}\bigsqcup_{}^{}{g_{n}H} = \bigsqcup_{i=1}^{n}{g_{i}H}\quad\text{e}\quad H = \bigsqcup_{j=1}^{m}{h_{j}K}.
    $$
  A multliplica\c c\~ao por $g_{i}$ \'e uma bije\c c\~ao 
    $$
      m_{g_{i}}:h_{j}K\rightarrow g_{i}h_{j}K,\quad x\mapsto g_{i}x.
    $$
  Assim, $g_{i}H = \bigsqcup_{j=1}^{m}{g_{i}h_{j}K}.$ Portanto, 
    $$
    G = \bigsqcup_{i=1}^{n}{\biggl[\bigsqcup_{j=1}^{m}{g_{i}g_{j}K}\biggr]},
    $$
    de onde conclu\'i-se que $[G:K]=mn.$ \qedsymbol
\end{proof*}
\newpage

\section{Aula 07 - 13/04/2023}
\subsection{Motiva\c c\~oes}
\begin{itemize}
  \item Relacionando morfismos com as estruturas de subgrupos;
  \item Teorema da Correspond\^encia.
\end{itemize}

\subsection{Mais Sobre Morfismos}
\begin{prop*}
  Se $\varphi:G\rightarrow G'$ \'e um morfismo de grupos finitos e $H\leq{G}$ \'e tal que  $\gcd{(|H|, |G'|)} = 1$, ent\~ao $\ker{\varphi}\supseteq{H}.$
\end{prop*}
\begin{proof*}
  Tome $\varphi_{H}:H\rightarrow G'$. Pela propriedade da aula passada, 
    $$
    |im{\varphi_{H}}|\biggl|_{}^{}\biggr. |H|\text{ e } |G'|.
    $$
    Por hip\'otese, $|im(\varphi_{H})|=1$, i.e., $im \varphi_{H} = \{1'\}$. Assim, $\varphi(H)\subseteq{\{1'\}}.$
    Portanto, $H\subseteq{\ker{\varphi}}.$ \qedsymbol
\end{proof*}
\begin{example*}
  Se $H\leq{S_{n}}, |H| = 2k+1,$ ent\~ao $H\subseteq{A_{n}}.$ Considere $sgn:S_{n}\rightarrow \{\pm1\}$. Pela proposi\c c\~ao,
  $H\subseteq{\ker{sgn}}=A_{n}.$
\end{example*}
\begin{prop*}
  Seja $\varphi:G\rightarrow G'$ um morfismo de grupos, $K=\ker{\varphi}, H'\leq{G'}$ e $H = \varphi^{-1}(H').$ Ent\~ao, 
 $H\leq{G}$ e $K\leq{H}.$ Al\'em disso, se $H'\trianglelefteq{G'}$, ent\~ao $H\trianglelefteq{G}$. Por fim, se $\varphi$ for
 sobrejetora e $H\trianglelefteq{G}$, temos $\varphi(H) = H'\trianglelefteq{G'}.$
\end{prop*}
\begin{proof*}
  Vamos come\c car mostrando que H \'e um subgrupo de G. Com efeito, como H' \'e subgrupo de G', ent\~ao 1' \'e um de seus elementos.
Assim, $\varphi^{-1}(1')\subseteq{H}$ e, em particular, 1 pertence a H e $K\leq{H}.$ Agora, sejam x, y elementos de H. Ent\~ao,
 $\varphi(x), \varphi(y)\in H'$, ou seja, $\varphi(x)\varphi(y)^{-1}\in H'$. Assim, $\varphi(xy^{-1})\in H'$ e $xy^{-1}\in H.$
Portanto, H \'e subgrupo de G.

  Daremos continuidade provando a segunda parte do resultado. Suponha que $H'\trianglelefteq{G'}$ e sejam $x\in gHg^{-1}, x = ghg^{-1}$ 
para algum h em H. 
  $$
    \varphi(x) = \varphi(ghg^{-1}) = \varphi(g)\varphi(h)\varphi(g)^{-1}\in H' \Rightarrow ghg^{-1}\in H.
  $$
Portanto, $H\trianglelefteq{G}.$

  Finalmente, mostremos que $\varphi(H)\trianglelefteq{G'}.$ Tome g' em G e $y\in\gamma(H).$ Como $\varphi$ \'e sobrejetora, existe
x em H e g em G tal que $\varphi(g)=g', \varphi(x) = y.$ Portanto,
  $$
  g'y(g')^{-1} = \varphi(g)\varphi(x)\varphi(g)^{-1} = \varphi(gxg^{-1})\in \varphi(H).\text{ \qedsymbol}
  $$
\end{proof*}
\begin{example*}
  Segue que $GL_{n}(\mathbb{R})^{+}\trianglelefteq{GL_{n}(\mathbb{R})}$. De fato, note que, se definirmos
    $$
      \det:GL_{n}(\mathbb{R})\rightarrow \mathbb{R}^{\times},
    $$
    ent\~ao $GL_{n}(\mathbb{R})^{+} = \det^{-1}(\mathbb{R}_{>0})\trianglelefteq{GL_{n}(\mathbb{R})}$. \qedsymbol
\end{example*}
\subsection{Teorema da Correspond\^encia}
  O resultado a seguir \'e conhecido como Teorema da Correspond\^encia.
\begin{theorem*}
  Seja $\varphi:G\rightarrow G'$ sobrejetora e $K=\ker{\varphi}.$ Ent\~ao,  
  \begin{align*}
    &\{H\leq{G}: K \subseteq{H}\} \longleftrightarrow \{N: N\leq{G'}\}\\
    &\quad\quad\quad\quad H\mapsto \varphi(H)\\
    &\quad\quad\quad\quad \varphi^{-1}(N)\mapsfrom N.
  \end{align*}
  Al\'em diso, se $H\leftrightarrow N$', ent\~ao $H\trianglelefteq{G} \Longleftrightarrow N\trianglelefteq{G'}$ e
  $|H| = |N|\cdot|K|.$
\end{theorem*}
\begin{proof*}
  Vamos mostrar as seguintes coisas - $H = \varphi^{-1}\varphi(H) N = \varphi \varphi^{-1}(N), |H|=|N||K|.$ 

  Nesta ordem, come\c camos observando que $H\leq{\varphi^{-1}\varphi(H)}$ \'e sempre verdadeira. Seja x em $\varphi^{-1}\varphi(H).$
  Ent\~ao, $\varphi(x)\in\varphi(H),$ isto \'e, existe h em H tal que $\varphi(x) = \varphi(h)$. Disto, temos 
    $$
      \varphi(xh^{-1}) = 1' \Longleftrightarrow xh^{-1}\in K\subseteq{H} \Rightarrow x\in H,
    $$
  concluindo o que desej\'avamos mostrar.

  Para a segunda parte, fica como exerc\'icio.

  Por fim, considere $\varphi_{|_{H}}:H\rightarrow \gamma(H) = N$. Pela aula anterior, portanto,
    $$
    |H| = |\ker{\varphi_{|_{H}}}||Im(\varphi_{|_{H}})| = |K||N|.\text{ \qedsymbol}
    $$
\end{proof*}
\begin{def*}
  Dados G e G' grupos, defina o quociente de G por G' como 
    $$
    G\times{G'} = \{(g, g'): g\in G, g'\in G'\}\quad\square
    $$
\end{def*}
  Afirmamos que $G\times{G'}$ com a opera\c c\~ao $(g_{1}, g_{1}')(g_{2}, g_{2}') = (g_{1}g_{1}', g_{2}g_{2}')$ \'e um grupo.
Al\'em disso, temos os seguintes morfismos:
 \begin{align*}
   & \pi:G\times{G'}\rightarrow G,\quad \pi':G\times{G'}\rightarrow G'\\
   &  i:G\rightarrow G\times{G'},\quad i':G'\rightarrow G\times{G'},
 \end{align*}
 Sendo eles chamados, respectivamente, de proje\c c\~oes e inje\c c\~oes. Ademais, $(1, 1')$ \'e o elemento neutro de $G\times{G'}.$
 \begin{prop*}
   Se $\gcd{(r, s)} = 1$, ent\~ao o grupo c\'iclico de ordem rs \'e o produto $C_{r}\times{C_{s}}$, sendo $C_{r}$ o grupo
 c\'iclico de ordem r (E $C_{s}$ o c\'iclico de ordem s).
 \end{prop*}
\begin{proof*}
  Seja $C_{rs}$ o grupo c\'iclico de ordem rs. Se $C_{r}=<x>, C_{s} = <y>.$ Ent\~ao, $C_{r}\times{C_{s}} = <(x, y)>.$ De fato, 
    $$
    (x, y)^{rs} = (x^{rs}, y^{rs}) = ((x^{r})^{s}, (y^{s})^{r}) = (1, 1'),
    $$
  tal que $k = ord(xy)\biggl|_{}^{}\biggr.rs.$ Como $(r, s) = 1,$ existem $a, b\in \mathbb{Z}$ tais que 
   \begin{align*}
     &1 = ar + bs\\
     &k = ark + bsk.
   \end{align*}
  Observe que $(x, y)^{k} = (x^{k}, y^{k}) = (1, 1')$, o que implica que $r \biggl|_{}^{}\biggr.k$ e $s \biggl|_{}^{}\biggr.k$, ou seja,
  $k = rr'$ e $k = ss'.$ Substituindo isso na segunda f\'ormula, temos $k = rsar' + rsbs' = rs(\text{outros termos})$. Portanto,
  $rs \biggl|_{}^{}\biggr.k.$ \qedsymbol
\end{proof*}
\begin{example*}
  Segue que $C_{2}\times C_{2}\neq C_{4}$ (Verifique!).
\end{example*}
\begin{prop*}
  Sejam $H, K \subseteq{G}$ e $f:H\times{K}\rightarrow G, (h,k)\mapsto hk$ com $im F = HK\coloneqq\{hk: h\in H, k\in K\}.$ Ent\~ao,
 \begin{itemize}
   \item[a)] f \'e injetora se, e somente se, $H\cap{K} = \{1\};$
   \item[b)] f \'e morfismo se, e somente se, $hk=kh$ para todo $h\in H, k\in K;$
   \item[c)] Se $H\trianglelefteq{G},$ ent\~ao $HK\leq{G};$
   \item[d)] f \'e isomorfismo se, e somente se, $H\cap{K}=\{1\}, HK = G\text{ e }H, K\trianglelefteq{G}.$
\end{itemize}
\end{prop*}
\begin{proof*}
  a) $(\Rightarrow)$ Se $H\cap{K}\neq\{1\}$, ent\~ao existe x em $H\cap{K}, x\neq 1$, ou seja, 
    $$
      f(xx^{-1}) = xx^{-1} = 1 f(1.1),
    $$
    o que implica que f n\~ao \'e injetora, uma contradi\c c\~ao.

    $(\Leftarrow)$ Se $f(h_{1},k_{1})=h_{1}k_{1}=h_{2}k_{2} = f(h_{2}k_{2}) \Longleftrightarrow h_{2}^{-1}h_{1} = k_{2}k_{1}^{-1}\in H\cap K$. Logo,
    $h_{2}^{-1}h_{1} = 1$ e $k_{2}k_{1}^{-1} = 1,$ ou seja, f \'e injetora.

  b) $(\Leftarrow)$ Segue que $f((h_{1},k_{1})(h_{2},k_{2})) = f(h_{1}h_{2}, k_{1}k_{2}) = h_{1}h_{2}k_{1}k_{2} = h_{1}k_{1}h_{2}k_{2} =
f(h_{1}, k_{1})f(h_{2},k_{2}).$

    $(\Rightarrow)$ Se f \'e um morfismo, ent\~ao $h_{1}h_{2}k_{1}k_{2} = h_{1}k_{1}h_{2}k_{2}$ para todos $h_{1},h_{2}\in H, k_{1},k_{2}\in K.$
  Em part\'icular, para $h_{1}=k_{2}=1,$ temos $h_{2}k_{1} = k_{1}h_{2}$ para todos $h_{1}\in H, k_{2}\in K.$

  c) Sejam $h_{1}, h_{2}\in H$ e $k_{1}, k_{2}\in K$. Ent\~ao, 
    $$
    (h_{1}k_{1})(h_{2}k_{2})^{-1} = h_{1}k_{1}k_{2}^{-1}h_{2}^{-1} = h_{1}h_{2}'h_{1}k_{1}^{-1}\in HK
    $$

    d) $(\Leftarrow)$ Por $H\cap K = \{1\}$ e $HK = G,$ f \'e bijetora. Assim, basta mostrar que f \'e morfismo. 
    Sejam $h_{2}, h_{1}\in H$ e $k_{1}, k_{2}\in K$. Segue que 
      $$
        f((h_{1},k_{1})(h_{2}, k_{2})) = f((h_{1}h_{2}, k_{1}k_{2})) = h_{1}h_{2}k_{1}k_{2}k = h_{1}k_{1}h_{2}k_{2} = f(h_{1}, k_{1})f(h_{2}, k_{2}).
      $$
      Sejam, tamb\'em, $h\in H, k\in K$ 
        $$
          hkh^{-1}k^{-1} = hk(kh)^{-1}.
        $$
        Como $H\trianglelefteq{G},$ segue que 
          $$
            hh'kk^{-1} = hh'\in H
          $$
        Como $K\trianglelefteq{G},$ temos tamb\'em 
          $$
            hkh^{-1}k^{-1} = hh^{-1}k'k^{-1}\in K.
          $$
          Portanto, $hkh^{-1}k^{-1}\in H\cap K=\{1\}$ \qedsymbol
\end{proof*}
\begin{prop*}
  Os grupos de ordem 4 s\~ao $C_{4}$ ou $C_{2}\times C_{2}.$
\end{prop*}
\begin{proof*}
  Seja G um grupo de ordem 4 dado por $G=\{1, g_{1}, g_{2}, g_{3}\}$. Pelo teorema de Lagrange, $ord(g_{i})\biggl|_{}^{}\biggr.4$,
ou seja, $ord g_{i} = 2$ ou 4. Se existe $g_{i}\in G$ tal que $ord(g_{i}) = 4,$ ent\~ao $G\approx C_{4}.$

Caso contr\'ario, todo $g_{i}$ \'e tal que $|g_{i}|=2.$ Assim, defina 
  $$
    f:<g_{1}>\times<g_{2}>\rightarrow G,\quad (x,y)\mapsto xy.
  $$
  Pelo item d da proposi\c c\~ao anterior, f \'e isomorfismo. \qedsymbol
\end{proof*}
\newpage

\section{Aula 08 - 18/04/2023}
\end{document}





