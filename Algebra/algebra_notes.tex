 \documentclass{article}
 \usepackage{amsmath}
 \usepackage{amsthm}
 \usepackage{amssymb}
 \usepackage{pgfplots}
 \usepackage{amsfonts}
 \usepackage[margin=2.5cm]{geometry}
 \usepackage{graphicx}
 \usepackage[export]{adjustbox}
 \usepackage{fancyhdr}
 \usepackage[portuguese]{babel}
 \usepackage{hyperref}
 \usepackage{lastpage}
 \usepackage{mathtools}

 \pagestyle{fancy}
 \fancyhf{}

 \pgfplotsset{compat = 1.18}

 \hypersetup{
     colorlinks,
     citecolor=black,
     filecolor=black,
     linkcolor=black,
     urlcolor=black
 }
 \newtheorem*{def*}{\underline{Defini\c c\~ao}}
 \newtheorem*{prop*}{\underline{Proposi\c c\~ao}}
 \newtheorem*{theorem*}{\underline{Teorema}}
 \newtheorem*{crl*}{\underline{Corol\'ario}}
 \newtheorem{example*}{\underline{Exemplo}}
 \newtheorem*{proof*}{\underline{Prova}}
 \newtheorem*{lemma*}{\underline{Lema}}
 \renewcommand\qedsymbol{$\blacksquare$}
 \newcommand{\Lin}[1]{Lin_{\mathbb{K}}({#1})}

 \rfoot{P\'agina \thepage \hspace{1pt} de \pageref{LastPage}}

 \title{Notas de }
 \author{Renan Wenzel}
 \date{\today}

 \begin{document}
 \begin{figure}[ht]
	\minipage{0.76\textwidth}
		\includegraphics[width=4cm]{../icmc.png}
		\hspace{7cm}
		\includegraphics[height=4.9cm,width=4cm]{../brasao_usp_cor.jpg}
	\endminipage	
\end{figure}

\begin{center}
	\vspace{1cm}
	\LARGE
	UNIVERSIDADE DE S\~AO PAULO

	\vspace{1.3cm}
	\LARGE
	INSTITUTO DE CI\^ENCIAS MATEM\'ATICAS E COMPUTACIONAIS - ICMC

	\vspace{1.7cm}
	\Large
	\textbf{Notas de Aula de \'Algebra}

	\vspace{1.3cm}
	\large
	\textbf{Renan Wenzel - 11169472}

	\vspace{1.3cm}
	\large
	\textbf{Roberto Carlos - alvarago@icmc.usp.br}

	\vspace{1.3cm}
	\today
\end{center}

 \newpage

 \tableofcontents

 \newpage

\section{Aula 01 - 14/03/2023}
\subsection{Motiva\c c\~oes}
\begin{itemize}
  \item Compreender o que ser\'a estudado ao longo do curso;
\end{itemize}
\subsection{Introdu\c c\~ao ao Curso}
  Este curso \'e sobre teoria de grupos, a qual possui origem no estudo de simetrias, sejam elas de figuras ou de objetos alg\'ebricos.
Um exemplo de grupo seria o seguinte: 

  Considere um tri\^angulo equil\'atero. Existem algumas formas de olharmos para as simetrias
do tri\^angulo, como rotacionando-o, refletindo-o com rela\c c\~ao a um ponto m\'edio e um v\'ertice fixo. Contabilizando todas as poss\'iveis
formas delas acontecerem, h\'a seis simetrias deste ret\^angulo. Ademais, compondo simetrias resulta em outra, i.e., rotacionar e refletir
um certo v\'ertice continuar\'a sendo uma simetria do tri\^angulo. Al\'em disto, \'e um fato (futuramente visto) que essas seis simetrias 
totalizam todas as poss\'iveis simetrias de um tri\^angulo equil\'atero. De fato, dado um pol\'igono regular de n lados, ele possui
n! simetrias.

\subsection{Grupos e Opera\c c\~oes}
 \begin{def*}
   Seja S um conjunto n\~ao-vazio. Uma opera\c c\~ao em S \'e um mapa
   \begin{align*}
     \mu:&S \times S\rightarrow S\\
         &(a, b)\mapsto{\mu(a, b)}
   \end{align*}
 \end{def*}
\begin{example*}
  A opera\c c\~ao soma em $\mathbb{Z}, +:\mathbb{Z}\times{\mathbb{Z}}\rightarrow \mathbb{Z}, (a, b)\mapsto a + b$ \'e uma opera\c c\~ao.
\end{example*}
\begin{example*}
  Uma opera\c c\~ao em $\mathbb{R}$ \'e a multiplica\c c\~ao $.:\mathbb{R}\times{\mathbb{R}}\rightarrow \mathbb{R}, (a, b)\mapsto ab.$
\end{example*}
\begin{example*}
  Um exemplo do que n\~ao \'e opera\c c\~ao seria a subtra\c c\~ao dos naturais, $-:\mathbb{N}\times{\mathbb{N}}\rightarrow \mathbb{N}, (a, b)\mapsto a - b.$ (Consegue responder por que n\~ao \'e?)
\end{example*}
\begin{example*}
  Se S \'e o conjunto de simetrias de um tri\^angulo equil\'atero, ent\~ao a composi\c c\~ao
  \begin{align*}
    \circ: &S\times{S}\rightarrow S \\
           &(\sigma, \tau)\mapsto \sigma\circ \tau
  \end{align*}
  \'e uma opera\c c\~ao bin\'aria.
\end{example*}
  Faremos a conven\c c\~ao de denotar $\mu(a, b)$ por $a.b$ ou $a + b$, com base no contexto.
 \begin{def*}
   Uma opera\c c\~ao $\mu$ em S n\~ao-vazio, denotada pelo produto, \'e dita associativa se, para todos a, b, c em S,
   $$
      (a.b).c = a.(b.c), \quad \biggl(\mu(a, \mu(b, c)) = \mu(\mu(a, b), c)\biggr).
   $$
   Por outro lado, ser\'a dita comutativa se 
   $$
      a.b = b.a, \quad \biggl(\mu(a, b) = \mu(b, a)\biggr).
   $$
   Diremos, tamb\'em, que ela tem elemento neutro (ou identidade) se existe um elemento e em S tal que 
   $$
    a.e = e.a = a,\forall a\in{S}.
   $$
   Neste caso, diremos que e \'e o elemento neutro, ou a identidade, para $\mu$.
 \end{def*}
  Utilizaremos a nota\c c\~ao 1 para a identidade no caso em que $\mu$ \'e denotada por um produto e 0 pro caso em que \'e denotada por adi\c c\~ao.

 \begin{example*}
   A multiplica\c c\~ao de matrizes \'e associativa, n\~ao \'e comutativa e possui identidade.
 \end{example*}
\begin{example*}
  A soma de n\'umeros inteiros \'e associativa, comutativa e possui identidade.
\end{example*}
\begin{example*}
  A pot\^encia nos n\'umeros reais \'e n\~ao associativa, nem comutativa, mas possui identidade: $a^{(b^{c})} \neq (a^{b})^{c} = a^{bc}$
\end{example*}

\begin{prop*}
  Seja S um conjunto n\~ao-vazio e $\mu$ uma opera\c c\~ao em S denotada pelo produto. Ent\~ao, existe um \'unico jeito de definir o
produto (denotado temporariamente por $[a_{1}, \cdots, a_{n}]$) de n elementos em S tal que 
\begin{align*}
  &(i)\quad [a_{1}] = a_{1}; \\
  &(ii)\quad [a_{1}, a_{2}] = \mu(a_{1}, a_{2}) = a_{1}a_{2}; \\
  &(iii)\quad\forall 1\leq{i}<n, [a_{1}, \cdots, a_{n}] = [a_{1}, \cdots, a_{i}][a_{i+1}, \cdots, a_{n}].
\end{align*}
\end{prop*}
\begin{proof*}
  (iii)$\Rightarrow$ Para o caso $n\leq{2}$ \'e ok. Agora, suponha o produto bem-definido de r elementos em S, $r\leq{n=1}$. Ent\~ao,
  defina $[a_{1}, \cdots, a_{n}]\coloneqq [a_{1}, \cdots, a_{n-1}][a_{n}]. $ Como a defini\c c\~ao acima satisfaz a condi\c c\~ao (iii) para
i=n-1, se ela estiver bem-definida, ela ser\'a \'unica. Com efeito, seja $1\leq{i}<n-1$, tal que 
  \begin{align*}
    [a_{1}, \cdots, a_{n}] &= [a_{1}, \cdots, a_{n-1}][a_{n}] = [a_{1}, \cdots, a_{i}][a_{i+1}, \cdots, a_{n-1}][a_{n}]\\
                           &= \biggl([a_{1}, \cdots, a_{i}]\biggr)\biggl([a_{i+1}, \cdots, a_{n-1}][a_{n}]\biggr) \\
                           &= [a_{1}, \cdots, a_{i}][a_{i+1}, \cdots, a_{n}].\text{ \qedsymbol}
  \end{align*}
\end{proof*}
\begin{def*}
  Seja S n\~ao-vazio e $\mu$ uma opera\c c\~ao em S com identidade 1. Um elemento a de S \'e dito invers\'ivel se existe b em S tal que
 $ab = ba = 1.$ Neste caso, b \'e o inverso de a, denotado por $b\coloneqq a^{-1}$.
\end{def*}
  Note que tanto o elemento inverso quanto o elemento neutro, se existirem, s\~ao \'unicos (c.f. Lema abaixo). Al\'em disso, o inverso da adi\c c\~ao \'e
denotad por -a.
\begin{lemma*}
  Seja S n\~ao-vazio, $\mu$ uma opera\c c\~ao associativa denotada pelo produto. Ent\~ao, 
 \begin{itemize}
   \item[i)] Existe no m\'aximo um elemento neutro para S e $\mu$;
   \item[ii)] Se o elemento neutro existe, ent\~ao para cada elemento de S, existe no m\'aximo um inverso;
   \item[iii)] Se um elemento a de S tem inverso \`a esquerda l e \`a direita r, i.e. $l.a = 1 \text{ e } a.r = 1$, ent\~ao
a \'e invers\'ivel com inverso l = r.
   \item[iv)] Se a, b em S s\~ao invers\'iveis, ent\~ao o produto ab \'e invers\'ivel, com inverso $b^{-1}a^{-1}.$
 \end{itemize}
\end{lemma*}
  Antes de provar, observe que a exist\^encia de um elemento inverso \`a esquerda ou \`a direita n\~ao garante que um elemento
seja invers\'ivel (exerc\'icio), eles devem coincidir.
\begin{proof*}
 $(i)\Rightarrow)$ Suponha que existem 1, 1' em S como seus elementos neutros. Basta mostramos que eles coincidem. Com efeito,
 $$
    1 = 1.1' = 1'.1 = 1'.
 $$
 Portanto, o elemento neutro \'e \'unico.

 $(ii)\Rightarrow)$ Assuma a exist\^encia de dois elementos inversos em S para um elemento a, denotados por b, b'. Ent\~ao, como
ab = ba = 1, temos
  $$
    b = b1 = b(ab') = (ba)b' = 1b' = b'.
  $$
  Portanto, o elemento inverso \'e \'unico. Os itens (iii) e (iv) s\~ao exerc\'icios. \qedsymbol
\end{proof*}
  
 \begin{def*}
   Um monoide \'e um par (G, $\mu$), em que G \'e um conjunto n\~ao-vazio e $\mu$ uma opera\c c\~ao associativa e com elemento neutro em G.
  Se, ainda por cima, $\mu$ for comutativa, (G, $\mu$) \' e um monoide abeliano (ou comutativo).
 \end{def*}
 \begin{def*}
   Um grupo \'e um par (G, $\mu$) \'e um monoide (G, $\mu$) com a condi\c c\~ao extra que todo elemento de G possui inverso. Caso
 $\mu$ seja comutativa, chamamos G de grupo abeliano.
 \end{def*}
\begin{example*}
  Os inteiros com a soma, $(\mathbb{Z}, +)$, \'e um grupo comutativo, enquanto $(\mathbb{Z}, .)$ n\~ao \'e um grupo, mas sim um monoide.
\end{example*}
\begin{example*}
  O grupo das matrizes com entradas reais e sua multiplica\c c\~ao, $(\mathbb{M}_{n}(\mathbb{R}), .)$, \'e um grupo n\~ao-abeliano.
\end{example*}

\section{Aula 02 - 16/03/2023}
\subsection{Motiva\c c\~oes}
\begin{itemize}
  \item Outras estruturas alg\'ebricas e exemplos;
  \item Tamanho de um grupo;
  \item Subgrupos
\end{itemize}
\subsection{Usos de Grupos}
  Podemos usar grupos para definir outras constru\c c\~oes alg\'ebricas, como segue.
 \begin{def*}
   Um anel \'e uma terna $(A, \mu, \phi)$, em que $(A, \mu)$ \'e um grupo abeliano e $(A, \phi)$ \'e um monoide. Al\'em disso,
vale a distributiva.
   $$
    \phi(a, \mu(c, d)) = \phi(\mu(a, c), \mu(a, d)), \quad (a(b + c) = ab + ac).
   $$
   Usualmente, escrevemos $(A, \mu, \phi) = (A, +, \cdot).\square$
 \end{def*}
\begin{def*}
  Um corpo \'e um anel $(A, +, \cdot)$ tal que $(A-\{0\}, .)$ \'e um grupo abeliano. $\square$
\end{def*}
  Deste ponto em diante, abandonaremos as letras gregas para usar apenas os s\'imbolos ``+'' ou ``.'' para um grupo com adi\c c\~ao
ou com multiplica\c c\~ao. Vejamos alguns exemplos.
\begin{example*}
  \begin{center}
  \begin{tabular}{||c c c c c||}
  \hline
  Conjunto & Monoide & Monoide Comutativo & Grupo & Grupo Comutativo \\ [1ex] 
 \hline\hline
 $(GL_{n}, \cdot)$ & Sim & N\~ao & Sim & N\~ao \\ 
 \hline
 $(SL_{n}, \cdot)$ & Sim & N\~ao & Sim & N\~ao \\
 \hline
 $(\mathbb{Z}, +)$ & Sim & Sim & Sim & Sim \\
 \hline
 $(\mathbb{Z}, \cdot)$ & Sim & Sim & N\~ao & N\~ao \\
 \hline
 $(\mathbb{Q}, +)$ & Sim & Sim & Sim & Sim \\ [1ex] 
 \hline 
 $(\mathbb{Q}, \cdot)$ & Sim & Sim & N\~ao & N\~ao \\ [1ex] 
 \hline 
 $(S = \{z\in \mathbb{C}: |z| = 1\}, \cdot)$ & Sim & Sim & Sim & Sim \\ [1ex] 
 \hline 
 $(\mathbb{M}_{n}(\mathbb{R}), +)$ & Sim & Sim & Sim & Sim \\ [1ex] 
 \hline 
 $(\mathbb{M}_{n}(\mathbb{R}), .)$ & Sim & N\~ao & N\~ao & N\~ao \\ [1ex] 
 \end{tabular}
\qedsymbol
\end{center}
\end{example*}
\begin{example*}
  Seja T um conjunto qualquer e 
  $$
    G = \{f:T\rightarrow T: f \text{ bijetora.}\}
  $$
  Ent\~ao, $(G, \circ)$ \'e um grupo, chamado grupo das permuta\c c\~oes ou simetrias de T. Se T \'e um conjunto finito, e.g.
  $T = \{1, \cdots, n\}, $ ent\~ao denotamos $(G, \circ)$ por $(S_{n}, \circ).$ \qedsymbol
\end{example*}
\begin{def*}
  A ordem de um grupo $(G, \cdot)$ \'e a cardinalidade de G: $|G|$. Caso $|G| < \infty,$ dizemos que $(G, +)$ \'e um grupo finito. $\square$
\end{def*}
\begin{example*}
  A ordem de $|\mathbb{Z}| = \infty$ e $|S_{n}| = n!$ \qedsymbol
\end{example*}
\begin{prop*}
  Se $(G, \cdot)$ \'e um grupo e a, b, c s\~ao elementos de G tais que ab = ac ou ba = ca, ent\~ao b = c. Al\'em disso, se ab = a ou
ba = a, ent\~ao b = 1.
\end{prop*}
\begin{proof*}
  Seja $a^{-1}$ o inverso de a, ent\~ao $a^{-1}(ab) = a^{-1}(ac).$ Mais ainda, se ab = a, ent\~ao $b = (a^{-1}a)b = a^{-1}(ab) = a^{-1}a = 1.$
\end{proof*}
  Fica de exerc\'icio mostrar que s\'o existe um grupo de ordem 2 e que $S_{3}$ \'e um grupo n\~ao-comutativo.
\subsection{Subgrupos}
 \begin{def*}
   Um subgrupo H de um grupo $(G, \cdot)$ \'e um subconjunto H de contido em G tal que
  \begin{align*}
    &1) 1\in H; \\
    &2) a, b\in H\Rightarrow ab \in H; \\
    &3) a\in H\Rightarrow a^{-1}\in H.
  \end{align*}
  Denotaremos subrupos por $H \leq{G}$ ou $(H, \cdot) \leq{(G, \cdot)}. \square$
 \end{def*}
\begin{prop*}
 Com opera\c c\~ao induzida pela multiplica\c c\~ao de G restrita a H, $(H, \cdot)$ \'e um grupo. 
\end{prop*}
\begin{proof*}
  Como H est\'a contido em G, podemos restringir o produto de G a H para 
  $$
    ._H:H \times H\rightarrow H.
  $$
  Afirmamos que $(H, \cdot)$ \'e um grupo. Com efeito, a restri\c c\~ao de . a H est\'a bem-definida pelo segundo item da defini\c c\~ao
de subgrupo. Mais ainda, ela \'e associativa em H por ser em G e todo elemento em H tem inverso pela condi\c c\~ao 3. Por fim,
ela tem elemento neutro pela primeira requisi\c c\~ao ao definir subgrupo. Portanto, $(H, \cdot)$ \'e um grupo.
\end{proof*}
Observe que todo grupo tem ao menos dois subgrupos, chamados triviais, sendo eles $\{1\}$ e ele mesmo. Qualquer outro leva o nome de
subgrupo pr\'oprio.
\begin{example*}
  $(SL_{n}, \cdot) \leq{(GL_{n}, \cdot)}$ e $(\mathbb{Z}, +) \leq{(\mathbb{Q}, +)}.$ \qedsymbol
\end{example*}
\begin{example*}
  Seja n um inteiro, ent\~ao $n\mathbb{Z}\coloneqq \{nk: k\in \mathbb{Z}\}$ \'e um subgrupo dos inteiros. De fato, 0 = n0 pertence
  a $n\mathbb{Z}.$ Al\'em disso, se nk e nk' pertencem a $n\mathbb{Z},$ ent\~ao
  $$
    nk + nk' = n(k + k')\in n \mathbb{Z}.
  $$
  Por fim, se nk pertence a $ n\mathbb{Z}$, ent\~ao n(-k) tamb\'em pertence a $n \mathbb{Z}$ e nk + n(-k) = 0. \qedsymbol
\end{example*}
\begin{prop*}
  Todo subgrupo de $(\mathbb{Z}, +)$ \'e da forma $n \mathbb{Z}$ para algum n inteiro.
\end{prop*}
\begin{proof*}
  Caso n seja 1, $n \mathbb{Z} = \mathbb{Z}$ e, se n = 0, ent\~ao $n \mathbb{Z} = \{0\}. $ Agora, seja H um subgrupo pr\'oprio dos
inteiros e n o menor inteiro positivo em H. Afirmamos que $n \mathbb{Z} = H$. 

  De fato, $n \mathbb{Z} \leq{H}$, pois n \'e um elemento de H, ent\~ao $nk = \underbrace{n + \ldots + n}_{\text{k-vezes}} \in H$. Al\'em disso, $-n\in H,$ de forma
que $-nk = \underbrace{(-n) + \ldots + (-n)}_{\text{k-vezes}}\in H.$ Portanto, $n \mathbb{Z}\in H.$

  Por outro lado, seja m um inteiro de H e considere $m = nq + r, 0 \leq{r} < n, q\in \mathbb{Z}.$ Pelo algoritmo de divis\~ao de Euclides,
  $$
  m - nq = r \Rightarrow r\in H\Rightarrow r = 0.
  $$
  e, assim, m = nq pertence a $n\mathbb{Z}$. Portanto, $H = n \mathbb{Z}$. \qedsymbol
\end{proof*}
\begin{prop*}
 \begin{itemize}
   \item[1)] $n \mathbb{Z} + m \mathbb{Z}$ \'e subgrupo de $\mathbb{Z};$
   \item[2)] $n \mathbb{Z} + m \mathbb{Z} = d \mathbb{Z}$, em que d \'e tal que 
    \item[2.1)] $d | n$ e $d | m$;
      \item[2.2)] Se $l | n$ e $l | m$, ent\~ao $l | d;$
        \item[2.3)] Existem r, s inteiros tais que $rn + sm = d.$
 \end{itemize}
 Definimos $d = \gcd{(n, m)}$ como o m\'aximo divisor comum de m e n.
\end{prop*}
\begin{proof*}
  A prova do item 1 fica como exerc\'icio. 

  $2.1)\Rightarrow n \mathbb{Z} + m \mathbb{Z} = d \mathbb{Z}.$ Em particular, se n, m pertencem a $d \mathbb{Z}$, ent\~ao
$d | n \text{ e } d | m.$

  $2.2)\Rightarrow$ Suponha que $n = lq_{1}, m = lq_{2} $. Se x pertence a $n \mathbb{Z} + m \mathbb{Z},$ ent\~ao
  \begin{align*}
    &x = nk_{1} + mk_{2} = lq_{1}k_{1} + lk_{2}q_{2} \in l \mathbb{Z} \\
    &\Rightarrow n \mathbb{Z} + m \mathbb{Z} \subseteq{l \mathbb{Z}} \Rightarrow l | d.
  \end{align*}
  tamb\'em \'e poss\'ivel mostrar isso usando o item 3 da proposi\c c\~ao.

  $2.3)\Rightarrow$ imediato. \qedsymbol
\end{proof*}
\begin{prop*}
 \begin{itemize}
   \item[1)] $m \mathbb{Z}\cap n \mathbb{Z}$ \'e subgrupo dos inteiros;
   \item[2)] $m \mathbb{Z}\cap n \mathbb{Z} = l \mathbb{Z}$, em que l \'e tal que 
     \item[2.1)] $m | l, n | l;$
       \item[2.2)] Se $m | l'$ e $n | l'$, ent\~ao $l | l'.$
 \end{itemize}
 Definimos $l = mmc(m, n)$ o m\'inimo m\'ultiplo comum de m e n.
\end{prop*}
\begin{proof*}
  Fica como exerc\'icio.
\end{proof*}
\begin{crl*}
  Se m, n s\~ao inteiros, ent\~ao $mn = mmc(m, n)\gcd{(m, n)}$.
\end{crl*}
\newpage

\section{Aula 03 - 21/03/2023}
\subsection{Motiva\c c\~oes}
\begin{itemize}
  \item Outros exemplos de subgrupos;
  \item Subgrupos gerado por subconjuntos;
  \item Grupo c\'iclico e ordem de elementos.
\end{itemize}
\subsection{Subgrupos - Outras Propriedades}
  Quando o conjunto candidato a subgrupo \'e n\~ao-vazio, n\~ao \'e necessario exigir que a identidade seja parte dele. De fato,
\begin{prop*}
  Se G \'e um grupo e $H\subseteq{G}, H \neq\emptyset$, ent\~ao $H\leq{G}$ se, e somente se,
 \begin{align*}
   &1) ab\in H, \quad a, b\in H\\
   &2) a^{-1}\in H, \quad a\in H.
 \end{align*}
\end{prop*}
\begin{proof*}
  $\Rightarrow$ Segue da defini\c c\~ao de subgrupo (ab e $a^{-1}$ pertencem a H por defini\c c\~ao);

  $\Leftarrow$ Sendo H n\~ao-vazio, existe $a\in H$. Atrav\'es de (2), $a^{-1}\in H$ e, por (1), $aa^{-1} = 1\in H.$ Portanto,
H \'e subgrupo de G. \qedsymbol
\end{proof*}
  Outra formula\c c\~ao de subgrupo requer apenas uma condi\c c\~ao:
 \begin{prop*}
   Se G \'e um grupo e $H\subseteq{G}, H \neq\emptyset$, ent\~ao $H\leq{G}$ se, e s\'o se, $ab^{-1}\in H$ para todos $a, b\in H.$
 \end{prop*}
\begin{proof*}
  $\Rightarrow$ Suponha que H \'e um subgrupo de G e sejam $a, b\in H$. Ent\~ao, por defini\c c\~ao, $a^{-1}, b^{-1}\in H$. Assim,
segue da defini\c c\~ao de subgrupo que $ab^{-1}\in H$.

  $\Leftarrow$ Se $H \neq\emptyset$, existe ao menos um a em H. Por hip\'otese, $1 = aa^{-1}\in H$. Assim, $a^{-1} = 1a^{-1}\in H$.
Por fim, se a, b s\~ao membros de H, ent\~ao $b^{-1}\in H$, tal que $ab = a(b^{-1})^{-1}\in H$. Portanto, H \'e subgrupo de G. \qedsymbol
\end{proof*}

 \begin{def*}
   Se G \'e um grupo, ent\~ao $Z(G) = \{g\in G: ga = ag\forall a\in{G}\} $ \'e um subgrupo de G chamado centro de G.
 \end{def*}
  Provemos que Z(G) \'e de fato um subgrupo. De fato, $1\in Z(G)$ pela defini\c c\~ao de elemento neutro. Al\'em disso, se $g, h\in Z(G),$
ent\~ao gha = gah = agh, tal que $gh\in Z(G).$ Al\'em disso, $g^{-1}\in Z(G),$ pois $g^{-1}a = (a^{-1}g)^{-1} = (ga^{-1})^{-1} = ag^{-1}.$
  Uma propriedade interessante \'e que G ser\'a um grupo abeliano se, e somente se, $Z(G) = G.$ 
 \begin{example*}
 Exerc\'icio: Dados $G_{1}, G_{2}$ grupo, defina o grupo produto como $G_{1}\times G_{2} = \{(g_{1}, g_{2}): g_{i}\in G_{i}\}$. Encontre
uma opera\c c\~ao que torne este conjunto um grupo de fato.
 \end{example*}
 \begin{example*}
  \begin{itemize}
    \item[1)]  Se V \'e um subespa\c co vetorial de um corpo qualquer $\mathbb{K}$, ent\~ao $V<\leq{\mathbb{K}}$.  
    \item[2)] O conjunto 
      $$
      SU_{2}(\mathbb{C}) = \biggl\{ \begin{bmatrix}
          \alpha & -\overline{\beta} \\
          \beta & \overline{\alpha}
    \end{bmatrix}: \alpha, \beta \in \mathbb{C}, |\alpha|^{2} + |\beta|^{2} = 1\biggr\} 
    $$
    \'e um subgrupo de $GL_{2}(\mathbb{C})$.
    \item[3)] O conjunto
      $$
      SO_{2}(\mathbb{R}) = \biggl\{\begin{bmatrix}
          \cos{(\theta)} & \sin{(\theta)}\\
          \sin{(\theta)} & \cos{(\theta)}
      \end{bmatrix}: \theta\in \mathbb{R}\biggr\} \leq{GL_{2}(\mathbb{R})}
      $$
  \end{itemize}
 \end{example*}
\begin{prop*}
  Se G \'e um grupo abeliano, ent\~ao todo subgrupo de G \'e tamb\'em abeliano
\end{prop*}
\begin{proof*}
  Se $H\leq{G}, a, b\in H$, em particular a, b tamb\'em pertencem a G, tal que $ab = ba.$ \qedsymbol
\end{proof*}
  Observe que a rec\'iproca e falsa. Com efeito, o subgrupo 
  $$
  \biggl\{\begin{bmatrix}
      1 & a \\
      0 & 1
    \end{bmatrix}\in GL_{2}(\mathbb{R}): a\in \mathbb{R}\biggr\}
  $$
  \'e subgrupo abeliano de $GL_{2}(\mathbb{R}).$ Al\'em disso, a rec\'iproca n\~ao vale nem mesmo se todo subgrupo pr\'oprio de um
grupo for abeliano, visto que todo subgrupo de $S_{3}$ \'e abeliano, mas o pr\'oprio $S_{3}$ n\~ao \'e.

 \begin{def*}
   Seja G um grupo e $S\subseteq{G}$ um subconjunto n\~ao-vazio. Definimos o conjunto gerado por S como 
   $$
   <S>\coloneqq \biggl\{a_{1}\cdots a_{n}: a_{i}\in S\text{ ou }a_{i}^{-1}\in{S}\biggr\}, \quad n\in \mathbb{N}\square.
   $$
 \end{def*}
\begin{prop*}
  $<S>$ \'e um subgrupo de G.
\end{prop*}
\begin{proof*}
  \'E claro que $<S> \neq\emptyset$. Agora, se $a_{1}\cdots a_{n} (= x), b_{1}\cdots b_{m}(= y)\in <S>$, ent\~ao
  $$
  xy^{-1} = a_{1} \cdots a_{n}(b_{1}\cdots b_{m})^{-1} = a_{1}\cdots a_{n}b_{m}^{-1}\cdots b_{1}^{-1}\in <S>.
  $$
  Portanto, pela segunda defini\c c\~ao equivalente de subgrupo, $<S>\leq{G}.$ \qedsymbol
\end{proof*}
\begin{def*}
  Nas condi\c c\~oes da propsi\c c\~ao, $<S>$ \'e o subgrupo gerado por S. Caso S seja finito, digamos $S=\{g_{1},\cdots, g_{n}\}$, denotamos
  $<S>$ por $<g_{1}, \cdots, g_{n}>.\square$
\end{def*}
\begin{def*}
  Sejam G um grupo e g um elemento seu. Se $G=<g>$, diremos que G \'e um grupo c\'iclico. $\square$
\end{def*}
\begin{def*}
  Se G \'e um grupo e g seu elemento, definimos a ordem de g (nota\c c\~ao: $|g|$ ou ord(g)) como a ordem de $<g>$.
\end{def*}
\begin{example*}
  $\mathbb{Z} = (1)$ \'e um grupo c\'iclico infinito, $S_{2}$ \'e um grupo c\'iclico finito e $S_{3}$ n\~ao \'e c\'iclico. \qedsymbol
\end{example*}
Atente-se ao fato de que $<g>\coloneqq\{\cdots, g^{-2}, g^{-1}, g^{0}=1, g, g^{2}, \cdots\} = \{g^{\mathbb{Z}}\} $
\begin{example*}
  Exerc\'icio: Calcule as ordens dos elementos de $S_{2}, S_{3}.$
\end{example*}
  Note que todo subgrupo de $\mathbb{Z}$ \'e c\'iclico. Al\'em disso, se $|G|<\infty,$ segue que $|g|<\infty$. Em particular,
$|g|\leq{|G|}$. Vale mencionar tamb\'em que mesmo se o grupo tem ordem infinita, o grupo c\'iclico pode ter ordem finita. De fato,
se $(G, \cdot) = (\mathbb{R}^{\times}, \cdot)$, tome g = 1. Ent\~ao, $<g> = \{-1, 1\}$, que \'e finito de ordem 2.
 \begin{prop*}
   Sejam G um grupo e g um elemento seu. Denotemos por S o conjunto dos inteiros n tais que $g^{n} = 1.$ Ent\~ao,
  \begin{itemize}
    \item[i)]$S\leq{\mathbb{Z}}$;
    \item[ii)]As pot\^encias $g^{m}, g^{n}, m\geq{n}$ s\~ao iguais se, e somente se, $g^{m-n} = 1(i.e. m-n\in S);$
    \item[iii)] Se $S\neq0 \mathbb{Z},$ ent\~ao $S=n\mathbb{Z}$ e as pot\^encias $1, g, g^{2}, \cdots, g^{n-1}$ s\~ao distintas e 
  s\~ao todos os elementos em $<g>.$ Em particular, $|g|=n.$
  \end{itemize}
 \end{prop*}
\begin{proof*}
  $(i)\Rightarrow$ Se m, n pertence a S, ent\~ao $g^{m-n} = g^{m}(g^{n})^{-1} = 1$, logo m-n pertence a S. \'E claro que S \'e n\~ao-vazio, pois 0 sempre \'e um elemento seu.

  $(ii)\Rightarrow$ \'E a lei do cancelamento.

  $(iii)\Rightarrow$ Se $S = \{0\}$, \'e autom\'atico. Como $S\leq{\mathbb{Z}}$, pela classifica\c c\~ao dos subgrupos de $\mathbb{Z},$
existe n em $\mathbb{Z}$ tal que $S = n \mathbb{Z}$. Agora, seja k um inteiro qualquer. Segue da divis\~ao Euclidiana que
$k = nq + r, 0\leq{r}<n$. Assim, $g^{k} = g^{nq}g^{r} = 1g^{r},$ tal que $<g> \subseteq{\{g^{0}=1, \cdots, g^{n-1}\}}$. Finalmente, pelo
item (ii) e da minimalidade de n. \qedsymbol
\end{proof*}
\begin{crl*}
  $<g> = \{1, g, g^{2}, \cdots, g^{n-1}\} $ 
\end{crl*}
\begin{crl*}
  Se a ordem de g \'e diferente de zero, ent\~ao ela \'e o menor inteiro positivo n tal que $g^{n} = 1$.
\end{crl*}
\begin{crl*}
  Se a ordem de g \'e $n>{0}$, ent\~ao $g^{k} = 1,$ se, e somente se, $n|k.$
\end{crl*}
\begin{crl*}
  Se a ordem de g \'e $n>0, k\in \mathbb{Z}$, ent\~ao $|g^{k}|=\displaystyle \frac{n}{mdc(n, k)}$.
\end{crl*}
\newpage

\section{Aula 04 - 23/03/2023}
\begin{itemize}
  \item Ciclos e Grupo de Permuta\c c\~oes
  \item Morfismo de Grupos
  \item Classes laterais
\end{itemize}
\subsection{Ciclos e Grupos de Permuta\c c\~ao}
Introduzimos a seguir o grupo das permuta\c c\~oes, denotado $S_{n}.$
\begin{def*}
  Uma permuta\c c\~ao $\sigma\in S_{n}$ \'e um r-ciclo se existem $a_{1},\cdots, a_{r}\in\{1,\cdots, n\}$ tais que $\sigma(a_{1})=a_{2},
  \sigma(a_{2})=a_{3}, \cdots, \sigma(a_{r-1})=a_{r}, \sigma(a_{r}) = a_{1}$ e, al\'em disso, $\sigma(j) = j$ para todo j em $\{1,\cdots, n\}/\{a_{1},\cdots,a_{r}\}$. 
  Dizemos que r \'e o comprimento de r, e denotamos $\sigma$ por $\sigma=(a_{1}\cdots a_{r}).\square$
\end{def*}
\begin{def*}
  Um 2-ciclo \'e chamado transposi\c c\~ao. $\square$
\end{def*}
 \begin{example*}
   Seja $\sigma\in S_{5}.$ Um 5-ciclo \'e , por exemplo, $\sigma(1)=2, \sigma(2)=3, \sigma(3)=4, \sigma(4)=5, \sigma(5)=1,$ ou $\sigma=(12345)=(34512).$
Um 3-ciclo seria $\sigma(1)=4, \sigma(2)=2, \sigma(3)=1, \sigma(4)=3, \sigma(5)=5$ e uma transposi\c c\~ao seria $\sigma(1)=2, \sigma(2)=1, \sigma(3)=3, \sigma(4)=4, \sigma(5)=5.$
 \end{example*}
\begin{def*}
  Duas permuta\c c\~oes $\sigma, \tau\in S_{n}$ s\~ao disjuntas se para todo $j\in\{1, \cdots, n\}, \sigma(j) = j$ ou, exclusivamente, $\tau(j) = j.$
\end{def*}
\begin{example*}
  $\tau\in S_{5}, \tau=(34), \sigma(12) \Rightarrow \tau, \sigma$ s\~ao disjuntas.
\end{example*}
  Observe que nem toda permuta\c c\~ao \'e um r-ciclo. De fato, $\sigma\in S_{5}$ dada por $\sigma(1)=3, \sigma(2) = 4, \sigma(3) = 5, \sigma(4) = 2$
e $\sigma(5)=1$ n\~ao \'e um r-ciclo. O fato \'e que toda permuta\c c\~ao \'e o produto de ciclos disjuntos de comprimento maior ou igual a 2.
Assim, $\sigma = (135)(24)$ descreve a permuta\c c\~ao enviando 1 pra 3, 2 pra 4, 3 pra 5, 4 pra 2 e 5 pra 1. 
\begin{example*}
  Seja $\sigma\in S_{5}, \sigma=(12)(13)(15) = (1532)$. Note que l\^e-se o produto de permuta\c c\~oes como a composi\c c\~ao de fun\c c\~oes, isto \'e,
come\c ca-se pela direita e termina na esquerda (afinal, \'e a composi\c c\~ao de permuta\c c\~oes, que s\~ao, particularmente, fun\c c\~oes!). Deste produt\'orio, vimos que o n\'umero 1 \'e o \'unico que ser\'a alterado, i.e., uma permuta\c c\~ao ap\'os o 1 demarca o fim da a\c c\~ao. Assim, este
exemplo indica que 1 se torna 5 e permanece assim (a primeira a\c c\~ao torna 1 no elemento 5). 5 se torna 1, depois 3 e permanece assim ($1->5->3->3$), 2 se torna 1
no final ($2->2->2->1$), 3 se torna eventualmente 2 ($3->2->1->2$) e 4 permanece constante. (P.S. Se essa parte ficar confusa, me chamem no celular pra eu explicar melhor).
\end{example*}
\begin{prop*}
  Toda permuta\c c\~ao em $S_{n}$ \'e um produto de transposi\c c\~oes (2-ciclos). Isto \'e, $S_{n}=\langle\text{transposi\c c\~oes}\rangle$. 
Al\'em disso, se $\sigma\in S_{n}, \sigma=\tau_{1}\cdots\tau_{r} = \rho_{1}\cdots\rho_{s}$ fatora\c c\~oes em transposi\c c\~oes, ent\~ao $2|r-s.$
\end{prop*}
\begin{proof*}
  Observe que $Id = (12)(21)\in\langle\text{ transposi\c c\~oes }\rangle$. Se $\sigma\in S_{n}$ \'e uma permuta\c c\~ao qualquer, ent\~ao $\sigma$ \'e 
o produto de ciclos. Logo, basta verificar a proposi\c c\~ao para um r-ciclo $\sigma.$ Suponha, assim, que $\sigma = (a_{1}\cdots a_{r}$ \'e um r-ciclo.
Com isso, $r=(a_{1}a_{2})(a_{1}a_{3})\cdots(a_{1}a_{r}).$ \qedsymbol
\end{proof*}
\begin{prop*}
  Exerc\'icio: Mostre que qualquer fatora\c c\~ao de um r-ciclo em transposi\c c\~oes tem mesma paridade.
\end{prop*}
\begin{def*}
  Seja $\sigma\in S_{n}$. Ent\~ao, a matriz de permuta\c c\~ao $\sigma$ \'e
  $$
    U(\sigma)\coloneqq \begin{pmatrix}
      e_{\sigma(1)}\\
      \vdots\\
      e_{\sigma(n)}
    \end{pmatrix}
  $$
  em que $e_{i}$ \'e o i-\'esimo vetor can\^onico de $\mathbb{R}^{n}.\square$
\end{def*}
\begin{example*}
  Seja $\sigma = (135)(24)\in S_{5}.$ Para esta permuta\c c\~ao, a matriz \'e
  $$
    U(\sigma) =
    \begin{pmatrix}
      e_{3}\\
      e_{4}\\
      e_{5}\\
      e_{2}\\
      e_{1}
    \end{pmatrix} =
    \begin{pmatrix}
      0 & 0 & 1 & 0 & 0\\
      0 & 0 & 0 & 1 & 0\\
      0 & 0 & 0 & 0 & 1\\
      0 & 1 & 0 & 0 & 0\\
      1 & 0 & 0 & 0 & 0
    \end{pmatrix}
  $$
  Note que 
  $$
    U(\sigma) = \begin{pmatrix}
      1\\
      2\\
      3\\
      4\\
      5
    \end{pmatrix} =
    \begin{pmatrix}
      5\\
      4\\
      3\\
      2\\
      1
    \end{pmatrix}
  $$
\end{example*}
 \begin{prop*}
   Sejam $\sigma, \tau\in S_{n}$ e $U(\sigma), U(\tau)$ as matrizes associadas respectivas. Ent\~ao,
  \begin{itemize}
    \item[1)] $U(\sigma)(12 \cdots n)^{T} = a_{1}e_{1} + \cdots + a_{n}e_{n} \Longleftrightarrow \sigma(j) = a_{j},\quad j = 1, \cdots, n.$
    \item[2)] $U(\sigma)$ sempre tem um \'unico 1 em cada linha e em cada coluna. Reciprocamente, toda matriz desse tipo
    \'e uma matriz de alguma permuta\c c\~ao.
    \item[3)] $\det{U(\sigma)}\in\{-1, 1\}.$
    \item[4)] A matriz de permuta\c c\~ao de $\tau\sigma$ \'e $U(\tau)\cdot U(\sigma)$.
  \end{itemize}
 \end{prop*}
\begin{proof*}
  Exerc\'icio.
\end{proof*}
\begin{def*}
  Se $\sigma\in S_{n}$ e $U(\sigma)$ \'e a matriz associada, definimos o sinal de $\sigma (sgn(\sigma))$ como sendo o 
$\det{(U(\sigma)}.$Al\'em disso, diremos que $\sigma$ \'e uma permuta\c c\~ao par quando $sgn(\sigma) = 1$ e \'impar quando
 $sgn(\sigma) = -1.\quad\square$ 
\end{def*}
  Observe que \'e poss\'ivel demonstrar que $sgn(\sigma) = (-1)^{r},$ em que r \'e o n\'umero de transposi\c c\~oes que aparecem
na decomposi\c c\~ao de $\sigma.$

\subsection{Morfismos de Grupos}
  Morfismos de grupos funcionam como fun\c c\~oes entre conjuntos, mas que levam em conta a opera\c c\~ao existente nos grupos.
 \begin{def*}
   Sejam G, G' dois grupos. Um morfismo de grupos \'e um mapa $\phi:G\rightarrow G'$ tal que $\phi(gh)=\phi(g)\phi(h)$ para todo
  g, h em G. $\quad\square$
 \end{def*}
 \begin{example*}
   S\~ao morfismos:
  \begin{align*}
    &1) sgn:S_{n}\rightarrow \{+1, -1\}, \sigma\mapsto sgn(\sigma),&&sgn(\sigma\tau) = \det(U(\sigma)U(\tau)) = \det(U(\sigma))\det(U(\tau)) = sgn(\sigma)sgn(\tau)\\
    &2) \det:GL_{n}\rightarrow \mathbb{R}^{\times}, A\mapsto\det(A)\\
    &3) \exp:(\mathbb{R}, +)\rightarrow (\mathbb{R}, \cdot), x\mapsto e^{x}\\
    &3) \phi:G\rightarrow G', g\mapsto 1', &&\text{ em que }1'\text{ \'e o elemento neutro de G'.}\\
    &3) \text{ Se }H\leq{G}, \text{ ent\~ao a inclus\~ao} i:H\rightarrow G, h\mapsto h\text{ \'e um morfismo.}\\
    &&3.1) \text{ Em particular, } U:S_{n}\rightarrow GL_{n}, \sigma\mapsto U(\sigma)\\
    &3) \mathbb{Z}\rightarrow G, n\mapsto g^{n}, g\in G\text{ fixo.}
  \end{align*}
 \end{example*}
 \begin{prop*}
   Seja $\phi:G\rightarrow G'$ um morfismo. Ent\~ao,
  \begin{align*}
    &1)g_{1}\cdots g_{n}\in G, \phi(g_{1}\cdots g_{n}) = \phi(g_{1})\cdots\phi(g_{n}).\\
    &2)\text{ Se 1 \'e o elemento neutro de G e 1' o elemento neutro de G', } \phi(1)=1'.\\
    &3)\phi(g^{-1}) = \phi(g)^{-1}.
  \end{align*}
 \end{prop*}
\begin{proof*}
  1.) Os casos 1 e 2 s\~ao ok. Assim, vamos mostrar por indu\c c\~ao. Suponha que vale para
n-1. Ent\~ao, 
  $$
  \phi(g_{1}\cdots\phi_{n})=\phi((g_{1}\cdots g_{n-1})g_{n}) = \phi(g_{1}\cdots g_{n-1})\phi(g_{n}) = \phi(g_{1})\cdots\phi(g_{n-1})\phi(g_{n}).
  $$

  2.) $\phi(1) = \phi(1.1)\coloneqq \phi(1)\phi(1) \Rightarrow 1' = \phi(1)\phi(1)^{-1} = \phi(1)$

  3.) $1' = \phi(1) = \phi(gg^{-1}) = \phi(g)\phi(g^{-1}) \Rightarrow \phi(g)^{-1} = \phi(g^{-1}).$ \qedsymbol
\end{proof*}
\begin{def*}
  Se $\phi:G\rightarrow G'$ \'e um morfismo, defina a imagem de $\phi$ por $Im \phi\coloneqq\{u\in G': \exists x\in G, \phi(x)=y\}$ 
  e o n\'ucleo (ou kernel) de $\phi$ por $\ker{(\phi)}\coloneqq\{x\in G: \phi(x) = 1'\} $, em que 1' \'e o elemento neutro de G'.  
\end{def*}
\begin{prop*}
  A imagem de um morfismo $\phi:G\rightarrow G'$ \'e um subgrupo de G' e o kernel de $\phi$ \'e um de G.
\end{prop*}
\begin{proof*}
  Se y, y' pertencem a $Im\phi$, ent\~ao existem x, x' em G tais que $\phi(x) = y, \phi(x')=y'.$ Assim, 
  $$
    yy' = \phi(x)\phi(x') = \phi(xx') \Rightarrow yy'\in Im\phi.
  $$
  Al\'em disso, \'e claro que $\phi(1) = 1'\in Im\phi.$ Finalmente, se y pertence a $Im \phi,$ ent\~ao
 $\phi(x^{-1}) = \phi(x)^{-1} = y^{-1} \Rightarrow y^{-1}\in Im \phi.$ A prova de que $\ker\phi\leq{G}$ fica como
 exerc\'icio. \qedsymbol
\end{proof*}
\begin{def*}
  Seja $sgn:S_{n}\rightarrow \{+1, -1\}$. Definimos $A_{n} = \ker{(sgn)}$ como o grupo alternado. $\quad\square$
\end{def*}
\begin{def*}
  Se H \'e um subgrupo de G e g um elemento de G, defina a classe lateral \`a esquerda de G em H como
  $$
    gH\coloneqq \{gh: h\in H\}.\quad\square
  $$
\end{def*}
\begin{prop*}
  Seja $\phi:G\rightarrow G'$ um morfismo e K $= \ker{(\phi)}.$ Se a, b s\~ao elementos de G, s\~ao equivalentes:
 \begin{align*}
   &1) \phi(a) = \phi(b)\\
   &2) a^{-1}b\in K\\
   &3) b\in aH\\
   &4) aK = bK.
 \end{align*}
\end{prop*}
\begin{proof*}
  \begin{align*}
  &1)\Rightarrow2): \phi(a) = \phi(b) \Rightarrow \phi(a^{-1}b) = 1' \Rightarrow a^{-1}b\in K;\\ 
  &2)\Rightarrow1): a^{-1}b\in K \Rightarrow \phi(a^{-1}b) = 1' \Rightarrow \phi(a)=\phi(b);\\
  &1)\Rightarrow3): a^{-1}b\in K \text{ se } \exists h\in K \text{ tais que } a^{-1}h = bh \Rightarrow b\in aK;\\
  &3)\Rightarrow1): \text{ Suponha que }b\in aH, b = ah \Rightarrow \phi(b) = \phi(a)\phi(h) = \phi(a);\\
  &(1)\Longleftrightarrow(4): \text{Exerc\'icio. \qedsymbol}
  \end{align*}
\end{proof*}
\end{document}

