\documentclass{article}
\usepackage{amsmath}
\usepackage{multirow}
\usepackage{amsthm}
\usepackage{amssymb}
\usepackage{pgfplots}
\usepackage{amsfonts}
\usepackage[margin=2.5cm]{geometry}
\usepackage{graphicx}
\usepackage[export]{adjustbox}
\usepackage{fancyhdr}
\usepackage[portuguese]{babel}
\usepackage{hyperref}
\usepackage{lastpage}
\usepackage{mathtools}
\usepackage{tikz}
\usepackage{tikz-3dplot}
\usetikzlibrary{angles,quotes}

\pagestyle{fancy}
\fancyhf{}

\pgfplotsset{compat = 1.18}

\hypersetup{
    colorlinks,
    citecolor=black,
    filecolor=black,
    linkcolor=black,
    urlcolor=black
}
\newtheorem*{def*}{\underline{Defini\c c\~ao}}
\newtheorem*{theorem*}{\underline{Teorema}}
\newtheorem*{lemma*}{\underline{Lema}}
\newtheorem*{prop*}{\underline{Proposi\c c\~ao}}
\newtheorem{example}{\underline{Exemplo}}
\newtheorem*{proof*}{\underline{Prova}}
\renewcommand\qedsymbol{$\blacksquare$}
\newcommand{\Lin}[1]{Lin_{\mathbb{K}}({#1})}

\rfoot{P\'agina \thepage \hspace{1pt} de \pageref{LastPage}}

\begin{document}
\begin{figure}[ht]
\minipage{0.76\textwidth}
  \includegraphics[width=4cm]{../icmc.png}
  \hspace{7cm}
  \includegraphics[height=4.9cm,width=4cm]{../brasao_usp_cor.jpg}
\endminipage  
\end{figure}

\begin{center}
\vspace{1cm}
\LARGE
UNIVERSIDADE DE S\~AO PAULO

\vspace{1.3cm}
\LARGE
INSTITUTO DE CI\^ENCIAS MATEM\'ATICAS E COMPUTACIONAIS - ICMC

\vspace{1.7cm}
\Large
\textbf{Notas de F\'isica}

\vspace{1.3cm}
\large
\textbf{Renan Wenzel - 11169472}

\vspace{1.3cm}
\large
\textbf{Patr\'icia Christina Marques Castilho - patricia.castilho@ifsc.usp.br}

\vspace{1.3cm}
\today
\end{center}

\newpage

\tableofcontents

\newpage

\section{Aula 00 - 23/03/2023}
  (Revis\~ao Unidades de Medidas)

\section{Aula 01 - 27/03/2023}
\begin{itemize}
  \item Revisar propriedades de derivadas;
  \item Aplicar derivadas em movimento 1D. 
\end{itemize}
\subsection{Movimentos 1D}
  Dada uma part\'icula com posi\c c\~ao descrita por $x = x(t)$, em que t \'e a vari\'avel de tempo, denotamos seu deslocamento
por $\Delta x = x_{2} - x_{1} = x(t_{2}) - x(t_{1}).$ Analogamente, o intervalo de tmepo \'e definido por $\Delta t = t_{2} - t_{1}$.
Com essas ferramentas, j\'a podemos definir a velocidade m\'edia de um objeto em uma dimens\~ao como $\vec{v} = \frac{\Delta x}{\Delta t}.$
Observe que, quanto menor o intervalo de tempo, mais moment\^aneo se torna essa defini\c c\~ao, de modo que a velocidade instant\^anea
pode ser encontrada como 
  $$
    \lim_{\Delta t\to0}\frac{x(t + \Delta t) - x(t)}{\Delta t} = \vec{v}(t).
  $$
Regras de derivadas:
\begin{align*}
  &f(t) = c \Rightarrow \frac{df}{dt} = 0 \text{ Derivada de uma constante \'e sempre nula;}\\
  &f(t) = x^{n} \Rightarrow \frac{df}{dt} = nx^{n-1} \text{ Regra do tombo;}\\
  &f(t) = A\sin{(t)} \Rightarrow \frac{df}{dt} = A\cos{(t)};\\
  &f(t) = B\cos{(t)} \Rightarrow \frac{df}{dt} = -B\sin{(t)};\\
  &f(t) = C e^{t} \Rightarrow \frac{df}{dt} = C e^{t}.
\end{align*}
\begin{example}
 \begin{align*}
   &i)f(t) = 3t^{4} + t^{2} \Rightarrow \frac{df}{dt} = 12t^{3} + 2t\\
   &ii) f(t) = 5\sin{(t)} + 3(t^{2}+1) = 5\sin{(t)} + 3t^{2} + 3 \Rightarrow \frac{df}{dt} = 5\cos{(t)} + 6t 
 \end{align*} 
\end{example}

  A partir deste ponto, tome t como tempo, x(t) como posi\c c\~ao e v(t) a velocidade instant\^anea.

\begin{tikzpicture}
\begin{axis}[
    axis lines = left,
    xlabel = \(t\),
    ylabel = {\(x(t)\)},
]
%Below the red parabola is defined
\addplot [
    domain=0:20, 
    samples=100, 
    color=red,
]
{3*x - 20};
\addlegendentry{\(x(t) = 3t - 20\)}
%Here the blue parabola is defined
\addplot [
    domain=20:40, 
    samples=100, 
    color=blue,
    ]
    {-3*x + 100};
\addlegendentry{\(x(t) = -3t + 100\)}
\end{axis}
\end{tikzpicture}

  Esse movimento em que a velocidade \'e descrita por uma linha reta \'e conhecido como movimento retil\'ineo uniforme, pois 
a velocidade $v(t)$ muda de forma linear, i.e., $\frac{dx}{dt} = c$, em que c \'e uma constante.

  Por outro lado, h\'a outro tipo de movimento, o movimento retil\'ineo uniformemente variado, em que a velocidade n\~ao \'e constante.
A a\c c\~ao respons\'avel por mudar a velocidade \'e conhecida como acelera\c c\~ao, e os gr\'aficos tendem a assumir o seguinte formato

\begin{tikzpicture}
\begin{axis}[
    axis lines = left,
    xlabel = \(t\),
    ylabel = {\(x(t)\)},
]
%Below the red parabola is defined
\addplot [
    domain=0:20, 
    samples=100, 
    color=red,
]
{-x^2 + 10*x};
\addlegendentry{\(x(t) = -x^2+10x\)}
%Here the blue parabola is defined
\end{axis}
\end{tikzpicture}

Ou, caso a velocidade cres\c ca com o tempo,

\begin{tikzpicture}
\begin{axis}[
    axis lines = left,
    xlabel = \(t\),
    ylabel = {\(x(t)\)},
]
%Below the red parabola is defined
\addplot [
    domain=0:20, 
    samples=100, 
    color=red,
]
{x^2};
\addlegendentry{\(x(t) = x^2\)}
%Here the blue parabola is defined
\end{axis}
\end{tikzpicture}

H\'a ainda o caso em que a velocidade cresce por um tempo e diminui depois, com gr\'aficos como o que segue
 
\begin{tikzpicture}
\begin{axis}[
    axis lines = left,
    xlabel = \(t\),
    ylabel = {\(x(t)\)},
]
%Below the red parabola is defined
\addplot [
    domain=0:20, 
    samples=100, 
    color=red,
]
{x^2 - 20*x};
\addlegendentry{\(x(t) = x^2 - 20x\)}
%Here the blue parabola is defined
\end{axis}
\end{tikzpicture}

Nestes casos, para calcular o deslocamento da particula, precisamos somar muito mais intervalos de tempo. Para isso, observe que
cada instante, a posi\c c\~ao da part\'icula pode ser encontrada multiplicando-se o intervalo de tempo pela velocidade instan\^anea, i.e., 
 $\Delta x_{i}' = v_{i}'\Delta t_{i}'$. Quebrando os intervalos desta forma, o deslocamento de um ponto a outro \'e denotado por
 $$
   \Delta x_{1, 2} = x(t_{2}) - x(t_{1})\approx \sum\limits_{k=1}^{N}\Delta x_{i}' = \sum\limits_{k=1}^{N}v_{i}'\Delta t_{i}'
 $$
 Assim como para a velocidade instant\^anea, quanto menor tomarmos o intervalo de tempo, mais preciso \'e o valor encontrado para $\Delta x_{1, 2}$,
o que indica uma boa oportunidade para o uso do limite novamente. Com isso, definimos
 $$
  x(t_{2}) - x(t_{1}) = \lim_{\Delta t'\to0} \sum\limits_{i=1}^{N}v(t_{i}')\Delta t_{i}' = \int_{t_{1}}^{t_{2}}v(t)dt
 $$
 Este \'ultimo s\'imbolo, chamado integral, descreve a \'area ``embaixo'' da curva da fun\c c\~ao f(t) dentro do intervalo $[t_{1}, t_{2}].$
Supondo que c e k s\~ao constantes quaisquer, seguem abaixo algumas das regras de integra\c c\~ao: 
\begin{align*}
  &i)f(t) = ct^{n} \Rightarrow \frac{df}{dt} = nct^{n-1} \Rightarrow F(t) = \frac{ct^{n+1}}{n+1}\text{ (Primitiva de f)}\\
  &ii) \int_{t_{1}}^{t_{2}}f(t)dt = F(t_{2}) - F(t_{1}) = \frac{c}{n+1}t_{2}^{n+1} - \frac{c}{n+1}t_{1}^{n+1}\text{ (Integral definida de f)}\\
  &iii) \int_{}^{}f(t)dt = \frac{c}{n+1}t^{n+1} + k\text{ (Integral indefinida de f)}
\end{align*}
  Para conferir se a integral est\'a correta, \'e preciso derivar a fun\c c\~ao F e, se obter como resultado a fun\c c\~ao f, significa que est\'a correto.
Com este conhecimento em mente, segue que 
  $$
  \boxed{ x(t) = \int_{}^{}v_{0}dt = v_{0}t + x_{0}}
  $$
  Algumas outras regras importantes:
 \begin{align*}
   &iv) \frac{d\sin{(t)}}{dt} = \cos{(t)} \Rightarrow \int_{}^{}\cos{(t)}dt = \sin{(t)} + c\\
   &v) \frac{d\cos{(t)}}{dt} = -\sin{(t)} \Rightarrow \int_{}^{}\sin{(t)}dt = \cos{(t)} + c\\
   &vi) \frac{d e^{t}}{dt} = e^{t} \Rightarrow \int_{}^{} e^{t}dt = e^{t} + c
 \end{align*}
  Ou seja, em certo sentido, a integral e a derivada s\~ao dois lados da mesma moeda, assim como mulitplica\c c\~ao e divis\~ao ou adi\c c\~ao e subtra\c c\~ao.
  \newpage

\section{Aula 02 - 29/03/2023}
\subsection{Motiva\c c\~oes}
\begin{itemize}
  \item Estudar a acelera\c c\~ao;
  \item Entender o Movimento Retil\'ineo Uniformemente Variado.
\end{itemize}
\subsection{Acelera\c c\~ao}
  Definimos previamente a velocidade m\'edia como sendo a varia\c c\~ao de tempo dividindo o deslocamento, sendo, portanto,
uma quantidade representando a taxa de varia\c c\~ao da posi\c c\~ao em um intervalo de tempo. De forma an\'aloga,
definimos a acelera\c c\~ao como a taxa de varia\c c\~ao da velocidade em um intervalo de tempo, ou seja, 
  $$
    \vec{a_{m}} = \frac{\Delta \vec{v}}{\Delta t}.
  $$
  Ainda mais, se ela for positiva, a velocidade aumenta. Caso contr\'ario, ela diminui. Ainda repetindo o processo feito
para o caso da velocidade, podemos encontrar uma acelera\c c\~ao instan\^anea como 
    $$
      a(t) = \lim_{\Delta t\to0}\biggl[\frac{v(t + \Delta t) - v(t)}{\Delta t}\biggr] = \frac{dv(t)}{dt}
    $$
  Observe tamb\'em que 
    $$
      a(t) = \frac{d}{dt}\biggl(\frac{dx(t)}{dt}\biggr) = \frac{d^{2}x}{dt^{2}}.
    $$
  Utilizando a an\'alise dimensional, \'e poss\'ivel encontrar a dimens\~ao da acelera\c c\~ao como $[a]=\frac{[v]}{[t]} = \frac{\frac{[L]}{[t]}}{[t]} = \frac{[L]}{[t]^{2}}.$ Assim,
se o sistema de medida for o Sistema Internacional, $[a] = \frac{m}{s^{2}}$.

    \begin{tikzpicture}
    \begin{axis}[
        axis lines = left,
        xlabel = \(t\),
        ylabel = {\(x(t)\)},
    ]
    \addplot [
        domain=0:10,
        samples=100, 
        color=red,
    ]
    {-x^2 + 10*x};
    \addlegendentry{\(x(t) = -x^{2}+10x\)}
    \addplot[
      domain=0:10,
      samples=100,
      color=blue,
    ]
    {-2*x + 10};
    \addlegendentry{\(v(t) = -2x + 10\)},
      \addplot[
        domain=0:10,
        samples=100,
        color=green,
      ]
      {2};
    \addlegendentry{\(a(t) = 2\)},
    \end{axis}
  \end{tikzpicture}

\subsection{Movimento Retil\'ineo Uniformemente Variado.}
  Sabendo que $a =\displaystyle \frac{d^{2}x(t)}{dt^{2}}$, podemos fazer o caminho oposto para encontrar uma f\'ormula para
  a posi\c c\~ao sabendo a acelera\c c\~ao. De fato, dado um intervalo de tempo $[t_{0}, t],$
  $$
    v(t) = \int_{t_{0}}^{t}a(t)dt = at \biggl|_{t_{0}}^{t}\biggr. = at - at_{0}
  $$
  Sabemos, tamb\'em, que $v(t) - v(t_{0}) = \Delta v$, tal que 
    $$
      v(t) = v(t_{0}) + a(t-t_{0}) = v_{0} + a(t-t_{0})
    $$
  Al\'em disso, vimos que 
    $$
      \Delta x = x(t) - x(t_{0}) = \int_{t_{0}}^{t}v(t) dt.
    $$
  Juntando tudo, segue a f\'ormula dita: 
  \begin{align*}
    x(t) - x(t_{0}) &= \overbrace{\int_{t_{0}}^{t}[v_{0} + a(t-t_{0})]dt}^{\int_{}^{}f(t) + g(t)dt = \int_{}^{}f(t)dt + \int_{}^{}g(t)dt}= \overbrace{\int_{t_{0}}^{t}v_{0}dt}^{\int_{}^{}c dt = ct} + \overbrace{\int_{t_{0}}^{t}atdt}^{\int_{}^{}t^{n}dt = \frac{t^{n+1}}{n+1}} - \int_{t_{0}}^{t}at_{0}dt\\
                    & \Rightarrow x(t) - x(t_{0})  =  v_{0}t \biggl|_{t_{0}}^{t}\biggr. + a\frac{t^{2}}{2} \biggl|_{t_{0}}^{t}\biggr. - at_{0}t \biggl|_{t_{0}}^{t}\biggr. \\
                    &= v_{0}(t-t_{0}) + a \frac{(t^{2} - t_{0}^{2})}{2} - at_{0}(t-t_{0}) \\
                    &= v_{0}(t-t_{0}) + a \frac{t^{2}-t_{0}^{2}}{2} - at_{0}t + at_{0}^{2} = v_{0}t - v_{0}t_{0} + \frac{a}{2}(t^{2}-2t_{0}t + 2t_{0}^{2})\\
                    &= v_{0}(t-t_{0}) + \frac{a}{2}(t-t_{0})^{2}\\
                    & \Rightarrow x(t) = x(t_{0}) + v_{0}(t-t_{0}) + \frac{1}{2}a(t-t_{0})^{2}.
  \end{align*}
  Com isso, no caso em que $t_{0} = 0$, segue que 
   $$
    \boxed{x(t) = x_{0} + v_{0}t + \frac{at^{2}}{2}}     
   $$
  Uma coisa not\'avel \'e que todas essas f\'ormulas est\~ao dependentes de tempo. No entanto, ser\'a que \'e poss\'ivel
se livrar dessa vari\'avel e relacionar, por exemplo, velocidade e posi\c c\~ao? A resposta \'e sim! E vamos mostrar como a seguir,
na equa\c c\~ao conhecida como Equa\c c\~ao de Torricelli. Com efeito,
 \begin{align*}
   &(I)\quad (t-t_{0}) = \frac{v(t)-v_{0}}{a} = \frac{v-v_{0}}{a}\\
   &(II)\quad x(t) = x_{0} + v_{0}(t-t_{0}) + \frac{1}{2}a(t-t_{0})^{2}\\
   &(I\text{ com }II)\quad x = x_{0} + v_{0}\frac{v-v_{0}}{a} + \frac{1}{2}a \frac{v-v_{0}}{a} \\
   & \Rightarrow x = x_{0} + \frac{1}{a}\biggl\{v_{0}v - v_{0}^{2} + \frac{1}{2}(v^{2}-2vv_{0}+v_{0}^{2})\biggr\}\\
   & = x_{0} + \frac{1}{a}\biggl\{-v_{0}^{2} + \frac{v^{2}}{2} + \frac{v_{0}^{2}}{2}\biggr\}\\
   & \Rightarrow x - x_{0} = \frac{1}{2a}\biggl[v^{2}-v_{0}^{2}\biggr] \Longleftrightarrow [v^{2} - v_{0}^{2}] = 2a(x-x_{0}).
 \end{align*}
 Portanto, chegamos na Equa\c c\~ao de Torricelli
  $$
  \boxed{v^{2} = v_{0}^{2} + 2a(x-x_{0})}
  $$
  Para refor\c car o que foi visto at\'e agora, vejamos um exemplo.
 \begin{example}
   Suponha que um carro freia uniformemente, passando de 60km/h para 30km/h em 5 segundos. Qual \'e a dist\^ancia
que o carro percorrer\'a at\'e parar? Em quanto tempo?

  \textbf{Solu\c c\~ao}: Sabemos que $x(t) = x_{0} + v_{0}(t-t_{0}) + \frac{1}{2}a(t-t_{0})^{2}, v(t) = v_{0} + a(t-t_{0}),
\text{ e }v^{2} = v_{0}^{2} + 2a(x-x_{0}).$ Al\'em disso, como \'e at\'e o carro parar, a velocidade final \'e 0km/h, a varia\c c\~ao
de tempo at\'e o momento em que a velocidade atinge 30km/h (=8.333m/s) \'e dada como $\Delta t = 5 - 0 = 5$s , sendo a velocidade inicial 60km/h (=16,666m/s). 
Pela equa\c c\~ao dois, 
  $$
    a = \frac{v(t_{1}) - v_{0}}{t_{1}-t_{0}} = \frac{8.33 - 16.66}{5} = -1.66\frac{m}{s^{2}}
  $$
  Agora, para obter a dist\^ancia, sendo $v_{2} = 0km/h$ o valor da acelera\c c\~ao no tempo em que o carro para (o segundo percurso),
utilizamos Torricelli para obter o deslocamento no peda\c co final do percurso
  $$
    v_{2}^{2} = v_{1}^{2} + 2a(x_{2}-x_{1}) \Rightarrow 0 = 8.33^{2} + 2(-1.66)\Delta x_{2}
  $$
  Assim, isolando o $\Delta x_{2},$  
    $$
    \Delta x_{2} = \frac{8.33^{2}}{3.32} = \text{ Professora vai passar na pr\'oxima aula.}
    $$
  Ademais, para encontrar todo o caminho que o carro andou, temos 
    $$
      0 = v_{0}^{2} + 2a(x_{2} - x_{0}) = 16.66^{2} + 2(-1.66)\Delta x \Rightarrow \Delta x = \frac{16.66^{2}}{3.32}
    $$
  Finalmente, o instante de tempo pode ser encontrado fazendo 
    $$
    v_{2}(t) = v_{1} + a(t_{2}-t_{0}) \Rightarrow 0 = 8.33 - 1.66\Delta t_{2} \Rightarrow \Delta t_{2} = 5s.\text{ \qedsymbol}
    $$
 \end{example}
 \newpage

\section{ Aula 03 - 30/03/2023}
\subsection{Motiva\c c\~oes}
 \begin{itemize}
   \item Resolu\c c\~ao de Exerc\'icios.
 \end{itemize}
 \subsection{Exerc\'icio 29 - Tipler}
 ``Considere a trajet\'oria de dois carros, o Carro A e o Carro B. (a) Existe algum instante para o qual os carros est\~ao lado-a-lado? (b) Eles viajam sempre no mesmo sentido? (c) Eles viajam com a mesma velocidade em algum instante t? (d) Para que t os carros est\~ao mais distantes entre si? (e) Esboce os gr\'aficos de $v\times t$'
 \begin{center}
   \begin{tikzpicture}
   \begin{axis}[
       axis lines = left,
       xlabel = \(t\),
       ylabel = {\(x(t)\)},
   ]
   \addplot [
       domain=0:10, 
       samples=100, 
       color=red,
   ]
   {7*x/2 + 9/3};
 \addlegendentry{\(\text{Carro B}\)}
   %Here the blue parabola is defined
     \addplot[
       domain=0:10,
       samples=100,
       color=blue,
     ]
     {-x^2 + 13*x};
   \addlegendentry{\(\text{Carro A}\)}
     %Here the blue parabola is defined
   \end{axis}
 \end{tikzpicture}
\end{center}
  Os carros se encontram lado-a-lado quando os gr\'aficos se cruzam, ou seja, em t = 1s e t = 9s (Tipler mais acurado que meu gr\'afico.). \'E not\'avel
quer eles n\~ao est\~ao sempre no mesmo sentido, visto que, a partir de 6s, o gr\'afico do carro B passa a mudar o sentido. Em aproximadamente 5s,
ambos est\~ao com a reta tangente iguais, ou seja, est\~ao com a mesma velocidade, e dist\^ancia entre eles est\'a maior exatamente no ponto em que as 
velocidades est\~ao iguais. Finalmente, seguem os gr\'aficos:
\begin{center}
    \begin{tikzpicture}
    \begin{axis}[
        axis lines = left,
        xlabel = \(t\),
        ylabel = {\(v(t)\)},
    ]
    \addplot [
        domain=0:6, 
        samples=100, 
        color=red,
    ]
    {-3*x + 18};
  \addlegendentry{\(\text{Carro A}\)}
    %Here the blue parabola is defined
    \addplot[
      domain=0:6,
      samples=100,
      color=red,
    ]
    {5};
  \addlegendentry{\(\text{Carro B}\)}
    %Here the blue parabola is defined
    \end{axis}
  \end{tikzpicture}
\end{center}

\subsection{Exerc\'icio 44 - Tipler}
``Um carro viaja em linha reta com $\vec{v} = 80\text{km/h} $ durante $\Delta t_{1}=2.5$h. Depois, $\vec{v_{2}} = 40$km/h, $\Delta t_{2} = 1.5$h. Qual
\'e o deslocamento total? E qual \'e a velocidade $\vec{v}$ total?''
\begin{align*}
  &(a)\quad \Delta x = \Delta x_{1} + \Delta x_{2} = \vec{v_{1}}\Delta t_{1} + \vec{v_{2}}\Delta t_{2}  \Rightarrow \Delta x = 260\text{km.}\\
  &(b)\quad \vec{v} = \frac{\Delta x}{\Delta t} = \frac{260}{4} = 65\text{km/h}.
\end{align*}

\subsection{Exerc\'icio 58 - Tipler}
  ``Um carro acelera de 48.3km/h para 80.5km/h em 3.70s. Qual a acelera\c c\~ao m\'edia?''

Primeiramente, precisamos converter as unidades para medidas iguais. Com isso, note que $\vec{v_1} = 48.3km/h = 13.52m/s, \vec{v_{2}} = 80.5km/h = 22.54m/s$. Assim,
chegamos em 
  $$
  \vec{a} = \frac{\Delta v}{\Delta t} = \frac{v_{2} - v_{1}}{\Delta t}\approx 2.4\text{m/s}.
  $$

\subsection{Exerc\'icio 67 - Tipler}
``Um corpo est\'a em uma posi\c c\~ao inicial $x_{1}$ com velocidade inicial $\vec{v_{1}}$. Passado um tempo, ele se encontra na posi\c c\~ao $x_{2}$ com velocidade $\vec{v_{2}}$. Qual \'e a acelera\c c\~ao deste corpo?''

Utilizaremos Torricelli. sabemos que 
 \begin{align*}
   &(1):\quad x_{1} = 6m, \vec{v_{1}} = 10m/s\\
   &(2):\quad x_{2} = 10m, \vec{v_{2}} = 15m/s.
 \end{align*}
 Deste modo, $v^{2} = v_{0}^{2} + 2a\Delta x \Rightarrow v_{2}^{2} = v_{1}^{2} + 2a(x_{2} - x_{1}) \Rightarrow a \approx 16m/s^{2}$

\subsection{Exerc\'icio 72 - Tipler}
  ``Um parafuso se desprende de um elevador subindo a $v_{0} = 6m/s$. O parafuso atinge o fundo do po\c co em 3s. (a) Qual era a altura do elevador? (b) Qual \'e a velocidade do parafuso no ch\~ao? Tome g = 9.8 $m/s^{2}$''

Sabemos que $t_{0} = 0s, y(t_{0}) = h, v(t_{0}) = v_{0}.$ Com isso, podemos descrever $y(t) = h + v_{0}t + \frac{1}{2}gt^{2}$. Vamos responder, agora, o item a, isto \'e, qual \'e o valor
da altura h? Segue que, em $t=3s, y(t) = 0$. Utilizando a f\'ormula, 
  $$
    h = -v_{0}t + \frac{1}{2}gt^{2} = -6 \cdot3 + \frac{1}{2}9.8 \cdot 3^{2} = 26.1m
  $$
  Com rela\c c\~ao ao item (b), vimos que $v(t) = v_{0} + at.$ Deste modo, 
    $$
      v(3s) = 6 - 9.8 \cdot 3 = -23.4m/s
    $$
  Indo um pouco al\'em do que foi pedido, analisemos o movimento do parafuso. \'E poss\'ivel concluir que o parafuso atingir\'a a altura 
m\'axima no instante em que $t^{*} = \frac{v_{0}}{g} = 0.6s,$ visto que este momento ocorre quando $v(t) = v_{0} - gt = 0$. Com isso,
conclui-se que a altura m\'axima \'e $y(t^{*}) = h + v_{0}t^{*} - \frac{1}{2}gt^{*^{2}} \approx 27.5m.$ No gr\'afico,
  \begin{center}
    \begin{tikzpicture}
    \begin{axis}[
        axis lines = left,
        xlabel = \(t\),
        ylabel = {\(y(t)\)},
    ]
    \addplot [
      domain = 0:4,
      color = black,
      ]
    {0};
    \addplot [
        domain=0:3, 
        samples=100, 
        color=red,
    ]
    {-9.8*x^2/2 + 6*x + 26.1};
    \addlegendentry{\(y(t)\)}
    %Here the blue parabola is defined
      \addplot[
        domain=0:3,
        samples=100,
        color=blue,
      ]
      {27.955};
    \addlegendentry{\(h_{\text{max}} = \frac{dy}{dt} = 0\)}
      %Here the blue parabola is defined
    \end{axis}
  \end{tikzpicture}
\end{center}

\subsection{Exemplo - Aula 06 Vanderlei}
  ``Suponha que h\'a um trem parado no instante t=0 com acelera\c c\~ao a. Passados 6s, um passageiro chega ao local e observa o trem na posi\c c\~ao $x_{trem_{1}}$.
  Este passageiro sai correndo com velocidade $v_{0}$ para tentar alcan\c car o trem. Qual \'e a velocidade m\'inima que o passageiro precisa atingir
  para alcan\c c\'a-lo?''
    
  Com rela\c c\~ao ao trem, suas condi\c c\~oes iniciais s\~ao $t_{0} = 0, x_{trem} = 0, v_{trem} = 0,$ tal que $x_{trem}(t) = \frac{1}{2}at^{2}$. 
Por outro lado, quanto ao passageiro, quando $t=6s, x_{p} = 0$, de modo que $x_{p}(t) = x_{p_0} + v_{0}t$. Como temos a informa\c c\~ao da posi\c c\~ao
do passageiro aos 6s, 
  $$
    x_{p}(6) = x_{p_{0}} + v_{0} \cdot6 = 0 \Rightarrow x_{p_{0}} = -6v_{0} \Rightarrow x_{p}(t) = v_{0}(t-6).
  $$
  No momento em que o passageiro alcan\c ca o trem, eles possuem posi\c c\~oes iguais, isto \'e, $x_{p}(t) = x_{trem}(t)$. Graficamente,
 \begin{center}
     \begin{tikzpicture}
     \begin{axis}[
         axis lines = left,
         xlabel = \(t\),
         ylabel = {\(x(t)\)},
     ]
     \addplot [
         domain=0:12, 
         samples=100, 
         color=red,
     ]
     {x^2};
   \addlegendentry{\(\text{Trem}\)}
       \addplot[
         domain=0:6,
         samples=100,
         color=blue,
       ]
       {0};
     \addlegendentry{\(\text{Passageiro}\)}
       \addplot[
         domain=6:12,
         samples=100,
         color=green,
       ]
       {25*x-150};
       \addlegendentry{\(\frac{dx_{trem}}{dt} = \frac{dx_{p}}{dt}\)}
     \end{axis}
   \end{tikzpicture}
 \end{center}
  Ou seja, buscamos $t^{*}$ tal que $x_{p}(t^{*}) = x_{trem}(t^{*}), v_{p}(t^{*}) = v_{trem}(t^{*})$. Com efeito,
 \begin{align*}
   v_{0}(t^{*} - 6) &= \frac{at^{*^{2}}}{2} \Rightarrow v_{0} = at^{*} \Rightarrow t^{*} = \frac{v_{0}}{a}\\
                    &v_{0} = \frac{a}{2}\frac{(\frac{v_{0}}{a})^{2}}{\frac{v_{0}}{a}-6} \Rightarrow \frac{v_{0}^{2}}{2a} = 6v_{0} \Rightarrow v_{0} = 12a.
 \end{align*}
 Outra forma de resolver \'e utilizando o fato de que quando $\frac{dv}{dt} = 0$, a fun\c c\~ao est\'a num m\'inimo. Ou seja, basta encontrar
o valor m\'inimo de $v_{0}$ que satisfa\c ca o que buscamos. Temos 
  $$
    v_{0}(t-6) = \frac{at^{2}}{2} \Rightarrow v_{0}(t) = \frac{at^{2}}{2}\frac{1}{(t-6)}.
  $$ 
  Agora, derivando essa equa\c c\~ao para $v_{0},$ 
    $$
      \frac{dv_{0}}{dt} = \frac{d}{dt}\biggl(\frac{at^{2}}{2}\frac{1}{t-6}\biggr) = \frac{d}{dt}(f(t)g(t)),
    $$
    em que $f(t) = \frac{at^{2}}{2}, g(t) = (t-6)^{-1}$. Fazemos isso porque h\'a uma regra para derivar o produto de fun\c c\~oes,
  a Regra do Produto 
    $$
      \boxed{\frac{df(t)g(t)}{dt}= g(t)\frac{df(t)}{dt} + f(t)\frac{dg(t)}{dt}}
    $$
    Derivando individualmente f e g, 
      $$
        \frac{df(t)}{dt} = at, \quad \frac{dg(t)}{dt} = -(t-6)^{-2} = -\frac{1}{(t-6)^{2}}.
      $$
    Agora, vamos juntar tudo para obter a derivada de $v_{0}$: 
    \begin{align*}
      \frac{dv_{0}}{dt} &= \frac{df(t)}{dt}g(t) + \frac{dg(t)}{dt}f(t) = \frac{at}{t-6} - \frac{1}{2(t-6)^{2}}at^{2}\\
                        &= at\biggl(\frac{1}{t-6} - \frac{t}{2(t-6)^{2}}\biggr) = 0 \\
                        & \Rightarrow \frac{1}{t-6} = \frac{t}{2(t-6)} \Rightarrow 1 = \frac{t}{2(t-6)}\\
                        & \Rightarrow 2(t-6) = t \Rightarrow 2t - t = 12 \Rightarrow t = 12s.
    \end{align*}
\newpage

\section{Aula 04 - 10/04/2023}
\subsection{Motiva\c c\~oes}
 \begin{itemize}
   \item Iniciar os estudos de movimentos em um plano todo (duas dimens\~oes);
   \item Revisar vetores e sua manipula\c c\~ao.
 \end{itemize}

\subsection{Vetores}
  Come\c camos com um estudo das propriedades de veotres. Dados vetores $\vec{r_{1}}, \vec{r_{2}}$ e um n\'umero real $\lambda$, definimos:
 \begin{itemize}
  \item[i)] A soma dos vetores:
  \begin{center}
    \begin{tikzpicture}
      \coordinate (O) at (0,0); % origin
      \draw [->,red] (O) -- (2,1) node [right] {$\vec{r}_1$}; % vector r_1
      \draw [->,blue] (2,1) -- (3,3) node [right] {$\vec{r}_1$}; % vector r_1
      \draw [->,blue] (O) -- (1,2) node [above] {$\vec{r}_2$}; % vector r_2
      \draw [->,red] (1,2) -- (3,3) node [above] {$\vec{r}_1$}; % vector r_2
      \draw [->,black] (O) -- (3,3) node [above right] {$\vec{r}_1+\vec{r}_2$}; % sum of vectors
    \end{tikzpicture}
  \end{center}
  \item[ii)] A multiplica\c c\~ao por escalar: $\lambda(r_{1}+r_{2})$ (Essencialmente, o resultado \'e aumentar ou diminuir o tamanho da seta.)
    \begin{center}
      \begin{tikzpicture}
        \coordinate (O) at (0,0); % origin
        \draw [->,red] (O) -- (2,1) node [right] {$\vec{v}$}; % vector v
        \draw [->,blue] (O) -- (4,2) node [right] {$\lambda\vec{v}$}; % scaled vector
        \draw [dashed] (2,1) -- (4,2); % dotted line to show scaling
      \end{tikzpicture}
    \end{center}
 \end{itemize}
 A t\'itulo de curiosidade, a soma de vetores em tr\^es dimens\~oes seria desta forma: 
 \begin{center}
   \tdplotsetmaincoords{70}{135} %set the viewing angle
\begin{tikzpicture}[scale=2, tdplot_main_coords]
    \coordinate (O) at (0,0,0);
    \draw[thick,->] (O) -- (2,0,0) node[anchor=north east]{$x$};
    \draw[thick,->] (O) -- (0,2,0) node[anchor=north west]{$y$};
    \draw[thick,->] (O) -- (0,0,2) node[anchor=south]{$z$};

    \draw[->,red] (O) -- (1.5,0.5,1.5) node[midway, below right] {$\vec{u}$};
    \draw[->,blue] (1.5,0.5,1.5) -- (2,2,2) node[midway, above left] {$\vec{v}$};
    \draw[->,blue] (O) -- (0.5,1.5,0.5) node[midway, above] {$\vec{v}$};
    \draw[->,red] (0.5,1.5,0.5) -- (2,2,2) node[midway, below] {$\vec{u}$};
    \draw[->,black] (O) -- (2,2,2) node[midway, above right] {$\vec{u}+\vec{v}$};
\end{tikzpicture}
\end{center}
 Por\'em, n\~ao basta utilizar apenas representa\c c\~oes gr\'aficas para vetores. Desta forma, \'e comum definirmos um sistema de coordenadas
cartesiano para suas componentes. Assim, um vetor $\vec{u}$ pode ser decomposto em uma coordenada x e outra coordenada y: 
  $$
  \vec{u} = u_{x}\hat{i} + u_{y}\hat{j} (+u_{z}\hat{k})
  $$
  chamamos os valores $u_{x}, u_{y}, u_{z}$ de proje\c c\~oes, sendo a \'ultima um objeto presente apenas no caso de tr\^es coordenadas.
Com isso, definimos o m\'odulo do vetor, ou seja, seu tamanho, pela f\'ormula 
  $$
    |\vec{u}| = \sqrt{u_{x}^{2}+u_{y}^{2}},
  $$
  e, de brinde, ganhamos f\'ormulas para as proje\c c\~oes em cada coordenada:
 \begin{align*}
   &u_{x} = |u|\cos{\theta} \Rightarrow \cos{\theta} = \frac{u_{x}}{|\vec{u}|}\\
   &u_{y} = |u|\sin{\theta} \Rightarrow \sin{\theta} = \frac{u_{y}}{|\vec{u}|}\\
   &\tan{\theta} = \frac{u_{y}}{u_{x}}.
 \end{align*}
 \'E importante, tam\'em, darmos uma forma de obter um betor de m\'odulo 1, i.e., um vetor unit\'ario, visto que ele pode nos fornecer a informa\c c\~ao do
valor do \^angulo $\theta$, a dire\c c\~ao, etc. Ele \'e obtido reduzindo um vetor u pelo seu m\'odulo, 
  $$
  \hat{u} = \frac{\vec{u}}{|\vec{u}|}.
  $$
  Uma utilidade imediata da defini\c c\~ao em coordenadas \'e que agora temos um modo de tratar a soma de vetores algebricamente 
    \begin{align*}
     &\text{Soma: } \vec{u}+\vec{v} = (u_{x}\hat{i} + u_{y}\hat{j}) + (v_{x}\hat{i} + v_{y}\hat{j}) = (u_{x}+v_{x})\hat{i} + (u_{y}+v_{y})\hat{j}\\
     &\text{Multiplica\c c\~ao por Escalar: } \lambda \vec{u} = \lambda(u_{x}\hat{i} + u_{y}\hat{j}) = \lambda u_{x}\hat{i} + \lambda u_{y}\hat{j}\\
     &\theta = ctg{\biggl(\frac{\lambda u_{y}}{\lambda u_{x}}\biggr) = ctg{\biggl(\frac{u_{y}}{u_{x}}\biggr)}}.
    \end{align*}
    Agora podemos ir \`a aplica\c c\~ao f\'isica dessa discuss\~ao, o deslocamento de uma part\'icula no plano. Nesta configura\c c\~ao, normalmente
  ter\'a-se uma part\'icula com posi\c c\~ao $\vec{x}(t) = x(t)\hat{i} + y(t)\hat{j}\quad (+z(t)\hat{k}).$ Para realizar o estudo desses casos,
  vamos decompor o movimento dela em cada eixo, ou seja, quebramos o movimento no plano em dois movimentos independentes, um em cada eixo
  x ou y. Nestas condi\c c\~oes, o deslocamento de uma part\'icula de uma posi\c c\~ao 1 at\'e uma posi\c c\~ao 2 ser\'a 
    $$
    \vec{x_{2}} - \vec{x_{1}} = (x_{2}\hat{i} + y_{2}\hat{j}) - (x_{1}\hat{i}+y_{1}\hat{j}) = (x_{2}-x_{1})\hat{i} + (y_{2}-y_{1})\hat{j}.
    $$
    Com isso, podemos escrever que o deslocamento $\Delta \vec{r}$ \'e 
      $$
      \Delta \vec{r} = \Delta x\hat{i} + \Delta y\hat{j}.
      $$
    Ainda mais, se conhecemos o valor do \^angulo entre as posi\c c\~oes 1 e 2 e o m\'odulo dos vetores representando-as, 
      $$
        |\Delta r|^{2} = x_{1}^{2} + x_{2}^{2} - 2x_{1}x_{2}\cos{\theta}.
      $$
    Tendo o b\'asico do deslocamento, podemos repetir o racioc\'inio pr\'evio para trabalhar com acelera\c c\~ao e velocidade. De fato, 
      $$
        \vec{v_{m}} = \frac{\Delta \vec{r}}{\Delta t}, \quad \vec{a_{m}} = \frac{\Delta \vec{v_{m}}}{\Delta t}
      $$
      e os valores instant\^aneos ser\~ao dados por 
      \begin{align*}
        &\vec{v}(t) = \frac{d \vec{r}(t)}{dt} = \frac{d x(t)}{dt}\hat{i} + \frac{d y(t)}{dt}\hat{j} = v_{x}\hat{i} + v_{y}\hat{j}.\\
        &\vec{a}(t) = \frac{d \vec{v}(t)}{dt} = \frac{d v_{x}(t)}{dt}\hat{i} + \frac{d v_{y}(t)}{dt}\hat{j} = a_{x}\hat{i} + a_{y}\hat{j}.
      \end{align*}
      Al\'em disso, o m\'odulo e orienta\c c\~ao desses valores ser\~ao dados por 
     \begin{align*}
       &|\vec{v}| = \sqrt{v_{x}^{2}+v_{y}^{2}},\quad \theta_{v} = ctg\biggl(\frac{v_{y}}{v_{x}}\biggr)\\
       &|\vec{a}| = \sqrt{a_{x}^{2}+a_{y}^{2}},\quad \theta_{a} = ctg\biggl(\frac{a_{y}}{a_{x}}\biggr).
     \end{align*}
     Note que a acelera\c c\~ao n\~ao aponta na dire\c c\~ao da velocidade em si, mas sim na dire\c c\~ao da \textbf{varia\c c\~ao} da velocidade.

\subsection{Movimento Uniforme Bidimensional}
    Considere uma part\'icula com posi\c c\~ao $\vec{r}(t)$ e uma orienta\c c\~ao, tal que forma um \^angulo $\theta$ com o plano. Como
  estaremos considerando o movimento do tipo uniforme, a acelera\c c\~ao \'e nula e a velocidade $\vec{v}(t) = v_{0}$ \'e constante,
  tendo m\'odulo $v_{0}$ e orienta\c c\~ao $\theta$. Em outras palavras, as componentes desse vetor ser\~ao, tamb\'em, constantes, isto \'e, 
    $$
    \text{constantes}\left\{\begin{array}{ll}
          v_{x}(t) = v_{x_{0}}\\
          v_{y}(t) = v_{y_{0}}.
        \end{array}\right.
    $$
    Desta forma, a decomposi\c c\~ao da velocidade em coordenadas \'e tal que 
   \begin{align*}
     &\text{Eixo x: }v_{x}(t) = v_{x_{0}} \Rightarrow x(t) = x_{0} + v_{x_{0}}(t-t_{0}),\quad x_{0} = x(t_{0})\\
     &\text{Eixo y: }v_{y}(t) = v_{y_{0}} \Rightarrow y(t) = y_{0} + v_{y_{0}}(t-t_{0}),\quad y_{0} = y(t_{0}).
   \end{align*}
   Logo, a posi\c c\~ao da part\'icula no plano ser\'a dada por 
     $$
     \vec{r}(t) = x(t)\hat{i} + y(t)\hat{j} = (x_{0} + v_{x_{0}}(t-t_{0})\hat{i}) + (y_{0} + v_{y_{0}}\hat{j}).
     $$
  Note que, quando falamos de trajet\'oria de uma part\'icula ou objeto, buscamos uma rela\c c\~ao entre as componentes x(t) e y(t) que
independe do tempo, isto \'e, a rela\c c\~ao temporal \'e dada de forma implicita. Uma forma de fazer isso \'e a atrav\'es da tangente, pois 
  $$
  \frac{y(t) - t_{0}}{x(t) - x_{0}} = \frac{v_{y_{0}}(t-t_{0})}{v_{x_{0}}(t-t_{0})} = \frac{v_{y_{0}}}{v_{x_{0}}} = \tan{\theta_{0}}.
  $$
  Com isso, 
    $$
    y(t) - y_{0} = \tan{\theta_{0}}(x(t)-x_{0}) \Rightarrow y = \tan{(\theta_{0})}x - \tan{(\theta_{0})}x_{0} + y_{0},
    $$
    ou seja, y tem a forma de um equa\c c\~ao da reta com inclina\c c\~ao constante e igual a $\tan{\theta_{0}}.$
\newpage

\section{Aula 06 - 13/04/2023}
\subsection{Motiva\c c\~oes}
\begin{itemize}
\item Revisar movimento relativo;
\end{itemize}

\subsection{Movimento Relativo}
  Fixada uma origem O, a soma dos vetores representando os corpos A e B
    $$
      \vec{r}_{AO} + \vec{r}_{BA} = \vec{r}_{BO}
    $$
  nos fornece a dire\c c\~ao relativa do corpo B com rela\c c\~ao a A. Se A e B est\~ao se movendo, ou seja, os vetores deles 
possuem depend\^encia no tmepo ($\vec{r}_{AO} = \vec{r}_{AO}(t), \vec{r}_{BO} = \vec{r}_{BO}(t)$), ent\~ao 
  $$
    \vec{r}_{BA}(t) = \vec{r}_{BO}(t) + \vec{r}_{AO}(t),
  $$
  ou seja, a posi\c c\~ao relativa de B com rela\c c\~ao a A tamb\'em depender\'a do tempo. Al\'em de posi\c c\~ao relativa,
podemos definir outros conceitos, tais como a velocidade relativa: 
  $$
    \vec{v}_{BA}(t) = \frac{d \vec{r}_{BA}(t)}{dt} = \frac{d \vec{r}_{BO}}{dt} + \frac{d \vec{r}_{AO}}{dt} \Rightarrow \vec{v}_{BA}(t) = \vec{v}_{BO}(t)  + \vec{v}_{AO}(t)
  $$
  e acelera\c c\~ao relativa de modo an\'alogo, i.e., $\vec{a}_{BA}(t) = \vec{a}_{BO}(t) + \vec{a}_{AO}(t).$ Vejamos alguns exemplos
 \begin{example}
   Considere um sistema em que um carrinho viaja com velocidade $\vec{v}_{r}$ e tem um passageiro $\vec{v}_{p}$ com ele. 
Ambos se movem para a direita. 
  Neste caso, h\'a o sistema referencial de in\'erca da pessoa dentro do trem. Buscamos descobrir a velocidade da pessoa com rela\c c\~ao ao trem. 
De fato, segue que 
  $$
    \vec{v}_{p} = \vec{v}_{PT} + \vec{v}_{T}.
  $$
 \end{example}
 \begin{example}
  Considere um sistema an\'alogo ao anterior, mas, embaixo, h\'a uma plataforma se movendo para a esquerda com velocidade igual \`a do trem.
  Neste caso, h\'a o sistema referencial de in\'erca da pessoa dentro do trem. Buscamos descobrir a velocidade da pessoa com rela\c c\~ao
\`a plataforma. Obtemos
  $$
  \vec{v}_{PT} = \vec{v}_{PT}^{x}\hat{i} + \vec{v}_{PT}^{y}\hat{j}. \Rightarrow \vec{v}_{p} = (\vec{v}_{PT}^{x} + \vec{v}_{T})\hat{i} + \vec{v}_{P}^{y}\hat{j}
  $$
 \end{example}
\begin{example}
  (Exemplo 32 do Tipler): Considere um sistema de avi\~ao e vento, no qual o m\'odulo da velocidade do avi\~ao \'e de 200km/h e,
 o da velocidade do vento, \'e 90km/h. O vento \'e dado por um vetor apontando para a direito, enquanto o avi\~ao \'e um vetor apontando para cima. 
Pergunta-se: (a) Qual \'e a orienta\c c\~ao que o avi\~ao deve voar? (Ambos est\~ao sendo vistos do solo.) (b) Qual \'e o m\'odulo da
velocidade do avi\~ao visto do solo?
  (a) Segue que 
    $$
      \vec{v}_{AO} = \vec{v}_{A} - \vec{v}_{v} \Rightarrow \sin{\theta} = \frac{|\vec{v}_{v}|}{|\vec{v}_{av}|} = \frac{90}{200} = \frac{9}{20}\approx 27\deg
    $$

    (b) Sabemos, por pit\'agora, que 
      $$
      |\vec{v}_{AT}|^{2} = |\vec{v}_{v}|^{2} + |\vec{v}_{a}|^{2} \Rightarrow |\vec{v}_{a}| = \sqrt{|\vec{v}_{av}|^{2} - |\vec{v}_{v}|^{2}} = \sqrt{51900}\approx 178km/h
      $$
\end{example}
  Com rela\c c\~ao a este \'ultimo exemplo, por que a velocidade $\vec{v}_{a}$ tem valor 178km/h e n\~ao 200 - 90 = 110km/h?
A resposta est\'a na decomposi\c c\~ao de $\vec{v}_{av}$, pois 
\begin{align*}
  &v_{av}^{x} = |\vec{v}_{av}|\sin{\theta} = -200 \cdot 0.454\approx -90km/h\\
  &v_{av}^{y} = |\vec{v}_{av}|\cos{\theta} = 200 \cdot 0.891\approx 178km/h.
\end{align*}
 \begin{example}
   Suponha que, num instante $t_{0}$, dois trens est\~ao andando em dire\c c\~ao a uma plataforma. O trem um chegou nela, vindo do Norte, enquanto o trem dois,
   vindo pelo Leste, ainda se move, ambos com velocidade 
     $$
       |\vec{v}_{1}| = |\vec{v}_{2}| = 60km/h.
     $$
     Passados dois minutos, o trem 2 alcan\c ca a plataforma e continua andando na dire\c c\~ao Oeste com velocidade $\vec{v}_{2}$ e o trem
    um continuou sua viagem ao Sul com velocidade $\vec{v}_{1}$. Pede-se: (a) Determine o vetor $\vec{v}_{21}$ da velocidade relativa dos trens. (b) Encontre, para este vetor do item (a), 
    seu m\'odulo (c) Quando a dist\^ancia entre os vetores \'e m\'inima?

    Faremos o diagrama de velocidades. Nele, $|v_{1}| = |v_{2}|$. 
\begin{center}
  \begin{tikzpicture}[scale=2]
    \coordinate (A) at (0,0);
    \coordinate (B) at (-1.5,0);
    \coordinate (C) at (0, -1.5);
    \draw[-latex] (A) -- node[below] {$\vec{v}_1$} (B);
    \draw[-latex] (A) -- node[left] {$\vec{v}_2$} (C);
    \draw (B) -- node[above right] {$\vec{v}_{12}$} (C);
  \end{tikzpicture}
\end{center}
  Al\'em disso, pelo desenho, 
    $$
    \sin{\theta} = \frac{|\vec{v}_{2}|}{|\vec{v}_{21}|},\quad \cos{\theta} = \frac{|\vec{v}_{1}|}{|\vec{v}_{21}|} = \frac{|\vec{v}_{2}|}{|\vec{v}_{21}|} = \sin{\theta}.
    $$
    A igualdade entre seno e cosseno ocorre quando o \^angulo vale 45 graus, ou seja, $\theta = 45\deg.$ Assim,
     $$
     \vec{v}_{21} = \vec{v}_{2} - \vec{v}_{1} = -|\vec{v}_{2}|\hat{i} - (-|\vec{v}_{1}|\hat{j}) \Rightarrow \vec{v}_{21} = -|\vec{v}_{2}|\hat{i} + |\vec{v}_{1}|\hat{j}.
     $$
     Logo, 
       $$
       |\vec{v}_{21}| = \sqrt{|\vec{v}_{1}|^{2} + |\vec{v}_{2}|^{2}}\approx 85km/h
       $$

    Para resolver, agora, o item b, a comecemos pela posi\c c\~ao relativa 2 nos instantes t=0, t=2min e t=4min. Quanto ao trem 2, as
  informa\c c\~oes que temos indicam que ele se move no eixo x ($y_{2}(t) = 0$), em t=2min, ele est\'a na origem ($x_{2}(2min) = 0$) e, deste modo,
    $$
      \vec{r}_{2}(t) = x_{2}(t)\hat{i} + y_{2}(t)\hat{j} = x_{2}(t)\hat{i} \Rightarrow x_{2}(t) = x_{2O} + |\vec{v}_{2}|t
    $$
    Utilizando o valor que sabemos, i.e., x(2min), segue que, convertendo 2 minutos para horas ($2min\approx 0.03h$) ,
      $$
      x(0.03) = 0 = x_{2O} - 60 \cdot 0.03 \Rightarrow x_{2O} = 60 \cdot 0.03 = 2km.
      $$

      Agora, sobre o trem 1, sabe-se que ele se move no eixo y, ou seja, $x_{1}(t) - 0$, tal que 
        $$
          y_{1}(t) = -|\vec{v}_{1}|t = -60t
        $$
      
    Com essas informa\c c\~es, encontramos os valores 
   \begin{align*}
     &t = 0min:\quad x_{1}(0) = 0, y_{1}(0) = 0,\quad x_{2}(0) = 2km, y_{2}(0) = 0km\\
     &t = 2min:\quad x_{1}(2) = 0, y_{1}(2) = -2km,\quad x_{2}(2) = 0km, y_{2}(2) = 0km\\
     &t = 4min:\quad x_{1}(4) = 0, y_{1}(4) = -4km,\quad x_{2}(4) = -2, y_{2}(4) = 0km.
   \end{align*}
   Desta forma, 
     $$
     \vec{r}_{21}(t) = \vec{r}_{2}(t) - \vec{r}_{1}(t) \Rightarrow \vec{r}_{21}(t) = x_{2}(t)\hat{i} - y_{1}(t)\hat{j}.
     $$

    Finalmente, para o item c, calculamos a dis\^encia como 
      $$
      L_{21}(t) = \sqrt{x_{1}^{2} + (-y_{1})^{2}} = \sqrt{(x_{20}-|\vec{v}_{2}(t)|)^{2} + (|\vec{v}_{1}|)^{2}} = \sqrt{7200t^{2} - 240t + 4}.
      $$
    Para encontrar a dist\^ancia \textbf{m\'inima}, \'e preciso derivar esta f\'ormula, igualar a 0 e resolver para tempo. Coloque $l = 7200t^{2}-240t+4$,
  tal que 
    $$
      \frac{dl}{dt} = 2 \cdot 7200t - 240 + 0 = 0
    $$
    Resolvendo isso, encontramos o tempo em que a dist\^ancia \'e m\'inima, valendo $t^{*}\approx 0.017h\approx 1min$, tal que a dist\^ancia m\'inima \'e 
      $$
        L_{21}(t^{*})\approx 1.4.
      $$
 \end{example}
\newpage

\section{Aula 7 - 17/03/2023}
\subsection{Motiva\c c\~oes}
\begin{itemize}
  \item Come\c car a estudar o movimento circular;
  \item 
\end{itemize}

\subsection{Movimento Circular}
  Quando temos uma particular fazendo movimento circular em um c\'irculo de raio R num eixo x, y, diremos que ela, sua posi\c c\~ao em qualquer instante
ser\'a dada por um vetor r(t), sendo sua trajet\'oria limitada a este c\'irculo. Assim, obtemos o sistema $R = |\vec{r}(t)|$, sendo 
$\vec{r}(t) = x(t)\hat{i} + y(t)\hat{j}$ e 
  $$
     \left\{\begin{array}{ll}
         x(t) = R\cos{\theta(t)}\\
         y(t) = R\sin{\theta(t)}.
      \end{array}\right.
  $$
  \begin{center}
\begin{tikzpicture}[>=latex]
    \draw[->] (-1.2,0) -- (1.2,0) node[right] {$x$};
    \draw[->] (0,-1.2) -- (0,1.2) node[above] {$y$};
    \draw (0,0) circle [radius=1];
    \draw[->] (0, 0) -- ({-cos(45)}, {sin(45)}) node[above] {$R$};
    \draw[->] (0,0) -- ({cos(45)},{sin(45)}) node[above right] {$r(t)$};
\end{tikzpicture}
  \end{center}
  Utilizando o sistema e o desenho, obtemos 
    $$
    \vec{r}(t) = R\cos{\theta(t)}\hat{i} + R\sin{\theta(t)}\hat{j},
    $$
    donde segue o valor do m\'odulo do vetor $\vec{r}(t)$:
   \begin{align*}
     |\vec{r}(t)|^{2} = x^{2}(t) + y^{2}(t) &= (R\cos{\theta(t)})^{2} + (R\sin{\theta(t)})^{2} \\
                                            &= R^{2}(\cos{\theta(t)}^{2} + \sin{\theta(t)}^{2}) \\
                                            &= R^{2} \Rightarrow |\vec{r}(t)| = R.
   \end{align*}
   Conclu\'imos, assim, que todo o movimento da part\'icula \'e dado em termos do \^angulo $\theta(t).$ Al\'em disso, o deslocamento
  da part\'icula \'e feita em arcos de c\'irculo $s(t) = R\theta(t)$. Chamamos esta posi\c c\~ao de ``posi\c c\~ao escalar do corpo sobre o c\'irculo''.
  No entanto, no movimento circular, h\'a outra posi\c c\~ao, chamada ``posi\c c\~ao angular do corpo'', que \'e dada por $\theta(t) = \frac{s(t)}{R}.$
  
  Utilizando estes dois, podemos encontrar uma equa\c c\~ao para $\vec{r}(t)$: 
    $$
    \vec{r}(t) = R\cos{\theta(t)}\hat{i} + R\sin{\theta(t)}\hat{j} = R\underbrace{[\cos{\theta(t)}\hat{i} + \sin{\theta(t)}\hat{j}]}_{\hat{r}(t)}
= R\hat{r}(t),
    $$
    em que $\hat{r}(t)$ \'e um versor na dire\c c\~ao de $\vec{r}(t)$, isto \'e, um vetor com m\'odulo um. De fato, vamos verificar isto: 
    $$
    |\hat{r}(t)| = \sqrt[]{\cos^{2}(\theta(t)) + \sin^{2}(\theta(t))} = 1
    $$
    A seguir, vamos estudar como este versor $\hat{r}(t)$ varia, ou seja, vamos derivar este vetor com respeito ao tempo. Para isso,
  introduizremos outra regra de deriva\c c\~ao, a ``Regra da Cadeia''. Dada uma fun\c c\~ao $f(t) = u(v(t))$, ou seja, uma fun\c c\~ao
  definida como uma fun\c c\~ao composta, sua deriva\c c\~ao \'e feita de denro pra fora: Derivamos v(t) com respeito a t, depois derivamos
  u com rela\c c\~ao a v e multiplicamos, ou seja, 
    $$
    \boxed{\frac{df}{dt} = \frac{du}{dv}\frac{dv}{dt}}.
    $$
    Assim, no caso do versor $\hat{r}(t),$ 
    \begin{align*}
      \frac{d\hat{r}(t)}{dt} &= \frac{d}{dt}[\cos{\theta(t)}\hat{i} + \sin{\theta(t)}\hat{j}]\\
                             &= \frac{d}{dt}[\cos{\theta(t)}]\hat{i} + \frac{d}{dt}[\sin{\theta(t)}]\hat{j}\\
                             &= \frac{d\cos{\theta(t)}}{d\theta}\frac{d\theta(t)}{dt}\hat{i} + \frac{d\sin{\theta(t)}}{d\theta}\frac{d\theta(t)}{dt}\hat{j}\\
                             &= -\sin{\theta(t)}\frac{d\theta(t)}{dt}\hat{i} + \cos{\theta(t)}\frac{d\theta(t)}{dt}\hat{j}.
    \end{align*}
    Como $\theta(t)$ \'e a posi\c c\~ao angular, chamamos a sua derivada com respeito a tempo de velocidade \^angular 
      $$
      \boxed{\omega(t) = \frac{d\theta(t)}{dt}.}
      $$
    De brinde, conseguimos definir a velocidade escalar da part\'icula como 
   \begin{align*}
      \frac{ds(t)}{dt} &= \frac{dR\theta(t)}{dt} = R \frac{d\theta(t)}{dt} = R\omega(t).\\
      &\Rightarrow \boxed{v(t) = R\omega(t).}
   \end{align*}
   Vamos estudar a dimens\~ao dessa quantidade. Temos 
     $$
     [\omega] = \frac{[\theta]}{[t]} = \frac{1}{T} \quad \text{(Exemplo: rad/s (radianos por segundo.))}
     $$ 
    como unidades de velocidade angular e 
      $$
        [v] = [R][\omega] = LT^{-1} \quad \text{(Exemplo: m/s (metros por segundo))}
      $$
    como dimens\~ao da velocidade escalar. Com rela\c c\~ao ao versor definido, sua derivada \'e 
      $$
      \frac{d\hat{r}(t)}{dt} = \omega(t)\underbrace{[-\sin{\theta(t)}\hat{i}+\cos{\theta(t)}\hat{j}]}_{\hat{\theta(t)}},
      $$
      em que $\hat{\theta(t)}$ \'e um versor apontando na dire\c c\~ao do \^angulo. Com isso, definimos a velocidade vetorial por 
        $$
        \vec{v}(t) = \frac{d \vec{r}(t)}{dt} = R \frac{d\hat{r}(t)}{dt} \Rightarrow \vec{v}(t) = R\omega(t)\hat{\theta(t)}.
        $$
      Algo interessante de notar \'e que a velocidade escalar consiste do m\'odulo da velocidade vetorial, i.e., $v(t) = |\vec{v}(t)|.$
      Tamb\'em podemos representar a velocidade por meio das componentes em cada eixo: 
        $$
        \vec{v}(t) = \underbrace{-R\omega(t)\sin{\theta(t)}}_{v_{x}(t)}\hat{i} + \underbrace{R\omega(t)\cos{\theta(t)}}_{v_{y}(t)}\hat{j}.
        $$
  \subsection{Acelera\c c\~oes no Movimento Circular.}
      Agora que estamos mais familiariizados com a velocidade e posi\c c\~ao angular, podemos estudar a acelera\c c\~ao no movimento circular.
      Assim como antes, come\c camos definindo a acelera\c c\~ao angular: 
      $$
        \alpha(t) = \frac{d\omega(t)}{dt} = \frac{d^{2}\theta(t)}{dt^{2}},
      $$
      que possui dimens\~ao $[\alpha] = \frac{[\omega]}{[t]} = \frac{T^{-1}}{T} = T^{-2}$, sendo um exemplo a unidade $rad/s^{2}$, i.e.,
      radiano por segundo quadrado. Analogamente, definimos a acelera\c c\~ao vetorial por 
        $$
        \vec{a}(t) = \frac{d \vec{v}(t)}{dt} = \frac{d}{dt}[R\omega(t)\hat{\theta}(t)].
        $$
        Pela regra do produto, 
        $$
        \vec{a}(t) = R[\frac{d\omega(t)}{dt}\hat{\theta(t)} + \omega(t) \frac{d\hat{\theta(t)}}{dt}].
        $$
      Analisando termo a termo, a primeira derivada acontece no m\'odulo da velocidade, i.e., $R \frac{d\omega(t)}{dt}$, ou seja,
      representa a varia\c c\~ao no m\'odulo da velocdade. Por outro lado, o segundo termo representa a varia\c c\~ao da dire\c c\~ao 
      da velocidade. Como j\'a encontramos alguns desses termos antes, segue que 
        $$
        \vec{a}(t) = R[\alpha(t)\hat{\theta(t)} + v(t)\frac{d\hat{\theta}(t)}{dt}].
        $$
        Mas o que \'e este termo $\frac{d\hat{\theta}}{dt}$? Olhando pra ele com cuidado, vemos que 
       \begin{align*}
         \frac{d\hat{\theta}}{dt} &= \frac{d}{dt}[-\sin{\theta(t)}\hat{i} + \cos{\theta(t)}\hat{j}] \\
                                  &= -\frac{d}{dt}[\sin{\theta(t)}]\hat{i} + \frac{d}{dt}[\cos{\theta(t)}]\hat{j}\\
                                  &= -\cos{\theta(t)}\frac{d\theta(t)}{dt}\hat{i} -\sin{\theta(t)}\frac{d\theta(t)}{dt}\hat{j}\\
                                  &\Rightarrow \frac{d\hat{\theta(t)}}{dt} = \omega(t)[-\cos{\theta(t)}\hat{i} - \sin{\theta(t)}\hat{j}] = \omega(t)(-\hat{r}(t)).
       \end{align*}
     Portanto, 
       $$
         \vec{a}(t) = R[\alpha(t)\hat{\theta(t)} + \omega^{2}(t)(-\hat{r}(t))] = \underbrace{R\alpha(t)\hat{\theta(t)}}_{\text{acelera\c c\~ao tangencial }\vec{a}_{t}(t)} - \underbrace{R\omega^{2}(t)\hat{r}(t)}_{\text{acelera\c c\~ao centr\'ipeta }\vec{a}_{cp}(t)}
       $$
      Obtivemos disso tudo duas acelera\c c\~oes novas e que precisam ser mais compreendidas. Vamos come\c car pela tangencial.

      Com rela\c c\~ao ao m\'odulo da acelera\c c\~ao tangencial, note que $|\vec{a}_{t}(t)| = R\alpha(t) = \frac{dv(t)}{dt}$. Agora,
      quanto \`a acelera\c c\~ao centr\'ipeta, seu m\'odulo \'e dado por $|\vec{a}_{cp}(t)| = R\omega^{2}(t) \Rightarrow|\vec{a}_{cp}(t)| = \frac{v^{2}}{R}.$ 

  \subsection{Movimento Circular Uniforme}
      Resumindo o que temos at\'e o momento em forma de tabela, segue que 

    \begin{table}[h!]
    \centering
    \begin{tabular}{|c|c|c|}
        \hline
        & \textbf{Variáveis angulares} & \textbf{Variáveis escalares} \\
        \hline
        \textbf{Posição} & $\theta(t)$ & $s(t) = R\theta(t)$ \\
        \hline
        \textbf{Velocidade} & $\omega(t) = \frac{d \theta(t)}{dt}$ & $v(t) = \frac{d s(t)}{dt} = R\omega(t)$ \\
        \hline
        \textbf{Aceleração} & $\alpha(t) = \frac{d\omega(t)}{dt} = \frac{d^2\theta(t)}{dt^2}$ & $|\vec{a}(t)| = \frac{dv(t)}{dt} = R\alpha(t),\quad |\vec{a}_{cp}| = \frac{v^2}{R}$ \\
        \hline
    \end{tabular}
    \caption{Resumo movimento circular.}
    \label{tab:my_label}
  \end{table}
  No movimento circular uniforme, estudamos arcos iguais em tempos iguais, ou seja, 
    $$
      \Delta s_{1} = \Delta s_{2},\quad \Delta t_{1} = \Delta t_{2},
    $$
    tal que $\Delta \theta_{1} = \Delta \theta_{2}$. Al\'em disos, $\omega(t)\equiv \omega$ constante. Assim, 
      $$
      |\vec{v}(t)| = v(t) = R\omega(t)\equiv v,\text{ constante} \Rightarrow a_{t} = R\alpha(t) = 0.
      $$
    Com isso, as posi\c c\~oes s\~ao descritas por 
   \begin{align*}
     &\omega(t) = \omega \Rightarrow \theta(t) = \theta_{0} + \omega(t-t_{0})\\
     &v(t) = v \Rightarrow s(t) = s_{0} + v(t-t_{0})
   \end{align*}
   Neste caso, o movimento \'e peri\'odico, ou seja, ele volta a ter as mesmas propriedades ap\'os um per\'iodo T. Em forma matem\'atica, isso quer dizer que 
     $$
       \left\{\begin{array}{ll}
           \vec{r}(t + T) = \vec{r}(t)\\
           \vec{v}(t + T) = \vec{v}(t).
         \end{array}\right.
     $$ 
    Tendo isso em mente, definimos tamb\'em a frequ\^encia como o n\'umero de ocorr\^encias. Ele vale o inverso do per\'iodo T, i.e., $f = \frac{1}{T}$.
\newpage
\section{Aula 8 - 19/04/2023}
\subsection{Motiva\c c\~oes}
\begin{itemize}
  \item Come\c car os estudos de din\^amica
\end{itemize}

\subsection{Exemplo de MCU - 67 Tiples}
Suponha que a Terra tem velocidade angular $\omega$ constante, velocidade e acelera\c c\~ao angulares $\vec{v}_{\theta}(t), \vec{a}_{\theta}(t)$
e velocidade e acera\c c\~ao escalares $\vec{v}_{e}(t), \vec{a}_{e}(t)$.
\begin{center}
\begin{tikzpicture}[scale=3]
  \shade[ball color=blue!10!white,opacity=0.7] (0,0) circle (1cm);
  \shade[ball color=yellow!10!white,opacity=0.7] (0,0) circle (0.5cm);
  \draw[->,thick,red] (0,0) -- (1,0) arc (0:45:1) node[midway,above right]{$\theta$};
  \draw[dashed] (0,0) -- (-1,0);
  \draw[dashed] (0,0) -- (0,-1);
  \draw[dashed] (0,0) -- (0,1);
\draw[->, black](0,0) -- ({cos(45)}, {sin(45)}) ;
  \draw (0,0) -- (1,0) node[midway,above]{$r$};
  \draw (0.2,0) arc (0:45:0.2) node[midway,right]{$\phi$};
\end{tikzpicture}
\end{center}

 No equador, $R_{E}=R_{T}, \omega_{E}=\omega$, tal que 
   $$
      \left\{\begin{array}{ll}
         v_{E}=\omega_{E}R_{E} = \omega R_{T}\\
         a_{E_{cp}} = \frac{v_{E}^{2}}{R_{E}} = \frac{(\omega R_{T})^{2}}{R_{T}} = \omega^{2}R_{T}.
       \end{array}\right.
   $$
   Na latitude $\theta,$ $R_{\theta} = R_{T}\cos{\theta}, \omega_{\theta} = \omega$, de modo que 
     $$
        \left\{\begin{array}{ll}
           v_{\theta} = \omega R_{T}\cos{\theta}\\
           a_{\theta_{cp}} = \omega^{2}R_{T}\cos{\theta}
         \end{array}\right.
     $$
  Para a Terra dar uma volta em torno de si de novo, ela demora aproximadamente 24h. Assim, $T = 24h$ \'e o per\'iodo da Terra,
donde conclu\'imos que a frequ\^encia ser\'a $f = \frac{1}{T} = \frac{1}{86400}s^{-1}.$ Como $\omega = 2\pi f,$ segue que 
  $$
    \omega = \frac{2\pi}{86400} = 7.27 \cdot 10^{-5}rad/s.
  $$
  Pede-se: a) Quais s\~ao os valores de $v_{e}, a_{e}?$ b e d) Quais s\~ao as orienta\c c\~oes das acelera\c c\~oes? c) Quanto valem $v_{\theta}, a_{\theta}?$

  a.) Vemos que $v_{E} = 463.1m/s, a_{E} = 0.0337m/s^{2}, g = 9.8m/s^{2}$. Em particular, $a_{E} = 0.0034g.$
  
  b. e d.) O diagrama de v indica que o vetor acelera\c c\~ao aponta na vertical pra esquerda e levemente pra cima.

  c.) Temos $v_{\theta} = 379.4m/s, a_{\theta}=0.0276m/s^{2}.$

\subsection{Din\^amica e Leis de Newton}
\subsubsection{O que esperar}
 Quando estudamos os movimentos anteriores, est\'avamos vendo cinem\'atica, a descri\c c\~ao matem\'atica do movimento. No entanto,
nunca nos questionamos o que causa o movimento. Como ele surge, o que influencia-o, etc. Essa pergunta \'e respondida pela din\^amica,
q formula\c c\~ao matem\'atica que explic\'ita as causas do movimento. Ela nos fornece uma rela\c c\~ao entre as intera\c c\~oes, chamadas for\c cas,
que o corpo sofre e o seu movimento. A primeira formula\c c\~ao da din\^amica foi feita por Isaac Newton, sendo suas Leis nosso Ponto de partida.

\subsubsection{Leis de Newton}
  A primeira Lei de Newton, tamb\'em chamada de Lei da In\'ercia, afirma que 
  \begin{quote}
    ``Um corpo em repouso, ou em movimento retil\'ineo uniforme, permenecer\'a em seu estado de movimento a n\~ao ser que uma
    for\c ca externa atue sobre ele. ''
  \end{quote}
  Observe que velocidade constante significa que tanto seu m\'odulo ser\'a constante quanto a dire\c c\~ao o movimento precisa ser em linha reta.
  Uma consequ\^encia dessa Lei \'e que n\~ao tem distin\c c\~ao entre um corpo em repouso e um corpo se movendo com velocidade constante.
  Com isso, um sistema de referencial inercial ser\'a definido como um eixo de coordenadas que est\'a em repouso ou se movendo com
  velocidade constante.

  A segunda Lei de Newton surge para explicar como aparecem as for\c cas dentro do contexto da din\^amica, dizendo que 
 \begin{quote}
    ``A for\c ca resultante atuando em um corpo \'e igual \`a massa dele multiplicada pela acelera\c c\~ao''
 \end{quote}
  Matematicamente, isto significa que 
    $$
    \boxed{\vec{F}_{r} = m \cdot \vec{a} = m \cdot \frac{d \vec{v}}{dt}}
    $$
  O termo novo m \'e chamado de massa inercial, sendo interpretada como a grandeza f\'isica que expressa a resist\^encia do corpo
ao movimento. Quanto maior for a massa, maior vai ser a resist\^encia a se mover. De fato, se temos dois blocos de massas $m_{1}>m_{2},$ ent\~ao 
  $$
    \vec{a}_{1} = \frac{\vec{F}}{m_{1}},\quad \vec{a}_{2} = \frac{\vec{F}}{m_{2}} \Rightarrow a_{1} < a_{2}.
  $$
  A dimens\~ao dessa grandeza \'e $[m] = M$. No Sistema Internacional, a unidade de massa \'e o kilograma.

  Note que a for\c ca \'e uma grandeza vetorial que soma-se, ou seja, se h\'a v\'arias for\c cas agindo sobre um corpo, a resultante ser\'a
  a soma delas. Se temos for\c cas $\vec{F}_{21}, \vec{F}_{31}$ agindo sobre um corpo, ent\~ao a resultante ser\'a $\vec{F}_{R} = \vec{F}_{21} + \vec{F}_{31}.$
  Al\'em disso, suas coordenadas ser\~ao 
 \begin{align*}
   &x: F_{res}^{x} = -F_{21}\sin{\alpha} + F_{31}\sin{\beta}\\
   &y: F_{res}^{y} = -F_{21}\cos{\alpha} - F_{31}\cos{\beta}.
 \end{align*}

 \begin{center}
   \begin{tikzpicture}[scale=2]
    \draw[->,>=stealth, dashed] (-1.5,0) -- (1.5,0) node[right]{$x$}; % Draw the x-axis
    \draw[->,>=stealth, dashed] (0,-1.5) -- (0,1.5) node[above]{$y$}; % Draw the y-axis
    
    \draw (-0.75,-0.75) -- (0.75,-0.75) -- (0.75,0.75) -- (-0.75,0.75) -- cycle; % Draw the cube
    \draw[->,>=stealth] (0,0) -- (-1,-1) node[midway, below left]{$F_{21}$}; % Draw the F21 arrow
    \draw[->,>=stealth] (0,0) -- (0,-1) node[midway, right]{$mg$}; % Draw the gravity arrow
    \draw[->,>=stealth] (0,0) -- (1,-1) node[midway, below right]{$F_{31}$}; % Draw the F31 arrow
    
    \draw (0,-0.2) arc (-90:-45:0.2) node[midway, below]{$\beta$}; % Draw the angle beta
    
    \draw[dashed] (0,-0.75) -- (0,-1.5); % Draw the vertical dashed line for reference
    \end{tikzpicture}
  \end{center}
 A unidade da for\c ca \'e dada por $[F] = [ma] = MLT^{-2}$. No SI, sua unidade \'e $1kgms^{-2} = 1N$ o Newton.
 Um corpo ser\'a dito em equil\'ibrio quando a for\c ca resultante agindo sobre ele \'e nula, pois, neste caso, 
   $$
     \vec{F}_{res} = \sum\limits_{n}^{}\vec{F}_{n} = 0 \Rightarrow F_{res} = ma = 0 \Rightarrow a = 0.
   $$

  A terceira e \'ultima Lei de Newton \'e conhecida como Lei da A\c c\~ao e Rea\c c\~ao. Segue seu enunciado
 \begin{quote}
   ``Se um corpo faz uma for\c ca em outro, ent\~ao este segundo tamb\'em realizar\'a uma for\c ca no primeiro, sendo esta de
   mesmo m\'odulo, mas com dire\c c\~ao oposta. ''
 \end{quote}
  Em outras palavras, se um corpo 2 age sobre um corpo 1 com for\c ca $\vec{F}_{21},$ ent\~ao o corpo 1 far\'a uma for\c ca sobre
  o corpo 2 $\vec{F}_{12}$ tal que $\vec{F}_{21} = -\vec{F}_{12}, |\vec{F}_{21}| = |\vec{F}_{12}|$.

  \subsubsection{Exemplo 4.2 - Tipler}
  Os dados que temos \'e que h\'a uma pessoa que se moveu 2.25m em 3s e cuja massa \'e 68kg. Pede-se para encontrar o m\'odulo da for\c ca
agindo sobre ela. Segue que 
  $$
    x(t) = x_{0} + v_{0}t + \frac{1}{2}at^{2} \Rightarrow \Delta x = x(3) - x(0) = \frac{1}{2}a3^{2} = \frac{9}{2}a = 2.25m
  $$
  Isolando a equa\c c\~ao, encontra-se que $a = 0.5m/s^{2}.$ Com isso, como $|\vec{F}| = |m \vec{a}| = 68 \cdot 0.5 = 34N$. \qedsymbol
\end{document}
