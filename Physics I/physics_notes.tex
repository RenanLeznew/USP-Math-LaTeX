\documentclass{article}
\usepackage{amsmath}
\usepackage{multirow}
\usepackage{amsthm}
\usepackage{amssymb}
\usepackage{pgfplots}
\usepackage{amsfonts}
\usepackage[margin=2.5cm]{geometry}
\usepackage{graphicx}
\usepackage[export]{adjustbox}
\usepackage{fancyhdr}
\usepackage[portuguese]{babel}
\usepackage{hyperref}
\usepackage{lastpage}
\usepackage{mathtools}
\usepackage{tikz}
\usepackage{tikz-3dplot}
\usetikzlibrary{angles,quotes}
\usetikzlibrary{calc,arrows.meta,patterns}

\pagestyle{fancy}
\fancyhf{}

\pgfplotsset{compat = 1.18}

\hypersetup{
    colorlinks,
    citecolor=black,
    filecolor=black,
    linkcolor=black,
    urlcolor=black
}
\newtheorem*{def*}{\underline{Defini\c c\~ao}}
\newtheorem*{theorem*}{\underline{Teorema}}
\newtheorem*{lemma*}{\underline{Lema}}
\newtheorem*{prop*}{\underline{Proposi\c c\~ao}}
\newtheorem{example}{\underline{Exemplo}}
\newtheorem*{proof*}{\underline{Prova}}
\renewcommand\qedsymbol{$\blacksquare$}
\newcommand{\Lin}[1]{Lin_{\mathbb{K}}({#1})}

\rfoot{P\'agina \thepage \hspace{1pt} de \pageref{LastPage}}

\begin{document}
\begin{figure}[ht]
\minipage{0.76\textwidth}
  \includegraphics[width=4cm]{../icmc.png}
  \hspace{7cm}
  \includegraphics[height=4.9cm,width=4cm]{../brasao_usp_cor.jpg}
\endminipage  
\end{figure}

\begin{center}
\vspace{1cm}
\LARGE
UNIVERSIDADE DE S\~AO PAULO

\vspace{1.3cm}
\LARGE
INSTITUTO DE CI\^ENCIAS MATEM\'ATICAS E COMPUTACIONAIS - ICMC

\vspace{1.7cm}
\Large
\textbf{Notas de F\'isica}

\vspace{1.3cm}
\large
\textbf{Renan Wenzel - 11169472}

\vspace{1.3cm}
\large
\textbf{Patr\'icia Christina Marques Castilho - patricia.castilho@ifsc.usp.br}

\vspace{1.3cm}
\today
\end{center}

\newpage

\tableofcontents

\newpage

\section{Aula 00 - 23/03/2023}
  (Revis\~ao Unidades de Medidas)

\section{Aula 01 - 27/03/2023}
\begin{itemize}
  \item Revisar propriedades de derivadas;
  \item Aplicar derivadas em movimento 1D. 
\end{itemize}
\subsection{Movimentos 1D}
  Dada uma part\'icula com posi\c c\~ao descrita por $x = x(t)$, em que t \'e a vari\'avel de tempo, denotamos seu deslocamento
por $\Delta x = x_{2} - x_{1} = x(t_{2}) - x(t_{1}).$ Analogamente, o intervalo de tmepo \'e definido por $\Delta t = t_{2} - t_{1}$.
Com essas ferramentas, j\'a podemos definir a velocidade m\'edia de um objeto em uma dimens\~ao como $\vec{v} = \frac{\Delta x}{\Delta t}.$
Observe que, quanto menor o intervalo de tempo, mais moment\^aneo se torna essa defini\c c\~ao, de modo que a velocidade instant\^anea
pode ser encontrada como 
  $$
    \lim_{\Delta t\to0}\frac{x(t + \Delta t) - x(t)}{\Delta t} = \vec{v}(t).
  $$
Regras de derivadas:
\begin{align*}
  &f(t) = c \Rightarrow \frac{df}{dt} = 0 \text{ Derivada de uma constante \'e sempre nula;}\\
  &f(t) = x^{n} \Rightarrow \frac{df}{dt} = nx^{n-1} \text{ Regra do tombo;}\\
  &f(t) = A\sin{(t)} \Rightarrow \frac{df}{dt} = A\cos{(t)};\\
  &f(t) = B\cos{(t)} \Rightarrow \frac{df}{dt} = -B\sin{(t)};\\
  &f(t) = C e^{t} \Rightarrow \frac{df}{dt} = C e^{t}.
\end{align*}
\begin{example}
 \begin{align*}
   &i)f(t) = 3t^{4} + t^{2} \Rightarrow \frac{df}{dt} = 12t^{3} + 2t\\
   &ii) f(t) = 5\sin{(t)} + 3(t^{2}+1) = 5\sin{(t)} + 3t^{2} + 3 \Rightarrow \frac{df}{dt} = 5\cos{(t)} + 6t 
 \end{align*} 
\end{example}

  A partir deste ponto, tome t como tempo, x(t) como posi\c c\~ao e v(t) a velocidade instant\^anea.

\begin{tikzpicture}
\begin{axis}[
    axis lines = left,
    xlabel = \(t\),
    ylabel = {\(x(t)\)},
]
%Below the red parabola is defined
\addplot [
    domain=0:20, 
    samples=100, 
    color=red,
]
{3*x - 20};
\addlegendentry{\(x(t) = 3t - 20\)}
%Here the blue parabola is defined
\addplot [
    domain=20:40, 
    samples=100, 
    color=blue,
    ]
    {-3*x + 100};
\addlegendentry{\(x(t) = -3t + 100\)}
\end{axis}
\end{tikzpicture}

  Esse movimento em que a velocidade \'e descrita por uma linha reta \'e conhecido como movimento retil\'ineo uniforme, pois 
a velocidade $v(t)$ muda de forma linear, i.e., $\frac{dx}{dt} = c$, em que c \'e uma constante.

  Por outro lado, h\'a outro tipo de movimento, o movimento retil\'ineo uniformemente variado, em que a velocidade n\~ao \'e constante.
A a\c c\~ao respons\'avel por mudar a velocidade \'e conhecida como acelera\c c\~ao, e os gr\'aficos tendem a assumir o seguinte formato

\begin{tikzpicture}
\begin{axis}[
    axis lines = left,
    xlabel = \(t\),
    ylabel = {\(x(t)\)},
]
%Below the red parabola is defined
\addplot [
    domain=0:20, 
    samples=100, 
    color=red,
]
{-x^2 + 10*x};
\addlegendentry{\(x(t) = -x^2+10x\)}
%Here the blue parabola is defined
\end{axis}
\end{tikzpicture}

Ou, caso a velocidade cres\c ca com o tempo,

\begin{tikzpicture}
\begin{axis}[
    axis lines = left,
    xlabel = \(t\),
    ylabel = {\(x(t)\)},
]
%Below the red parabola is defined
\addplot [
    domain=0:20, 
    samples=100, 
    color=red,
]
{x^2};
\addlegendentry{\(x(t) = x^2\)}
%Here the blue parabola is defined
\end{axis}
\end{tikzpicture}

H\'a ainda o caso em que a velocidade cresce por um tempo e diminui depois, com gr\'aficos como o que segue
 
\begin{tikzpicture}
\begin{axis}[
    axis lines = left,
    xlabel = \(t\),
    ylabel = {\(x(t)\)},
]
%Below the red parabola is defined
\addplot [
    domain=0:20, 
    samples=100, 
    color=red,
]
{x^2 - 20*x};
\addlegendentry{\(x(t) = x^2 - 20x\)}
%Here the blue parabola is defined
\end{axis}
\end{tikzpicture}

Nestes casos, para calcular o deslocamento da particula, precisamos somar muito mais intervalos de tempo. Para isso, observe que
cada instante, a posi\c c\~ao da part\'icula pode ser encontrada multiplicando-se o intervalo de tempo pela velocidade instan\^anea, i.e., 
 $\Delta x_{i}' = v_{i}'\Delta t_{i}'$. Quebrando os intervalos desta forma, o deslocamento de um ponto a outro \'e denotado por
 $$
   \Delta x_{1, 2} = x(t_{2}) - x(t_{1})\approx \sum\limits_{k=1}^{N}\Delta x_{i}' = \sum\limits_{k=1}^{N}v_{i}'\Delta t_{i}'
 $$
 Assim como para a velocidade instant\^anea, quanto menor tomarmos o intervalo de tempo, mais preciso \'e o valor encontrado para $\Delta x_{1, 2}$,
o que indica uma boa oportunidade para o uso do limite novamente. Com isso, definimos
 $$
  x(t_{2}) - x(t_{1}) = \lim_{\Delta t'\to0} \sum\limits_{i=1}^{N}v(t_{i}')\Delta t_{i}' = \int_{t_{1}}^{t_{2}}v(t)dt
 $$
 Este \'ultimo s\'imbolo, chamado integral, descreve a \'area ``embaixo'' da curva da fun\c c\~ao f(t) dentro do intervalo $[t_{1}, t_{2}].$
Supondo que c e k s\~ao constantes quaisquer, seguem abaixo algumas das regras de integra\c c\~ao: 
\begin{align*}
  &i)f(t) = ct^{n} \Rightarrow \frac{df}{dt} = nct^{n-1} \Rightarrow F(t) = \frac{ct^{n+1}}{n+1}\text{ (Primitiva de f)}\\
  &ii) \int_{t_{1}}^{t_{2}}f(t)dt = F(t_{2}) - F(t_{1}) = \frac{c}{n+1}t_{2}^{n+1} - \frac{c}{n+1}t_{1}^{n+1}\text{ (Integral definida de f)}\\
  &iii) \int_{}^{}f(t)dt = \frac{c}{n+1}t^{n+1} + k\text{ (Integral indefinida de f)}
\end{align*}
  Para conferir se a integral est\'a correta, \'e preciso derivar a fun\c c\~ao F e, se obter como resultado a fun\c c\~ao f, significa que est\'a correto.
Com este conhecimento em mente, segue que 
  $$
  \boxed{ x(t) = \int_{}^{}v_{0}dt = v_{0}t + x_{0}}
  $$
  Algumas outras regras importantes:
 \begin{align*}
   &iv) \frac{d\sin{(t)}}{dt} = \cos{(t)} \Rightarrow \int_{}^{}\cos{(t)}dt = \sin{(t)} + c\\
   &v) \frac{d\cos{(t)}}{dt} = -\sin{(t)} \Rightarrow \int_{}^{}\sin{(t)}dt = \cos{(t)} + c\\
   &vi) \frac{d e^{t}}{dt} = e^{t} \Rightarrow \int_{}^{} e^{t}dt = e^{t} + c
 \end{align*}
  Ou seja, em certo sentido, a integral e a derivada s\~ao dois lados da mesma moeda, assim como mulitplica\c c\~ao e divis\~ao ou adi\c c\~ao e subtra\c c\~ao.
  \newpage

\section{Aula 02 - 29/03/2023}
\subsection{Motiva\c c\~oes}
\begin{itemize}
  \item Estudar a acelera\c c\~ao;
  \item Entender o Movimento Retil\'ineo Uniformemente Variado.
\end{itemize}
\subsection{Acelera\c c\~ao}
  Definimos previamente a velocidade m\'edia como sendo a varia\c c\~ao de tempo dividindo o deslocamento, sendo, portanto,
uma quantidade representando a taxa de varia\c c\~ao da posi\c c\~ao em um intervalo de tempo. De forma an\'aloga,
definimos a acelera\c c\~ao como a taxa de varia\c c\~ao da velocidade em um intervalo de tempo, ou seja, 
  $$
    \vec{a_{m}} = \frac{\Delta \vec{v}}{\Delta t}.
  $$
  Ainda mais, se ela for positiva, a velocidade aumenta. Caso contr\'ario, ela diminui. Ainda repetindo o processo feito
para o caso da velocidade, podemos encontrar uma acelera\c c\~ao instan\^anea como 
    $$
      a(t) = \lim_{\Delta t\to0}\biggl[\frac{v(t + \Delta t) - v(t)}{\Delta t}\biggr] = \frac{dv(t)}{dt}
    $$
  Observe tamb\'em que 
    $$
      a(t) = \frac{d}{dt}\biggl(\frac{dx(t)}{dt}\biggr) = \frac{d^{2}x}{dt^{2}}.
    $$
  Utilizando a an\'alise dimensional, \'e poss\'ivel encontrar a dimens\~ao da acelera\c c\~ao como $[a]=\frac{[v]}{[t]} = \frac{\frac{[L]}{[t]}}{[t]} = \frac{[L]}{[t]^{2}}.$ Assim,
se o sistema de medida for o Sistema Internacional, $[a] = \frac{m}{s^{2}}$.

    \begin{tikzpicture}
    \begin{axis}[
        axis lines = left,
        xlabel = \(t\),
        ylabel = {\(x(t)\)},
    ]
    \addplot [
        domain=0:10,
        samples=100, 
        color=red,
    ]
    {-x^2 + 10*x};
    \addlegendentry{\(x(t) = -x^{2}+10x\)}
    \addplot[
      domain=0:10,
      samples=100,
      color=blue,
    ]
    {-2*x + 10};
    \addlegendentry{\(v(t) = -2x + 10\)},
      \addplot[
        domain=0:10,
        samples=100,
        color=green,
      ]
      {2};
    \addlegendentry{\(a(t) = 2\)},
    \end{axis}
  \end{tikzpicture}

\subsection{Movimento Retil\'ineo Uniformemente Variado.}
  Sabendo que $a =\displaystyle \frac{d^{2}x(t)}{dt^{2}}$, podemos fazer o caminho oposto para encontrar uma f\'ormula para
  a posi\c c\~ao sabendo a acelera\c c\~ao. De fato, dado um intervalo de tempo $[t_{0}, t],$
  $$
    v(t) = \int_{t_{0}}^{t}a(t)dt = at \biggl|_{t_{0}}^{t}\biggr. = at - at_{0}
  $$
  Sabemos, tamb\'em, que $v(t) - v(t_{0}) = \Delta v$, tal que 
    $$
      v(t) = v(t_{0}) + a(t-t_{0}) = v_{0} + a(t-t_{0})
    $$
  Al\'em disso, vimos que 
    $$
      \Delta x = x(t) - x(t_{0}) = \int_{t_{0}}^{t}v(t) dt.
    $$
  Juntando tudo, segue a f\'ormula dita: 
  \begin{align*}
    x(t) - x(t_{0}) &= \overbrace{\int_{t_{0}}^{t}[v_{0} + a(t-t_{0})]dt}^{\int_{}^{}f(t) + g(t)dt = \int_{}^{}f(t)dt + \int_{}^{}g(t)dt}= \overbrace{\int_{t_{0}}^{t}v_{0}dt}^{\int_{}^{}c dt = ct} + \overbrace{\int_{t_{0}}^{t}atdt}^{\int_{}^{}t^{n}dt = \frac{t^{n+1}}{n+1}} - \int_{t_{0}}^{t}at_{0}dt\\
                    & \Rightarrow x(t) - x(t_{0})  =  v_{0}t \biggl|_{t_{0}}^{t}\biggr. + a\frac{t^{2}}{2} \biggl|_{t_{0}}^{t}\biggr. - at_{0}t \biggl|_{t_{0}}^{t}\biggr. \\
                    &= v_{0}(t-t_{0}) + a \frac{(t^{2} - t_{0}^{2})}{2} - at_{0}(t-t_{0}) \\
                    &= v_{0}(t-t_{0}) + a \frac{t^{2}-t_{0}^{2}}{2} - at_{0}t + at_{0}^{2} = v_{0}t - v_{0}t_{0} + \frac{a}{2}(t^{2}-2t_{0}t + 2t_{0}^{2})\\
                    &= v_{0}(t-t_{0}) + \frac{a}{2}(t-t_{0})^{2}\\
                    & \Rightarrow x(t) = x(t_{0}) + v_{0}(t-t_{0}) + \frac{1}{2}a(t-t_{0})^{2}.
  \end{align*}
  Com isso, no caso em que $t_{0} = 0$, segue que 
   $$
    \boxed{x(t) = x_{0} + v_{0}t + \frac{at^{2}}{2}}     
   $$
  Uma coisa not\'avel \'e que todas essas f\'ormulas est\~ao dependentes de tempo. No entanto, ser\'a que \'e poss\'ivel
se livrar dessa vari\'avel e relacionar, por exemplo, velocidade e posi\c c\~ao? A resposta \'e sim! E vamos mostrar como a seguir,
na equa\c c\~ao conhecida como Equa\c c\~ao de Torricelli. Com efeito,
 \begin{align*}
   &(I)\quad (t-t_{0}) = \frac{v(t)-v_{0}}{a} = \frac{v-v_{0}}{a}\\
   &(II)\quad x(t) = x_{0} + v_{0}(t-t_{0}) + \frac{1}{2}a(t-t_{0})^{2}\\
   &(I\text{ com }II)\quad x = x_{0} + v_{0}\frac{v-v_{0}}{a} + \frac{1}{2}a \frac{v-v_{0}}{a} \\
   & \Rightarrow x = x_{0} + \frac{1}{a}\biggl\{v_{0}v - v_{0}^{2} + \frac{1}{2}(v^{2}-2vv_{0}+v_{0}^{2})\biggr\}\\
   & = x_{0} + \frac{1}{a}\biggl\{-v_{0}^{2} + \frac{v^{2}}{2} + \frac{v_{0}^{2}}{2}\biggr\}\\
   & \Rightarrow x - x_{0} = \frac{1}{2a}\biggl[v^{2}-v_{0}^{2}\biggr] \Longleftrightarrow [v^{2} - v_{0}^{2}] = 2a(x-x_{0}).
 \end{align*}
 Portanto, chegamos na Equa\c c\~ao de Torricelli
  $$
  \boxed{v^{2} = v_{0}^{2} + 2a(x-x_{0})}
  $$
  Para refor\c car o que foi visto at\'e agora, vejamos um exemplo.
 \begin{example}
   Suponha que um carro freia uniformemente, passando de 60km/h para 30km/h em 5 segundos. Qual \'e a dist\^ancia
que o carro percorrer\'a at\'e parar? Em quanto tempo?

  \textbf{Solu\c c\~ao}: Sabemos que $x(t) = x_{0} + v_{0}(t-t_{0}) + \frac{1}{2}a(t-t_{0})^{2}, v(t) = v_{0} + a(t-t_{0}),
\text{ e }v^{2} = v_{0}^{2} + 2a(x-x_{0}).$ Al\'em disso, como \'e at\'e o carro parar, a velocidade final \'e 0km/h, a varia\c c\~ao
de tempo at\'e o momento em que a velocidade atinge 30km/h (=8.333m/s) \'e dada como $\Delta t = 5 - 0 = 5$s , sendo a velocidade inicial 60km/h (=16,666m/s). 
Pela equa\c c\~ao dois, 
  $$
    a = \frac{v(t_{1}) - v_{0}}{t_{1}-t_{0}} = \frac{8.33 - 16.66}{5} = -1.66\frac{m}{s^{2}}
  $$
  Agora, para obter a dist\^ancia, sendo $v_{2} = 0km/h$ o valor da acelera\c c\~ao no tempo em que o carro para (o segundo percurso),
utilizamos Torricelli para obter o deslocamento no peda\c co final do percurso
  $$
    v_{2}^{2} = v_{1}^{2} + 2a(x_{2}-x_{1}) \Rightarrow 0 = 8.33^{2} + 2(-1.66)\Delta x_{2}
  $$
  Assim, isolando o $\Delta x_{2},$  
    $$
    \Delta x_{2} = \frac{8.33^{2}}{3.32} = \text{ Professora vai passar na pr\'oxima aula.}
    $$
  Ademais, para encontrar todo o caminho que o carro andou, temos 
    $$
      0 = v_{0}^{2} + 2a(x_{2} - x_{0}) = 16.66^{2} + 2(-1.66)\Delta x \Rightarrow \Delta x = \frac{16.66^{2}}{3.32}
    $$
  Finalmente, o instante de tempo pode ser encontrado fazendo 
    $$
    v_{2}(t) = v_{1} + a(t_{2}-t_{0}) \Rightarrow 0 = 8.33 - 1.66\Delta t_{2} \Rightarrow \Delta t_{2} = 5s.\text{ \qedsymbol}
    $$
 \end{example}
 \newpage

\section{ Aula 03 - 30/03/2023}
\subsection{Motiva\c c\~oes}
 \begin{itemize}
   \item Resolu\c c\~ao de Exerc\'icios.
 \end{itemize}
 \subsection{Exerc\'icio 29 - Tipler}
 ``Considere a trajet\'oria de dois carros, o Carro A e o Carro B. (a) Existe algum instante para o qual os carros est\~ao lado-a-lado? (b) Eles viajam sempre no mesmo sentido? (c) Eles viajam com a mesma velocidade em algum instante t? (d) Para que t os carros est\~ao mais distantes entre si? (e) Esboce os gr\'aficos de $v\times t$'
 \begin{center}
   \begin{tikzpicture}
   \begin{axis}[
       axis lines = left,
       xlabel = \(t\),
       ylabel = {\(x(t)\)},
   ]
   \addplot [
       domain=0:10, 
       samples=100, 
       color=red,
   ]
   {7*x/2 + 9/3};
 \addlegendentry{\(\text{Carro B}\)}
   %Here the blue parabola is defined
     \addplot[
       domain=0:10,
       samples=100,
       color=blue,
     ]
     {-x^2 + 13*x};
   \addlegendentry{\(\text{Carro A}\)}
     %Here the blue parabola is defined
   \end{axis}
 \end{tikzpicture}
\end{center}
  Os carros se encontram lado-a-lado quando os gr\'aficos se cruzam, ou seja, em t = 1s e t = 9s (Tipler mais acurado que meu gr\'afico.). \'E not\'avel
quer eles n\~ao est\~ao sempre no mesmo sentido, visto que, a partir de 6s, o gr\'afico do carro B passa a mudar o sentido. Em aproximadamente 5s,
ambos est\~ao com a reta tangente iguais, ou seja, est\~ao com a mesma velocidade, e dist\^ancia entre eles est\'a maior exatamente no ponto em que as 
velocidades est\~ao iguais. Finalmente, seguem os gr\'aficos:
\begin{center}
    \begin{tikzpicture}
    \begin{axis}[
        axis lines = left,
        xlabel = \(t\),
        ylabel = {\(v(t)\)},
    ]
    \addplot [
        domain=0:6, 
        samples=100, 
        color=red,
    ]
    {-3*x + 18};
  \addlegendentry{\(\text{Carro A}\)}
    %Here the blue parabola is defined
    \addplot[
      domain=0:6,
      samples=100,
      color=red,
    ]
    {5};
  \addlegendentry{\(\text{Carro B}\)}
    %Here the blue parabola is defined
    \end{axis}
  \end{tikzpicture}
\end{center}

\subsection{Exerc\'icio 44 - Tipler}
``Um carro viaja em linha reta com $\vec{v} = 80\text{km/h} $ durante $\Delta t_{1}=2.5$h. Depois, $\vec{v_{2}} = 40$km/h, $\Delta t_{2} = 1.5$h. Qual
\'e o deslocamento total? E qual \'e a velocidade $\vec{v}$ total?''
\begin{align*}
  &(a)\quad \Delta x = \Delta x_{1} + \Delta x_{2} = \vec{v_{1}}\Delta t_{1} + \vec{v_{2}}\Delta t_{2}  \Rightarrow \Delta x = 260\text{km.}\\
  &(b)\quad \vec{v} = \frac{\Delta x}{\Delta t} = \frac{260}{4} = 65\text{km/h}.
\end{align*}

\subsection{Exerc\'icio 58 - Tipler}
  ``Um carro acelera de 48.3km/h para 80.5km/h em 3.70s. Qual a acelera\c c\~ao m\'edia?''

Primeiramente, precisamos converter as unidades para medidas iguais. Com isso, note que $\vec{v_1} = 48.3km/h = 13.52m/s, \vec{v_{2}} = 80.5km/h = 22.54m/s$. Assim,
chegamos em 
  $$
  \vec{a} = \frac{\Delta v}{\Delta t} = \frac{v_{2} - v_{1}}{\Delta t}\approx 2.4\text{m/s}.
  $$

\subsection{Exerc\'icio 67 - Tipler}
``Um corpo est\'a em uma posi\c c\~ao inicial $x_{1}$ com velocidade inicial $\vec{v_{1}}$. Passado um tempo, ele se encontra na posi\c c\~ao $x_{2}$ com velocidade $\vec{v_{2}}$. Qual \'e a acelera\c c\~ao deste corpo?''

Utilizaremos Torricelli. sabemos que 
 \begin{align*}
   &(1):\quad x_{1} = 6m, \vec{v_{1}} = 10m/s\\
   &(2):\quad x_{2} = 10m, \vec{v_{2}} = 15m/s.
 \end{align*}
 Deste modo, $v^{2} = v_{0}^{2} + 2a\Delta x \Rightarrow v_{2}^{2} = v_{1}^{2} + 2a(x_{2} - x_{1}) \Rightarrow a \approx 16m/s^{2}$

\subsection{Exerc\'icio 72 - Tipler}
  ``Um parafuso se desprende de um elevador subindo a $v_{0} = 6m/s$. O parafuso atinge o fundo do po\c co em 3s. (a) Qual era a altura do elevador? (b) Qual \'e a velocidade do parafuso no ch\~ao? Tome g = 9.8 $m/s^{2}$''

Sabemos que $t_{0} = 0s, y(t_{0}) = h, v(t_{0}) = v_{0}.$ Com isso, podemos descrever $y(t) = h + v_{0}t + \frac{1}{2}gt^{2}$. Vamos responder, agora, o item a, isto \'e, qual \'e o valor
da altura h? Segue que, em $t=3s, y(t) = 0$. Utilizando a f\'ormula, 
  $$
    h = -v_{0}t + \frac{1}{2}gt^{2} = -6 \cdot3 + \frac{1}{2}9.8 \cdot 3^{2} = 26.1m
  $$
  Com rela\c c\~ao ao item (b), vimos que $v(t) = v_{0} + at.$ Deste modo, 
    $$
      v(3s) = 6 - 9.8 \cdot 3 = -23.4m/s
    $$
  Indo um pouco al\'em do que foi pedido, analisemos o movimento do parafuso. \'E poss\'ivel concluir que o parafuso atingir\'a a altura 
m\'axima no instante em que $t^{*} = \frac{v_{0}}{g} = 0.6s,$ visto que este momento ocorre quando $v(t) = v_{0} - gt = 0$. Com isso,
conclui-se que a altura m\'axima \'e $y(t^{*}) = h + v_{0}t^{*} - \frac{1}{2}gt^{*^{2}} \approx 27.5m.$ No gr\'afico,
  \begin{center}
    \begin{tikzpicture}
    \begin{axis}[
        axis lines = left,
        xlabel = \(t\),
        ylabel = {\(y(t)\)},
    ]
    \addplot [
      domain = 0:4,
      color = black,
      ]
    {0};
    \addplot [
        domain=0:3, 
        samples=100, 
        color=red,
    ]
    {-9.8*x^2/2 + 6*x + 26.1};
    \addlegendentry{\(y(t)\)}
    %Here the blue parabola is defined
      \addplot[
        domain=0:3,
        samples=100,
        color=blue,
      ]
      {27.955};
    \addlegendentry{\(h_{\text{max}} = \frac{dy}{dt} = 0\)}
      %Here the blue parabola is defined
    \end{axis}
  \end{tikzpicture}
\end{center}

\subsection{Exemplo - Aula 06 Vanderlei}
  ``Suponha que h\'a um trem parado no instante t=0 com acelera\c c\~ao a. Passados 6s, um passageiro chega ao local e observa o trem na posi\c c\~ao $x_{trem_{1}}$.
  Este passageiro sai correndo com velocidade $v_{0}$ para tentar alcan\c car o trem. Qual \'e a velocidade m\'inima que o passageiro precisa atingir
  para alcan\c c\'a-lo?''
    
  Com rela\c c\~ao ao trem, suas condi\c c\~oes iniciais s\~ao $t_{0} = 0, x_{trem} = 0, v_{trem} = 0,$ tal que $x_{trem}(t) = \frac{1}{2}at^{2}$. 
Por outro lado, quanto ao passageiro, quando $t=6s, x_{p} = 0$, de modo que $x_{p}(t) = x_{p_0} + v_{0}t$. Como temos a informa\c c\~ao da posi\c c\~ao
do passageiro aos 6s, 
  $$
    x_{p}(6) = x_{p_{0}} + v_{0} \cdot6 = 0 \Rightarrow x_{p_{0}} = -6v_{0} \Rightarrow x_{p}(t) = v_{0}(t-6).
  $$
  No momento em que o passageiro alcan\c ca o trem, eles possuem posi\c c\~oes iguais, isto \'e, $x_{p}(t) = x_{trem}(t)$. Graficamente,
 \begin{center}
     \begin{tikzpicture}
     \begin{axis}[
         axis lines = left,
         xlabel = \(t\),
         ylabel = {\(x(t)\)},
     ]
     \addplot [
         domain=0:12, 
         samples=100, 
         color=red,
     ]
     {x^2};
   \addlegendentry{\(\text{Trem}\)}
       \addplot[
         domain=0:6,
         samples=100,
         color=blue,
       ]
       {0};
     \addlegendentry{\(\text{Passageiro}\)}
       \addplot[
         domain=6:12,
         samples=100,
         color=green,
       ]
       {25*x-150};
       \addlegendentry{\(\frac{dx_{trem}}{dt} = \frac{dx_{p}}{dt}\)}
     \end{axis}
   \end{tikzpicture}
 \end{center}
  Ou seja, buscamos $t^{*}$ tal que $x_{p}(t^{*}) = x_{trem}(t^{*}), v_{p}(t^{*}) = v_{trem}(t^{*})$. Com efeito,
 \begin{align*}
   v_{0}(t^{*} - 6) &= \frac{at^{*^{2}}}{2} \Rightarrow v_{0} = at^{*} \Rightarrow t^{*} = \frac{v_{0}}{a}\\
                    &v_{0} = \frac{a}{2}\frac{(\frac{v_{0}}{a})^{2}}{\frac{v_{0}}{a}-6} \Rightarrow \frac{v_{0}^{2}}{2a} = 6v_{0} \Rightarrow v_{0} = 12a.
 \end{align*}
 Outra forma de resolver \'e utilizando o fato de que quando $\frac{dv}{dt} = 0$, a fun\c c\~ao est\'a num m\'inimo. Ou seja, basta encontrar
o valor m\'inimo de $v_{0}$ que satisfa\c ca o que buscamos. Temos 
  $$
    v_{0}(t-6) = \frac{at^{2}}{2} \Rightarrow v_{0}(t) = \frac{at^{2}}{2}\frac{1}{(t-6)}.
  $$ 
  Agora, derivando essa equa\c c\~ao para $v_{0},$ 
    $$
      \frac{dv_{0}}{dt} = \frac{d}{dt}\biggl(\frac{at^{2}}{2}\frac{1}{t-6}\biggr) = \frac{d}{dt}(f(t)g(t)),
    $$
    em que $f(t) = \frac{at^{2}}{2}, g(t) = (t-6)^{-1}$. Fazemos isso porque h\'a uma regra para derivar o produto de fun\c c\~oes,
  a Regra do Produto 
    $$
      \boxed{\frac{df(t)g(t)}{dt}= g(t)\frac{df(t)}{dt} + f(t)\frac{dg(t)}{dt}}
    $$
    Derivando individualmente f e g, 
      $$
        \frac{df(t)}{dt} = at, \quad \frac{dg(t)}{dt} = -(t-6)^{-2} = -\frac{1}{(t-6)^{2}}.
      $$
    Agora, vamos juntar tudo para obter a derivada de $v_{0}$: 
    \begin{align*}
      \frac{dv_{0}}{dt} &= \frac{df(t)}{dt}g(t) + \frac{dg(t)}{dt}f(t) = \frac{at}{t-6} - \frac{1}{2(t-6)^{2}}at^{2}\\
                        &= at\biggl(\frac{1}{t-6} - \frac{t}{2(t-6)^{2}}\biggr) = 0 \\
                        & \Rightarrow \frac{1}{t-6} = \frac{t}{2(t-6)} \Rightarrow 1 = \frac{t}{2(t-6)}\\
                        & \Rightarrow 2(t-6) = t \Rightarrow 2t - t = 12 \Rightarrow t = 12s.
    \end{align*}
\newpage

\section{Aula 04 - 10/04/2023}
\subsection{Motiva\c c\~oes}
 \begin{itemize}
   \item Iniciar os estudos de movimentos em um plano todo (duas dimens\~oes);
   \item Revisar vetores e sua manipula\c c\~ao.
 \end{itemize}

\subsection{Vetores}
  Come\c camos com um estudo das propriedades de veotres. Dados vetores $\vec{r_{1}}, \vec{r_{2}}$ e um n\'umero real $\lambda$, definimos:
 \begin{itemize}
  \item[i)] A soma dos vetores:
  \begin{center}
    \begin{tikzpicture}
      \coordinate (O) at (0,0); % origin
      \draw [->,red] (O) -- (2,1) node [right] {$\vec{r}_1$}; % vector r_1
      \draw [->,blue] (2,1) -- (3,3) node [right] {$\vec{r}_1$}; % vector r_1
      \draw [->,blue] (O) -- (1,2) node [above] {$\vec{r}_2$}; % vector r_2
      \draw [->,red] (1,2) -- (3,3) node [above] {$\vec{r}_1$}; % vector r_2
      \draw [->,black] (O) -- (3,3) node [above right] {$\vec{r}_1+\vec{r}_2$}; % sum of vectors
    \end{tikzpicture}
  \end{center}
  \item[ii)] A multiplica\c c\~ao por escalar: $\lambda(r_{1}+r_{2})$ (Essencialmente, o resultado \'e aumentar ou diminuir o tamanho da seta.)
    \begin{center}
      \begin{tikzpicture}
        \coordinate (O) at (0,0); % origin
        \draw [->,red] (O) -- (2,1) node [right] {$\vec{v}$}; % vector v
        \draw [->,blue] (O) -- (4,2) node [right] {$\lambda\vec{v}$}; % scaled vector
        \draw [dashed] (2,1) -- (4,2); % dotted line to show scaling
      \end{tikzpicture}
    \end{center}
 \end{itemize}
 A t\'itulo de curiosidade, a soma de vetores em tr\^es dimens\~oes seria desta forma: 
 \begin{center}
   \tdplotsetmaincoords{70}{135} %set the viewing angle
\begin{tikzpicture}[scale=2, tdplot_main_coords]
    \coordinate (O) at (0,0,0);
    \draw[thick,->] (O) -- (2,0,0) node[anchor=north east]{$x$};
    \draw[thick,->] (O) -- (0,2,0) node[anchor=north west]{$y$};
    \draw[thick,->] (O) -- (0,0,2) node[anchor=south]{$z$};

    \draw[->,red] (O) -- (1.5,0.5,1.5) node[midway, below right] {$\vec{u}$};
    \draw[->,blue] (1.5,0.5,1.5) -- (2,2,2) node[midway, above left] {$\vec{v}$};
    \draw[->,blue] (O) -- (0.5,1.5,0.5) node[midway, above] {$\vec{v}$};
    \draw[->,red] (0.5,1.5,0.5) -- (2,2,2) node[midway, below] {$\vec{u}$};
    \draw[->,black] (O) -- (2,2,2) node[midway, above right] {$\vec{u}+\vec{v}$};
\end{tikzpicture}
\end{center}
 Por\'em, n\~ao basta utilizar apenas representa\c c\~oes gr\'aficas para vetores. Desta forma, \'e comum definirmos um sistema de coordenadas
cartesiano para suas componentes. Assim, um vetor $\vec{u}$ pode ser decomposto em uma coordenada x e outra coordenada y: 
  $$
  \vec{u} = u_{x}\hat{i} + u_{y}\hat{j} (+u_{z}\hat{k})
  $$
  chamamos os valores $u_{x}, u_{y}, u_{z}$ de proje\c c\~oes, sendo a \'ultima um objeto presente apenas no caso de tr\^es coordenadas.
Com isso, definimos o m\'odulo do vetor, ou seja, seu tamanho, pela f\'ormula 
  $$
    |\vec{u}| = \sqrt{u_{x}^{2}+u_{y}^{2}},
  $$
  e, de brinde, ganhamos f\'ormulas para as proje\c c\~oes em cada coordenada:
 \begin{align*}
   &u_{x} = |u|\cos{\theta} \Rightarrow \cos{\theta} = \frac{u_{x}}{|\vec{u}|}\\
   &u_{y} = |u|\sin{\theta} \Rightarrow \sin{\theta} = \frac{u_{y}}{|\vec{u}|}\\
   &\tan{\theta} = \frac{u_{y}}{u_{x}}.
 \end{align*}
 \'E importante, tam\'em, darmos uma forma de obter um betor de m\'odulo 1, i.e., um vetor unit\'ario, visto que ele pode nos fornecer a informa\c c\~ao do
valor do \^angulo $\theta$, a dire\c c\~ao, etc. Ele \'e obtido reduzindo um vetor u pelo seu m\'odulo, 
  $$
  \hat{u} = \frac{\vec{u}}{|\vec{u}|}.
  $$
  Uma utilidade imediata da defini\c c\~ao em coordenadas \'e que agora temos um modo de tratar a soma de vetores algebricamente 
    \begin{align*}
     &\text{Soma: } \vec{u}+\vec{v} = (u_{x}\hat{i} + u_{y}\hat{j}) + (v_{x}\hat{i} + v_{y}\hat{j}) = (u_{x}+v_{x})\hat{i} + (u_{y}+v_{y})\hat{j}\\
     &\text{Multiplica\c c\~ao por Escalar: } \lambda \vec{u} = \lambda(u_{x}\hat{i} + u_{y}\hat{j}) = \lambda u_{x}\hat{i} + \lambda u_{y}\hat{j}\\
     &\theta = ctg{\biggl(\frac{\lambda u_{y}}{\lambda u_{x}}\biggr) = ctg{\biggl(\frac{u_{y}}{u_{x}}\biggr)}}.
    \end{align*}
    Agora podemos ir \`a aplica\c c\~ao f\'isica dessa discuss\~ao, o deslocamento de uma part\'icula no plano. Nesta configura\c c\~ao, normalmente
  ter\'a-se uma part\'icula com posi\c c\~ao $\vec{x}(t) = x(t)\hat{i} + y(t)\hat{j}\quad (+z(t)\hat{k}).$ Para realizar o estudo desses casos,
  vamos decompor o movimento dela em cada eixo, ou seja, quebramos o movimento no plano em dois movimentos independentes, um em cada eixo
  x ou y. Nestas condi\c c\~oes, o deslocamento de uma part\'icula de uma posi\c c\~ao 1 at\'e uma posi\c c\~ao 2 ser\'a 
    $$
    \vec{x_{2}} - \vec{x_{1}} = (x_{2}\hat{i} + y_{2}\hat{j}) - (x_{1}\hat{i}+y_{1}\hat{j}) = (x_{2}-x_{1})\hat{i} + (y_{2}-y_{1})\hat{j}.
    $$
    Com isso, podemos escrever que o deslocamento $\Delta \vec{r}$ \'e 
      $$
      \Delta \vec{r} = \Delta x\hat{i} + \Delta y\hat{j}.
      $$
    Ainda mais, se conhecemos o valor do \^angulo entre as posi\c c\~oes 1 e 2 e o m\'odulo dos vetores representando-as, 
      $$
        |\Delta r|^{2} = x_{1}^{2} + x_{2}^{2} - 2x_{1}x_{2}\cos{\theta}.
      $$
    Tendo o b\'asico do deslocamento, podemos repetir o racioc\'inio pr\'evio para trabalhar com acelera\c c\~ao e velocidade. De fato, 
      $$
        \vec{v_{m}} = \frac{\Delta \vec{r}}{\Delta t}, \quad \vec{a_{m}} = \frac{\Delta \vec{v_{m}}}{\Delta t}
      $$
      e os valores instant\^aneos ser\~ao dados por 
      \begin{align*}
        &\vec{v}(t) = \frac{d \vec{r}(t)}{dt} = \frac{d x(t)}{dt}\hat{i} + \frac{d y(t)}{dt}\hat{j} = v_{x}\hat{i} + v_{y}\hat{j}.\\
        &\vec{a}(t) = \frac{d \vec{v}(t)}{dt} = \frac{d v_{x}(t)}{dt}\hat{i} + \frac{d v_{y}(t)}{dt}\hat{j} = a_{x}\hat{i} + a_{y}\hat{j}.
      \end{align*}
      Al\'em disso, o m\'odulo e orienta\c c\~ao desses valores ser\~ao dados por 
     \begin{align*}
       &|\vec{v}| = \sqrt{v_{x}^{2}+v_{y}^{2}},\quad \theta_{v} = ctg\biggl(\frac{v_{y}}{v_{x}}\biggr)\\
       &|\vec{a}| = \sqrt{a_{x}^{2}+a_{y}^{2}},\quad \theta_{a} = ctg\biggl(\frac{a_{y}}{a_{x}}\biggr).
     \end{align*}
     Note que a acelera\c c\~ao n\~ao aponta na dire\c c\~ao da velocidade em si, mas sim na dire\c c\~ao da \textbf{varia\c c\~ao} da velocidade.

\subsection{Movimento Uniforme Bidimensional}
    Considere uma part\'icula com posi\c c\~ao $\vec{r}(t)$ e uma orienta\c c\~ao, tal que forma um \^angulo $\theta$ com o plano. Como
  estaremos considerando o movimento do tipo uniforme, a acelera\c c\~ao \'e nula e a velocidade $\vec{v}(t) = v_{0}$ \'e constante,
  tendo m\'odulo $v_{0}$ e orienta\c c\~ao $\theta$. Em outras palavras, as componentes desse vetor ser\~ao, tamb\'em, constantes, isto \'e, 
    $$
    \text{constantes}\left\{\begin{array}{ll}
          v_{x}(t) = v_{x_{0}}\\
          v_{y}(t) = v_{y_{0}}.
        \end{array}\right.
    $$
    Desta forma, a decomposi\c c\~ao da velocidade em coordenadas \'e tal que 
   \begin{align*}
     &\text{Eixo x: }v_{x}(t) = v_{x_{0}} \Rightarrow x(t) = x_{0} + v_{x_{0}}(t-t_{0}),\quad x_{0} = x(t_{0})\\
     &\text{Eixo y: }v_{y}(t) = v_{y_{0}} \Rightarrow y(t) = y_{0} + v_{y_{0}}(t-t_{0}),\quad y_{0} = y(t_{0}).
   \end{align*}
   Logo, a posi\c c\~ao da part\'icula no plano ser\'a dada por 
     $$
     \vec{r}(t) = x(t)\hat{i} + y(t)\hat{j} = (x_{0} + v_{x_{0}}(t-t_{0})\hat{i}) + (y_{0} + v_{y_{0}}\hat{j}).
     $$
  Note que, quando falamos de trajet\'oria de uma part\'icula ou objeto, buscamos uma rela\c c\~ao entre as componentes x(t) e y(t) que
independe do tempo, isto \'e, a rela\c c\~ao temporal \'e dada de forma implicita. Uma forma de fazer isso \'e a atrav\'es da tangente, pois 
  $$
  \frac{y(t) - t_{0}}{x(t) - x_{0}} = \frac{v_{y_{0}}(t-t_{0})}{v_{x_{0}}(t-t_{0})} = \frac{v_{y_{0}}}{v_{x_{0}}} = \tan{\theta_{0}}.
  $$
  Com isso, 
    $$
    y(t) - y_{0} = \tan{\theta_{0}}(x(t)-x_{0}) \Rightarrow y = \tan{(\theta_{0})}x - \tan{(\theta_{0})}x_{0} + y_{0},
    $$
    ou seja, y tem a forma de um elua\c c\~ao da reta com inclina\c c\~ao constante e igual a $\tan{\theta_{0}}.$
\newpage

\section{Aula 05 - 12/04/2023}
\subsection{O que esperar?}
\begin{itemize}
  \item Lançamento de Projéteis;
  \item Alcance e altura máximos.
\end{itemize}
\subsection{Lançamento de Projéteis}
  Para descrevermos a trajetória de um objeto lançado, utilizamos a equação 
    $$
    y = v_{0}\sin{(\theta )}\frac{x}{v_{0}\cos{(\theta _0)}j} - \frac{1}{2}g\biggl[\frac{x}{v_{0}\cos{(\theta _0)}}\biggr]^{2} = \tan{(\theta _{0})}x-\frac{1}{2}\frac{g}{v_{0}\cos{(\theta _{0})}}x^{2}.
    $$
  Desta equação, podemos deduzir algumas informações sobre o movimento de tal objeto. Por exemplo, qual é a altura máxima
  que um objeto atingirá? E onde ele estará neste instante? Quanto tempo levará para atingir este ponto máximo de altura?  
  Vamos responder cada uma dessas situações. Para a terceira, note que, ao chegar no ponto máximo, $v_{y}=0.$ Assim, 
    $$
    0 = v_{0}\sin{(\theta_{0})}-gt_{m} \Rightarrow \boxed{t_{m} = \frac{v_{0}\sin{(\theta_{0})}}{g}.}
    $$
    Sabendo o tempo, podemos encontrar o valor dela com reação ao eixo y: 
      $$
      y_{m} = v_{0}\sin{(\theta )}\biggl[\frac{v_{0}\sin{(\theta )}}{g}\biggr] - \frac{1}{2}g\biggl[\frac{v_{0}\sin{(\theta )}}{g}\biggr]^{2} \Rightarrow 
      $$
      $$
      \boxed{y_{m}=\frac{v_{0}^{2}\sin{(\theta )}}{g}}
      $$
    Tendo estas duas informações, somos capazes de encontrar a informação restante - qual o valor da posição no eixo x. 
      $$
      x_{m} = v_{0}\cos{(\theta )}\frac{v_{0}\sin{(\theta )}}{g} \Rightarrow \boxed{x_{m} = \frac{v_{0}^{2}}{g}\cos{(\theta )}\sin{(\theta )}}
      $$
    Ao lançar um objeto, não apenas terá uma altura máxima, mas alguma hora ele atingirá o chão ou um obstáculo (espero). Esse valor é
    conhecido como alcance e descreve, como indica o nome, o alcance que o arremesso terá. Com um processo similar ao da
    altura máxima, podemos encontrar valores para o tempo que ele leva até atingir este alcance e a posição. A ideia por trás deste
    raciocínio é pegar o caso em que o objeto atinge o chão, pois, assim, teremos y=0. Com isso, 
      $$
        0 = v_{0}\sin{(\theta )}t - \frac{1}{2}gt^{2} \Rightarrow t_{a}(v_{0}\sin{(\theta )}-\frac{1}{2}gt)
      $$
    Para esta equação mais à direita zerar, duas coisas podem ocorrer - Ou $t_{a} = 0$, ou $v_{0}\sin{(\theta )} - \frac{1}{2}gt_{a}=0.$
    Portanto, 
      $$
      \boxed{t_{a} = \frac{2v_{0}\sin{(\theta )}}{g}}
      $$
    E quanto à posição? Agora que encontramos o valor do tempo, é possível resolver este problema sem muitas complicações, pois 
      $$
        x_{a} = v_{0}\cos{(\theta )}2\frac{v_{0}\sin{(\theta )}}{g} \Rightarrow x_{a} = \frac{2v_{0}^{2}}{g}\cos{(\theta )}\sin{(\theta )} = 2x_{m}.
      $$
    Isso responde uma parte das nossas questões com relação ao deslocamento no espaço. No entanto, além disso, como o ângulo que
    lançamos o objeto altera o alcance? 

    Dada uma função qualquer f(t), no ponto p' tal que f(t') tenha seu valor máximo, podemos utilizar a derivada, especificamente
    o momento em que a derivada é nula. Neste raciocínio, o anglo máximo pode ser encontrado através de 
      $$
        \frac{dx_{a}}{d\theta _{0}} \biggl|_{\theta_{0}=\theta_{ao}}^{}\biggr. = 0.
      $$
    Logo, utilizando $x_{a}(\theta_{0}) = \frac{2v_{0}^{2}}{g}\cos{(\theta_{0})}\sin{(\theta_{0})}$ juntamente da regra do produto,
   \begin{align*}
     \frac{dx_{a}}{d\theta _{0}} &= -\frac{2v_{0}^{2}}{g}\sin{(\theta_{0})}\cdot \sin{(\theta_{0})} + \frac{2v_{0}^{2}}{g}\cos{(\theta_{0})}\cos{(\theta_{0})}\\
                                 &= \biggl[\frac{2v_{0}^{2}}{g}(-\sin^{2}{(\theta_{0})}+\cos^{2}{(\theta_{0})})\biggr]\biggl|_{\theta_{0}=\theta_{am}}^{}\biggr. = 0
                                 &\Rightarrow \sin^{2}{(\theta_{am})}=\pm\cos^{2}{(\theta_{am})} \Rightarrow \boxed{\theta =45^{\circ}}
   \end{align*}
    Tendo as informações de deslocamento, conseguimos, por fim, encontrar a velocidade com que atingirá o solo. Temos
    $t_{A} = \frac{2v_{0}\sin{(\theta_{0})}}{g}$, tal que 
   \begin{align*}
     &v_{x}(t_{a}) = v_{x_{0}} = v_{0}\cos{(\theta_{0})}\\
     &v_{y}(t_{a}) = v_{0}\sin{(\theta_{0})}-gt_{a} = v_{0}\sin{(\theta )} - \frac{g2v_{0}\sin{(\theta _0)}}{g}\\
     &\Rightarrow  v_{y}(t_{a}) = -v_{0}\sin{(\theta )} = -v_{y_{0}} \Rightarrow \vec{v}(t_{a}) = v_{0}\cos{(\theta_{0})}\hat{i} - v_{0}\sin{(\theta_{0})}\hat{j} \\
     & |\vec{v}(t_{a})| = \sqrt[]{v_{0}^{2}[\cos^{2}{(\theta )}+\sin^{2}{(\theta )}]} = v_{0}.
   \end{align*}

   \subsection{Movimento Uniformemente Variado em Duas Dimensões}


\newpage
\section{Aula 06 - 13/04/2023}
\subsection{Motiva\c c\~oes}
\begin{itemize}
\item Revisar movimento relativo;
\end{itemize}

\subsection{Movimento Relativo}
  Fixada uma origem O, a soma dos vetores representando os corpos A e B
    $$
      \vec{r}_{AO} + \vec{r}_{BA} = \vec{r}_{BO}
    $$
  nos fornece a dire\c c\~ao relativa do corpo B com rela\c c\~ao a A. Se A e B est\~ao se movendo, ou seja, os vetores deles 
possuem depend\^encia no tmepo ($\vec{r}_{AO} = \vec{r}_{AO}(t), \vec{r}_{BO} = \vec{r}_{BO}(t)$), ent\~ao 
  $$
    \vec{r}_{BA}(t) = \vec{r}_{BO}(t) + \vec{r}_{AO}(t),
  $$
  ou seja, a posi\c c\~ao relativa de B com rela\c c\~ao a A tamb\'em depender\'a do tempo. Al\'em de posi\c c\~ao relativa,
podemos definir outros conceitos, tais como a velocidade relativa: 
  $$
    \vec{v}_{BA}(t) = \frac{d \vec{r}_{BA}(t)}{dt} = \frac{d \vec{r}_{BO}}{dt} + \frac{d \vec{r}_{AO}}{dt} \Rightarrow \vec{v}_{BA}(t) = \vec{v}_{BO}(t)  + \vec{v}_{AO}(t)
  $$
  e acelera\c c\~ao relativa de modo an\'alogo, i.e., $\vec{a}_{BA}(t) = \vec{a}_{BO}(t) + \vec{a}_{AO}(t).$ Vejamos alguns exemplos
 \begin{example}
   Considere um sistema em que um carrinho viaja com velocidade $\vec{v}_{r}$ e tem um passageiro $\vec{v}_{p}$ com ele. 
Ambos se movem para a direita. 
  Neste caso, h\'a o sistema referencial de in\'erca da pessoa dentro do trem. Buscamos descobrir a velocidade da pessoa com rela\c c\~ao ao trem. 
De fato, segue que 
  $$
    \vec{v}_{p} = \vec{v}_{PT} + \vec{v}_{T}.
  $$
 \end{example}
 \begin{example}
  Considere um sistema an\'alogo ao anterior, mas, embaixo, h\'a uma plataforma se movendo para a esquerda com velocidade igual \`a do trem.
  Neste caso, h\'a o sistema referencial de in\'erca da pessoa dentro do trem. Buscamos descobrir a velocidade da pessoa com rela\c c\~ao
\`a plataforma. Obtemos
  $$
  \vec{v}_{PT} = \vec{v}_{PT}^{x}\hat{i} + \vec{v}_{PT}^{y}\hat{j}. \Rightarrow \vec{v}_{p} = (\vec{v}_{PT}^{x} + \vec{v}_{T})\hat{i} + \vec{v}_{P}^{y}\hat{j}
  $$
 \end{example}
\begin{example}
  (Exemplo 32 do Tipler): Considere um sistema de avi\~ao e vento, no qual o m\'odulo da velocidade do avi\~ao \'e de 200km/h e,
 o da velocidade do vento, \'e 90km/h. O vento \'e dado por um vetor apontando para a direito, enquanto o avi\~ao \'e um vetor apontando para cima. 
Pergunta-se: (a) Qual \'e a orienta\c c\~ao que o avi\~ao deve voar? (Ambos est\~ao sendo vistos do solo.) (b) Qual \'e o m\'odulo da
velocidade do avi\~ao visto do solo?
  (a) Segue que 
    $$
      \vec{v}_{AO} = \vec{v}_{A} - \vec{v}_{v} \Rightarrow \sin{\theta} = \frac{|\vec{v}_{v}|}{|\vec{v}_{av}|} = \frac{90}{200} = \frac{9}{20}\approx 27\deg
    $$

    (b) Sabemos, por pit\'agora, que 
      $$
      |\vec{v}_{AT}|^{2} = |\vec{v}_{v}|^{2} + |\vec{v}_{a}|^{2} \Rightarrow |\vec{v}_{a}| = \sqrt{|\vec{v}_{av}|^{2} - |\vec{v}_{v}|^{2}} = \sqrt{51900}\approx 178km/h
      $$
\end{example}
  Com rela\c c\~ao a este \'ultimo exemplo, por que a velocidade $\vec{v}_{a}$ tem valor 178km/h e n\~ao 200 - 90 = 110km/h?
A resposta est\'a na decomposi\c c\~ao de $\vec{v}_{av}$, pois 
\begin{align*}
  &v_{av}^{x} = |\vec{v}_{av}|\sin{\theta} = -200 \cdot 0.454\approx -90km/h\\
  &v_{av}^{y} = |\vec{v}_{av}|\cos{\theta} = 200 \cdot 0.891\approx 178km/h.
\end{align*}
 \begin{example}
   Suponha que, num instante $t_{0}$, dois trens est\~ao andando em dire\c c\~ao a uma plataforma. O trem um chegou nela, vindo do Norte, enquanto o trem dois,
   vindo pelo Leste, ainda se move, ambos com velocidade 
     $$
       |\vec{v}_{1}| = |\vec{v}_{2}| = 60km/h.
     $$
     Passados dois minutos, o trem 2 alcan\c ca a plataforma e continua andando na dire\c c\~ao Oeste com velocidade $\vec{v}_{2}$ e o trem
    um continuou sua viagem ao Sul com velocidade $\vec{v}_{1}$. Pede-se: (a) Determine o vetor $\vec{v}_{21}$ da velocidade relativa dos trens. (b) Encontre, para este vetor do item (a), 
    seu m\'odulo (c) Quando a dist\^ancia entre os vetores \'e m\'inima?

    Faremos o diagrama de velocidades. Nele, $|v_{1}| = |v_{2}|$. 
\begin{center}
  \begin{tikzpicture}[scale=2]
    \coordinate (A) at (0,0);
    \coordinate (B) at (-1.5,0);
    \coordinate (C) at (0, -1.5);
    \draw[-latex] (A) -- node[below] {$\vec{v}_1$} (B);
    \draw[-latex] (A) -- node[left] {$\vec{v}_2$} (C);
    \draw (B) -- node[above right] {$\vec{v}_{12}$} (C);
  \end{tikzpicture}
\end{center}
  Al\'em disso, pelo desenho, 
    $$
    \sin{\theta} = \frac{|\vec{v}_{2}|}{|\vec{v}_{21}|},\quad \cos{\theta} = \frac{|\vec{v}_{1}|}{|\vec{v}_{21}|} = \frac{|\vec{v}_{2}|}{|\vec{v}_{21}|} = \sin{\theta}.
    $$
    A igualdade entre seno e cosseno ocorre quando o \^angulo vale 45 graus, ou seja, $\theta = 45\deg.$ Assim,
     $$
     \vec{v}_{21} = \vec{v}_{2} - \vec{v}_{1} = -|\vec{v}_{2}|\hat{i} - (-|\vec{v}_{1}|\hat{j}) \Rightarrow \vec{v}_{21} = -|\vec{v}_{2}|\hat{i} + |\vec{v}_{1}|\hat{j}.
     $$
     Logo, 
       $$
       |\vec{v}_{21}| = \sqrt{|\vec{v}_{1}|^{2} + |\vec{v}_{2}|^{2}}\approx 85km/h
       $$

    Para resolver, agora, o item b, a comecemos pela posi\c c\~ao relativa 2 nos instantes t=0, t=2min e t=4min. Quanto ao trem 2, as
  informa\c c\~oes que temos indicam que ele se move no eixo x ($y_{2}(t) = 0$), em t=2min, ele est\'a na origem ($x_{2}(2min) = 0$) e, deste modo,
    $$
      \vec{r}_{2}(t) = x_{2}(t)\hat{i} + y_{2}(t)\hat{j} = x_{2}(t)\hat{i} \Rightarrow x_{2}(t) = x_{2O} + |\vec{v}_{2}|t
    $$
    Utilizando o valor que sabemos, i.e., x(2min), segue que, convertendo 2 minutos para horas ($2min\approx 0.03h$) ,
      $$
      x(0.03) = 0 = x_{2O} - 60 \cdot 0.03 \Rightarrow x_{2O} = 60 \cdot 0.03 = 2km.
      $$

      Agora, sobre o trem 1, sabe-se que ele se move no eixo y, ou seja, $x_{1}(t) - 0$, tal que 
        $$
          y_{1}(t) = -|\vec{v}_{1}|t = -60t
        $$
      
    Com essas informa\c c\~es, encontramos os valores 
   \begin{align*}
     &t = 0min:\quad x_{1}(0) = 0, y_{1}(0) = 0,\quad x_{2}(0) = 2km, y_{2}(0) = 0km\\
     &t = 2min:\quad x_{1}(2) = 0, y_{1}(2) = -2km,\quad x_{2}(2) = 0km, y_{2}(2) = 0km\\
     &t = 4min:\quad x_{1}(4) = 0, y_{1}(4) = -4km,\quad x_{2}(4) = -2, y_{2}(4) = 0km.
   \end{align*}
   Desta forma, 
     $$
     \vec{r}_{21}(t) = \vec{r}_{2}(t) - \vec{r}_{1}(t) \Rightarrow \vec{r}_{21}(t) = x_{2}(t)\hat{i} - y_{1}(t)\hat{j}.
     $$

    Finalmente, para o item c, calculamos a dis\^encia como 
      $$
      L_{21}(t) = \sqrt{x_{1}^{2} + (-y_{1})^{2}} = \sqrt{(x_{20}-|\vec{v}_{2}(t)|)^{2} + (|\vec{v}_{1}|)^{2}} = \sqrt{7200t^{2} - 240t + 4}.
      $$
    Para encontrar a dist\^ancia \textbf{m\'inima}, \'e preciso derivar esta f\'ormula, igualar a 0 e resolver para tempo. Coloque $l = 7200t^{2}-240t+4$,
  tal que 
    $$
      \frac{dl}{dt} = 2 \cdot 7200t - 240 + 0 = 0
    $$
    Resolvendo isso, encontramos o tempo em que a dist\^ancia \'e m\'inima, valendo $t^{*}\approx 0.017h\approx 1min$, tal que a dist\^ancia m\'inima \'e 
      $$
        L_{21}(t^{*})\approx 1.4.
      $$
 \end{example}
\newpage

\section{Aula 7 - 17/03/2023}
\subsection{Motiva\c c\~oes}
\begin{itemize}
  \item Come\c car a estudar o movimento circular;
\end{itemize}

\subsection{Movimento Circular}
  Quando temos uma particular fazendo movimento circular em um c\'irculo de raio R num eixo x, y, diremos que ela, sua posi\c c\~ao em qualquer instante
ser\'a dada por um vetor r(t), sendo sua trajet\'oria limitada a este c\'irculo. Assim, obtemos o sistema $R = |\vec{r}(t)|$, sendo 
$\vec{r}(t) = x(t)\hat{i} + y(t)\hat{j}$ e 
  $$
     \left\{\begin{array}{ll}
         x(t) = R\cos{\theta(t)}\\
         y(t) = R\sin{\theta(t)}.
      \end{array}\right.
  $$
  \begin{center}
\begin{tikzpicture}[>=latex]
    \draw[->] (-1.2,0) -- (1.2,0) node[right] {$x$};
    \draw[->] (0,-1.2) -- (0,1.2) node[above] {$y$};
    \draw (0,0) circle [radius=1];
    \draw[->] (0, 0) -- ({-cos(45)}, {sin(45)}) node[above] {$R$};
    \draw[->] (0,0) -- ({cos(45)},{sin(45)}) node[above right] {$r(t)$};
\end{tikzpicture}
  \end{center}
  Utilizando o sistema e o desenho, obtemos 
    $$
    \vec{r}(t) = R\cos{\theta(t)}\hat{i} + R\sin{\theta(t)}\hat{j},
    $$
    donde segue o valor do m\'odulo do vetor $\vec{r}(t)$:
   \begin{align*}
     |\vec{r}(t)|^{2} = x^{2}(t) + y^{2}(t) &= (R\cos{\theta(t)})^{2} + (R\sin{\theta(t)})^{2} \\
                                            &= R^{2}(\cos{\theta(t)}^{2} + \sin{\theta(t)}^{2}) \\
                                            &= R^{2} \Rightarrow |\vec{r}(t)| = R.
   \end{align*}
   Conclu\'imos, assim, que todo o movimento da part\'icula \'e dado em termos do \^angulo $\theta(t).$ Al\'em disso, o deslocamento
  da part\'icula \'e feita em arcos de c\'irculo $s(t) = R\theta(t)$. Chamamos esta posi\c c\~ao de ``posi\c c\~ao escalar do corpo sobre o c\'irculo''.
  No entanto, no movimento circular, h\'a outra posi\c c\~ao, chamada ``posi\c c\~ao angular do corpo'', que \'e dada por $\theta(t) = \frac{s(t)}{R}.$
  
  Utilizando estes dois, podemos encontrar uma equa\c c\~ao para $\vec{r}(t)$: 
    $$
    \vec{r}(t) = R\cos{\theta(t)}\hat{i} + R\sin{\theta(t)}\hat{j} = R\underbrace{[\cos{\theta(t)}\hat{i} + \sin{\theta(t)}\hat{j}]}_{\hat{r}(t)}
= R\hat{r}(t),
    $$
    em que $\hat{r}(t)$ \'e um versor na dire\c c\~ao de $\vec{r}(t)$, isto \'e, um vetor com m\'odulo um. De fato, vamos verificar isto: 
    $$
    |\hat{r}(t)| = \sqrt[]{\cos^{2}(\theta(t)) + \sin^{2}(\theta(t))} = 1
    $$
    A seguir, vamos estudar como este versor $\hat{r}(t)$ varia, ou seja, vamos derivar este vetor com respeito ao tempo. Para isso,
  introduizremos outra regra de deriva\c c\~ao, a ``Regra da Cadeia''. Dada uma fun\c c\~ao $f(t) = u(v(t))$, ou seja, uma fun\c c\~ao
  definida como uma fun\c c\~ao composta, sua deriva\c c\~ao \'e feita de denro pra fora: Derivamos v(t) com respeito a t, depois derivamos
  u com rela\c c\~ao a v e multiplicamos, ou seja, 
    $$
    \boxed{\frac{df}{dt} = \frac{du}{dv}\frac{dv}{dt}}.
    $$
    Assim, no caso do versor $\hat{r}(t),$ 
    \begin{align*}
      \frac{d\hat{r}(t)}{dt} &= \frac{d}{dt}[\cos{\theta(t)}\hat{i} + \sin{\theta(t)}\hat{j}]\\
                             &= \frac{d}{dt}[\cos{\theta(t)}]\hat{i} + \frac{d}{dt}[\sin{\theta(t)}]\hat{j}\\
                             &= \frac{d\cos{\theta(t)}}{d\theta}\frac{d\theta(t)}{dt}\hat{i} + \frac{d\sin{\theta(t)}}{d\theta}\frac{d\theta(t)}{dt}\hat{j}\\
                             &= -\sin{\theta(t)}\frac{d\theta(t)}{dt}\hat{i} + \cos{\theta(t)}\frac{d\theta(t)}{dt}\hat{j}.
    \end{align*}
    Como $\theta(t)$ \'e a posi\c c\~ao angular, chamamos a sua derivada com respeito a tempo de velocidade \^angular 
      $$
      \boxed{\omega(t) = \frac{d\theta(t)}{dt}.}
      $$
    De brinde, conseguimos definir a velocidade escalar da part\'icula como 
   \begin{align*}
      \frac{ds(t)}{dt} &= \frac{dR\theta(t)}{dt} = R \frac{d\theta(t)}{dt} = R\omega(t).\\
      &\Rightarrow \boxed{v(t) = R\omega(t).}
   \end{align*}
   Vamos estudar a dimens\~ao dessa quantidade. Temos 
     $$
     [\omega] = \frac{[\theta]}{[t]} = \frac{1}{T} \quad \text{(Exemplo: rad/s (radianos por segundo.))}
     $$ 
    como unidades de velocidade angular e 
      $$
        [v] = [R][\omega] = LT^{-1} \quad \text{(Exemplo: m/s (metros por segundo))}
      $$
    como dimens\~ao da velocidade escalar. Com rela\c c\~ao ao versor definido, sua derivada \'e 
      $$
      \frac{d\hat{r}(t)}{dt} = \omega(t)\underbrace{[-\sin{\theta(t)}\hat{i}+\cos{\theta(t)}\hat{j}]}_{\hat{\theta(t)}},
      $$
      em que $\hat{\theta(t)}$ \'e um versor apontando na dire\c c\~ao do \^angulo. Com isso, definimos a velocidade vetorial por 
        $$
        \vec{v}(t) = \frac{d \vec{r}(t)}{dt} = R \frac{d\hat{r}(t)}{dt} \Rightarrow \vec{v}(t) = R\omega(t)\hat{\theta(t)}.
        $$
      Algo interessante de notar \'e que a velocidade escalar consiste do m\'odulo da velocidade vetorial, i.e., $v(t) = |\vec{v}(t)|.$
      Tamb\'em podemos representar a velocidade por meio das componentes em cada eixo: 
        $$
        \vec{v}(t) = \underbrace{-R\omega(t)\sin{\theta(t)}}_{v_{x}(t)}\hat{i} + \underbrace{R\omega(t)\cos{\theta(t)}}_{v_{y}(t)}\hat{j}.
        $$
  \subsection{Acelera\c c\~oes no Movimento Circular.}
      Agora que estamos mais familiariizados com a velocidade e posi\c c\~ao angular, podemos estudar a acelera\c c\~ao no movimento circular.
      Assim como antes, come\c camos definindo a acelera\c c\~ao angular: 
      $$
        \alpha(t) = \frac{d\omega(t)}{dt} = \frac{d^{2}\theta(t)}{dt^{2}},
      $$
      que possui dimens\~ao $[\alpha] = \frac{[\omega]}{[t]} = \frac{T^{-1}}{T} = T^{-2}$, sendo um exemplo a unidade $rad/s^{2}$, i.e.,
      radiano por segundo quadrado. Analogamente, definimos a acelera\c c\~ao vetorial por 
        $$
        \vec{a}(t) = \frac{d \vec{v}(t)}{dt} = \frac{d}{dt}[R\omega(t)\hat{\theta}(t)].
        $$
        Pela regra do produto, 
        $$
        \vec{a}(t) = R[\frac{d\omega(t)}{dt}\hat{\theta(t)} + \omega(t) \frac{d\hat{\theta(t)}}{dt}].
        $$
      Analisando termo a termo, a primeira derivada acontece no m\'odulo da velocidade, i.e., $R \frac{d\omega(t)}{dt}$, ou seja,
      representa a varia\c c\~ao no m\'odulo da velocdade. Por outro lado, o segundo termo representa a varia\c c\~ao da dire\c c\~ao 
      da velocidade. Como j\'a encontramos alguns desses termos antes, segue que 
        $$
        \vec{a}(t) = R[\alpha(t)\hat{\theta(t)} + v(t)\frac{d\hat{\theta}(t)}{dt}].
        $$
        Mas o que \'e este termo $\frac{d\hat{\theta}}{dt}$? Olhando pra ele com cuidado, vemos que 
       \begin{align*}
         \frac{d\hat{\theta}}{dt} &= \frac{d}{dt}[-\sin{\theta(t)}\hat{i} + \cos{\theta(t)}\hat{j}] \\
                                  &= -\frac{d}{dt}[\sin{\theta(t)}]\hat{i} + \frac{d}{dt}[\cos{\theta(t)}]\hat{j}\\
                                  &= -\cos{\theta(t)}\frac{d\theta(t)}{dt}\hat{i} -\sin{\theta(t)}\frac{d\theta(t)}{dt}\hat{j}\\
                                  &\Rightarrow \frac{d\hat{\theta(t)}}{dt} = \omega(t)[-\cos{\theta(t)}\hat{i} - \sin{\theta(t)}\hat{j}] = \omega(t)(-\hat{r}(t)).
       \end{align*}
     Portanto, 
       $$
         \vec{a}(t) = R[\alpha(t)\hat{\theta(t)} + \omega^{2}(t)(-\hat{r}(t))] = \underbrace{R\alpha(t)\hat{\theta(t)}}_{\text{acelera\c c\~ao tangencial }\vec{a}_{t}(t)} - \underbrace{R\omega^{2}(t)\hat{r}(t)}_{\text{acelera\c c\~ao centr\'ipeta }\vec{a}_{cp}(t)}
       $$
      Obtivemos disso tudo duas acelera\c c\~oes novas e que precisam ser mais compreendidas. Vamos come\c car pela tangencial.

      Com rela\c c\~ao ao m\'odulo da acelera\c c\~ao tangencial, note que $|\vec{a}_{t}(t)| = R\alpha(t) = \frac{dv(t)}{dt}$. Agora,
      quanto \`a acelera\c c\~ao centr\'ipeta, seu m\'odulo \'e dado por $|\vec{a}_{cp}(t)| = R\omega^{2}(t) \Rightarrow|\vec{a}_{cp}(t)| = \frac{v^{2}}{R}.$ 

  \subsection{Movimento Circular Uniforme}
      Resumindo o que temos at\'e o momento em forma de tabela, segue que 

    \begin{table}[h!]
    \centering
    \begin{tabular}{|c|c|c|}
        \hline
        & \textbf{Variáveis angulares} & \textbf{Variáveis escalares} \\
        \hline
        \textbf{Posição} & $\theta(t)$ & $s(t) = R\theta(t)$ \\
        \hline
        \textbf{Velocidade} & $\omega(t) = \frac{d \theta(t)}{dt}$ & $v(t) = \frac{d s(t)}{dt} = R\omega(t)$ \\
        \hline
        \textbf{Aceleração} & $\alpha(t) = \frac{d\omega(t)}{dt} = \frac{d^2\theta(t)}{dt^2}$ & $|\vec{a}(t)| = \frac{dv(t)}{dt} = R\alpha(t),\quad |\vec{a}_{cp}| = \frac{v^2}{R}$ \\
        \hline
    \end{tabular}
    \caption{Resumo movimento circular.}
    \label{tab:my_label}
  \end{table}
  No movimento circular uniforme, estudamos arcos iguais em tempos iguais, ou seja, 
    $$
      \Delta s_{1} = \Delta s_{2},\quad \Delta t_{1} = \Delta t_{2},
    $$
    tal que $\Delta \theta_{1} = \Delta \theta_{2}$. Al\'em disos, $\omega(t)\equiv \omega$ constante. Assim, 
      $$
      |\vec{v}(t)| = v(t) = R\omega(t)\equiv v,\text{ constante} \Rightarrow a_{t} = R\alpha(t) = 0.
      $$
    Com isso, as posi\c c\~oes s\~ao descritas por 
   \begin{align*}
     &\omega(t) = \omega \Rightarrow \theta(t) = \theta_{0} + \omega(t-t_{0})\\
     &v(t) = v \Rightarrow s(t) = s_{0} + v(t-t_{0})
   \end{align*}
   Neste caso, o movimento \'e peri\'odico, ou seja, ele volta a ter as mesmas propriedades ap\'os um per\'iodo T. Em forma matem\'atica, isso quer dizer que 
     $$
       \left\{\begin{array}{ll}
           \vec{r}(t + T) = \vec{r}(t)\\
           \vec{v}(t + T) = \vec{v}(t).
         \end{array}\right.
     $$ 
    Tendo isso em mente, definimos tamb\'em a frequ\^encia como o n\'umero de ocorr\^encias. Ele vale o inverso do per\'iodo T, i.e., $f = \frac{1}{T}$.
\newpage

\section{Aula 8 - 19/04/2023}
\subsection{Motiva\c c\~oes}
\begin{itemize}
  \item Come\c car os estudos de din\^amica
\end{itemize}

\subsection{Exemplo de MCU - 67 Tiples}
Suponha que a Terra tem velocidade angular $\omega$ constante, velocidade e acelera\c c\~ao angulares $\vec{v}_{\theta}(t), \vec{a}_{\theta}(t)$
e velocidade e acera\c c\~ao escalares $\vec{v}_{e}(t), \vec{a}_{e}(t)$.
\begin{center}
\begin{tikzpicture}[scale=3]
  \shade[ball color=blue!10!white,opacity=0.7] (0,0) circle (1cm);
  \shade[ball color=yellow!10!white,opacity=0.7] (0,0) circle (0.5cm);
  \draw[->,thick,red] (0,0) -- (1,0) arc (0:45:1) node[midway,above right]{$\theta$};
  \draw[dashed] (0,0) -- (-1,0);
  \draw[dashed] (0,0) -- (0,-1);
  \draw[dashed] (0,0) -- (0,1);
\draw[->, black](0,0) -- ({cos(45)}, {sin(45)}) ;
  \draw (0,0) -- (1,0) node[midway,above]{$r$};
  \draw (0.2,0) arc (0:45:0.2) node[midway,right]{$\phi$};
\end{tikzpicture}
\end{center}

 No equador, $R_{E}=R_{T}, \omega_{E}=\omega$, tal que 
   $$
      \left\{\begin{array}{ll}
         v_{E}=\omega_{E}R_{E} = \omega R_{T}\\
         a_{E_{cp}} = \frac{v_{E}^{2}}{R_{E}} = \frac{(\omega R_{T})^{2}}{R_{T}} = \omega^{2}R_{T}.
       \end{array}\right.
   $$
   Na latitude $\theta,$ $R_{\theta} = R_{T}\cos{\theta}, \omega_{\theta} = \omega$, de modo que 
     $$
        \left\{\begin{array}{ll}
           v_{\theta} = \omega R_{T}\cos{\theta}\\
           a_{\theta_{cp}} = \omega^{2}R_{T}\cos{\theta}
         \end{array}\right.
     $$
  Para a Terra dar uma volta em torno de si de novo, ela demora aproximadamente 24h. Assim, $T = 24h$ \'e o per\'iodo da Terra,
donde conclu\'imos que a frequ\^encia ser\'a $f = \frac{1}{T} = \frac{1}{86400}s^{-1}.$ Como $\omega = 2\pi f,$ segue que 
  $$
    \omega = \frac{2\pi}{86400} = 7.27 \cdot 10^{-5}rad/s.
  $$
  Pede-se: a) Quais s\~ao os valores de $v_{e}, a_{e}?$ b e d) Quais s\~ao as orienta\c c\~oes das acelera\c c\~oes? c) Quanto valem $v_{\theta}, a_{\theta}?$

  a.) Vemos que $v_{E} = 463.1m/s, a_{E} = 0.0337m/s^{2}, g = 9.8m/s^{2}$. Em particular, $a_{E} = 0.0034g.$
  
  b. e d.) O diagrama de v indica que o vetor acelera\c c\~ao aponta na vertical pra esquerda e levemente pra cima.

  c.) Temos $v_{\theta} = 379.4m/s, a_{\theta}=0.0276m/s^{2}.$

\subsection{Din\^amica e Leis de Newton}
\subsubsection{O que esperar}
 Quando estudamos os movimentos anteriores, est\'avamos vendo cinem\'atica, a descri\c c\~ao matem\'atica do movimento. No entanto,
nunca nos questionamos o que causa o movimento. Como ele surge, o que influencia-o, etc. Essa pergunta \'e respondida pela din\^amica,
q formula\c c\~ao matem\'atica que explic\'ita as causas do movimento. Ela nos fornece uma rela\c c\~ao entre as intera\c c\~oes, chamadas for\c cas,
que o corpo sofre e o seu movimento. A primeira formula\c c\~ao da din\^amica foi feita por Isaac Newton, sendo suas Leis nosso Ponto de partida.

\subsubsection{Leis de Newton}
  A primeira Lei de Newton, tamb\'em chamada de Lei da In\'ercia, afirma que 
  \begin{quote}
    ``Um corpo em repouso, ou em movimento retil\'ineo uniforme, permenecer\'a em seu estado de movimento a n\~ao ser que uma
    for\c ca externa atue sobre ele. ''
  \end{quote}
  Observe que velocidade constante significa que tanto seu m\'odulo ser\'a constante quanto a dire\c c\~ao o movimento precisa ser em linha reta.
  Uma consequ\^encia dessa Lei \'e que n\~ao tem distin\c c\~ao entre um corpo em repouso e um corpo se movendo com velocidade constante.
  Com isso, um sistema de referencial inercial ser\'a definido como um eixo de coordenadas que est\'a em repouso ou se movendo com
  velocidade constante.

  A segunda Lei de Newton surge para explicar como aparecem as for\c cas dentro do contexto da din\^amica, dizendo que 
 \begin{quote}
    ``A for\c ca resultante atuando em um corpo \'e igual \`a massa dele multiplicada pela acelera\c c\~ao''
 \end{quote}
  Matematicamente, isto significa que 
    $$
    \boxed{\vec{F}_{r} = m \cdot \vec{a} = m \cdot \frac{d \vec{v}}{dt}}
    $$
  O termo novo m \'e chamado de massa inercial, sendo interpretada como a grandeza f\'isica que expressa a resist\^encia do corpo
ao movimento. Quanto maior for a massa, maior vai ser a resist\^encia a se mover. De fato, se temos dois blocos de massas $m_{1}>m_{2},$ ent\~ao 
  $$
    \vec{a}_{1} = \frac{\vec{F}}{m_{1}},\quad \vec{a}_{2} = \frac{\vec{F}}{m_{2}} \Rightarrow a_{1} < a_{2}.
  $$
  A dimens\~ao dessa grandeza \'e $[m] = M$. No Sistema Internacional, a unidade de massa \'e o kilograma.

  Note que a for\c ca \'e uma grandeza vetorial que soma-se, ou seja, se h\'a v\'arias for\c cas agindo sobre um corpo, a resultante ser\'a
  a soma delas. Se temos for\c cas $\vec{F}_{21}, \vec{F}_{31}$ agindo sobre um corpo, ent\~ao a resultante ser\'a $\vec{F}_{R} = \vec{F}_{21} + \vec{F}_{31}.$
  Al\'em disso, suas coordenadas ser\~ao 
 \begin{align*}
   &x: F_{res}^{x} = -F_{21}\sin{\alpha} + F_{31}\sin{\beta}\\
   &y: F_{res}^{y} = -F_{21}\cos{\alpha} - F_{31}\cos{\beta}.
 \end{align*}

 \begin{center}
   \begin{tikzpicture}[scale=2]
    \draw[->,>=stealth, dashed] (-1.5,0) -- (1.5,0) node[right]{$x$}; % Draw the x-axis
    \draw[->,>=stealth, dashed] (0,-1.5) -- (0,1.5) node[above]{$y$}; % Draw the y-axis
    
    \draw (-0.75,-0.75) -- (0.75,-0.75) -- (0.75,0.75) -- (-0.75,0.75) -- cycle; % Draw the cube
    \draw[->,>=stealth] (0,0) -- (-1,-1) node[midway, below left]{$F_{21}$}; % Draw the F21 arrow
    \draw[->,>=stealth] (0,0) -- (0,-1) node[midway, right]{$mg$}; % Draw the gravity arrow
    \draw[->,>=stealth] (0,0) -- (1,-1) node[midway, below right]{$F_{31}$}; % Draw the F31 arrow
    
    \draw (0,-0.2) arc (-90:-45:0.2) node[midway, below]{$\beta$}; % Draw the angle beta
    
    \draw[dashed] (0,-0.75) -- (0,-1.5); % Draw the vertical dashed line for reference
    \end{tikzpicture}
  \end{center}
 A unidade da for\c ca \'e dada por $[F] = [ma] = MLT^{-2}$. No SI, sua unidade \'e $1kgms^{-2} = 1N$ o Newton.
 Um corpo ser\'a dito em equil\'ibrio quando a for\c ca resultante agindo sobre ele \'e nula, pois, neste caso, 
   $$
     \vec{F}_{res} = \sum\limits_{n}^{}\vec{F}_{n} = 0 \Rightarrow F_{res} = ma = 0 \Rightarrow a = 0.
   $$

  A terceira e \'ultima Lei de Newton \'e conhecida como Lei da A\c c\~ao e Rea\c c\~ao. Segue seu enunciado
 \begin{quote}
   ``Se um corpo faz uma for\c ca em outro, ent\~ao este segundo tamb\'em realizar\'a uma for\c ca no primeiro, sendo esta de
   mesmo m\'odulo, mas com dire\c c\~ao oposta. ''
 \end{quote}
  Em outras palavras, se um corpo 2 age sobre um corpo 1 com for\c ca $\vec{F}_{21},$ ent\~ao o corpo 1 far\'a uma for\c ca sobre
  o corpo 2 $\vec{F}_{12}$ tal que $\vec{F}_{21} = -\vec{F}_{12}, |\vec{F}_{21}| = |\vec{F}_{12}|$.

  \subsubsection{Exemplo 4.2 - Tipler}
  Os dados que temos \'e que h\'a uma pessoa que se moveu 2.25m em 3s e cuja massa \'e 68kg. Pede-se para encontrar o m\'odulo da for\c ca
agindo sobre ela. Segue que 
  $$
    x(t) = x_{0} + v_{0}t + \frac{1}{2}at^{2} \Rightarrow \Delta x = x(3) - x(0) = \frac{1}{2}a3^{2} = \frac{9}{2}a = 2.25m
  $$
  Isolando a equa\c c\~ao, encontra-se que $a = 0.5m/s^{2}.$ Com isso, como $|\vec{F}| = |m \vec{a}| = 68 \cdot 0.5 = 34N$. \qedsymbol
\newpage

\section{Aula 10 - 24/04/2023}
\subsection{O que esperar?}
\begin{itemize}
  \item Continua\c c\~ao de Din\^amica;
  \item Tipos de for\c cas.
\end{itemize}
\subsection{Motiva\c c\~ao}
Neste resumo, discutiremos diferentes tipos de forças em física, suas fórmulas e algumas propriedades. Elas ser\~ao abordadas ao longo
das pr\'oximas aulas, n\~ao apenas nesta primeira. \textbf{Nota ao Leitor: as f\'ormulas aqui apresentadas s\~ao apenas
para o m\'odulo das for\c cas, mas esta palavra foi omitida para n\~ao ficar muito repetitivo. As for\c cas s\~ao estudadas vetorialmente,
n\~ao esque\c ca-se disto.}

A força gravitacional é a atração entre dois objetos devido à sua massa. A fórmula da força gravitacional é:

\begin{equation}
F_g = G \frac{m_1 m_2}{r^2}
\end{equation}

onde $F_g$ é a força gravitacional, $G$ é a constante gravitacional, $m_1$ e $m_2$ são as massas dos dois objetos e $r$ é a distância entre seus centros de massa.
\begin{center}
\begin{tikzpicture}
    \node[draw, circle] (m1) at (0,0) {$m_1$};
    \node[draw, circle] (m2) at (4,0) {$m_2$};
    \draw[->, >=stealth, thick] (m1) -- node[above] {$\vec{F}_{g_{1,2}}$} (m2);
    \draw[->, >=stealth, thick] (m2) -- node[below] {$\vec{F}_{g_{2,1}}$} (m1);
\end{tikzpicture}
\end{center}

A força normal é a força que um objeto exerce perpendicularmente à sua superfície de contato com outro objeto. Esta força é responsável por impedir que os objetos atravessem uns aos outros. A força normal tem m\'odulo igual à componente do peso do objeto perpendicular à superfície.

\begin{center}
\begin{tikzpicture}
    \draw (0,0) -- (3,0) -- (3,2) -- cycle;
    \draw[->, >=stealth, thick] (1.5, 1) -- node[right] {$\vec{F}_N$} (1.5, 2);
    \draw[->, >=stealth, thick] (1.5, 1) -- node[below] {$\vec{F}_g$} (1.5, 0);
    \node at (1.5, 1) [circle, fill, inner sep=1pt] {};
\end{tikzpicture}
\end{center}

A força de atrito é a força que age oposta ao movimento relativo entre duas superfícies em contato. Existem dois tipos de força de atrito: estático e cinético.
A força de atrito estático age entre duas superfícies em repouso relativo e é responsável por impedir que objetos comecem a se mover. A força de atrito estático pode variar de zero até o valor máximo dado por:

\begin{equation}
F_{s_{max}} = \mu_s F_N
\end{equation}

onde $F_{s_{max}}$ é a força de atrito estático máxima, $\mu_s$ é o coeficiente de atrito estático e $F_N$ é a força normal.
A força de atrito cinético age entre duas superfícies em movimento relativo e é responsável por reduzir a velocidade dos objetos. A força de atrito cinético é dada por:

\begin{equation}
F_k = \mu_k F_N
\end{equation}

onde $F_k$ é a força de atrito cinético, $\mu_k$ é o coeficiente de atrito cinético e $F_N$ é a força normal.
\begin{center}
\begin{tikzpicture}
\draw (0,0) -- (4,0);
\node[draw, rectangle] (box) at (2,0.5) {Caixa};
\draw[->, >=stealth, thick] (box.north) -- ++(0,1.5) node[above] {$\vec{F}_N$};
\draw[->, >=stealth, thick] (box.south) -- ++(0,-1.5) node[below] {$\vec{F}g$};
\draw[->, >=stealth, thick] (box.west) -- ++(-1.5,0) node[left] {$\vec{F}{k}$};
\end{tikzpicture}
\end{center}

  Estas duas \'ultimas for\c cas, ou seja, for\c cas normal e atrito, encaixam-se na classe de for\c cas de contato.

  A força tensional é a força que atua ao longo de um cabo, corda ou fio quando esticado por forças opostas aplicadas em suas extremidades. A força tensional é transmitida ao longo do comprimento do cabo, corda ou fio.
  \begin{center}
    \begin{tikzpicture}
    \node[draw, circle] (mass) at (0,-1) {$m$};
    \draw[->, >=stealth, thick] (mass) -- ++(0,1.5) node[above] {$\vec{T}$};
    \draw (0,0) -- (0,1);
    \end{tikzpicture}
  \end{center}

A força elástica é a força de restitui\c c\~ao que atua quando um objeto é deformado, como uma mola esticada ou comprimida. A força elástica segue a Lei de Hooke:

\begin{equation}
F_e = -k \Delta x
\end{equation}

onde $F_e$ é a força elástica, $k$ é a constante elástica e $\Delta x$ é o deslocamento da posição de equilíbrio da mola.
\begin{center}
\begin{tikzpicture}
\node[draw, rectangle] (block) at (0,0) {Bloco};
\draw[->, >=stealth, thick] (block.east) -- ++(1.5,0) node[right] {$\vec{F}_e$};
\draw[decoration={coil, segment length=3mm, aspect=0.5, amplitude=3mm}, decorate] (block.west) -- ++(-2,0);
\end{tikzpicture}
\end{center}

A força de arrasto é a força de resistência que um objeto experimenta ao se mover através de um fluido (líquido ou gás). A força de arrasto é proporcional ao quadrado da velocidade do objeto em relação ao fluido e é dada por:

\begin{equation}
F_D = \frac{1}{2} \rho v^2 C_D A
\end{equation}

onde $F_D$ é a força de arrasto, $\rho$ é a densidade do fluido, $v$ é a velocidade do objeto, $C_D$ é o coeficiente de arrasto e $A$ é a área da seção transversal do objeto.

\begin{center}
\begin{tikzpicture}
\node[draw, circle] (sphere) at (0,0) {Esfera};
\draw[->, >=stealth, thick] (sphere.east) -- ++(-1.5,0) node[left] {$\vec{F}_D$};
\end{tikzpicture}
\end{center}

\subsection{Tipos de For\c ca}
\subsubsection{As Quatro For\c cas Fundamentais de Forma Breve}
O primeiro tipo de for\c ca que veremos \'e a Intera\c c\~ao gravitacional. Ela ocorre na forma da intera\c c\~ao de corpos massivos,
sendo representada pela for\c ca peso. Al\'em disso, ela \'e uma for\c ca atrativa e de longo alcance. Em particular, o m\'odulo da 
for\c ca gravitacoinal \'e 
  $$
    |\vec{F}_{21}| = |\vec{F}_{12}| = \frac{Gm_{1}m_{2}}{r^{2}}.
  $$
  Aqui, G \'e a constante de gravitacional, cujo valor \'e $G=6.67 \cdot 10^{-11}Nkg^{-2}m^{2}.$

  A segunda forma de intera\c c\~ao \'e a eletromagn\'etica, estando presente no contexto de part\'iculas carregadas. Ela ser\'a 
  atrativa dado que as duas cargas $q_{1}, q_{2}$ possuam sinais opostos ($q_{1}\cdot q_{2}<0$) e repulsivas caso possuam sinais
  iguais ($q_{1}\cdot q_{2}>0$). Seu m\'odulo \'e dado por 
    $$
      |\vec{F}_{12}|=|\vec{F}_{21}|=k\frac{q_{1}q_{2}}{r^{2}}.
    $$

  Outra for\c ca que existe \'e a intera\c c\~ao forte, que age em curto alcance, aproximadamente do tamanho do n\'ucleo at\^omico.
  Ela \'e respons\'avel por manter os pr\'otons e neutron no n\'ucleo.

  A \'ultima for\c ca fundamental \'e a intera\c c\~ao fraca, sendo uma que ocorre em curt\'issimo alcance e presente em alguns
  decaimento radioativos.

  Em din\^amica, estudaremos principalmente intera\c c\~oes gravitacionais na forma da For\c ca Peso e intera\c c\~oes eletromagn\'eticas,
  que se manifrestar\~ao nas formas da for\c ca normal, for\c ca de tra\c c\~ao, for\c ca de atrito, entre outras.

  \subsubsection{For\c ca Peso}
    A for\c ca peso \'e descrita como a atra\c c\~ao de um corpo por um objeto celeste quando o corpo est\'a em sua superf\'icie.
    Como vimos previamente, ela \'e dada, na Terra, por 
      $$
        F_{g} = \frac{GMm}{R_{T}^{2}} = mg.
      $$
    Isso ajuda-nos a descobrir, em particular, como obter o valor da acelera\c c\~ao na superf\'icie. Considerando a massa
    de humanos como desprez\'ivel em compara\c c\~ao \`a da Terra, obtemos 
      $$
        g = \frac{GM_{T}}{R_{T}^{2}}\approx 9.81m/s^{2}.
      $$
      A for\c ca peso que iremos considerar \'e definida como $\vec{P} = m \vec{g}$. Consideremos um exemplo.

 \begin{example}
   Considere uma pessoa pulando. A Terra exerce uma for\c ca atrativa nela de volta para a superf\'icie, a for\c ca Peso, tal que
   a for\c ca resultante nela \'e dada por 
     $$
       \vec{F}_r = \vec{P} = m \vec{g} - m \vec{a}.
     $$
     Caso a pessoa esteja na superf\'icie, ela estar\'a em equil\'ibrio, ou seja, $\vec{F}_{r} = 0.$ Com isso, segue que 
       $$
       \vec{F}_{r} = \vec{P} + \vec{N} = 0.
       $$
    Essa for\c ca oposta \`a Peso, que denotamos por N, ser\'a estudada a seguir.
   \end{example}
  
  \subsubsection{For\c ca Normal}
    A for\c ca normal \'e a resultante do contato dos corpos, com dire\c c\~ao perpendicular \`as superf\'icies de contato. Ela n\~ao
    possui uma f\'ormula exata, pois \'e uma intera\c c\~ao muito complexa relacionada a mol\'eculas e for\c cas eletromagn\'eticas.
   \begin{example}
     Considere uma escada encostada a uma parede. Ent\~ao, haver\'a uma for\c ca normal conseguinte do contato entre a escada
     e a parede perpendicular \`a parede.
     \begin{center}
     \begin{tikzpicture}
        \draw (0,0) -- (0,4);
        \draw[->, >=stealth, thick](0, 2)-- node[right] {$\vec{N}$} (2, 2);
        \draw (3,0) -- (0.1,0) -- (0.1,2) -- cycle;
        \node at (1.5, 1) [circle, fill, inner sep=1pt] {};
    \end{tikzpicture}
    \end{center}
   \end{example}
  \begin{example}
    Considere um bloco no elevador. Qual \'e a for\c ca normal? Na situa\c c\~ao (a), considere-o [o elevador] parado. Na
    situa\c c\~ao (b), ele possui velocidade constante. No caso (c), o elevador est\'a acelerando para cima. Finalmente,
    no item (d), o elevador acelera para baixo.

    Estando ele parado, a acelera\c c\~ao e a velocidade s\~ao nulos. Com isso, sabe-se o valor da resultante: 
      $$
        \vec{F}_{r} = m \vec{a} = \vec{0}.
      $$
    Observando a esquem\'atica do problema, nota-se que nada ocorre no eixo x, al\'em de que $\vec{a}_{x} = F_{r}^{x} = 0.$
    No entanto, com rela\c c\~ao ao eixo y, observamos que, apesar de $\vec{a}_{y} = F_{r}^{y} = 0,$ existem for\c cas atuando
    sobre o bloco. Assim, 
      $$
        0 = F_{r}^{y} = -P + N = -mg + N = 0 \Rightarrow N = mg = P.
      $$

    Quanto \`a situa\c c\~ao (b), note que, novamente, a for\c ca resultante ser\'a 0, pois a velocidade \'e constante.
    analisando as for\c cas em y, assim, obtemos o mesmo resultado que no item (a): 
      $$
        0 = F_{r}^{y} = -P + N \Rightarrow N = P = mg.
      $$

    No item (c), como o elevador est\'a acelerado, existe uma acelera\c c\~ao para o bloquinho, tal que 
      $$
        \vec{F}_{r} = m \cdot \vec{a} \neq 0.
      $$
    No eixo y, as duas for\c cas continuam as mesmas, ou seja, 
      $$
        F_{r}^{y} = -P + N = m \cdot a\neq 0.
      $$
    Assim, a for\c ca normal ser\'a dada por 
      $$
        N = m(a + g).
      $$

    Por fim, no caso (d), novamente, h\'a uma acelera\c c\~ao, ent\~ao a resultante \'e n\~ao-nula. No entanto, como a acelera\c c\~ao \'e para baixo, 
      $$
        \vec{F}_{r} = m \cdot \vec{(-a)}\neq 0.
      $$
      Com isso, 
        $$
          F_{r}^{y} = -P + N = m \cdot \vec{(-a)} \Rightarrow N = m(g - a).
        $$
  \end{example}
  \begin{example}
    Considere o um plano inclinado e um bloco deslizando nele. Qual \'e a acelera\c c\~ao $\vec{a}?$
   \begin{center}
    \begin{tikzpicture}
        \draw (0,2) -- (3,0) -- (0,0) -- cycle;
        \node[rectangle, draw] at (1.5,1.5){}; 
        \draw[thick] (0, 0) -- (4, 0);
    \end{tikzpicture}
    \end{center}
    Temos $\vec{N} = |\vec{N}|\hat{j}, |\vec{P}|=mg.$ Lembre-se que 
      $$
        \left\{\begin{array}{ll}
            P_{x}' = |\vec{P}|\sin{\theta}\\
            P_{y}' = |\vec{P}|\cos{\theta}
          \end{array}\right.
      $$
      Com isso, as componentes de for\c ca s\~ao  
     \begin{align*}
       &x') \Rightarrow F_{r}^{x'} = P_{x'} = |\vec{P}|\sin{\theta} = ma_{x'}\\
       &y') \Rightarrow F_{r}^{y'} = -P_{y'} + N = -|\vec{P}|\cos{\theta} + N = 0.
     \end{align*}
     Desta \'ultima equa\c c\~ao, tiramos que a for\c ca normal \'e $N = |\vec{P}|\cos{\theta} = mg\cos{\theta}.$ Al\'em disso,
     da primeira equa\c c\~ao, temos 
       $$
         a_{x}' = \frac{mg\sin{\theta}}{m} = g\sin{\theta}.
       $$
       Portanto, $\vec{a} = a_{x'}\hat{i} = g\sin{\theta}\hat{i}.$
  \end{example}
  \begin{example}
    Exemplo 4.7 do Tipler: Temos $v_{y}\leq 2.5m/s$ e uma altura de 1m. Pergunta-se: Qual \'e o maior \^angulo poss\'ivel?

    Ap\'os deslocar-se em $\Delta x$, o pacote tem $v_{x'}$ dado por 
      $$
      v_{x'}^{2} = v_{0_{x'}}^{2} + 2 a_{x'}\Delta x' = 2a_{x'}\Delta x' \Rightarrow v_{x'}^{2} = \frac{2g\sin{\theta}h}{\sin{\theta}} = 2gh.
      $$
      Logo, $v_{x'} = \sqrt[]{2gh}$. Assim, 
        $$
          \left\{\begin{array}{ll}
              v_{x} = v_{x}'\cos{\theta}\\
              v_{y} = v_{x'}\sin{\theta}\leq 2.50m/s.
            \end{array}\right.
        $$
        Disso segue que 
          $$
          \sqrt[]{2gh}\sin{\theta}\leq 2.5m/s \Rightarrow \sqrt[]{2gh}\sin{\theta_{max}} = 2.5m/s
          $$
        Portanto, $\theta_{max} = 34.4^{\circ}.$
  \end{example}
\newpage

\section{Aula 11 - 25/04/2023}
\subsection{O que esperar?}
\begin{itemize}
  \item For\c ca de Tens\~ao;
  \item Roldanas e Polias.
\end{itemize}

\subsection{For\c ca de Tens\~ao}
  A for\c ca de tens\~ao \'e a for\c ca sofirda por cordas e roldanas. Assume-se que a corda \'e inextens\'ivel e possui massa 
  desprez\'ivel. Um exemplo em que ela aparece bastante \'e quando h\'a roldanas/polias, ou seja, mecanismos que mudam a dire\c c\~ao
  da for\c ca. Vejamos exemplos
 \begin{example}
   Suponha que h\'a um bloco suspenso ao teto por uma corda com massa m e em equil\'ibrio. Qual \'e a for\c ca de tra\c c\~ao?

   Como o bloco est\'a em repouso, sabemos que a resultante vale 0. Note que as for\c cas que agem sobre o bloco s\~ao a tra\c c\~ao
   da corda e a fo\c ca peso. Logo, 
     $$
       0 = \vec{F}_{r} = T - P \Rightarrow T = P = mg.
     $$
 \end{example}
\begin{example}
    Agora, imagine que h\'a um bloco suspenso por duas cordas pr\'oximo \`a quinta de um teto. As duas cordas encontram-se num
  ponto p em que h\'a um n\'o. A primeira delas sai horizontalmente da parede, fazendo um \^angulo reto com a mesma. A segunda
  surge pelo teto com um \^angulo $\theta .$ Qual \'e a for\c ca de tra\c c\~ao que age sobre a corda que sai do ponto de n\'o
  delas e prende-se ao bloco? E nas cordas 1 e 2?

  Quanto ao primeiro item, note que o bloco est\'a em equil\'ibrio, logo a for\c ca resultante em p e no bloco deve ser 0.
  Assim, 
    $$
      T_{3}-P = 0 \Rightarrow  T_{3} = m \cdot g.
    $$

  Para o segundo, decomp\~oe-se a tra\c c\~ao nas suas componentes x e y:
 \begin{itemize}
   \item[Em x:] $T_{1}\cos{(\theta )}-T_{2}=0$
   \item[Em y:] $T_{1}\sin{(\theta )}-T_{3}=0.$
 \end{itemize}
 Pela segunda equação, utilizando o valor para $T_{3}$ previamente encontrado, segue que 
   $$
      T_{1}\sin{(\theta )} - T_{3}\Rightarrow T_{1}=\frac{mg}{\sin{(\theta )}}.
   $$
   Logo, combinando isso com a primeira equação, 
     $$
       T_{2}=T_{1}\cos{(\theta )} - \frac{mg}{\sin{(\theta )}}\cos{(\theta )} = \frac{mg}{\tan{(\theta )}}.
     $$
  Portanto, usando $\alpha =90^{\circ} - \theta $, segue que 
    $$
       \left\{\begin{array}{ll}
          T_{1}=\frac{mg}{\sin{(\theta )}} \Rightarrow 1.29mg\\
          T_{2}=\frac{mg}{\tan{(\theta )}} \Rightarrow 0.83mg\\
          T_{3}=mg.
        \end{array}\right.
    $$
\end{example}
  O próximo exemplo ilustra as situações de polias num de seus contextos mais comuns - Máquina de Atwood.
 \begin{example}
   A máquina de Atwood consiste em uma polia presa ao teto com um bloco em cada ponta. O primeiro bloco tem massa
   $m_{1}$ e o segundo tem $m_{2}$. A corda exerce uma tração T e os blocos têm pesos $P_{1}, P_{2}.$ 

   Se $m_{1}>m_{2}$, segue que $m_{1}$ desce e $m_{2}$ sobe. Também é possível relacionar cada quantidade.
   Se o deslocamento, a velocidade e a aceleração do primeiro bloco são $\Delta d_{1}, v_{1}, a_{1}$ e, do segundo bloco,
   $\Delta d_{2}, v_{2}, a_{2}$, essas quantias vinculam-se da seguinte forma:
  \begin{align*}
    &|\Delta d_{1}| = |\Delta d_{2}|\\
    &|v_{1}| = |v_{2}|\\
    &|a_{1}| = |a_{2}|.
  \end{align*}
    As forças atuando no bloco 1 são a tração para cima, a peso $P_{1}$ para baixo e a aceleração como consequência da
    diferença de massas entre os blocos, a qual aponta para baixo também. Por outro lado, no bloco 2, as forças são
    a tração para cima, a peso $P_{2}$ para baixo e, como consequência da diferença de massas, a aceleração que surge,
    dessa vez apontando para cima, já que este sobe. Em outras palavras, as seguintes equações descrevem o movimento no
    eixo y:
   \begin{align*}
     &1:\quad T-P_{1} = -m_{1}a\\
     &2:\quad T-P_{2} = m_{2}a.
   \end{align*}
   As equações que descrevem o bloco subindo podem ser encontradas notando que $-P_{1}+P_{2}=-m_{1}a-m_{2}a \Longleftrightarrow 
   -m_{1}g + m_{2}g = -(m_{1}+m_{2})a \Longleftrightarrow  g(m_{1}-m_{2}) = (m_{1}+m_{2})a.$ Disto, concluímos que 
     $$
       a = \frac{(m_{1}-m_{2})}{(m_{1}+m_{2})},\quad g>0, m_{1}>m_{2}.
     $$
     Além disso, 
       $$
         T = \frac{2m_{1}m_{2}}{(m_{1}+m_{2})}g
       $$
 \end{example}
\begin{example}
   O próximo exemplo que veremos é o de blocos numa mesa: Há dois blocos presos por uma corda que passa por uma polia.
  O primeiro bloco encontra-se na parte superior da mesa, possuindo massa $m_{1}$. O segundo bloco está suspenso, sendo
  sua massa $m_{2}.$ Nesse contexto, o bloco 2 puxa o bloco 1, que desloca-se $\Delta x_{1}$ unidades com aceleração em sua direção
  $a_{1},$ resultando no próprio movimento de $\Delta y_{2}$ com uma aceleração de $a_{2}$ para baixo.

  Para lidar com esse sistema, observe as forças que atuam em cada bloco - No bloco 1, a gravidade resulta na força
  peso $P_{1}$ para baixo e, pela Terceira Lei, há a consequência da normal $N_{1}.$ Além delas, há a tração T da corda.
  Com relação ao bloco 2, não há força de contato, então ele é governado pelo par ação reação da Tração T e da força
  peso $P_{2}.$

  Agora, analisando os vínculos dos valores, observa-se que $|\Delta x_{1}|=|\Delta y_{2}|$ e $|a_{1}| = |a_{2}| = a.$
  Podemos dar continuidade e estudar as equações. Começando pelo bloco 1, 
 \begin{align*}
   &\text{Eixo x: } F_{r} = T = m_{1}\cdot a\\
   &\text{Eixo y: } N_{1} - P_{1} = 0 (\text{Sem movimento.})
 \end{align*}
 Estudando, então, o bloco 2, o único movimento sendo no eixo y: 
\begin{align*}
  F_{r} &= T - P_{2} = -m_{2}a\\
        &\Rightarrow m_{1}a - P_{2} = -m_{2}a\\
        &\Rightarrow m_{1}a - m_{2}g = -m_{2}a\\
        &\Rightarrow (m_{1}+m_{2})a = m_{2}g\\
        &\Rightarrow a = \frac{m_{2}}{m_{1}+m_{2}}g.
\end{align*}

\end{example}
 \newpage

\section{Aula 12 - 27/04/2023}
\subsection{O que esperar?} 
\begin{itemize}
  \item For\c ca de Tens\~ao;
  \item For\c ca de mola/For\c ca El\'astica.
\end{itemize}
\subsection{For\c ca de Mola}
  Sendo $x_{0}$ a posi\c c\~ao da equa\c c\~ao da mola, ao aplicarmos uma for\c ca na mola e deformamos, a rea\c c\~ao
  \'e uma for\c ca chamada restauradora, pois ela \'e respons\'avel por recuperar o estado de equil\'ibrio da mola. Sua f\'ormula \'e 
    $$
      \vec{F} = -k\Delta \vec{x},
    $$
    em que k \'e a constante el\'astica da mola. Podemos elaborar nessa f\'ormula um pouco mais utilizando a segunda lei de Newton, pois
    $F = -kx = ma = m \frac{dv}{dt}.$ Assim, $-kx = m \frac{d^{2}x}{dt^{2}}$, de onde encontramos uma f\'ormula para a posi\c c\~ao em fun\c c\~ao de tempo 
      $$
      \frac{d^{2}x}{dt^{2}} = -\frac{k}{m}x = -\omega^{2}x \Rightarrow \boxed{x(t) = \frac{d^{2}x}{dt^{2}} = A\sin{\omega t}}
      $$
      Se derivarmos essa fun\c c\~ao, a velocidade do movimento ser\'a 
        $$
        \frac{d}{dt}\sin{\omega t} = \frac{d\sin{u}}{du}\frac{du}{dt} = \cos{u}\omega = \omega \cos{\omega t}
        $$
      e a acelera\c c\~ao 
      $$
      \frac{d}{dt}\cos{u} \omega = \frac{d}{du}\cos{u}\omega \frac{du}{dt} = \omega^{2}\sin{\omega t}.
      $$
      Vejamos um exemplo
     \begin{example}
        Considere um objeto em repouso preso a uma mola com posi\c c\~ao de equil\'ibrio $x_{0}$. Ocorre um
        deslocamento da mola em $\Delta y$, aplicando uma for\c ca $\vec{F}_{el}$ ao bloquinho. Qual \'e a for\c ca normal?

        Note que todas as for\c cas est\~ao atuando no eixo y. Como o corpo est\'a parado, a resultante no eixo y \'e nula, tal que
      $F_{r}^{y} = |F_{el}| + N - P$. Com isso, $N = P - |F_{d}| = mg - k|\Delta y|.$
     \end{example}
 \subsection{For\c ca de Tra\c c\~ao}
   \begin{example}
    Considere um elevador manual segurado por uma roldana com uma pessoa dentro dela. Essa pessoa est\'a puxando a corda, exercendo
    uma for\c ca $\vec{F}$ para baixo. O sistema pessoa-elevador tem uma massa de $m=95kg$. Utilize $g=9.81m/s^{2}$.
   \begin{itemize}
     \item[a)] Qual \'e o m\'odulo de $\vec{F}$ para subir com velocidade constante?
       \item[b)] Qual \'e a for\c ca resultante para uma acelera\c c\~ao $a_{y}=1.3m/s^{2}?$
   \end{itemize}

   (a) Observe que, ao puxar a corda, a rea\c c\~ao gerada \`a for\c ca do pux\~ao \'e a for\c ca de tra\c c\~ao. Al\'em disso,
   a resultante no ponto em que a pessoa puxa a corda \'e igual \`a tra\c c\~ao. Assim, como h\'a a tra\c c\~ao no elevador tamb\'em, 
     $$
   \vec{F}_{R}^{y}= -P + T + T = -P + 2T 
     $$
     Como \'e pedido que o elevador suba com velocidade constante, $\vec{F}_{R}^{y} = 0$, donde segue que 
       $$
         T = \frac{P}{2} = \frac{95 \cdot 9.8}{2} = 466N.
       $$

  (b) Como encontramos a forma da resultante no \'ultimo item, temos 
    $$
      \vec{F}_{R}^{y} = - P + 2T = ma_{y} \Rightarrow T = F = \frac{ma_{y}+P}{2}.
    $$
    Assim, $F = \frac{m}{2}(a_{y}+g)^{2} = 528N.$
   \end{example} 

  \begin{example}
    Dado um bloco e multiplas polias, sendo o bloco com um peso de $2670N.$ Suponha que o sistema est\'a em equil\'ibrio.
   \begin{itemize}
     \item[a)] Quanto vale a tra\c c\~ao no sistema da primeira roldana?
     \item[b)] Agora, quanto vale a tra\c c\~ao considerando tamb\'em a segunda roldana no sistema como separada?
     \item[c)] Suponha que a polia passa por uma nova polia ligada ao teto antes de chegar ao bloco. Qual \'e a tra\c c\~ao nesse caso?
   \end{itemize}

   (a) Como a polia redireciona a dire\c c\~ao da tra\c c\~ao uma vez, $F_{r}^{y} = 2T - P = 0,$ segue que $T = \frac{P}{2} = 1385N.$

   (b) Apesar de ter adicionado uma nova polia, ainda assim, como s\'o h\'a dois pontos de sustenta\c c\~ao, $F_{r}^{y} = 2T - P \Rightarrow T = \frac{P}{2}.$

   (c) Diferente do item (b), como ela passa por mais uma polia, h\'a um novo ponto de sustenta\c c\~ao al\'em do segundo, sendo eles
   o teto, a polia presa ao teto e a polia preso ao bloco. Logo, $F_{r}^{y} = 3T - P = 0$, ou seja, $T = \frac{P}{3} = 890N.$
  \end{example}

\section{Aula 13 - 03/05/2023}
\subsection{O que Esperar?}
\begin{itemize}
  \item Revis\~ao Pr\'e-Prova
\end{itemize}
\subsection{Cinem\'atica}
\subsubsection{Movimento Unidimensional}
  As equa\c c\~oes do movimento unidimensional s\~ao 
 \begin{align*}
   &\vec{r}(t)\Rightarrow \vec{v}(t)=\frac{d \vec{r}(t)}{dt} \Rightarrow \vec{a}(t) = \frac{d \vec{v}(t)}{dt} \\
   &\vec{v}(t)-\vec{v}(t_{0}) = \int_{t_{0}}^{t} \vec{a}(t)dt \Rightarrow \vec{r}(t) - \vec{r}(t_{0}) = \int_{t_{0}}^{t} \vec{v}(t)dt
 \end{align*}
 \subsubsection{Movimento Bidimensional}
  Em duas dimens\~oes ou mais, 
    $$
      \vec{r}(t) = x(t)\hat{i} + y(t)\hat{j} + z(t)\hat{k} (\text{ O mesmo para v(t) e a(t)}).
    $$
    Neste caso, decompomos o movimento em cada eixo e tratamos como movimento unidimensional. No caso do movimento bidimensional uniforme,
    as velocidades em ambos os eixos s\~ao constantes, tal que 
      $$
         \left\{\begin{array}{ll}
            x(t) = x_{0} + v_{x}t - t = \frac{x-x_{0}}{v_{x}}\\
            y(t) = y_{0} + v_{y}t.
          \end{array}\right.
      $$
      A trajet\'oria ser\'a $y=y_{0} + \frac{v_{y}}{v_{x}}(x-x_{0}).$ 
\subsubsection{Trajet\'oria de Proj\'eteis}
      Vimos tamb\'em sobre o movimento de proj\'eteis. Nele, a velocidade no eixo x \'e constante; No eixo y, $v_{y}(t) = v_{y}^{0} + at$,
      $a\neq0$. Assim, $x(t) = x_{0} + v_{x}^{0}t$ e $y(t) = y_{0} + v_{y}^{0}t$. Al\'em disso, a trajet\'oria \'e dada por 
        $$
        y = \tan{\theta }x - \frac{gx^{2}}{2|v_{0}|^{2}\cos^{2}{(\theta )}}
        $$
      Utilizando esses dados, a altura m\'axima de um objeto em trajet\'oria de proj\'etil, $(x_{m}, y_{m})$, ocorre quando
      $v_{y}(t_{m}) = 0$, tal que $t_{m}=\frac{|\vec{v}_{0}|\sin{(\theta )}}{g}, y_{m}=\frac{1}{2}\frac{|v_{0}|^{2}\sin^{2}{(\theta )}}{g}.$ 
      Finalmente, vimos o alcance de um objeto em lan\c camento, o qual \'e dado pelo ponto cartesiano $(x_{r}, 0)$: 
        $$
        x_{r} = \frac{v_{0}^{2}}{g}\sin{\theta }\cos{\theta },\quad t_{m} = \frac{2|\vec{v}_{0}|\sin{(\theta)}}{g} = 2t_{m}
        $$
\subsubsection{Movimento Circular}
        Al\'em desses tipos de movimentos, aprendemos sobre o movimento circular, sumarizado na tabela
    \begin{table}[h!]
    \centering
    \begin{tabular}{|c|c|c|c|}
        \hline
        & \textbf{Vetoriais} & \textbf{Variáveis angulares} & \textbf{Variáveis escalares} \\
        \hline
        \textbf{Posição} & $\vec{r}(t)$ & $\theta(t)$ & $s(t) = R\theta(t)$ \\
        \hline
        \textbf{Velocidade} & $\vec{v}(t)$ & $\omega(t) = \frac{d \theta(t)}{dt}$ & $v(t) = \frac{d s(t)}{dt} = R\omega(t)$ \\
        \hline
        \textbf{Aceleração} & $\vec{a}(t)$ & $\alpha(t) = \frac{d\omega(t)}{dt} = \frac{d^2\theta(t)}{dt^2}$ & $|\vec{a}(t)| = \frac{dv(t)}{dt} = R\alpha(t),\quad |\vec{a}_{cp}| = \frac{v^2}{R}$ \\
        \hline
    \end{tabular}
    \caption{Resumo movimento circular.}
    \label{Resumo2MCU}
  \end{table}

\subsection{Revis\~ao de Din\^amica}
  A din\^amica \'e a rela\c c\~ao entre o movimento e as intera\c c\~oes/for\c cas.
  \subsubsection{Leis de Newton}
  A primeira Lei de Newton \'e a seguinte:
 \begin{quote}
   \hypertarget{first_newton}{``Um corpo em repouso ou em Movimento Uniforme matn\'em o seu estado a n\~ao ser que haja uma for\c ca.''}
    \end{quote}
  A segunda relaciona massa e acelera\c c\~ao para descrever a intera\c c\~ao: 
    $$
    \hypertarget{second_newton}{\vec{F}_{r} = m \vec{a}.}
    $$
  A terceira e \'ultima \'e a chamada lei da a\c c\~ao e rea\c c\~ao 
 \begin{quote}
   \hypertarget{third_newton}{``Para toda for\c ca que \'e aplicada, ocorrer\'a uma for\c ca de rea\c c\~ao com mesma intensidade, mas dire\c c\~ao oposta''}
  \end{quote}
 \subsubsection{Tipos de For\c ca}
 Em seguida, vimos os tipos de for\c ca, sendo elas 
\begin{itemize}
  \item[1)] For\c ca Peso: $\vec{P} = m \vec{g}.$
  \item[2)] For\c ca Normal: Aparece em corpos que est\~ao em contato, respons\'avel por impedir objetos de entrarem um no outro.
  Caso o contato suma, a normal tamb\'em desparece, ou seja, se perde o contato, $\vec{N} = 0.$
  \item[3)] For\c ca de Tra\c c\~ao $(\vec{T})$: Atua no contexto de cordas, polias, etc.
  \item[4)] For\c ca El\'astica: For\c ca restaurativa quando uma mola \'e distorcida. $\vec{F} = -k \vec{x}.$
\end{itemize}
\subsection{Exemplos}
\begin{example}
  Considere um forno esquentado at\'e 200 graus. Num peda\c co deste forno, h\'a um pequeno buraco. Al\'em disso, h\'a algo dentro do forno.
Atrav\'es desse buraco, h\'a um feixe de part\'iculas saindo com velocidade $\vec{v}_{0}$. Na dire\c c\~ao pela qual elas est\~ao saindo,
h\'a um gradiente de campo magn\'etico $B'$ com uma for\c ca de origem magn\'etica agindo nele, dada por $\vec{F}\approx \mu_{B}\vec{B}'$,
em que $\mu_{B}$ \'e uma constante chamada magneton de Bohr com valor $\mu_{B}=9.27 \cdot 10^{-24}J/T.$ Quando a part\'icula passa pelo
campo ap\'os percorrer a dist\^ancia D, a for\c ca far\'a com que ela mude sua trajet\'oria e comece a mover-se para baixo. No fim do sistema, tem uma m\'aquina
que registrar\'a a posi\c c\~ao em que a part\'icula ir\'a parar, ou seja, um desvio $\Delta y$ de onde ela originalmente
teria parado caso continuasse a seguir reto.
Temos $|v_{0}|=400m/s, D = 10m, \Delta y = 19cm, B' = 2.5T/m.$ Qual \'e o \'atomo? 

Vamos separar em eixos. No eixo x, 
  $$
    v_{x} = v_{0}^{x} = |\vec{v}_{0}|,\quad x(t) = |\vec{v}_{0}|t.
  $$
  Disto segue que $t = \frac{D}{|\vec{v}_{0}|} = 0.025s.$

  No eixo y, 
    $$
      y(t) = \frac{1}{2}at^{2}.
    $$
  Podemos encontrar a acelera\c c\~ao notando que $F = ma = \mu_{B}B',$ ou seja, $\vec{a} = \frac{\mu_{B}B'}{m}$. Assim, 
    $$
      y(t) = \frac{1}{2}\frac{\mu_{B}B'}{m}t^{2}.
    $$
  No instante final, 
    $$
      \Delta y=\frac{1}{2}\frac{\mu_{B}B'}{m}t_{f}^{2} \Rightarrow m = \frac{1}{2}\frac{\mu_{B}B'}{\Delta y}t_{f}^{2}.
    $$
  Podemos resolver essa conta com os dados do enunciado: 
    $$
      m = \frac{1}{2}\frac{9.27 \cdot 10^{-24}\cdot 2.5}{0.19}(0.025)^{2} = 3.81 \cdot 10^{-26}kg.
    $$
  Pela tabela peri\'odica, utilizando a unidade de massa at\^omica como $1.66 \cdot 10^{-27}kg,$ temos 
    $$
      m = m^{*}\cdot u,
    $$
  sendo $m^{*}$ o n\'umero de massa at\^omica. Assim, 
    $$
      m^{*} = \frac{m}{u}\approx 22.9\approx 2 \cdot 11.
    $$
  Portanto, os \'atomos saindo pelo fog\~ao s\~ao de s\'odio.
\end{example}
\begin{example}
  Seuponha que tem um jogador de futebol a uma dist\^ancia d do gol. Ele chuta a bola, a qual faz um \^angulo $\theta $ com o 
eixo horizontal. O gol possui uma altura h. Qual \'e a menor e a maior velocidade para fazer o gol? Dados d = 9m, h = 2.4m, $\theta =30^{\circ}, g=9.8m/s^{2}.$

  Para entrar no gol, a bola deve percorrer toda a d e ficar no ch\~ao, ou acertar um lugar menor que h. Em outras palavras,
  temos uma restri\c c\~ao com rela\c c\~ao \`a trajet\'oria (Sempre que houver uma restri\c c\~ao de trajet\'oria, utilize a equa\c c\~ao da trajet\'oria.).
  Segue que 
    $$
    y = \tan{(\theta)}x - \frac{gx^{2}}{2v_{0}^{2}\cos^{2}{(\theta )}}.
    $$
  Em x = d, $0\leq y\leq h$, tal que 
    $$
    0\leq \tan{(\theta)}d - \frac{gd^{2}}{2v_{0}^{2}\cos^{2}{(\theta )}}\leq h.
    $$
  Vamos quebrar o problema em duas partes. Na primeira, 
 \begin{align*}
   &\frac{gd^{2}}{2v_{0}^{2}\cos^{2}{\theta }}\leq \tan{(\theta )}d \Rightarrow gd^{2}\leq \tan{(\theta )}d \cdot 2v_{0}^{2}\cos^{2}{(\theta)}\\
   & \Rightarrow \frac{gd^{2}}{\tan{(\theta )}d \cdot 2\cos^{2}{(\theta )}}\leq v_{0}^{2} \Rightarrow  \frac{gd}{2\sin{\theta }\cos{\theta }}\leq v_{0}^{2}\\
   & \Rightarrow v_{0}\geq 10.09m/s.
 \end{align*}  
 Para a parte 2,
 \begin{align*}
   &\tan{(\theta )}d - \frac{gd^{2}}{2v_{0}^{2}\cos^{2}{(\theta )}}\leq  h \Rightarrow \tan{(\theta )}d - h\leq \frac{gd^{2}}{2v_{0}^{2}\cos^{2}{(\theta )}}\\
   & \Rightarrow v_{0}^{2}(\tan{(\theta )}d - h)\leq \frac{gd^{2}}{2\cos^{2}{(\theta )}} \Rightarrow  v_{0}^{2}\leq \frac{gd^{2}}{2\cos^{2}{(\theta )}(\tan{(\theta )}d-h)}\\
   & v_{0}\leq 13.76m/s.
 \end{align*}
 Portanto, $10.09m/s\leq v_{0}\leq 13.76m/s.$
\end{example}
\newpage

\section{Aula 14 - 08/05/2023}
\subsection{O que esperar?}
\begin{itemize}
  \item Força de Atrito;
  \item Diferença entre atrito estático e dinâmico.
\end{itemize}
\subsection{Força de Atrito.}
  A força de atrito ocorre como uma forma de resistência ao movimento, é uma interação que precisa ser superada para que ocorra
  o movimento de dado objeto. Uma de suas característica é que ela será sempre tangencial à superfície de contato.
  Por experiência na vida real, é mais fácil mover um objeto que está em movimento do que um parado, o que sugere dois tipos diferentes
  de forças de atrito. De fato, ela pode ser categorizada em ``Força de Atrito Estático'' e ``Força de Atrito Cinético''.
\begin{center}
\begin{tikzpicture}[scale=1, transform shape]

% Draw surface
  \draw[thick] (0,0) -- (10,0) node[right] {$\mu_e \text{ é igual nos dois blocos.}$};

% Draw the first block (smaller base, taller)
\draw[fill=blue!30] (2,0) rectangle (4,4);
\node at (3,2) {Bloco 1};

% Draw the second block (larger base, shorter)
\draw[fill=red!30] (6,0) rectangle (9,3);
\node at (7.5,1.5) {Bloco 2};

% Draw weight forces (W1 and W2)
\draw[-{Latex[length=3mm]}, thick] (3,2) -- (3,-1) node[midway, left] {$\vec{P_1}$};
\draw[-{Latex[length=3mm]}, thick] (7.5,1.5) -- (7.5,-1) node[midway, left] {$\vec{P_2}$};

% Draw normal forces (N1 and N2)
\draw[-{Latex[length=3mm]}, thick] (3,4) -- (3,6) node[midway, right] {$\vec{N_1}$};
\draw[-{Latex[length=3mm]}, thick] (7.5,3) -- (7.5,5) node[midway, right] {$\vec{N_2}$};

\end{tikzpicture}
\end{center}

\subsubsection{Força de Atrito Estático}
  Esta forma de atrito surge em objetos que têm resistência ao começo do movimento. Numericamente, segue que 
    $$
      F_{e} = \mu_{e}|\vec{N}|,
    $$
  em que $\mu_{e} < 1$ é um coeficiente chamado ``coeficiente de atrito estático'', um valo que altera com base na natureza
  dos materiais de contato. Note que a força de atrito independe da área de contato! Além disso, o coeficiente é um valor
  puramente numérico. Observemos essas situações mais a fundo.

   Considere um bloco em contato com uma superfície tal que há uma força $\vec{F}$ agindo sobre ele. Além disso,
   suponha que essa força tem um valor em módulo menor que a força de atrito estático $\vec{F}_{e}.$ Vamos analisar as forças:
  \begin{align*}
    &y:\quad -P + n = 0 \Rightarrow N = P\\
    &x:\quad F - F_{at} = 0 \Rightarrow F_{at} = F.
  \end{align*}
  Quando $|\vec{F}|<F_{e},$ o bloco mantém-se ausente de movimento, mas, quando $|\vec{F}| > F_{e},$ o bloco anda!
 \subsubsection{Força de Atrito Cinético}
 Essa versão da força surge para objetos que já estão em movimento, mas continuam em contato com uma superfície. Neste caso,
 novamente, a descrição dessa interação é dada por um coeficiente adimensional, dessa vez conhecido como ``coeficiente de atrito
 cinético'', $\mu_{c} < 1$. A força será dada por 
   $$
     F_{at}^{c} = \mu_{c}|\vec{N}|.
   $$
  O fenômeno previamente mencionado de haver uma facilidade maior para mover algo que já estava em movimento é explicado observando
  que $\mu_{c} < \mu_{e}!$

 \begin{example}
   Considere um bloco em um plano inclinado que tenha ângulo $\theta $ com a superfície. Qual é o ângulo máximo para que não haja movimento?

   Começamos decompondo as forças do bloco: 
  \begin{align*}
    &y:\quad -P_{y} + N = 0 \Rightarrow -P\cos{(\theta )} = N\\
    &x: P_{x} - F_{at} = 0 \Rightarrow F_{at} = P\sin{(\theta )} = F_{e}
  \end{align*}
  Para resolver o problema, podemos escrever $F_{e} = P\sin{(\theta_{max} )}$, visto que o movimento ocorrerá
  apenas se a a força em x for maior que o atrito estático. Com isso, 
 \begin{align*}
   \mu_{e}N &= P\sin{(\theta_{max})} \Rightarrow \mu_{e}P\cos{(\theta_{max})} = P\sin{(\theta_{max})}\\
            &\Rightarrow \tan{(\theta_{max})} = \mu_{e}.
 \end{align*}
 \end{example}
\begin{example}
  Considere um piso escorregadio e um trenó em cima dele com massa m. Ele é puxado com força $\vec{F}$ por uma pessoa utilizando uma corda,
  a qual faz um ângulo horizontal $\theta $. Para qual valor de $\vec{F}$ o trenó começa a andar?
 \begin{center}
   \begin{tikzpicture}[scale=1, transform shape]

% Draw slippery floor
\draw[thick] (0,0) -- (10,0);

% Draw sled
\draw[fill=blue!30] (4,0) rectangle (7,2);
\node at (5.5,1) {$m$};

% Draw force vector and rope
\coordinate (A) at (7,2);
\draw[dashed] (A) -- ++(30:4);
\draw[-{Latex[length=3mm]}, thick] (A) -- ++(30:3) node[midway, above] {$\vec{F}$};

% Draw angle theta
\draw[-{Latex[length=2mm]}, thick] (A) -- ++(0:2);
\draw (A) ++(30:1) arc (30:0:1);
\node[anchor=west] at (7.9,2.3) {$\theta$};

\end{tikzpicture}
 \end{center}

  Novamente decompondo o problema em suas forças componentes, temos 
 \begin{align*}
   &y: -P + N + F_{y} = 0 \Rightarrow N = P - F_{y} = P - F\sin{(\theta )}\\
   &x: -F_{at} + F_{x} = 0 \Rightarrow F_{x} = F_{at} \text{ (Até que } F_{at}=\mu_{e}|\vec{N}|).
 \end{align*}
 Para que o movimento aconteça, é preciso que $F_{x} > F_{at}^{e} = F_{e}$, ou seja, 
   $$
     F\cos{(\theta )} > \mu_{e}(P-F\sin{(\theta )}) \Rightarrow  F > \frac{\mu_{e}P}{\cos{(\theta )}+\mu_{e}\sin{(\theta )}}.
   $$
   Agora podemos calcular, dados os valores m=50kg, $\mu_{e}=0.20, \mu_{c}=0.15, \theta =40^{\circ}$, podemos encontrar
   $|\vec{F}_{at}|$ e a aceleração do trenó nos casos 

  \begin{itemize}
    \item[a)] $|\vec{F}|=100N;$
    \item[b)] $|\vec{F}|=140N$
  \end{itemize}
  
  Antes de mais nada, observe que, com a fórmula que encontramos, para que haja movimento é preciso que 
    $$
      F_{c} > \frac{0.20 \cdot 50 \cdot 9.8}{0.766+0.20 \cdot 0.693} \cong{109N}
    $$

  a) $|\vec{F}|= 100N < F_{c}$, ou seja, o bloco está parado! Assim, sua aceleração será 0 e o valor da força de atrito será
    $$
      -F_{at} + F_{x} = 0 \Rightarrow  F_{at} = 100\cos{(\theta )} = 77N.
    $$

  b) $|\vec{F}| = 140N > F_{c}$, tal que o bloco, neste caso, anda. Com isso, ele terá aceleração não-nula e força de atrito com valor 
    $$
      F_{at} = F_{at}^{c} = \mu_{c}|\vec{N}| = \mu_{c}(|P-F_{y}|) = \mu_{c}(|P-F\sin{(\theta )}|) = 0.15(50 \cdot 9.8 - 140\sin{(40^{\circ})}) = 60N.
    $$
    Em particular, note o valor da força de atrito inferior ao do item (a). Para encontrarmos a aceleração, segue que, no eixo x,
   \begin{align*}
     x:\quad -F_{at}^{c} + F_{x} &= ma\neq0 \Rightarrow a = -\frac{F_{at}^{c}+F_{x}}{m} =\\
                                 &\frac{-60+140\cos{(40^{\circ})}}{50} = 0.94m/s^{2}.
   \end{align*}

  \begin{example}
    Considere um sistema de caixas dentro de um trem em movimento, sua velocidade inicial sendo $v_{0}$. Passado um tempo,
    o trem passa a desacelerar constantemente com aceleração a. Qual é a menor distância para parar o trem sem mover as caixas? Tome
    $\mu_{e}=0.25, v_{0}=48km/h.$

    Para as caixas continuarem paradas, é preciso que a força $F$ atuando numa caixa satisfaça $F\leq F_{at}^{e} = \mu_{e}N$.
    Quebrando a situação em eixos x e y, 
   \begin{align*}
     &y: N = P \\
     &x: F_{at} = F.
   \end{align*}
   Utilizando essas informações, obtemos que a condição torna-se $F\leq \mu_{e}mg.$ Elaborando mais, segue que 
     $$
       ma\leq \mu_{e}mg \Rightarrow  a\leq \mu_{e}g.
     $$
    Através da Equação de Torricelli, 
      $$
        v_{f}^{2} = v_{0}^{2} + 2a\Delta x \Rightarrow 0 = v_{0}^{2} - 2|\vec{A}|\Delta x \Rightarrow \Delta x = \frac{v_{0}^{2}}{2|\vec{A}|}.
      $$
      Portanto, 
        $$
          \Delta x^{(min)} = \frac{v_{0}^{2}}{2\mu_{e}g} = 36m.
        $$
  \end{example}
  \begin{example}
    Considere um chão com coeficiente de atrito estático $\mu_{e}.$ Nele, há dois blocos, A e B, cujas massas são
    $m_{A}, m_{B}$. Eles estão presos por uma corda e sendo puxados por outra corda com força $\vec{F}.$ Qual é a força 
    cinética $F_{c}$ para entrar em movimento?
   \begin{center}
     \begin{tikzpicture}[scale=1, transform shape]

% Draw ground
\draw[thick, pattern=north east lines] (0,0) rectangle (12,0.25);
\node at (6,-0.5) {$\mu_e$};

% Draw block A
\draw[fill=blue!30] (3,0.25) rectangle (5,2.25);
\node at (4,1.25) {$m_A$};

% Draw block B
\draw[fill=red!30] (7,0.25) rectangle (9,3.25);
\node at (8,1.75) {$m_B$};

% Draw ropes
\draw (5,1.25) -- (7,1.25);
\draw[-{Latex[length=3mm]}, thick] (9,1.75) -- ++(0:2) node[midway, above] {$\vec{F}$};

% Draw normal force and weight for block A
\draw[-{Latex[length=3mm]}, thick] (4,2.25) -- ++(90:1.5) node[midway, left] {$\vec{N_A}$};
\draw[-{Latex[length=3mm]}, thick] (4,0.25) -- ++(270:1.5) node[midway, left] {$\vec{P_A}$};

% Draw normal force and weight for block B
\draw[-{Latex[length=3mm]}, thick] (8,3.25) -- ++(90:1.5) node[midway, left] {$\vec{N_B}$};
\draw[-{Latex[length=3mm]}, thick] (8,0.25) -- ++(270:1.5) node[midway, left] {$\vec{P_B}$};

% Draw force F_c for block A
\draw[-{Latex[length=3mm]}, thick] (3,1.25) -- ++(180:1.5) node[midway, above] {$\vec{F_c}$};

% Draw force F_c for block B
\draw[-{Latex[length=3mm]}, thick] (7,1.75) -- ++(180:1.5) node[midway, above] {$\vec{F_c}$};

\end{tikzpicture}
   \end{center}

    Analisaremos as forças para cada um dos blocos, decompondo-as em coordenadas.

    \textbf{Bloco A:} 
   \begin{align*}
     &y: N_{A} = P_{A}\\
     &x: T - F_{at}^{A} = 0 \Rightarrow F_{at}^{A} = T \Rightarrow T_{e} = F_{at}^{a}=\mu_{e}m_{A}g
   \end{align*}

   \textbf{Bloco B:}
  \begin{align*}
    &y: N_{B} = P_{B}\\
    &x: F - T - F_{at}^{B} = 0 \Rightarrow F_{c} = T_{c} + F_{at}^{e(B)} = \mu_{e}m_{A}g + \mu_{e}m_{B}g.
  \end{align*}
  Portanto, 
    $$
      F_{c} =\mu_{e}g(m_{A}+m_{B}).
    $$
  Em particular, note como seria possível tratar os dois blocos como um só com massa $m_{A}+m_{B}!$
  \end{example}
\end{example}
\newpage

\section{Aula 15 - 10/05/2023}
\subsection{O que esperar?}
\begin{itemize}
  \item For\c ca de Arrasto;
  \item Resistência do Ar.
\end{itemize}
\subsection{For\c ca de Arrasto}
  Imagina um objeto se movendo por um fluído. Para o objeto mover-se, ele deve deslocar-se o fluído também. O que influencia na dificuldade para fazer isso?
Por experiências cotidianas, sabemos que a forma do objeto afetará nisso. Além disso, a rapidez com que o objeto anda também é um fator que influencia, junto com o próprio tipo de fluído.
Assim, a forma de arraste é oposta ao movimento! Podemos definí-la com melhor precisão agora: 
  \[
    |\vec{F}_{a}| \varpropto v^{n}.
  \]
  Para baixas velocidades, n =1, enquanto que, para altas velocidades, normalmente n=2. Uma das mais conhecidas formas de for\c ca de arrasto é a resistência do ar: 
    \[
      |\vec{F}_{a}| = b |\vec{v}|^{2}.
    \]
    Temos 
      \[
        P - F_{a} = ma_{y} \Rightarrow a_{y} = \frac{mg - bv^{2}}{m}.
      \]
    Observe que, por conta da diferen\c ca de for\c cas na fórmula da acelera\c cão, pode chegar um momento em que ela passa a valer zero. Ao chegar neste momento,
    diremos que o objeto atingiu sua velocidade terminal. Definimos ela como 
      \[
        \hypertarget{terminal_velocity}{\boxed{v_{T}=\sqrt[]{\frac{mg}{b}}}}
      \]
    Mas o que é essa constante b que está aparecendo? Explicitamente, ela vale 
      \[
        b=\frac{1}{2}C_{a}\rho_{f}A,
      \]
      em que \(C_{a}\) é o coeficiente de arrasto, \(\rho_{f}\) é a densidade do fluído e A é a se\c cão de área transversal do objeto. Disso, conseguimos tirar
    algumas conclusões. Por exemplo, quanto mais denso o fluído ou quanto maior o objeto, mais ele terá resistência ao movimento e menor será sua velocidade terminal.

   \begin{example}
     Considere uma gotinha de chuva perfeitamente esférica. Consideremos seu raio como aproximadamente 1 milímetro.
     Sua massa será dada por \(m_{\text{gota}}=\rho_{\text{água}}V_{\text{gota}}\), e vale aproximadamente \(4 \cdot 10^{-6}kg\). Assuma o coeficiente de arrasto como
     1 e a densidade do ar como \(\rho_{\text{ar}}=1.27kg/m^3\) e a área de se\c cão transversal da gota valendo \(A_{\text{gota}=\pi r^2\approx3 \cdot 10^{-6}m^2}\).

     Com esses dados, segue que 
       \[
         b = \frac{1}{2}C_{A}\rho_{\text{ar}}A_{\text{gota}}\approx 1.9 \cdot 10^{-6} \Rightarrow v_{T} = \sqrt[]{\frac{mg}{b}}\approx 4m/s.
       \]
   \end{example}
  \begin{example}
    Exepmlo 58 Tipler. Considere uma pessoa caindo com um paraquedas. Sabe-se que sua massa é de 64kg e que a velocidade terminal vale \(v_{T}=180km/h\). Qual é
    \(|\vec{F}_{a}|\)? Quanto vale b?

    Com efeito, 
      \[
        F_{R}^{y} = 0 = P-F_{a} \Rightarrow F_{a} = P = mg = 628N.
      \]

    Quanto ao item  b, temos 
      \[
        |\vec{F}_{a}| = b |\vec{v}| \Rightarrow  b = \frac{|\vec{F}_{a}|}{|\vec{v_{T}}^2|} \Rightarrow b = \frac{628}{50^2} = 0.251 \frac{kg}{m}.
      \]
  \end{example}
  O próximo exemplo é o de uma outra for\c ca de arrasto.
 \begin{example}
  Suponha um barco na água. Como evolui a velocidade dele com o tempo?

  Como ele está se movendo no fluído água, a for\c ca responsável por movê-lo para
  a frente resulta no surgimento de uma for\c ca resposta, justamente a de arrasto.
  Pela segunda \hyperlink{second_newton}{lei de Newton}, 
    \[
      F_{R} = F_{a} = -bv = m \cdot a = m \frac{dv}{dt}.
    \] 
  Assim, 
    \[
      -b dt = m \frac{dv}{v} \Rightarrow \int_{0}^{t}-\frac{b}{m}dt = \int_{v_{0}}^{v(t)}\frac{dv}{v}
      \Rightarrow -\frac{b}{m}\int_{0}^{t}dt = \ln{(v)}\biggl|_{v_{0}}^{v(t)}\biggr. = \ln{(v(t))} - \ln{v_{0}} = \ln{\frac{v(t)}{v_{0}}}
    \]
    (Nota: A integral \(\int_{}^{}\frac{1}{x}dx = \ln{(x)}\) é definida assim.)

  Faremos as contas a seguir: 
 \begin{align*}
   &-\frac{b}{m}t \biggl|_{0}^{t}\biggr. = \ln{(\frac{v(t)}{v_{0}})}\\
   &-\frac{b}{m}t = \ln{(\frac{v(t)}{v_{0}})}\\
   &e^{-\frac{b}{m}t} = \frac{v(t)}{v_{0}}\\
   &v(t) = v_{0}e^{-\frac{b}{m}t}
 \end{align*}
  
 Dados os valores \(v_{f}=44km/h, v_{0} = 97km/h, m=900kg, b=68\), por exemplo,
 o tempo para o barco parar será 
   \[
     t_{f} = -\frac{m}{b}\ln{(\frac{v(t_{f})}{v_{0}})} = \frac{900}{68}\ln{(\frac{12}{2t})}.
   \]
 \end{example}

\subsection{Movimentos Curvos}
  Quando um objeto realiza um movimento de curva, ele tem uma tendência
  a ter uma rota restrita. Essa tendência pode ser descrita pro uma for\c ca ``imaginária´´
  que surge da restri\c cão da trajetória do objeto, conhecida como
  for\c ca centrípeta, a qual foi estudada previamente, sendo ela descrita como 
    \[
      \vec{F}_{cp} = m \vec{a}_{cp}.
    \]
    A dire\c cão da varia\c cão da velocidade em dois pontos do movimento curvo 
    é paralelo à acelera\c cão centrípeta mencionada.
   \begin{example}
     Considere uma bolinha que gira e está presa por duas cordas a uma haste.
     As for\c cas atuando sobre ela incluem a Peso P, a tensão da corda 1 \(T_{1}\)
     e a tensão da corda 2 \(T_{2}\). Como a bolinha realiza um movimento circular,
     há uma acelera\c cão resultante que aponta para o centro do raio que a bolinha faz,
     a chamada acelera\c cão centípeta. Sendo o comprimento das cordas L e L, o ângulo
     da bolinha com a horizontal da haste \(\theta \), descreva sua dinâmica.

     Vamos descrever essa situa\c cão para cada eixo. Com efeito, segue que 

     \textbf{Eixo x:}
    \begin{align*}
      &-T_{1}\cos{(\theta )} - T_{2}\cos{(\theta )} = F_{R}^{x} = m(-a_{cp})\\
      &\Rightarrow (T_{1}+T_{2})\cos{(\theta )}=m \frac{v^2}{R} = mR\omega ^2.
    \end{align*}
    \textbf{Eixo y:}
   \begin{align*}
     &-P + T_{1}\sin{(\theta )}-T_{2}\sin{(\theta )}=F_{R}^y = 0\\
     &\Rightarrow T_{2} = T_{1} - \frac{mg}{\sin{(\theta )}} = 9.5N.
   \end{align*}
   
   Além disso, 
     \[
       \omega ^2 = \frac{(T_{1}+T_{2})}{mR}\cos{(\theta )}\approx 20 \Rightarrow \omega \approx 4.5rad/s.
     \]


   \end{example}

\section{Aula 17 - 15/05/2023}
\subsection{O que esperar?}
\begin{itemize}
  \item Continua\c cão de Energia Cinética e Trabalho;
  \item Exemplo de trabalho com molas;
  \item Movimento tridimensional.
\end{itemize}
\subsection{Trabalho e Energia Cinética}
  Na aula passada, vimos o Teorema da Energia Cinética, escrito como 
    \[
      \hypertarget{kin_en_theo}{W_{1\rightarrow 2} = \int_{x_1}^{x_2}Fdx = \Delta E_{\text{cin}}}
    \]
    e que afirma que o trabalho realizado por uma intera\c cão é calculado segundo a variação
    da energia cinética. Caso a for\c ca seja constante, note que 
      \[
        \hypertarget{fcon_kin}{W_{1\rightarrow 2} = \int_{x_1}^{x_2} Fdx = F\int_{x_1}^{x_2}dx = F\Delta x.}
      \]
    Por outro lado, se F for uma for\c ca variável, como a integral representa a área
    abaixo da curva da fun\c cão, o Trabalho será dado pela área debaixo da curva do 
    gráfico de F(x)

\begin{tikzpicture}
\begin{axis}[
    axis lines = left,
    xlabel = $x$,
    ylabel = {$f(x)$},
]

\addplot [
    domain=2:4, 
    samples=100, 
    color=red,
    pattern=north east lines,
    pattern color=red,
] {x^3} \closedcycle;

\addplot[domain=0:5, samples=100, color=black]{x^3};

\draw [dashed, thick] (axis cs: 2,0) -- (axis cs:2,{2^3});
\draw [dashed, thick] (axis cs: 4,0) -- (axis cs:4,{4^3});

\node at (axis cs: 3,30) {$W_{1\rightarrow 2}$};

\end{axis}
\end{tikzpicture}

\begin{example}
   Considere um bloquinho preso a uma parede por uma mola. A posi\c cão de equilíbrio 
   desta mola é \(x_{0}=0\) e a for\c ca resultante é, também, 0. Qual é o trabalho 
   da for\c ca restauradora quando puxamos o bloco até uma posi\c cão \(x_{1}\) para ela
   voltar até um ponto \(x_{2}\)?

   Temos 
     \[
       W_{1\rightarrow 2} = \int_{x_1}^{x_2}F_{\text{mola}}dx = \int_{x_1}^{x_2} -kxdx = 
       k \int_{x_1}^{x_2}xdx = -k \frac{x^{2}}{2}\biggl|_{x_1}^{x_2}\biggr. = -\frac{k}{2}(x_{2}^{2} - x_{1}^{2}.)
     \]
     Em particular, se \(|x_{1} > |x_{2}|,\) a deforma\c cão da mola irá aumentar e o trabalho será negativo. 
     Por outro lado, se \(|x_{1}| < |x_{2}|, \) o trabalho é positivo. 

    \begin{tikzpicture}
\begin{axis}[
    axis lines = middle,
    xlabel = $x$,
    ylabel = {$F_{\text{mola}}$},
    xmin=-5, xmax=5,
    ymin=-5, ymax=5,
]

\addplot [
    domain=-4:-2, 
    samples=100, 
    color=yellow,
    pattern=north east lines,
    pattern color=yellow,
] {-x} \closedcycle;

\addplot [
    domain=2:4, 
    samples=100, 
    color=orange,
    pattern=north east lines,
    pattern color=orange,
] {-x} \closedcycle;

\addplot[domain=-5:5, samples=100, color=black]{-x};

\draw [dashed, thick] (axis cs: -4,0) -- (axis cs:-4,4);
\draw [dashed, thick] (axis cs: -2,0) -- (axis cs:-2,2);
\draw [dashed, thick] (axis cs: 2,0) -- (axis cs:2,-2);
\draw [dashed, thick] (axis cs: 4,0) -- (axis cs:4,-4);

\node at (axis cs: -3,2) {$F_{\text{mola}}$};
\node at (axis cs: 3,-2) {$F_{\text{mola}}$};

\end{axis}
\end{tikzpicture}
  Neste gráfico, a parte amarela demonstra um momento em que o trabalho é positivo,
  tal que há uma compressão da mola. A parte laranja, por outro lado, representa um
  trabalho negativo e, assim, uma disten\c cão da mola.
\end{example}

\subsection{Movimento Tridimensional.}
  O objetivo desta se\c cão é estudar os conceitos de trabalho, energia cinética,
etc. Num contexto de eixos x, y e z. Aqui, vamos utilizar deslocamentos infinitesimais
ao longo de um caminho \(C\), denotado \(\vec{dr}\), mais rigorosamente dado por 
  \[
    \vec{dr} = dx\hat{i} + dy\hat{j} + dz\hat{k}.
  \]
  Em três dimensões, a for\c ca será dada por 
    \[
      \vec{F} = F_{x}\hat{i} + F_{y}\hat{j} + F_{z}\hat{k}.
    \]
  Vejamos como o trabalho funciona neste caso: 
 \begin{align*}
   W_{1\rightarrow 2} &= \int_{\vec{r_{1}}}^{\vec{r_{2}}}\underbrace{\vec{F}\cdot \vec{dr}}_{\text{produto escalar}}\\
                      & \left\{\begin{array}{ll}
                          \vec{F}\cdot \vec{dr} = f_{x}dx + F_{y}dy + F_{z}dz\\
                          \vec{F}\cdot \vec{dr} = |\vec{F}||\vec{dr}|\cos{(\theta )}
                      \end{array}\right.
 \end{align*}
  Então, 
    \[
      W_{1\rightarrow 2} = \int_{\vec{r_{1}}}^{\vec{r_{2}}}(F_{x}dx + F_{y}d_{y} + F_{z}dz) = \int_{\vec{r_{1}}}^{\vec{r_{2}}}|\vec{F}|\cos{(\theta )}|\vec{dr}|.
    \]
  Este tipo de integral é conhecido como integral de linha, pois ela é restrita ao caminho 
  C

\begin{center}
\tdplotsetmaincoords{70}{110}
\begin{tikzpicture}[scale=3,tdplot_main_coords]
\draw[thick,->] (0,0,0) -- (1,0,0) node[anchor=north east]{$x$};
\draw[thick,->] (0,0,0) -- (0,1,0) node[anchor=north west]{$y$};
\draw[thick,->] (0,0,0) -- (0,0,1) node[anchor=south]{$z$};

\tdplotsetcoord{P}{1.414213}{54.735610}{45}

\draw[-stealth,color=red] (O) -- (P) node[above right] {$C$};

\draw[dashed, color=red] (O) -- (Pxy);
\draw[dashed, color=red] (P) -- (Pxy);

\draw (1,0,0) arc (0:45:1);
\tdplotdrawarc{(O)}{0.2}{0}{54.735610}{anchor=north}{$\theta$}

\tdplotsetthetaplanecoords{45}
\tdplotdrawarc[tdplot_rotated_coords]{(0,0,0)}{0.5}{0}{54.735610}{anchor=south west}{$\phi$}

\end{tikzpicture}
\end{center}

 \begin{example}
   Um exemplo simples é o de uma partícula que move-se apenas no eixo z, ou seja,
   \(x = y = 0.\) Neste caso, o deslocamento será \(\vec{dr} = dz \hat{k}\). Suponha,
   também, que a for\c ca é constante e que \(\theta \) seja constante. Então, o
   trabalho realizado por esta for\c ca será 
     \[
       W_{1\rightarrow2} = \int_{\vec{r_{1}}}^{\vec{r_{2}}}\vec{F}\cdot \vec{dr} = \int_{z_{1}}^{z_{2}}|\vec{F}|\cos{(\theta )}|dz| = |\vec{F}|\cos{(\theta )}\Delta z.
     \]
 \end{example}
 Quando fazemos o produto escalar, essencialmente estamos fazendo a proje\c cão de um
vetor no outro. Por isso que há o termo cosseno, estamos projetando a for\c ca F
no eixo desejado.

  Além disso, note que, sempre que a for\c ca e \(\vec{dr}\) forem perpendiculares,
o trabalho será 0, pois o cosseno será nulo! Em particular, com rela\c cão
ao eixo y sem movimento, então \(\vec{P}, \vec{N}\) são perpendiculares a \(\vec{v}, \vec{dr}\), tal que
o trabalho vale 0. No eixo x, \(F_{r} = F_{x} = |\vec{F}|\cos{(\theta )} = ma,\)
de modo que \(F_{x}\) é paralelo a \(\vec{v}, \vec{dr}\). Assim, 
  \[
    W_{1\rightarrow 2} = \int_{x_{1}}^{x_{2}}F_{x}dx = \int_{x_{1}}^{x_{2}}|\vec{F}|\cos{(\theta )}dx.
  \]

  No quesito da energia cinética, o trabalho resultante num caminho C será 
    \[
      W_{R_{1\rightarrow 2}} = \int_{\vec{r}_{1}}^{\vec{r}_{2}}\vec{F}_{R}\cdot \vec{dr} = \int_{\vec{r}_{1}}^{\vec{r}_{2}}m \frac{\vec{dv}}{dt}\cdot \vec{dr}.
    \]
  Note o uso da \hyperlink{second_law}{segunda lei de newton} para decompor a resultante em \(m \frac{\vec{dv}}{dt}.\)
  Agora, ``multiplicando'' a parte de dentro da integral por \(\frac{dt}{dt},\) 
    \[
      W_{1\rightarrow 2} = \int_{\vec{r}_{1}}^{\vec{r}_{2}}m \frac{\vec{dv}}{dt}\cdot \frac{\vec{dr}}{dt}dt = \int_{\vec{v}_{1}}^{\vec{v}_{2}}m \vec{dv}\vec{v(t)}.
    \]
  Logo, 
    \[
      W_{1\rightarrow 2} = \int_{\vec{v}_{1}}^{\vec{v}_{2}}m \vec{dv}\vec{v} = \int_{\vec{v}_{1}}^{\vec{v}_{2}}m \vec{v}\vec{dv}.
    \]
  No entanto, observe que 
    \[
       \left\{\begin{array}{ll}
          \vec{v} = v_{x}\hat{i} + v_{y}\hat{j} + v_{z}\hat{k}\\
          \vec{dv} = dv_{x}\hat{i} + dv_{y}\hat{j} + dv_{z}\hat{k}.
        \end{array}\right.
    \]
  Reescrevendo o trabalho utilizando isso, obtemos 
 \begin{align*}
   W_{1\rightarrow 2} &= \int_{\vec{v}_{1}}^{\vec{v}_{2}}m\biggl[v_{x}dv_{x} + v_{y}dv_{y}+v_{z}dv_{z}\biggr]\\
                      &= \int_{\vec{v}_{1}}^{\vec{v}_{2}}mv_{x}dv_{x} + \int_{\vec{v}_{1}}^{\vec{v}_{2}}mv_{y}dv_{y} + \int_{\vec{v}_{1}}^{\vec{v}_{2}}mv_{z}dv_{z}\\
                      &= m\biggl[\int_{\vec{v}_{1}}^{\vec{v}_{2}}v_{x}dv_{x}+v_{y}dv_{y}+v_{z}dv_{z}\biggr]\\
                      &= m\biggl[\frac{v_{x}^{2}}{2}\biggl|_{v_{1}}^{v_{2}}\biggr. + \frac{v_{y}^{2}}{2}\biggl|_{v_{1}}^{v_{2}}\biggr. + \frac{v_{z}^{2}}{2}\biggl|_{v_{1}}^{v_{2}}\biggr.\biggr]\\
                      &= \frac{m}{2}\biggl[(v_{x_{2}}^{2} + v_{y_{2}}^{2} + v_{z_{2}}^{2}) - (v_{x_{1}}^{2} + v_{y_{1}}^{2} + v_{z_{1}}^{2})\biggr]\\
                      &= \frac{m}{2}|\vec{v}_{2}|^{2} - \frac{m}{2}|\vec{v_{1}}^{2}| = \Delta E_{cin}
 \end{align*}
 Em outras palavras, 
\begin{quote}
  \hypertarget{work_kin_3d}{``O trabalho realizado pela for\c ca resultante em um corpo é igual à varia\c cão da sua energia cinética''.}
\end{quote}
\begin{example}
  Vamos ver um exemplo de integral de linha. Considere uma bolinha caindo de um ponto I até
  outro ponto II. Primeiro, ela come\c ca caindo até o chão e depois empurrada até o ponto. Na segunda op\c cão de caminho, ela cai em linha diagonal até o ponto. No terceiro,
  ela é lan\c cada de forma parabólica. Estudemos o trabalho em cada caso.

  Para o primeiro, decomponha o caminho todo em \(C_{I} = c_{1} + c_{2}\), sendo 
  \(c_{1}\) a queda até o chão e \(c_{2}\) a empurrada até o ponto. Temos 
    \[
      W_{1\rightarrow 2 }^{C_{I}} = \int_{c_{1}}^{}\vec{F}\vec{dr} + \int_{c_{2}}^{}\vec{F}\vec{dr} = \int_{c_{1}}^{}\vec{F}dy \hat{j} + \int_{c_{2}}^{}\vec{F}dx \hat{i}
    \]
    Calculando, segue que
   \begin{align*}
     W_{1\rightarrow 2} &= \int_{c_{1}}^{}-mg \hat{j}\cdot dy \hat{j} + \int_{c_{2}}^{}-mg \hat{j}\cdot dx \hat{i}.\\
                        &= \int_{c_{1}}^{}-mg dy + 0 = -mg y\biggl|_{c_{1}(1)}^{c_{2}(2)}\biggr. = -mg(0-h)mgh.
   \end{align*}

   Para o segundo, precisamos descrever a trajetória do caminho \(C_{II}\). Com isso,
   podemos definir \(y(x) = a + bx\), já que é uma diagonal. Sabemos que 
     \[
       y(0) = h\quad\text{ e } y(d) = 0.
     \]
    Com isso, segue que \(a = h\) e \(h + bd = 0\), tal que \(b = \frac{-h}{d}.\) Utilizando estes dados,
    reescrevemos y como \(y(x) = h - \frac{h}{d}x\). Assim, 
      \[
        W_{1\rightarrow 2}^{C_{II}} = \int_{\vec{r}_{1}}^{\vec{r}_{2}}\vec{F} \cdot \vec{dr}.
      \]
      Note também que \(dy = -\frac{h}{d}dx\). Destarte, 
     \begin{align*}
       W_{1\rightarrow 2}^{C_{II}} &= \int_{\vec{r}_{1}}^{\vec{r}_{2}}-mg \hat{j}\cdot (dx \hat{i} - \frac{h}{d}dx \hat{j})\\
                                   &= \int_{\vec{r}_{1}}^{\vec{r}_{2}}-mg \hat{j}dx \hat{i} + mg \hat{j}\frac{h}{d}dx \hat{j}\\
                                   &= \int_{\vec{r}_{1}}^{\vec{r}_{2}}mg \frac{h}{d} dx = mg \frac{h}{d}x \biggl|_{x_{1}}^{x_{2}}\biggr. = mg \frac{h}{d}(d-0) = mgh.
     \end{align*}
\end{example}
\newpage

\section{Aula 18 - 17/05/2023}
\subsection{O que esperar?}
\begin{itemize}
  \item For\c cas Conservativas;
\end{itemize}
\subsection{Trabalho da For\c ca Peso}
  De forma geral, qual é o trabalho da for\c ca peso? Temos 
    \[
      \vec{r}(t) = x(t)\hat{i} + y(t)\hat{j} + z(t)\hat{k},\quad \vec{dr} = dx \hat{i} + dy \hat{j} + dz \hat{k}.
    \]
  Considerando pontos \(\vec{r}_{1} = (x_{1}, y_{1}, z_{1})\) inicial e \(\vec{r}_{2} = (x_{2}, y_{2}, z_{2})\) final, o trabalho da for\c ca peso será 
    \[
      W_{P}^{C} = \int_{C}^{} \vec{P} \vec{dr} = \int_{C}^{}(-mg \vec{k})\cdot (dx \hat{i} + dy \hat{j} + dz \hat{k}) = - \int_{C}^{}mgdz = -mg\Delta z.
    \]
  Em outras palavras, a for\c ca peso não depende do caminho!

  Esse raciocínio acima é um exemplo de um tipo de for\c ca conhecida como ``For\c ca Conservativa'', aquelas para as quais o trabalho realizado entre dois pontos
  não depende do caminho e apenas das posi\c cões deles.  Outro conceito importante é o de caminho fechado, que consiste na classe de caminhos nos quais o ponto inicial
  e final coincidem (Por exemplo, um círculo fechado). A importância destes dois conceitos está em sua união, pois eles permitem deduzir que a for\c ca conservativa em
  um caminho fechado sempre realizará um trabalho nulo. Matematicamente, 
    \[
      \hypertarget{neccessary_conservative}{W_{P}^{C} = \oint_{C}\vec{F}\cdot \vec{dr} = 0}
    \]
    Exemplos de for\c cas conservativas incluem a for\c ca gravitacional, a for\c ca elástica, uma for\c ca constante, entre outras.
    Em uma dimensão, F = F(x) é conservativa.
   \begin{example}
     Exemplo 7.1 do Tipler - Integral em um caminho fechado.

     a) Calcule o trabalho realizado por \(\vec{F} = Ax \hat{i}\). 

     Existem quatro possíveis caminhos. O primeiro, definiremos por \(\vec{r}_{1} = x\hat{i} + 0\hat{j}, dr_{1} = dx\hat{i}\). Para o segundo, consideraremos
     \(\vec{r}_{2} = x_{m}\hat{i} + y\hat{j}, dr_{2} = dy\hat{j}.\) No terceiro, faremos \(\vec{r}_{3} = x\hat{i} + y_{m}\hat{j}\). Finalmente, para o quarto,
     tomaremos \(\vec{r}_{4} = 0\hat{i} + y\hat{j}, dr_{4} = dy\hat{j}.\) Assim, 
    \begin{align*}
      W_{C} &= \oint_{C}\vec{F}\cdot \vec{dr} = \int_{c_{1}}^{}Ax\hat{i}\cdot dx\hat{i} + \int_{c_{2}}^{}Ax\hat{i}\cdot dy\hat{j} + \int_{c_{3}}^{}Ax\hat{i}\cdot dx\hat{i}+ \int_{c_{4}}^{}Ax\hat{i}\cdot dy\hat{j}\\
            &= A \frac{x^{2}}{2}\biggl|_{0}^{x_{m}}\biggr. + \frac{Ax^{2}}{2}\biggl|_{x_{m}}^{0}\biggr.\\
            &= \frac{Ax_{m}^{2}}{2} - 0 + 0 - \frac{Ax_{m}^{2}}{2} = 0.
    \end{align*}

    v) Fa\c ca o mesmo para \(\vec{F} = Bxy\hat{i}\).

      Com o mesmo raciocínio de antes, 
        \[
          W_{C} = \int_{c_{1}}^{}Bxy\hat{i}\cdot dx\hat{i} + \int_{c_{3}}^{}Bxy\hat{i}\cdot dx\hat{i} = \frac{Byx^{2}}{2}\biggl|_{c_{1}}^{}\biggr. + \frac{byx^{2}}{2}\biggl|_{c_{3}}^{}\biggr. .
        \]
        Assim, concluimos que, como \(c_{1} \coloneqq (0, 0)->(x_{m}, 0)\) e \(c_{3} = (x_{m}, y_{m})->(0, y_{m})\),
          \[
            W_{C} = \frac{-By_{m}x_{m}^{2}}{2}\neq0,
          \]
          ou seja, a for\c ca nõa é conesrvativa!!
   \end{example}
   \begin{example}
      Seja \(\vec{F} = (\frac{F_{0}}{r}(y\hat{i}-x\hat{j}\). a) Qual é o valor de \(|\vec{F}|\)? b) Mostre que a dire\c cão de F é perpendicular à \(\vec{r}\).
      c) Quanto vale \(W_{F}\) ao longo de um círculo?

      A priori, note que 
     \begin{align*}
       |\vec{F}| &= [F_{x}^{2} + F_{y}^{2}]^{\frac{1}{2}} = \biggl[\biggl(\frac{F_{0}}{x}\biggr)^{2}y^{2} + \biggl(\frac{-F_{0}}{r}x\biggr)^{2}\biggr]^{2}\\
                 &= \frac{F_{0}}{r}(y^{2}+x^{2})^{\frac{1}{2}} = F_{0}.
     \end{align*}
     Em outras palavras, F tem módulo constante.

     Por outro lado, considere o produto escalar entre \(\vec{F}\) e \(\vec{r}\): 
       \[
         \vec{F}\cdot \vec{r} = |\vec{F}||\vec{r}|\cos{(\theta )}.
       \]
      Temos que mostrar que \(\theta = 90^{\circ}\), pois ser perpendicular significa que o produto interno anula-se. Segue que 
        \[
          \vec{F}\cdot \vec{r} = F_{x}x + F_{y}y = (\frac{F_{0}}{r})yx - \frac{(F_{0}}{r})xy = 0.
        \]
      Portanto, \(\cos{(\theta )} = 0 \Rightarrow \theta =90^{\circ}\).

      Por fim, colocamos \(\vec{r}(t) = R\cos{(\theta (t))}\hat{i} + R\sin{(\theta (t))}\hat{j}, 0\leq \theta \leq 2\pi \) para descrever um caminho circular 
      no plano xy num caminho que depende de apenas uma variável \(\theta (t)\). Com isso,
      \[
        \vec{dr} = R(-\sin{(\theta )})d\theta \hat{i} + R\cos{(\theta )}d\theta \hat{j}.
      \]
      Portanto, 
     \begin{align*}
       W_{C} = \oint_{C}\vec{F}\cdot \vec{dr} &= \oint_{C}[(\frac{f_{0}}{r})(y\hat{i}-x\hat{j})]\cdot [-R\sin{(\theta )}d\theta \hat{i} + R\cos{(\theta )}d\theta \hat{j}] \\
                                              &= \int_{0}^{2\pi }\biggl(\frac{F_{0}}{R}\biggr)y(-R\sin{(\theta )d\theta } + \biggl(\frac{F_{0}}{R}\biggr)(-x)R\cos{(\theta )}d\theta \\
                                              &= \int_{0}^{2\pi }-\biggl(\frac{F_{0}}{R}\biggr)R^{2}\sin^{2}{(\theta )}d\theta  + \biggl(\frac{F_{0}}{R}\biggr)(-R^{2}\cos^{2}{(\theta )}d\theta \\
                                              &= -F_{0}R \int_{}^{}(\sin^{2}{(\theta )} + \cos^{2}{(\theta )})d\theta  = -F_{0}R\theta \biggl|_{0}^{2\pi }\biggr. = -F_{0}2\pi R.
     \end{align*}
   \end{example}

\section{Aula 19 - 18/05/2023}
\subsection{O que esperar?}
\begin{itemize}
  \item Outras for\c cas na perspectiva de trabalho;
  \item Energia Potencial;
\end{itemize}
\subsection{Trabalho e Energia Cinética - Parte II}
  Vimos o trabalho de algumas for\c cas explicitamente, como a for\c ca gravitacional,
  e estudamos também o tipo de for\c ca conhecida como conservativa. Analisaremos
  a situa\c cão de algumas outras a seguir.

  Para come\c car, qual é o trabalho da for\c ca atrito? Lembrando que existem dois tipos,
é notável que, como a for\c ca de atrito estático está relacionada com a ausência de movimento,
i.e., um objeto estático, o trabalho dela será nulo. Com rela\c cão ao
atrito cinético, no entanto, a questão deve ser elaborada.

  Suponha que um bloco realiza duas trajetórias - primeiro de \(x_1\) até \(x_2\)
e em segundo de \(x_2\) até \(x_1\) - tal que a trajetória total é a ``soma das duas trajetórias.''
Assim, 
  \[
    W_{F_{at}}^{TT} = W_{F_{at}}^{T_1} + W_{F_{at}}^{T_2} = \int_{x_{1}}^{x_{2}}(-F_{at})\cdot dx + \int_{x_{2}}^{x_{1}}F_{at}\cdot dx
  \]
Como \(|F_{at}^{c}| = \mu_{c}N = \mu_{c}mg\) , concluímos que 
\begin{align*}
    W_{F_{at}}^{TT} &= -\int_{x_{1}}^{x_{2}}\mu_{c}mgdx + \int_{x_{2}}^{x_{1}}\mu_{c}mgdx\\
                    &= -\mu_{c}mg(x_{2}-x_{1}) + \mu_{c}mg(x_{1}-x_{2}) = -2\mu_{c}mg\Delta x\neq 0.
\end{align*}
Portanto, a for\c ca de atrito não é conservativa, ou seja, ela é dissipativa.

  Voltando às for\c cas conservativas, podemos representá-la como a varia\c cão de uma quantidade.
  Essa quantidade será chamada de Energia Potencial, e escrevemos 
    \[
      W_{1\rightarrow2}^{C} = \int_{C}^{}\vec{F}\cdot \vec{dr} = -\Delta U = -(U(\vec{r}_{2}) - U(\vec{r}_{1})).
    \]
    Formalmente, iremos definir a energia potencial como 
      \[
      \hypertarget{potential_energy}{\boxed{U(\vec{r}) = -\int_{\vec{r}_{0}}^{\vec{r}}\vec{F}\cdot \vec{dr}}}
      \]
      sendo \(\vec{r}_{0}\) o ponto no espa\c co para o qual a energia potencial é nula \((U(\vec{r}_{0} = 0))\).
     \begin{example}
       Vamos considerar um objeto caindo livremente de uma altura h. Encontraremos sua energia potencial.
       Nesta situa\c cão, defina \(\vec{r}_{0}\) = (0, 0), tal que \(U(0, 0) = 0.\) Assim,
       \[
        U(P) = -\int_{0}^{h}\vec{P}\cdot dy\hat{j} = \int_{0}^{h}(-mg)dy = mgh.
      \]
      Vejamos o que ocorre se mudarmos o ponto inicial para metade do caminho - \(\vec{r}_{0} = \frac{h}{2}\). Neste caso,  
        \[
          U(P) = -\int_{\frac{h}{2}}^{h}(-mg)dy = mgy \biggl|_{\frac{h}{2}}^{h}\biggr. = mg \frac{h}{2}.
        \]
      No entanto, apesar do resultado diferente, ao calcularmos a varia\c cão da energia potencial,
      sempre chegaremos ao mesmo resultado - \(\Delta U = mgh.\)
     \end{example}
    \begin{example}
      Agora, suponha um sistema massa-mola, com ponto \(x_{0} = 0, U(x_{0}) = 0\)
      e F = 0. Então, como a for\c ca da mola é -kx, a energia potencial será 
        \[
          U(x) = -\int_{0}^{x}F_{mola}dx'= - \int_{0}^{x}(-kx')dx' = \frac{kx^{2}}{2}
        \]
    \end{example} 
      
      Observe que, pelo \hyperlink{work_kin_3d}{Teorema da Energia Cinética}, 
        \[
          W_{1\rightarrow2}^{C} = -\Delta U = \Delta E_{cin}.
        \]
      Manipulando esta equa\c cão, obtemos 
        \[
          \Delta E_{cin} + \Delta U = 0 \Rightarrow (E_{cin}^{x_2} + U^{x_2}) - (E_{cin}^{x_1} + U^{x_1}) = 0
        \]
        Definimos os termos da forma \((E_{cin}^{x_2} + U^{x_2})\) como a Energia Mecânica do sistema. Mais precisamente, 
        \[
          \hypertarget{mechanical_energy}{E_{T} = E_{cin} + U_{R}}
        \] 
        Em particular, observe que \(\Delta E_{T} = 0\), ou seja, a energia
        mecânica é conservada no sistema. A quantidade \(U_{R}\) é análoga à 
        forma que escrevemos a for\c ca resultante, no sentido de que ela é a soma
        de todas as energias potenciais. Vamos ver agora como aplicar a conserva\c cão de energia
       \begin{example}
         Considere uma bolinha que come\c ca com \(v_{0} = 0m/s\) e atinge o chão com velocidade
         \(v_{2}\). Buscamos o valor dessa velocidade 2. Temos 
           \[
             \Delta E = E_{2} - E_{1} = 0.
           \]
          Sabemos que 
            \[
              E_{1} = E_{cin} + U_{1} = 0 + mgh = mgh
            \]
          e 
            \[
              E_{2} = E_{cin} + U_{2} = \frac{1}{2}mv_{2}^{2}.
            \]
          Logo, 
            \[
              E_{1} = \frac{1}{2}mv_{2}^{2} - mgh = 0 \Rightarrow v_{2}^{2} = 2gh \Rightarrow v_{2} = \sqrt[]{2gh}.
            \]
       \end{example}
\end{document}
