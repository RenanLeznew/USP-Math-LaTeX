\documentclass{article}
\usepackage{amsmath}
\usepackage{amsthm}
\usepackage{amssymb}
\usepackage{pgfplots}
\usepackage{amsfonts}
\usepackage[margin=2.5cm]{geometry}
\usepackage{graphicx}
\usepackage[export]{adjustbox}
\usepackage{fancyhdr}
\usepackage[portuguese]{babel}
\usepackage{hyperref}
\usepackage{lastpage}
\usepackage{mathtools}

\pagestyle{fancy}
\fancyhf{}

\pgfplotsset{compat = 1.18}

\hypersetup{
    colorlinks,
    citecolor=black,
    filecolor=black,
    linkcolor=black,
    urlcolor=black
}
\newtheorem*{def*}{\underline{Defini\c c\~ao}}
\newtheorem*{theorem*}{\underline{Teorema}}
\newtheorem*{lemma*}{\underline{Lema}}
\newtheorem*{prop*}{\underline{Proposi\c c\~ao}}
\newtheorem{example}{\underline{Exemplo}}
\newtheorem*{proof*}{\underline{Prova}}
\renewcommand\qedsymbol{$\blacksquare$}
\newcommand{\Lin}[1]{Lin_{\mathbb{K}}({#1})}

\rfoot{P\'agina \thepage \hspace{1pt} de \pageref{LastPage}}

\begin{document}
\begin{figure}[ht]
\minipage{0.76\textwidth}
  \includegraphics[width=4cm]{../icmc.png}
  \hspace{7cm}
  \includegraphics[height=4.9cm,width=4cm]{../brasao_usp_cor.jpg}
\endminipage  
\end{figure}

\begin{center}
\vspace{1cm}
\LARGE
UNIVERSIDADE DE S\~AO PAULO

\vspace{1.3cm}
\LARGE
INSTITUTO DE CI\^ENCIAS MATEM\'ATICAS E COMPUTACIONAIS - ICMC

\vspace{1.7cm}
\Large
\textbf{Notas de F\'isica}

\vspace{1.3cm}
\large
\textbf{Renan Wenzel - 11169472}

\vspace{1.3cm}
\large
\textbf{Patr\'icia Christina Marques Castilho - patricia.castilho@ifsc.usp.br}

\vspace{1.3cm}
\today
\end{center}

\newpage

\tableofcontents

\newpage

\section{Aula 00 - 23/03/2023}
  (Revis\~ao Unidades de Medidas)

\section{Aula 01 - 27/03/2023}
\begin{itemize}
  \item Revisar propriedades de derivadas;
  \item Aplicar derivadas em movimento 1D. 
\end{itemize}
\subsection{Movimentos 1D}
  Dada uma part\'icula com posi\c c\~ao descrita por $x = x(t)$, em que t \'e a vari\'avel de tempo, denotamos seu deslocamento
por $\Delta x = x_{2} - x_{1} = x(t_{2}) - x(t_{1}).$ Analogamente, o intervalo de tmepo \'e definido por $\Delta t = t_{2} - t_{1}$.
Com essas ferramentas, j\'a podemos definir a velocidade m\'edia de um objeto em uma dimens\~ao como $\vec{v} = \frac{\Delta x}{\Delta t}.$
Observe que, quanto menor o intervalo de tempo, mais moment\^aneo se torna essa defini\c c\~ao, de modo que a velocidade instant\^anea
pode ser encontrada como 
  $$
    \lim_{\Delta t\to0}\frac{x(t + \Delta t) - x(t)}{\Delta t} = \vec{v}(t).
  $$
Regras de derivadas:
\begin{align*}
  &f(t) = c \Rightarrow \frac{df}{dt} = 0 \text{ Derivada de uma constante \'e sempre nula;}\\
  &f(t) = x^{n} \Rightarrow \frac{df}{dt} = nx^{n-1} \text{ Regra do tombo;}\\
  &f(t) = A\sin{(t)} \Rightarrow \frac{df}{dt} = A\cos{(t)};\\
  &f(t) = B\cos{(t)} \Rightarrow \frac{df}{dt} = -B\sin{(t)};\\
  &f(t) = C e^{t} \Rightarrow \frac{df}{dt} = C e^{t}.
\end{align*}
\begin{example}
 \begin{align*}
   &i)f(t) = 3t^{4} + t^{2} \Rightarrow \frac{df}{dt} = 12t^{3} + 2t\\
   &ii) f(t) = 5\sin{(t)} + 3(t^{2}+1) = 5\sin{(t)} + 3t^{2} + 3 \Rightarrow \frac{df}{dt} = 5\cos{(t)} + 6t 
 \end{align*} 
\end{example}

  A partir deste ponto, tome t como tempo, x(t) como posi\c c\~ao e v(t) a velocidade instant\^anea.

\begin{tikzpicture}
\begin{axis}[
    axis lines = left,
    xlabel = \(t\),
    ylabel = {\(x(t)\)},
]
%Below the red parabola is defined
\addplot [
    domain=0:20, 
    samples=100, 
    color=red,
]
{3*x - 20};
\addlegendentry{\(x(t) = 3t - 20\)}
%Here the blue parabola is defined
\addplot [
    domain=20:40, 
    samples=100, 
    color=blue,
    ]
    {-3*x + 100};
\addlegendentry{\(x(t) = -3t + 100\)}
\end{axis}
\end{tikzpicture}

  Esse movimento em que a velocidade \'e descrita por uma linha reta \'e conhecido como movimento retil\'ineo uniforme, pois 
a velocidade $v(t)$ muda de forma linear, i.e., $\frac{dx}{dt} = c$, em que c \'e uma constante.

  Por outro lado, h\'a outro tipo de movimento, o movimento retil\'ineo uniformemente variado, em que a velocidade n\~ao \'e constante.
A a\c c\~ao respons\'avel por mudar a velocidade \'e conhecida como acelera\c c\~ao, e os gr\'aficos tendem a assumir o seguinte formato

\begin{tikzpicture}
\begin{axis}[
    axis lines = left,
    xlabel = \(t\),
    ylabel = {\(x(t)\)},
]
%Below the red parabola is defined
\addplot [
    domain=0:20, 
    samples=100, 
    color=red,
]
{-x^2 + 10*x};
\addlegendentry{\(x(t) = -x^2+10x\)}
%Here the blue parabola is defined
\end{axis}
\end{tikzpicture}

Ou, caso a velocidade cres\c ca com o tempo,

\begin{tikzpicture}
\begin{axis}[
    axis lines = left,
    xlabel = \(t\),
    ylabel = {\(x(t)\)},
]
%Below the red parabola is defined
\addplot [
    domain=0:20, 
    samples=100, 
    color=red,
]
{x^2};
\addlegendentry{\(x(t) = x^2\)}
%Here the blue parabola is defined
\end{axis}
\end{tikzpicture}

H\'a ainda o caso em que a velocidade cresce por um tempo e diminui depois, com gr\'aficos como o que segue
 
\begin{tikzpicture}
\begin{axis}[
    axis lines = left,
    xlabel = \(t\),
    ylabel = {\(x(t)\)},
]
%Below the red parabola is defined
\addplot [
    domain=0:20, 
    samples=100, 
    color=red,
]
{x^2 - 20*x};
\addlegendentry{\(x(t) = x^2 - 20x\)}
%Here the blue parabola is defined
\end{axis}
\end{tikzpicture}

Nestes casos, para calcular o deslocamento da particula, precisamos somar muito mais intervalos de tempo. Para isso, observe que
cada instante, a posi\c c\~ao da part\'icula pode ser encontrada multiplicando-se o intervalo de tempo pela velocidade instan\^anea, i.e., 
 $\Delta x_{i}' = v_{i}'\Delta t_{i}'$. Quebrando os intervalos desta forma, o deslocamento de um ponto a outro \'e denotado por
 $$
   \Delta x_{1, 2} = x(t_{2}) - x(t_{1})\approx \sum\limits_{k=1}^{N}\Delta x_{i}' = \sum\limits_{k=1}^{N}v_{i}'\Delta t_{i}'
 $$
 Assim como para a velocidade instant\^anea, quanto menor tomarmos o intervalo de tempo, mais preciso \'e o valor encontrado para $\Delta x_{1, 2}$,
o que indica uma boa oportunidade para o uso do limite novamente. Com isso, definimos
 $$
  x(t_{2}) - x(t_{1}) = \lim_{\Delta t'\to0} \sum\limits_{i=1}^{N}v(t_{i}')\Delta t_{i}' = \int_{t_{1}}^{t_{2}}v(t)dt
 $$
 Este \'ultimo s\'imbolo, chamado integral, descreve a \'area ``embaixo'' da curva da fun\c c\~ao f(t) dentro do intervalo $[t_{1}, t_{2}].$
Supondo que c e k s\~ao constantes quaisquer, seguem abaixo algumas das regras de integra\c c\~ao: 
\begin{align*}
  &i)f(t) = ct^{n} \Rightarrow \frac{df}{dt} = nct^{n-1} \Rightarrow F(t) = \frac{ct^{n+1}}{n+1}\text{ (Primitiva de f)}\\
  &ii) \int_{t_{1}}^{t_{2}}f(t)dt = F(t_{2}) - F(t_{1}) = \frac{c}{n+1}t_{2}^{n+1} - \frac{c}{n+1}t_{1}^{n+1}\text{ (Integral definida de f)}\\
  &iii) \int_{}^{}f(t)dt = \frac{c}{n+1}t^{n+1} + k\text{ (Integral indefinida de f)}
\end{align*}
  Para conferir se a integral est\'a correta, \'e preciso derivar a fun\c c\~ao F e, se obter como resultado a fun\c c\~ao f, significa que est\'a correto.
Com este conhecimento em mente, segue que 
  $$
  \boxed{ x(t) = \int_{}^{}v_{0}dt = v_{0}t + x_{0}}
  $$
  Algumas outras regras importantes:
 \begin{align*}
   &iv) \frac{d\sin{(t)}}{dt} = \cos{(t)} \Rightarrow \int_{}^{}\cos{(t)}dt = \sin{(t)} + c\\
   &v) \frac{d\cos{(t)}}{dt} = -\sin{(t)} \Rightarrow \int_{}^{}\sin{(t)}dt = \cos{(t)} + c\\
   &vi) \frac{d e^{t}}{dt} = e^{t} \Rightarrow \int_{}^{} e^{t}dt = e^{t} + c
 \end{align*}
  Ou seja, em certo sentido, a integral e a derivada s\~ao dois lados da mesma moeda, assim como mulitplica\c c\~ao e divis\~ao ou adi\c c\~ao e subtra\c c\~ao.
  \newpage

\section{Aula 02 - 29/03/2023}
\subsection{Motiva\c c\~oes}
\begin{itemize}
  \item Estudar a acelera\c c\~ao;
  \item Entender o Movimento Retil\'ineo Uniformemente Variado.
\end{itemize}
\subsection{Acelera\c c\~ao}
  Definimos previamente a velocidade m\'edia como sendo a varia\c c\~ao de tempo dividindo o deslocamento, sendo, portanto,
uma quantidade representando a taxa de varia\c c\~ao da posi\c c\~ao em um intervalo de tempo. De forma an\'aloga,
definimos a acelera\c c\~ao como a taxa de varia\c c\~ao da velocidade em um intervalo de tempo, ou seja, 
  $$
    \vec{a_{m}} = \frac{\Delta \vec{v}}{\Delta t}.
  $$
  Ainda mais, se ela for positiva, a velocidade aumenta. Caso contr\'ario, ela diminui. Ainda repetindo o processo feito
para o caso da velocidade, podemos encontrar uma acelera\c c\~ao instan\^anea como 
    $$
      a(t) = \lim_{\Delta t\to0}\biggl[\frac{v(t + \Delta t) - v(t)}{\Delta t}\biggr] = \frac{dv(t)}{dt}
    $$
  Observe tamb\'em que 
    $$
      a(t) = \frac{d}{dt}\biggl(\frac{dx(t)}{dt}\biggr) = \frac{d^{2}x}{dt^{2}}.
    $$
  Utilizando a an\'alise dimensional, \'e poss\'ivel encontrar a dimens\~ao da acelera\c c\~ao como $[a]=\frac{[v]}{[t]} = \frac{\frac{[L]}{[t]}}{[t]} = \frac{[L]}{[t]^{2}}.$ Assim,
se o sistema de medida for o Sistema Internacional, $[a] = \frac{m}{s^{2}}$.

    \begin{tikzpicture}
    \begin{axis}[
        axis lines = left,
        xlabel = \(t\),
        ylabel = {\(x(t)\)},
    ]
    \addplot [
        domain=0:10,
        samples=100, 
        color=red,
    ]
    {-x^2 + 10*x};
    \addlegendentry{\(x(t) = -x^{2}+10x\)}
    \addplot[
      domain=0:10,
      samples=100,
      color=blue,
    ]
    {-2*x + 10};
    \addlegendentry{\(v(t) = -2x + 10\)},
      \addplot[
        domain=0:10,
        samples=100,
        color=green,
      ]
      {2};
    \addlegendentry{\(a(t) = 2\)},
    \end{axis}
  \end{tikzpicture}

\subsection{Movimento Retil\'ineo Uniformemente Variado.}
  Sabendo que $a =\displaystyle \frac{d^{2}x(t)}{dt^{2}}$, podemos fazer o caminho oposto para encontrar uma f\'ormula para
  a posi\c c\~ao sabendo a acelera\c c\~ao. De fato, dado um intervalo de tempo $[t_{0}, t],$
  $$
    v(t) = \int_{t_{0}}^{t}a(t)dt = at \biggl|_{t_{0}}^{t}\biggr. = at - at_{0}
  $$
  Sabemos, tamb\'em, que $v(t) - v(t_{0}) = \Delta v$, tal que 
    $$
      v(t) = v(t_{0}) + a(t-t_{0}) = v_{0} + a(t-t_{0})
    $$
  Al\'em disso, vimos que 
    $$
      \Delta x = x(t) - x(t_{0}) = \int_{t_{0}}^{t}v(t) dt.
    $$
  Juntando tudo, segue a f\'ormula dita: 
  \begin{align*}
    x(t) - x(t_{0}) &= \overbrace{\int_{t_{0}}^{t}[v_{0} + a(t-t_{0})]dt}^{\int_{}^{}f(t) + g(t)dt = \int_{}^{}f(t)dt + \int_{}^{}g(t)dt}= \overbrace{\int_{t_{0}}^{t}v_{0}dt}^{\int_{}^{}c dt = ct} + \overbrace{\int_{t_{0}}^{t}atdt}^{\int_{}^{}t^{n}dt = \frac{t^{n+1}}{n+1}} - \int_{t_{0}}^{t}at_{0}dt\\
                    & \Rightarrow x(t) - x(t_{0})  =  v_{0}t \biggl|_{t_{0}}^{t}\biggr. + a\frac{t^{2}}{2} \biggl|_{t_{0}}^{t}\biggr. - at_{0}t \biggl|_{t_{0}}^{t}\biggr. \\
                    &= v_{0}(t-t_{0}) + a \frac{(t^{2} - t_{0}^{2})}{2} - at_{0}(t-t_{0}) \\
                    &= v_{0}(t-t_{0}) + a \frac{t^{2}-t_{0}^{2}}{2} - at_{0}t + at_{0}^{2} = v_{0}t - v_{0}t_{0} + \frac{a}{2}(t^{2}-2t_{0}t + 2t_{0}^{2})\\
                    &= v_{0}(t-t_{0}) + \frac{a}{2}(t-t_{0})^{2}\\
                    & \Rightarrow x(t) = x(t_{0}) + v_{0}(t-t_{0}) + \frac{1}{2}a(t-t_{0})^{2}.
  \end{align*}
  Com isso, no caso em que $t_{0} = 0$, segue que 
   $$
    \boxed{x(t) = x_{0} + v_{0}t + \frac{at^{2}}{2}}     
   $$
  Uma coisa not\'avel \'e que todas essas f\'ormulas est\~ao dependentes de tempo. No entanto, ser\'a que \'e poss\'ivel
se livrar dessa vari\'avel e relacionar, por exemplo, velocidade e posi\c c\~ao? A resposta \'e sim! E vamos mostrar como a seguir,
na equa\c c\~ao conhecida como Equa\c c\~ao de Torricelli. Com efeito,
 \begin{align*}
   &(I)\quad (t-t_{0}) = \frac{v(t)-v_{0}}{a} = \frac{v-v_{0}}{a}\\
   &(II)\quad x(t) = x_{0} + v_{0}(t-t_{0}) + \frac{1}{2}a(t-t_{0})^{2}\\
   &(I\text{ com }II)\quad x = x_{0} + v_{0}\frac{v-v_{0}}{a} + \frac{1}{2}a \frac{v-v_{0}}{a} \\
   & \Rightarrow x = x_{0} + \frac{1}{a}\biggl\{v_{0}v - v_{0}^{2} + \frac{1}{2}(v^{2}-2vv_{0}+v_{0}^{2})\biggr\}\\
   & = x_{0} + \frac{1}{a}\biggl\{-v_{0}^{2} + \frac{v^{2}}{2} + \frac{v_{0}^{2}}{2}\biggr\}\\
   & \Rightarrow x - x_{0} = \frac{1}{2a}\biggl[v^{2}-v_{0}^{2}\biggr] \Longleftrightarrow [v^{2} - v_{0}^{2}] = 2a(x-x_{0}).
 \end{align*}
 Portanto, chegamos na Equa\c c\~ao de Torricelli
  $$
  \boxed{v^{2} = v_{0}^{2} + 2a(x-x_{0})}
  $$
  Para refor\c car o que foi visto at\'e agora, vejamos um exemplo.
 \begin{example}
   Suponha que um carro freia uniformemente, passando de 60km/h para 30km/h em 5 segundos. Qual \'e a dist\^ancia
que o carro percorrer\'a at\'e parar? Em quanto tempo?

  \textbf{Solu\c c\~ao}: Sabemos que $x(t) = x_{0} + v_{0}(t-t_{0}) + \frac{1}{2}a(t-t_{0})^{2}, v(t) = v_{0} + a(t-t_{0}),
\text{ e }v^{2} = v_{0}^{2} + 2a(x-x_{0}).$ Al\'em disso, como \'e at\'e o carro parar, a velocidade final \'e 0km/h, a varia\c c\~ao
de tempo at\'e o momento em que a velocidade atinge 30km/h (=8.333m/s) \'e dada como $\Delta t = 5 - 0 = 5$s , sendo a velocidade inicial 60km/h (=16,666m/s). 
Pela equa\c c\~ao dois, 
  $$
    a = \frac{v(t_{1}) - v_{0}}{t_{1}-t_{0}} = \frac{8.33 - 16.66}{5} = -1.66\frac{m}{s^{2}}
  $$
  Agora, para obter a dist\^ancia, sendo $v_{2} = 0km/h$ o valor da acelera\c c\~ao no tempo em que o carro para (o segundo percurso),
utilizamos Torricelli para obter o deslocamento no peda\c co final do percurso
  $$
    v_{2}^{2} = v_{1}^{2} + 2a(x_{2}-x_{1}) \Rightarrow 0 = 8.33^{2} + 2(-1.66)\Delta x_{2}
  $$
  Assim, isolando o $\Delta x_{2},$  
    $$
    \Delta x_{2} = \frac{8.33^{2}}{3.32} = \text{ Professora vai passar na pr\'oxima aula.}
    $$
  Ademais, para encontrar todo o caminho que o carro andou, temos 
    $$
      0 = v_{0}^{2} + 2a(x_{2} - x_{0}) = 16.66^{2} + 2(-1.66)\Delta x \Rightarrow \Delta x = \frac{16.66^{2}}{3.32}
    $$
  Finalmente, o instante de tempo pode ser encontrado fazendo 
    $$
    v_{2}(t) = v_{1} + a(t_{2}-t_{0}) \Rightarrow 0 = 8.33 - 1.66\Delta t_{2} \Rightarrow \Delta t_{2} = 5s.\text{ \qedsymbol}
    $$
 \end{example}

\section{ Aula 03 - 30/03/2023}
\subsection{Motiva\c c\~oes}
 \begin{itemize}
   \item Resolu\c c\~ao de Exerc\'icios.
 \end{itemize}
 \subsection{Exerc\'icio 29 - Tipler}
 ``Considere a trajet\'oria de dois carros, o Carro A e o Carro B. (a) Existe algum instante para o qual os carros est\~ao lado-a-lado? (b) Eles viajam sempre no mesmo sentido? (c) Eles viajam com a mesma velocidade em algum instante t? (d) Para que t os carros est\~ao mais distantes entre si? (e) Esboce os gr\'aficos de $v\times t$'
 \begin{center}
   \begin{tikzpicture}
   \begin{axis}[
       axis lines = left,
       xlabel = \(t\),
       ylabel = {\(x(t)\)},
   ]
   \addplot [
       domain=0:10, 
       samples=100, 
       color=red,
   ]
   {7*x/2 + 9/3};
 \addlegendentry{\(\text{Carro B}\)}
   %Here the blue parabola is defined
     \addplot[
       domain=0:10,
       samples=100,
       color=blue,
     ]
     {-x^2 + 13*x};
   \addlegendentry{\(\text{Carro A}\)}
     %Here the blue parabola is defined
   \end{axis}
 \end{tikzpicture}
\end{center}
  Os carros se encontram lado-a-lado quando os gr\'aficos se cruzam, ou seja, em t = 1s e t = 9s (Tipler mais acurado que meu gr\'afico.). \'E not\'avel
quer eles n\~ao est\~ao sempre no mesmo sentido, visto que, a partir de 6s, o gr\'afico do carro B passa a mudar o sentido. Em aproximadamente 5s,
ambos est\~ao com a reta tangente iguais, ou seja, est\~ao com a mesma velocidade, e dist\^ancia entre eles est\'a maior exatamente no ponto em que as 
velocidades est\~ao iguais. Finalmente, seguem os gr\'aficos:
\begin{center}
    \begin{tikzpicture}
    \begin{axis}[
        axis lines = left,
        xlabel = \(t\),
        ylabel = {\(v(t)\)},
    ]
    \addplot [
        domain=0:6, 
        samples=100, 
        color=red,
    ]
    {-3*x + 18};
  \addlegendentry{\(\text{Carro A}\)}
    %Here the blue parabola is defined
    \addplot[
      domain=0:6,
      samples=100,
      color=red,
    ]
    {5};
  \addlegendentry{\(\text{Carro B}\)}
    %Here the blue parabola is defined
    \end{axis}
  \end{tikzpicture}
\end{center}

\subsection{Exerc\'icio 44 - Tipler}
``Um carro viaja em linha reta com $\vec{v} = 80\text{km/h} $ durante $\Delta t_{1}=2.5$h. Depois, $\vec{v_{2}} = 40$km/h, $\Delta t_{2} = 1.5$h. Qual
\'e o deslocamento total? E qual \'e a velocidade $\vec{v}$ total?''
\begin{align*}
  &(a)\quad \Delta x = \Delta x_{1} + \Delta x_{2} = \vec{v_{1}}\Delta t_{1} + \vec{v_{2}}\Delta t_{2}  \Rightarrow \Delta x = 260\text{km.}\\
  &(b)\quad \vec{v} = \frac{\Delta x}{\Delta t} = \frac{260}{4} = 65\text{km/h}.
\end{align*}

\subsection{Exerc\'icio 58 - Tipler}
  ``Um carro acelera de 48.3km/h para 80.5km/h em 3.70s. Qual a acelera\c c\~ao m\'edia?''

Primeiramente, precisamos converter as unidades para medidas iguais. Com isso, note que $\vec{v_1} = 48.3km/h = 13.52m/s, \vec{v_{2}} = 80.5km/h = 22.54m/s$. Assim,
chegamos em 
  $$
  \vec{a} = \frac{\Delta v}{\Delta t} = \frac{v_{2} - v_{1}}{\Delta t}\approx 2.4\text{m/s}.
  $$

\subsection{Exerc\'icio 67 - Tipler}
``Um corpo est\'a em uma posi\c c\~ao inicial $x_{1}$ com velocidade inicial $\vec{v_{1}}$. Passado um tempo, ele se encontra na posi\c c\~ao $x_{2}$ com velocidade $\vec{v_{2}}$. Qual \'e a acelera\c c\~ao deste corpo?''

Utilizaremos Torricelli. sabemos que 
 \begin{align*}
   &(1):\quad x_{1} = 6m, \vec{v_{1}} = 10m/s\\
   &(2):\quad x_{2} = 10m, \vec{v_{2}} = 15m/s.
 \end{align*}
 Deste modo, $v^{2} = v_{0}^{2} + 2a\Delta x \Rightarrow v_{2}^{2} = v_{1}^{2} + 2a(x_{2} - x_{1}) \Rightarrow a \approx 16m/s^{2}$

\subsection{Exerc\'icio 72 - Tipler}
  ``Um parafuso se desprende de um elevador subindo a $v_{0} = 6m/s$. O parafuso atinge o fundo do po\c co em 3s. (a) Qual era a altura do elevador? (b) Qual \'e a velocidade do parafuso no ch\~ao? Tome g = 9.8 $m/s^{2}$''

Sabemos que $t_{0} = 0s, y(t_{0}) = h, v(t_{0}) = v_{0}.$ Com isso, podemos descrever $y(t) = h + v_{0}t + \frac{1}{2}gt^{2}$. Vamos responder, agora, o item a, isto \'e, qual \'e o valor
da altura h? Segue que, em $t=3s, y(t) = 0$. Utilizando a f\'ormula, 
  $$
    h = -v_{0}t + \frac{1}{2}gt^{2} = -6 \cdot3 + \frac{1}{2}9.8 \cdot 3^{2} = 26.1m
  $$
  Com rela\c c\~ao ao item (b), vimos que $v(t) = v_{0} + at.$ Deste modo, 
    $$
      v(3s) = 6 - 9.8 \cdot 3 = -23.4m/s
    $$
  Indo um pouco al\'em do que foi pedido, analisemos o movimento do parafuso. \'E poss\'ivel concluir que o parafuso atingir\'a a altura 
m\'axima no instante em que $t^{*} = \frac{v_{0}}{g} = 0.6s,$ visto que este momento ocorre quando $v(t) = v_{0} - gt = 0$. Com isso,
conclui-se que a altura m\'axima \'e $y(t^{*}) = h + v_{0}t^{*} - \frac{1}{2}gt^{*^{2}} \approx 27.5m.$ No gr\'afico,
  \begin{center}
    \begin{tikzpicture}
    \begin{axis}[
        axis lines = left,
        xlabel = \(t\),
        ylabel = {\(y(t)\)},
    ]
    \addplot [
      domain = 0:4,
      color = black,
      ]
    {0};
    \addplot [
        domain=0:3, 
        samples=100, 
        color=red,
    ]
    {-9.8*x^2/2 + 6*x + 26.1};
    \addlegendentry{\(y(t)\)}
    %Here the blue parabola is defined
      \addplot[
        domain=0:3,
        samples=100,
        color=blue,
      ]
      {27.955};
    \addlegendentry{\(h_{\text{max}} = \frac{dy}{dt} = 0\)}
      %Here the blue parabola is defined
    \end{axis}
  \end{tikzpicture}
\end{center}

\subsection{Exemplo - Aula 06 Vanderlei}
  ``Suponha que h\'a um trem parado no instante t=0 com acelera\c c\~ao a. Passados 6s, um passageiro chega ao local e observa o trem na posi\c c\~ao $x_{trem_{1}}$.
  Este passageiro sai correndo com velocidade $v_{0}$ para tentar alcan\c car o trem. Qual \'e a velocidade m\'inima que o passageiro precisa atingir
  para alcan\c c\'a-lo?''
    
  Com rela\c c\~ao ao trem, suas condi\c c\~oes iniciais s\~ao $t_{0} = 0, x_{trem} = 0, v_{trem} = 0,$ tal que $x_{trem}(t) = \frac{1}{2}at^{2}$. 
Por outro lado, quanto ao passageiro, quando $t=6s, x_{p} = 0$, de modo que $x_{p}(t) = x_{p_0} + v_{0}t$. Como temos a informa\c c\~ao da posi\c c\~ao
do passageiro aos 6s, 
  $$
    x_{p}(6) = x_{p_{0}} + v_{0} \cdot6 = 0 \Rightarrow x_{p_{0}} = -6v_{0} \Rightarrow x_{p}(t) = v_{0}(t-6).
  $$
  No momento em que o passageiro alcan\c ca o trem, eles possuem posi\c c\~oes iguais, isto \'e, $x_{p}(t) = x_{trem}(t)$. Graficamente,
 \begin{center}
     \begin{tikzpicture}
     \begin{axis}[
         axis lines = left,
         xlabel = \(t\),
         ylabel = {\(x(t)\)},
     ]
     \addplot [
         domain=0:12, 
         samples=100, 
         color=red,
     ]
     {x^2};
   \addlegendentry{\(\text{Trem}\)}
       \addplot[
         domain=0:6,
         samples=100,
         color=blue,
       ]
       {0};
     \addlegendentry{\(\text{Passageiro}\)}
       \addplot[
         domain=6:12,
         samples=100,
         color=green,
       ]
       {25*x-150};
       \addlegendentry{\(\frac{dx_{trem}}{dt} = \frac{dx_{p}}{dt}\)}
     \end{axis}
   \end{tikzpicture}
 \end{center}
  Ou seja, buscamos $t^{*}$ tal que $x_{p}(t^{*}) = x_{trem}(t^{*}), v_{p}(t^{*}) = v_{trem}(t^{*})$. Com efeito,
 \begin{align*}
   v_{0}(t^{*} - 6) &= \frac{at^{*^{2}}}{2} \Rightarrow v_{0} = at^{*} \Rightarrow t^{*} = \frac{v_{0}}{a}\\
                    &v_{0} = \frac{a}{2}\frac{(\frac{v_{0}}{a})^{2}}{\frac{v_{0}}{a}-6} \Rightarrow \frac{v_{0}^{2}}{2a} = 6v_{0} \Rightarrow v_{0} = 12a.
 \end{align*}
 Outra forma de resolver \'e utilizando o fato de que quando $\frac{dv}{dt} = 0$, a fun\c c\~ao est\'a num m\'inimo. Ou seja, basta encontrar
o valor m\'inimo de $v_{0}$ que satisfa\c ca o que buscamos. Temos 
  $$
    v_{0}(t-6) = \frac{at^{2}}{2} \Rightarrow v_{0}(t) = \frac{at^{2}}{2}\frac{1}{(t-6)}.
  $$ 
  Agora, derivando essa equa\c c\~ao para $v_{0},$ 
    $$
      \frac{dv_{0}}{dt} = \frac{d}{dt}\biggl(\frac{at^{2}}{2}\frac{1}{t-6}\biggr) = \frac{d}{dt}(f(t)g(t)),
    $$
    em que $f(t) = \frac{at^{2}}{2}, g(t) = (t-6)^{-1}$. Fazemos isso porque h\'a uma regra para derivar o produto de fun\c c\~oes,
  a Regra do Produto 
    $$
      \boxed{\frac{df(t)g(t)}{dt}= g(t)\frac{df(t)}{dt} + f(t)\frac{dg(t)}{dt}}
    $$
    Derivando individualmente f e g, 
      $$
        \frac{df(t)}{dt} = at, \quad \frac{dg(t)}{dt} = -(t-6)^{-2} = -\frac{1}{(t-6)^{2}}.
      $$
    Agora, vamos juntar tudo para obter a derivada de $v_{0}$: 
    \begin{align*}
      \frac{dv_{0}}{dt} &= \frac{df(t)}{dt}g(t) + \frac{dg(t)}{dt}f(t) = \frac{at}{t-6} - \frac{1}{2(t-6)^{2}}at^{2}\\
                        &= at\biggl(\frac{1}{t-6} - \frac{t}{2(t-6)^{2}}\biggr) = 0 \\
                        & \Rightarrow \frac{1}{t-6} = \frac{t}{2(t-6)} \Rightarrow 1 = \frac{t}{2(t-6)}\\
                        & \Rightarrow 2(t-6) = t \Rightarrow 2t - t = 12 \Rightarrow t = 12s.
    \end{align*}
\end{document}
