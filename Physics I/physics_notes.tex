\documentclass{article}
\usepackage{amsmath}
\usepackage{amsthm}
\usepackage{amssymb}
\usepackage{pgfplots}
\usepackage{amsfonts}
\usepackage[margin=2.5cm]{geometry}
\usepackage{graphicx}
\usepackage[export]{adjustbox}
\usepackage{fancyhdr}
\usepackage[portuguese]{babel}
\usepackage{hyperref}
\usepackage{lastpage}
\usepackage{mathtools}

\pagestyle{fancy}
\fancyhf{}

\pgfplotsset{compat = 1.18}

\hypersetup{
    colorlinks,
    citecolor=black,
    filecolor=black,
    linkcolor=black,
    urlcolor=black
}
\newtheorem*{def*}{\underline{Defini\c c\~ao}}
\newtheorem*{theorem*}{\underline{Teorema}}
\newtheorem*{lemma*}{\underline{Lema}}
\newtheorem*{prop*}{\underline{Proposi\c c\~ao}}
\newtheorem{example}{\underline{Exemplo}}
\newtheorem*{proof*}{\underline{Prova}}
\renewcommand\qedsymbol{$\blacksquare$}
\newcommand{\Lin}[1]{Lin_{\mathbb{K}}({#1})}

\rfoot{P\'agina \thepage \hspace{1pt} de \pageref{LastPage}}

\begin{document}
\begin{figure}[ht]
\minipage{0.76\textwidth}
  \includegraphics[width=4cm]{../icmc.png}
  \hspace{7cm}
  \includegraphics[height=4.9cm,width=4cm]{../brasao_usp_cor.jpg}
\endminipage  
\end{figure}

\begin{center}
\vspace{1cm}
\LARGE
UNIVERSIDADE DE S\~AO PAULO

\vspace{1.3cm}
\LARGE
INSTITUTO DE CI\^ENCIAS MATEM\'ATICAS E COMPUTACIONAIS - ICMC

\vspace{1.7cm}
\Large
\textbf{Notas de F\'isica}

\vspace{1.3cm}
\large
\textbf{Renan Wenzel - 11169472}

\vspace{1.3cm}
\large
\textbf{Patr\'icia Christina Marques Castilho - patricia.castilho@ifsc.usp.br}

\vspace{1.3cm}
\today
\end{center}

\newpage

\tableofcontents

\newpage

\section{Aula 00 - 23/03/2023}
  (Revis\~ao Unidades de Medidas)

\section{Aula 01 - 27/03/2023}
\begin{itemize}
  \item Revisar propriedades de derivadas;
  \item Aplicar derivadas em movimento 1D. 
\end{itemize}
\subsection{Movimentos 1D}
  Dada uma part\'icula com posi\c c\~ao descrita por $x = x(t)$, em que t \'e a vari\'avel de tempo, denotamos seu deslocamento
por $\Delta x = x_{2} - x_{1} = x(t_{2}) - x(t_{1}).$ Analogamente, o intervalo de tmepo \'e definido por $\Delta t = t_{2} - t_{1}$.
Com essas ferramentas, j\'a podemos definir a velocidade m\'edia de um objeto em uma dimens\~ao como $\vec{v} = \frac{\Delta x}{\Delta t}.$
Observe que, quanto menor o intervalo de tempo, mais moment\^aneo se torna essa defini\c c\~ao, de modo que a velocidade instant\^anea
pode ser encontrada como 
  $$
    \lim_{\Delta t\to0}\frac{x(t + \Delta t) - x(t)}{\Delta t} = \vec{v}(t).
  $$
Regras de derivadas:
\begin{align*}
  &f(t) = c \Rightarrow \frac{df}{dt} = 0 \text{ Derivada de uma constante \'e sempre nula;}\\
  &f(t) = x^{n} \Rightarrow \frac{df}{dt} = nx^{n-1} \text{ Regra do tombo;}\\
  &f(t) = A\sin{(t)} \Rightarrow \frac{df}{dt} = A\cos{(t)};\\
  &f(t) = B\cos{(t)} \Rightarrow \frac{df}{dt} = -B\sin{(t)};\\
  &f(t) = C e^{t} \Rightarrow \frac{df}{dt} = C e^{t}.
\end{align*}
\begin{example}
 \begin{align*}
   &i)f(t) = 3t^{4} + t^{2} \Rightarrow \frac{df}{dt} = 12t^{3} + 2t\\
   &ii) f(t) = 5\sin{(t)} + 3(t^{2}+1) = 5\sin{(t)} + 3t^{2} + 3 \Rightarrow \frac{df}{dt} = 5\cos{(t)} + 6t 
 \end{align*} 
\end{example}

  A partir deste ponto, tome t como tempo, x(t) como posi\c c\~ao e v(t) a velocidade instant\^anea.

\begin{tikzpicture}
\begin{axis}[
    axis lines = left,
    xlabel = \(t\),
    ylabel = {\(x(t)\)},
]
%Below the red parabola is defined
\addplot [
    domain=0:20, 
    samples=100, 
    color=red,
]
{3*x - 20};
\addlegendentry{\(x(t) = 3t - 20\)}
%Here the blue parabola is defined
\addplot [
    domain=20:40, 
    samples=100, 
    color=blue,
    ]
    {-3*x + 100};
\addlegendentry{\(x(t) = -3t + 100\)}
\end{axis}
\end{tikzpicture}

  Esse movimento em que a velocidade \'e descrita por uma linha reta \'e conhecido como movimento retil\'ineo uniforme, pois 
a velocidade $v(t)$ muda de forma linear, i.e., $\frac{dx}{dt} = c$, em que c \'e uma constante.

  Por outro lado, h\'a outro tipo de movimento, o movimento retil\'ineo uniformemente variado, em que a velocidade n\~ao \'e constante.
A a\c c\~ao respons\'avel por mudar a velocidade \'e conhecida como acelera\c c\~ao, e os gr\'aficos tendem a assumir o seguinte formato

\begin{tikzpicture}
\begin{axis}[
    axis lines = left,
    xlabel = \(t\),
    ylabel = {\(x(t)\)},
]
%Below the red parabola is defined
\addplot [
    domain=0:20, 
    samples=100, 
    color=red,
]
{-x^2 + 10*x};
\addlegendentry{\(x(t) = -x^2+10x\)}
%Here the blue parabola is defined
\end{axis}
\end{tikzpicture}

Ou, caso a velocidade cres\c ca com o tempo,

\begin{tikzpicture}
\begin{axis}[
    axis lines = left,
    xlabel = \(t\),
    ylabel = {\(x(t)\)},
]
%Below the red parabola is defined
\addplot [
    domain=0:20, 
    samples=100, 
    color=red,
]
{x^2};
\addlegendentry{\(x(t) = x^2\)}
%Here the blue parabola is defined
\end{axis}
\end{tikzpicture}

H\'a ainda o caso em que a velocidade cresce por um tempo e diminui depois, com gr\'aficos como o que segue
 
\begin{tikzpicture}
\begin{axis}[
    axis lines = left,
    xlabel = \(t\),
    ylabel = {\(x(t)\)},
]
%Below the red parabola is defined
\addplot [
    domain=0:20, 
    samples=100, 
    color=red,
]
{x^2 - 20*x};
\addlegendentry{\(x(t) = x^2 - 20x\)}
%Here the blue parabola is defined
\end{axis}
\end{tikzpicture}

Nestes casos, para calcular o deslocamento da particula, precisamos somar muito mais intervalos de tempo. Para isso, observe que
cada instante, a posi\c c\~ao da part\'icula pode ser encontrada multiplicando-se o intervalo de tempo pela velocidade instan\^anea, i.e., 
 $\Delta x_{i}' = v_{i}'\Delta t_{i}'$. Quebrando os intervalos desta forma, o deslocamento de um ponto a outro \'e denotado por
 $$
   \Delta x_{1, 2} = x(t_{2}) - x(t_{1})\approx \sum\limits_{k=1}^{N}\Delta x_{i}' = \sum\limits_{k=1}^{N}v_{i}'\Delta t_{i}'
 $$
 Assim como para a velocidade instant\^anea, quanto menor tomarmos o intervalo de tempo, mais preciso \'e o valor encontrado para $\Delta x_{1, 2}$,
o que indica uma boa oportunidade para o uso do limite novamente. Com isso, definimos
 $$
  x(t_{2}) - x(t_{1}) = \lim_{\Delta t'\to0} \sum\limits_{i=1}^{N}v(t_{i}')\Delta t_{i}' = \int_{t_{1}}^{t_{2}}v(t)dt
 $$
 Este \'ultimo s\'imbolo, chamado integral, descreve a \'area ``embaixo'' da curva da fun\c c\~ao f(t) dentro do intervalo $[t_{1}, t_{2}].$
Supondo que c e k s\~ao constantes quaisquer, seguem abaixo algumas das regras de integra\c c\~ao: 
\begin{align*}
  &i)f(t) = ct^{n} \Rightarrow \frac{df}{dt} = nct^{n-1} \Rightarrow F(t) = \frac{ct^{n+1}}{n+1}\text{ (Primitiva de f)}\\
  &ii) \int_{t_{1}}^{t_{2}}f(t)dt = F(t_{2}) - F(t_{1}) = \frac{c}{n+1}t_{2}^{n+1} - \frac{c}{n+1}t_{1}^{n+1}\text{ (Integral definida de f)}\\
  &iii) \int_{}^{}f(t)dt = \frac{c}{n+1}t^{n+1} + k\text{ (Integral indefinida de f)}
\end{align*}
  Para conferir se a integral est\'a correta, \'e preciso derivar a fun\c c\~ao F e, se obter como resultado a fun\c c\~ao f, significa que est\'a correto.
Com este conhecimento em mente, segue que 
  $$
  \boxed{ x(t) = \int_{}^{}v_{0}dt = v_{0}t + x_{0}}
  $$
  Algumas outras regras importantes:
 \begin{align*}
   &iv) \frac{d\sin{(t)}}{dt} = \cos{(t)} \Rightarrow \int_{}^{}\cos{(t)}dt = \sin{(t)} + c\\
   &v) \frac{d\cos{(t)}}{dt} = -\sin{(t)} \Rightarrow \int_{}^{}\sin{(t)}dt = \cos{(t)} + c\\
   &vi) \frac{d e^{t}}{dt} = e^{t} \Rightarrow \int_{}^{} e^{t}dt = e^{t} + c
 \end{align*}
  Ou seja, em certo sentido, a integral e a derivada s\~ao dois lados da mesma moeda, assim como mulitplica\c c\~ao e divis\~ao ou adi\c c\~ao e subtra\c c\~ao.
 
\end{document}
