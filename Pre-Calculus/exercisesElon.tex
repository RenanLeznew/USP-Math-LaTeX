\documentclass{article}
\usepackage{amsmath}
\usepackage{amsthm}
\usepackage{amssymb}
\usepackage{amsfonts}
\usepackage{hyperref}

\newtheorem*{exercise*}{Enunciado}

\hypersetup{
    colorlinks,
    citecolor=black,
    filecolor=black,
    linkcolor=black,
    urlcolor=black
}

\title{Exerc\'icios do Elon}
\author{Renan Wenzel}
\date{\today}

\begin{document}
    \maketitle

    \newpage
    \tableofcontents
    \newpage
    
    \section{Cap\'itulo 3}
    \subsection{Exerc\'icio 1}
    \subsection{Exerc\'icio 2}
    \begin{exercise*}
        $f:X\rightarrow{Y}$ \'e injetora se, e somente se, existe uma fun\c c\~ao $g:Y\rightarrow{X}$ tal que g(f(x)) = x para todo $x\in{X}$.
    \end{exercise*}
    
    \paragraph{} Comecemos pela implica\c c\~ao 
    \textit{\begin{quote}"    
    Se $f:X\rightarrow{Y}$ \'e injetora, ent\~ao
    existe uma fun\c c\~ao $g:Y\rightarrow{X}$ tal que g(f(x)) = x para todo $x\in{X}$.
    "\end{quote}} 
    A ideia
    aqui \'e que n\'os definamos uma "coisa" $g:Y\rightarrow{X}$ tal que g(f(x)) = x para todo $x\in{X}$
    que n\'os ainda n\~ao sabemos se \'e uma fun\c c\~ao ou n\~ao. A partir disso, vamos mostrar que,
    de fato, essa g \'e uma fun\c c\~ao. Em outras palavras, mostrar que ela \'e bem-definida (o que
    significa que se $y_1 = y_2$, ent\~ao $g(y_1) = g(y_2)$).

    A priori, suponha que $y_1 = y_2$, mas $g(y_1) \neq g(y_2)$. Suponha tamb\'em que $y_1, y_2$ 
    pertencem \`a imagem da fun\c c\~ao f. Em outras palavras, $y_1 = f(x_1), y_2 = f(x_2)$ para 
    algum $x_1, x_2\in{X}$. Neste caso, se $g(y_1) \neq g(y_2)$, ent\~ao $g(f(x_1)) = x_1 \neq 
    x_2 = g(f(x_2))$. No entanto, isso \'e uma contradi\c c\~ao, pois f \'e injetora, tal que
    $y_1 = y_2$ implica que $f(x_1) = f(x_2)$. 
    
    Agora, lidemos com o caso em que $y_1, y_2$ n\~ao pertencem \`a imagem de f. Com isso, podemos
    definir a fun\c c\~ao g da maneira que desejarmos, pois o caso que importa \'e quando ela \'e
    aplicada a algum elemento da imagem de f. Assim, definindo, por exemplo, $g(y) = 1$ para todo
    y fora da imagem de f. Ent\~ao, se $y_1 = y_2,$ segue que $g(y_1) = 1 = g(y_2)$, tal que a fun\c c\~ao
    est\'a, de fato, bem-definida.

    Resta lidar com a outra implica\c c\~ao, isto \'e,
    \begin{quote}
        Se existe uma fun\c c\~ao $g:Y\rightarrow{X}$ tal que g(f(x)) = x para todo $x\in{X}$, 
        ent\~ao f \'e injetora.
    \end{quote}
    Explicitamente, precisamos mostrar que se $f(x_1) = f(x_2)$, ent\~ao $x_1 = x_2$. De fato,
    suponha que $f(x_1) = f(x_2)$. Aplicando g, segue que:
    $$
        x_2 = g(f(x_2)) = g(f(x_1)) = x_1.
    $$
    Portanto, a fun\c c\~ao \'e injetora.
    \qedsymbol
    
    \subsection{Exerc\'icio 3}
    \begin{exercise*}
        \begin{quote}"
            Se $f:X\rightarrow{Y}$ \'e injetora, ent\~ao
            existe uma fun\c c\~ao $g:Y\rightarrow{X}$ tal que g(f(x)) = x para todo $x\in{X}$.            
        "\end{quote}
    \end{exercise*}

    \subsection{Exerc\'icio 4}
    \subsection{Exerc\'icio 5}
    \subsection{Exerc\'icio 6}
\end{document}