\documentclass{article}
 \usepackage{amsmath}
 \usepackage{amsthm}
 \usepackage{amssymb}
 \usepackage{pgfplots}
 \usepackage[utf8]{inputenc}
 \usepackage{amsfonts}
 \usepackage[margin=2.5cm]{geometry}
 \usepackage{graphicx}
 \usepackage[export]{adjustbox}
 \usepackage{fancyhdr}
 \usepackage[portuguese]{babel}
 \usepackage{hyperref}
 \usepackage{lastpage}
 \usepackage{mathtools}
 \setcounter{section}{-1}

 \pagestyle{fancy}
 \fancyhf{}

 \pgfplotsset{compat = 1.18}

 \hypersetup{
     colorlinks,
     citecolor=black,
     filecolor=black,
     linkcolor=black,
     urlcolor=black
 }
 \newtheorem*{def*}{\underline{Defini\c c\~ao}}
 \newtheorem*{theorem*}{\underline{Teorema}}
 \newtheorem*{lemma*}{\underline{Lema}}
 \newtheorem*{prop*}{\underline{Proposi\c c\~ao}}
 \newtheorem*{crl*}{\underline{Corolário}}
 \newtheorem{example}{\underline{Exemplo}}
 \newtheorem*{proof*}{\underline{Prova}}
 \renewcommand\qedsymbol{$\blacksquare$}
 \newcommand{\Lin}[1]{Lin_{\mathbb{K}}({#1})}

 \rfoot{P\'agina \thepage \hspace{1pt} de \pageref{LastPage}}

 \begin{document}
 \begin{figure}[ht]
  \minipage{0.76\textwidth}
    \includegraphics[width=4cm]{../icmc.png}
    \hspace{7cm}
    \includegraphics[height=4.9cm,width=4cm]{../brasao_usp_cor.jpg}
  \endminipage  
\end{figure}

\begin{center}
  \vspace{1cm}
  \LARGE
  UNIVERSIDADE DE S\~AO PAULO

  \vspace{1.3cm}
  \LARGE
  INSTITUTO DE CI\^ENCIAS MATEM\'ATICAS E COMPUTACIONAIS - ICMC

  \vspace{1.7cm}
  \Large
  \textbf{Notas de Espaços Métricos}

  \vspace{1.3cm}
  \large
  \textbf{Renan Wenzel - 11169472}

  \vspace{1.3cm}
  \large
  \textbf{Professora - Thaís Jordão}

  \textbf{E-mail: tjordao@icmc.usp.br}

  \vspace{1.3cm}
  \today
\end{center}

 \newpage

 \tableofcontents
 \newpage
 \section{Informações (Possivelmente) Úteis}
 \subsection{Datas das Provas:}
\begin{itemize}
  \item[P1)] 31/08 - Peso 1;
  \item[P2)] 03/10 - Peso 2;
  \item[P3)] 31/10 - Peso 2;
  \item[P4)] 23/11 - Peso 3;
  \item[P5)] 14/12 - Peso 3.
\end{itemize}
\subsection{Bibliografia}
 \begin{itemize}
   \item LIMA, E. L.  ``Espaços Métricos'', Rio de Janeiro: Projeto Euclides, 2005.
   \item DOMINGUES, H. H. ``Espaços Métricos e Introdução à Topologia'', Atual Editora, 1982.
 \end{itemize}
\subsection*{Monitoria}
  A ser definido. 

 \newpage

\section{Aula 01 - 08/08/2023}
\subsection{Motivações}
\begin{itemize}
  \item Introdução ao Material do Curso.
\end{itemize}
\subsection*{O que é um espaço métrico?}
  Ao longo deste curso, trabalharemos com um conjunto M não-vazio.
\hypertarget{def_metric}{ \begin{def*}
   Uma função \(d:M\times M\rightarrow \mathbb{R}\) é dita ser uma métrica em M se: 
  \begin{itemize}
    \item[i)] \(d(x, y)\geq 0, x, y\in M\);
    \item[ii)] \(d(x, y) = 0 \Leftrightarrow x = y, x, y\in M\);
    \item[iii)] \(d(x, y) = d(y, x), x, y\in M\);
    \item[iv)] \(d(x, y)\leq d(x, z) + d(z, y), x, y, z\in M\).
  \end{itemize}
  Neste caso, o par \((M, d)\) é chamado espaço métrico.
 \end{def*}
}
\begin{example}
 \begin{itemize}
   
   \item[1)] \((\mathbb{R}, d)\), em que \(d:\mathbb{R}\times \mathbb{R}\rightarrow [0, \infty)\) é dado por 
    d(x, y) = \(|x-y|.\) É claro que, olhando para \(d(x,y),\) vale para quaisquer x, y reais que 
      \[
        d(x, y) = |x-y| = |-1(y-x)| = 1|y-x| = d(y, x).
      \]
    Assim, resta verificarmos os itens dois e quatro da \hyperlink{def_metric}{definição de métrica}. Para o item (ii), 
    \[
      |x-y| = 0 \Longleftrightarrow x = y.
    \]
    Com relação ao último item, observe que 
    \[
      |x+y|\leq |x| + |y|.
    \]
    De fato, como \(|a|\geq a\) para todo número real a, 
    \[
      |x+y|^{2} = (x+y)^{2} = x^{2} + 2xy +y^{2}\leq x^{2} + 2|x||y| + y^{2} = (|x| + |y|)^{2}.
    \]
    Logo, tomando a raíz dos dois lados, segue a afirmação:
    \[
        |x+y|\leq |x| + |y|.
    \]
    Com isso, temos 
      \[
        d(x, y) = |x-y| = |x-z+z-y| = |(x-z)-(y-z)|\leq |x-z| + |y-z|.
      \]
    Portanto, \(d(x, y)\leq d(x, z) + d(z, y)\), o que torna \((\mathbb{R}, d)\) um espaço métrico.

    \item[2)] Seja X um conjunto não-vazio. Definimos 
      \[
        d:X\times X\rightarrow [0, \infty)
      \]
      por 
        \[
          d(x, y) = \left\{\begin{array}{ll}
              0,\quad x = y\\
              1,\quad x\neq y.
            \end{array}\right.
        \]
      Esta métrica é conhecida como \textit{métrica discreta}. Verifiquemos as propriedades dela.

Com efeito, como a imagem dela pode ser apenas 0 ou 1, o item 1 é trivial. Por definição, a métrica vale 0 se,
e somente se, x e y são iguais, tal que o item (ii) está feito. O item (iii) segue automaticamente se x e y são iguais. Caso
eles sejam diferentes, temos \(d(x, y) = 1, d(y, x) = 1\), ou seja, o item (iii) é válido para todos os casos. Por fim, a desigualdade triangular fica como exercício.
 \end{itemize} 
\end{example}
\newpage

\section{Aula 02 - 10/08/2023}
\subsection{Motivações}
\begin{itemize}
  \item Exemplos de espaços métricos;
  \item Similaridade entre métricas;  
  \item Produto de espaços métricos.
\end{itemize}
\subsection{Uma nota histórica}
  Algumas dessas informações podem ser vistas com mais detalhe do livro de espaços métricos
de Jean Cerqueira, do IME. 

  Para um contexto temporal, em 1906, Maurice Fréchet publicou sua tese de doutorado, nomeada
``Sus quelques du calcul functionnel''. A seguir, em 1910, David Hilbert faz sua tentativa de axiomatizar
as ideias vistas na tese de Fréchet e, dois anos depois, Felix Hausdorff, trabalhando no conceito de separação de pontos
a partir do ponto de vista de conjuntos, acaba contribuindo com essa axiomatização. Antes disso, Henri Poincaré, havia sistematizado as ideias
de forma prototípica. Outro nome mencionável é o de Pavel Urysohnn, responsável por aprofundar-se na parte da separação de pontos.

\subsection{Exercício 1 da Lista 1}
  Considere \(d:\mathbb{R}^{2}\times \mathbb{R}^{2}\rightarrow [0, \infty) \) definida por 
    \[
      d((x_{1}, x_{2}), (y_{1}, y_{2})) = |x_{2}-y_{2}|.
    \]
  Então, d não é uma métrica. Chequemos este fato. 

  Com efeito, considere x um número real qualquer. Assim, 
    \[
      d((1, x), (2, x)) = |x-x| = 0.
    \]
  No entanto, (1, x) não é igual a (2, x), ou seja, já falha logo na primeira condição de métrica!
Portanto, conclui-se que d não pode ser métrica. Analogamente, definindo \(d':\mathbb{R}^{2}\times \mathbb{R}^{2}\rightarrow [0, \infty)\) por 
 \(d'((x_{1}, x_{2}), (y_{1}, y_{2}))=|x_{1}-y_{1}|\), ela também não será métrica.

 \textbf{Obs.:} Definiremos, ainda essa aula, a métrica \(d_{s}:\mathbb{R}^{2}\times \mathbb{R}^{2}\rightarrow [0, \infty)\) por 
   \[
     d_{s}((x_{1}, x_{2}), (y_{1}, y_{2})) = |x_{1}-y_{1}|+|x_{2}-y_{2}|,
   \]
   a qual torna \((\mathbb{R}^{2}, d_{s})\) num espaço métrico. Mostremos isso. Os itens \hyperlink{def_metric}{(i), (ii) e (iii)}
da definição de métrica estão trivialmente cumpridos. Para a desigualdade triangular, observe que 
\begin{align*}
  d((x_{1}, x_{2}), (y_{1}, y_{2})) &= |x_{1}-y_{1}| + |x_{2}-y_{2}| \\
                                    &\leq |x_{1}-z_{1}| + |z_{1}-y_{1}| + |x_{2}-z_{2}| + |x_{2}-y_{2}| \\
                                    &=d((x_{1}, x_{2}), (z_{1}, z_{2})) + d((z_{1}, z_{2}), (y_{1}, y_{2}))
\end{align*}
\subsection{Espaço de funções contínuas}
   Considere o intervalo I=[0, 1] e tome o conjunto 
     \[
       \mathcal{C}(I, \mathbb{R})\coloneqq \{f:I\rightarrow \mathbb{R}: f \text{ contínua}\}.
     \]
     Então, são métrica em \(\mathcal{C}(I, \mathbb{R})\) as funções \(d:\mathcal{C}(I, \mathbb{R})\rightarrow [0, \infty), (f, g)\mapsto \sup_{x\in I}\{|f(x)-g(x)|\}\)
e \(\rho :\mathcal{C}(I, \mathbb{R})\times \mathcal{C}(I, \mathbb{R})\rightarrow [0, +\infty)\) definida por 
  \[
    \rho(f, g) = \int_{0}^{1}|f(x)-g(x)|dx,\quad f, g\in \mathcal{C}(I, \mathbb{R}).
  \]
    Mostremos que elas são \hyperlink{def_metric}{métricas}, começando pela d.

    Observe que, se \(d(f, g) = 0,\), então 
      \[
        |f(y)-g(y)|\leq \sup_{x\in I}\{|f(x)-g(x)|\}\quad \forall y\in I.
      \]
    Logo, \(|f(y) - g(y)| = 0\) para todo y em I, garantindo que f e g são as mesmas.
    Para a desigualdade triangular, observe que, para qualquer y em I, e h em \(\mathcal{C}(I, \mathbb{R}),\) vale que
    \[
      |f(y)\pm h(y) - g(y)|\leq |f(y) - h(y)| + |h(y) - g(y)|.
    \]
    Tomando o supremo, obtemos 
      \[
        d(f, g)\leq d(f, h) + d(h, g).
      \]
    Assim, \((\mathcal{C}(I, \mathbb{R}), d)\) é um espaço métrico.

    Agora, analisando a questão de \(\rho\), se \(\rho (f, g) = 0,\) então 
      \[
        \int_{0}^{1}|f(x)-g(x)|dx = 0
      \]
    Como o Teorema da Conservação de Sinal implicaria em \(\int_{0}^{1}|f(x)-g(x)|dx > 0\) se \(|f(x)-g(x)| > 0\) para algum x em I.
    Logo, \(|f(x)-g(x)|=0\) para todo x em I. Para mostrar a desigualdade triangular, considere \(f, g, h\in \mathcal{C}(I, \mathbb{R}).\)
    Temos, para todo \(x\in I\), 
      \[
        |f(\overbrace{x)-g(}^{\pm h(x)}x)|\leq |f(x) - h(x)| + |h(x) - g(x)|
      \]
    Integrando os dois lados da desigualdade, 
      \[
        \int_{0}^{1}|f(x)-g(x)|\leq \int_{0}^{1}|f(x) - h(x)|dx + \int_{0}^{1}|h(x) - g(x)|dx,
      \]
      de onde segue a desigualdade triangular. Portanto, \((\mathcal{C}(I, \mathbb{R}), \rho)\) é um espaço métrico.
\subsection{Espaço euclidiano n-dimensional}
  Seja n um natural, \(n\geq 1.\) O espaço euclidiano é 
    \[
      \mathbb{R}^{n}\coloneqq \{(x_{1}, \cdots, x_{n}): x_{i}\in \mathbb{R}, i = 1, 2, \cdots, n\},
    \]
    munido da métrica usual/euclidiana, definida por 
      \[
        d(x, y) = \biggl(\sum\limits_{i=1}^{n}(x_{i}-y_{i})^{2}\biggr)^{\frac{1}{2}},
      \]
      em que \(x=(x_{1}, x_{2}, \cdots, x_{n})\) e \(y = (y_{1}, y_{2}, \cdots, y_{n}\) são elementos de \(\mathbb{R}^{n}.\)
  Sobre \(\mathbb{R}^{n}\), podemos considerar duas outras métricas, a métrica da soma e do máximo. Definimo-las, respectivamente, por
 \begin{align*}
   &d_{s}(x, y)\coloneqq \sum\limits_{i=1}^{n}|x_{i}-y_{i}|\\
   &d_{m}(x, y)\coloneqq \max\{|x_{i}-y_{i}|, i = 1, 2, \cdots, n\},
 \end{align*}
 para  \(x=(x_{1}, x_{2}, \cdots, x_{n})\) e \(y = (y_{1}, y_{2}, \cdots, y_{n}\) são elementos de \(\mathbb{R}^{n}.\) Vamos mostrar que a primeira d é métrica.
 
  Pra começar, não é difícil ver que os itens \hyperlink{def_metric}{(i) e (ii)} da definição são satisfeitos. Para mostrar a desigualdade triangular, será necessário
utilizar Cauchy-Schwarz: 
\begin{lemma*}[Desigualdade de Cauchy-Schwars]Dados \((x_{1}, \cdots, x_{n})\) e \((y_{1}, \cdots, y_{n})\), vale a desigualdade
  \[
    \hypertarget{cauchy_schwarz}{\sum\limits_{i=1}^{n}|x_{i}\cdot y_{i}|\leq \biggl(\sum\limits_{i=1}^{n}|x_{i}|^{2}\biggr)^{\frac{1}{2}}\biggl(\sum\limits_{i=1}^{n}|y_{i}|^{2}\biggr)^{\frac{1}{2}} }
  \]
\end{lemma*}
\begin{proof*}
  Observe que, dados \(x, y\in \mathbb{R}\), vale que \(2xy\leq x^{2} + y^{2}\).
  Temos, para \(x=(\sum\limits_{i=1}^{n}|x_{i}|^{2})^{\frac{1}{2}}\) e \(y=(\sum\limits_{i=1}^{n}|y_{i}|^{2})^{\frac{1}{2}}\). Com isso, aplicando a desigualdade vista para
  \(\frac{|x_{i}|}{x}\) e \(\frac{|y_{i}|}{y}\), ganhamos 
    \[
      2\frac{|x_{i}||y_{i}|}{xy}\leq \frac{|x_{i}|^{2}}{x^{2}} + \frac{|y_{i}|^{2}}{y^{2}}.
    \]
    Somando de \(i=1, \cdots, n,\) obtemos 
      \[
        \frac{2}{xy}\sum\limits_{i=1}^{n}|x_{i}||y_{i}| \leq \frac{1}{x^{2}}\sum\limits_{i=1}^{n}|x_{i}|^{2} + \frac{1}{y^{2}}\sum\limits_{i=1}^{n}|y_{i}|^{2} = 1 + 1 = 2.
      \]
    Portanto, isolando a soma à esquerda, temos 
      \[
        \sum\limits_{i=1}^{n}|x_{i}||y_{i}|\leq \frac{2}{2}\cdot xy = xy.\text{\qedsymbol}
      \]
    \end{proof*}
  A desigualdade triangular seguirá do seguinte 
  \begin{align*}
    \bigl[d(\overline{x}, \overline{y})\bigr]^{2} = \sum\limits_{i=1}^{n}(x_{i}-y_{i})^{2} &=\sum\limits_{i=1}^{n}(x_{i}-z_{i}+z_{i}-y_{i})^{2}\\
                                                                                           &=\sum\limits_{i=1}^{n}(x_{i}-z_{i})^{2} + 2\sum\limits_{i=1}^{n}(x_{i}-z_{i})(z_{i}-y_{i}) + \sum\limits_{i=1}^{n}(z_{i}-y_{i})^{2}\\
                                                                                           &\leq \sum\limits_{i=1}^{n}(x_{i}-z_{i})^{2} + 2\biggl(\sum\limits_{i=1}^{n}|x_{i}-z_{i}|^{2}\biggr)^{\frac{1}{2}}\biggl(\sum\limits_{i=1}^{n}|z_{i}-y_{i}|^{2}\biggr)^{\frac{1}{2}} + \sum\limits_{i=1}^{n}(z_{i}-y_{i})^{2}\\
                                                                                           &=\bigl[d(\overline{x}, \overline{y})\bigl]^{2} + 2 d(\overline{x}, \overline{z})d(\overline{z}, \overline{y}) + \bigl[d(\overline{z}, \overline{y})\bigr]^{2} \\
                                                                                           & = (d(\overline{x}, \overline{z})+d(\overline{z}, \overline{y}))^{2}.
  \end{align*}
  O fim da prova da desigualdade triangular é um exercício.
\newpage

\section{Aula 03 - 17/08/2023}
\subsection{Motivações}
\begin{itemize}
  \item Métricas similares;
  \item Produtos de Espaços Métricos;
  \item Espaços vetoriais normados;
  \item Desigualdade de Hölder e de Minkowski.
\end{itemize}
\subsection{Similaridade de Métricas}
 \begin{def*}
   Seja M um conjunto não vazio. Duas métricas d e \(\rho \) em M são similares se existem \(c_{1}, c_{2} > 0\) tais que 
     \[
       c_{1}d(x, y)\leq \rho (x, y)\leq c_{2}d(x, y)
     \]
 \end{def*}
\begin{example}
  Seja d a métrica usual e \(\delta  \) a métrica discreta em \(\mathbb{R}\). Para todo c positivo, existem
x, y em \(\mathbb{R}\) tais que \(d(x, y) > c\delta (x, y)\), ou seja, não vale \(d(z, w)\leq c\delta (z, w)\) para todos
z, w reais. 

  De fato, dado \(c > 0\), tome \(x=2c\) e \(y=c-1.\) Tem-se \(d(x, y) = c + 1\) e \(\delta (x, y) = 1\). Logo, não são similares.
\end{example}
\begin{prop*}
  Para quaisquer x, y em \(\mathbb{R}^{n},\) vale a desigualdade 
    \[
      d_{m}(x,y)\leq d(x,y)\leq d_{s}(x, y)\leq nd_{m}(x,y)
    \]
\end{prop*}
\begin{proof*}
  Seja \(x=(x_{1}, x_{2}, \cdots, x_{n}), y = (y_{1}, \cdots, y_{n})\). Temos: 
    \[
      d_{m}(x, y) = \max \biggl\{(|x_{i}-y_{i}|^{2})^{\frac{1}{2}}\biggr\}\leq \biggl(\sum\limits_{i=1}^{n}|x_{i}-y_{i}|^{2}\biggr)^{\frac{1}{2}} = d(x, y)
    \]
  e também \(d_{s}(x, y) = \sum\limits_{i=1}^{n}|x_{i}-y_{i}|\leq n\max \biggl\{|x_{i}-y_{i}|\biggr\}\),
  provando a desigualdade do meio  
  \[
    (d(x,y))^{2} = \sum\limits_{i=1}^{n}(x_{i}-y_{i})^{2} = \underbrace{\biggl(\sum\limits_{i=1}^{n}|x_{i}-y_{i}|\biggr)}_{d_{s}(x, y)^{2}} - A,\quad A\geq 0,
  \]
  de onde segue que \((d(x, y))^{2}\leq (d_{s}(x, y))^{2} - A\) e prova a desigualdade. \qedsymbol
\end{proof*}
\subsection{Produto de Espaços Métricos}
Considere \((M_{1}, d_{2}), (M_{2}, d_{2}), \cdots, (M_{n}, d_{n})\) espaços métricos. Para \(M=\Pi_{i=1}^{n}M_{i}\) e definimos as métricas anteriores
\begin{itemize}
  \item \(d(x, y) = \biggl\{\sum\limits_{i=1}^{n}[d_{i}(x_{i}, y_{i})]^{2}\biggr\}^{\frac{1}{2}}\)
  \item \(d_{s}(x, y) = \sum\limits_{i=1}^{n}d_{i}(x_{i}, y_{i})\);
  \item \(d_{max}(x, y) = \max\{d_{i}(x_{i}, y_{i})\}\).
\end{itemize}
\begin{example}
  Tome \(M = \mathcal{C}(I, \mathbb{R})\times \mathbb{R}\). Se \(x = (f, s), y=(h, t)\), então 
    \[
      d_{m}(x, y) = \max \biggl\{\rho (f, h), d(s, t)\biggr\}
    \]
\end{example}
\subsection{Espaços Vetoriais Normados}
\begin{def*}
  Dado V um espaço vetorial, uma norma em V é uma função \(||\cdot ||:V\rightarrow [0, +\infty)\) que satisfaz
 \begin{itemize}
   \item[i)] \(||\vec{v}|| = 0 \Longleftrightarrow \vec{v}=0\);
   \item[ii)] \(||\lambda \vec{v}|| = |\lambda |||\vec{v}||\);
   \item[iii)] \(||\vec{v}+\vec{w}||\leq ||\vec{v}|| + ||\vec{w}||\).
 \end{itemize}
\end{def*}
\begin{example}
  Seja \((\mathbb{R}^{2n}, +, \cdot )\) é um espaço vetorial e \(||\cdot ||:\mathbb{R}^{n}\rightarrow [0, \infty)\) dada por 
    \[
      x\mapsto ||x|| = \sqrt[]{\sum\limits_{i=1}^{n}x_{i}^{2}}
    \]
\end{example}
\begin{example}
  Considere \((\mathcal{C}(I, \mathbb{R}), +, \cdot )\) e defina \(||f||_{\infty} = \max \biggl\{|f(x)|: x\in I\biggr\}, f\in \mathcal{C}(I, \mathbb{R}).\) Isto define uma norma em
 \(\mathcal{C}(I, \mathbb{R}).\) 

  Note que \(||f||_{\infty} = 0 = \max\{|f(x)|\}.\) Logo, \(|f(y)| = |f|_{\infty} = 0\) para todo y em I e \(f(y) = 0\) para todo y em I.
Isso demonstra a primeira propriedade. Para a segunda propriedade, 
  \[
    ||\lambda f||_{\infty} = \max\{|\lambda f(x)|: x\in I\} = |\lambda |||f||_{\infty}
  \]
  pelas propriedades de módulo. Por fim, temos 
    \[
      ||f+g||_{\infty} = \max\{|f(x)+g(x)|: x\in I\}\leq \max\{|f(x)| + |g(x)|: x\in I\}\leq \max\{|f(x)|\} + \max\{|g(x)|\}\leq ||f||_{\infty} + ||g||_{\infty}.
    \]
\end{example}
\begin{example}
  Definiremos a norma p em \(\mathbb{R}^{n}\). Se \(p=\infty,\) coloquemos 
    \[
      ||x||_{\infty} = \sup\{|x_{i}|: 1\leq i\leq n\}.
    \]
  Se \(p\neq\infty\), definimos 
    \[
      ||x||_{p} = \biggl(\sum\limits_{i=1}^{n}|x_{i}|^{p}\biggr)^{\frac{1}{p}}
    \]
\end{example}
\subsection{A Desigualdade de Minkowski}
\begin{lemma*}
  Se \(p, q\in (1, \infty)\) é tal que \(\frac{1}{p} + \frac{1}{q} = 1\) e \(a, b\in [0, \infty),\) então 
    \[
      a^{\frac{1}{p}} + b^{\frac{1}{q}}\leq \frac{a}{q} + \frac{b}{q}.
    \]
\end{lemma*}
\begin{proof*}
  Se \(b > 0\), então \(\frac{a^{\frac{1}{p}}+b^{\frac{1}{q}}}{b} = \frac{a}{bp} + \frac{1}{q}\), isto é, 
    \[
      \biggl(\frac{a}{b}\biggr)^{\frac{1}{p}}\leq \frac{a}{bp} + \frac{1}{q}.
    \]
    Tomando \(z=\frac{a}{b}\) e \(\alpha =\frac{1}{p},\) segue que \(t^{\alpha }\leq \alpha t + 1 - \alpha.\) Com isso, defina 
  \(f_{\alpha }:\mathbb{R}^{+}\rightarrow \mathbb{R}\) por 
    \[
      f_{\alpha }(t) = \alpha t +1 - \alpha - z^{\alpha }.
    \]
    Segue que \(f_{\alpha }(t)\geq 0\) pela sua derivada. \qedsymbol
\end{proof*} 
\begin{lemma*}[Desigualdade de Hölder]
  Se \(p\in(1, \infty), q\in (1, \infty)\) é tal que \(\frac{1}{p}+\frac{1}{q}=1\) e \(a, b\in [0, \infty),\) então 
    \[
      \sum\limits_{i=1}^{n}|x_{i}y_{i}|\leq \biggl[\sum\limits_{i=1}^{n}|x_{i}|^{p}\biggr]^{\frac{1}{p}}\biggl[\sum\limits_{i=1}^{n}|y_{i}|^{q}\biggr]^{\frac{1}{q}}
    \] 
  para todo \(x=(x_{1}, \cdots, y_{n}), y=(y_{1}, \cdots, y_{n}) \in \mathbb{R}^{n}.\)
\end{lemma*}
\begin{proof*}
  Se x = 0 ou y = 0, a desigualdade é trivial. Se \(x\neq0\) e \(y\neq0\), então defina 
    \[
      a_{j} = \frac{|x_{j}|^{p}}{\sum\limits_{i=1}^{n}|x_{i}|^{p}}\quad b_{j}=\frac{|y_{j}|^{q}}{\sum\limits_{i=1}^{n}|y_{i}|^{q}}.
    \] 
  Observamos que \(\sum\limits_{j=1}^{n}a_{j} = \sum\limits_{j=1}^{n}b_{j}=1.\) Aplicando a desigualdade de Young, 
    \[
    a_{j}^{\frac{1}{p}}b_{j}^{\frac{1}{q}} = \frac{|x_{j}y_{j}|}{\biggl[\sum\limits_{i=1}^{n}|x_{i}|^{p}\biggr]^{\frac{1}{p}}\biggl[\sum\limits_{i=1}^{n}|y_{i}|^{q}\biggr]^{\frac{1}{q}}}\leq \frac{1}{p}a_{j} + \frac{1}{q}b_{j},
    \]
  para \(j=1, \cdots, n.\) Assim, 
    \[
      \frac{\sum\limits_{j=1}^{n}|x_{j}y_{j}|}{\biggl[\sum\limits_{i=1}^{n}|x_{i}|^{p}\biggr]^{\frac{1}{p}}\biggl[\sum\limits_{i=1}^{n}|y_{i}|^{q}\biggr]^{\frac{1}{q}}}\leq \frac{1}{p} + \frac{1}{q}=1,
    \]
  donde segue a desigualdade. \qedsymbol
\end{proof*}
\begin{prop*}
  Se \(p\in[1, \infty)\), então 
    \[
      \biggl[\sum\limits_{i=1}^{n}|x_{i}+y_{i}|^{p}\biggr]^{\frac{1}{p}}\leq \biggl[\sum\limits_{i=1}^{n}|x_{i}|^{p}\biggr]^{\frac{1}{p}} + \biggl[\sum\limits_{i=1}^{n}|y_{i}|^{p}\biggr]^{\frac{1}{p}},
    \]
  para todo \(x=(x_{1}, \cdots, x_{n}), y=(y_{1}, \cdots, y_{n})\in \mathbb{R}^{n}\).
\end{prop*}
\begin{proof*}
  Os casos \(p=1, \infty\) são deixados como exercício. Se \(p\in(1, \infty),\) então 
    \[
      \biggl[\sum\limits_{i=1}^{n}|x_{i}+y_{i}|^{p}\biggr]^{\frac{1}{p}}\leq \biggl[\sum\limits_{i=1}^{n}(|x_{i}|+|y_{i}|)^{p}\biggr]^{\frac{1}{p}}.
    \]
  Podemos escrever 
    \[
      (|x_{i}|+|y_{i}|)^{p} = (|x_{i}|+|y_{i}|)^{p-1}|x_{i}|+(|x_{i}|+|y_{i}|)^{p-1}|y_{i}|,\quad i = 1, \cdots, n.
    \]
    Somando os elementos à esquerda da desigualdade anterior, obtemos 
      \[
        \sum\limits_{i=1}^{n}(|x_{i}|+|y_{i}|)^{p} = x_{n}+y_{n}
      \]
    com 
      \[
        x_{n} + y_{n}\coloneqq \sum\limits_{i=1}^{n}(|x_{i}|+|y_{i}|)^{p-1}|x_{i}| + \sum\limits_{i=1}^{n}(|x_{i}|+|y_{i}|)^{p-1}|y_{i}|.
      \]
    Aplicando a Desigualdade de Hölder, temos 
   \begin{align*}
     x_{n}&\leq \biggl[\sum\limits_{i=1}^{n}|x_{i}|^{p}\biggr]^{\frac{1}{p}}\biggl[\sum\limits_{i=1}^{n}(|x_{i}|+|y_{i}|)^{(p-1)q}\biggr]^{\frac{1}{q}}\\
          &\leq \biggl[\sum\limits_{i=1}^{n}|x_{i}|^{p}\biggr]^{\frac{1}{p}}\biggl[\sum\limits_{i=1}^{n}(|x_{i}|+|y_{i}|)^{p}\biggr]^{\frac{1}{q}}.
   \end{align*}
   De forma análoga, temos 
     \[
       y_{n}\leq \biggl[\sum\limits_{i=1}^{n}|y_{i}|^{p}\biggr]^{\frac{1}{p}}\biggl[\sum\limits_{i=1}^{n}(|x_{i}|+|y_{i}|)^{p}\biggr]^{\frac{1}{q}}.
     \]
  Portanto, 
    \[
      \sum\limits_{i=1}^{n}(|x_{i}|+|y_{i}|)^{p} = x_{n} + y_{n}\leq \cdots.\quad\text{\qedsymbol}
    \]
\end{proof*}
\begin{example}
  Dado \(p\in[0, \infty], (\mathbb{R}^{n}, d_{p})\) é um espaço métrico, em que \(d_{p}\) está representando a métrica induzida por \(||\cdot ||_{p}.\) Então, para \(p=2\),
recuperamos o espaço euclidiano n-dimensional.
\end{example}
\newpage

\section{Aula 04 - 21/08/2023}
\subsection{Motivações}
\begin{itemize}
  \item Subespaços métricos e distância entre conjuntos;
  \item Conjuntos limitados, abertos e fechados;
  \item A topologia de espaços métricos.
\end{itemize}
\subsection{Subespaços Métricos e Distância entre Conjuntos}
\begin{def*}
  Seja (X, d) um espaço métrico e \(M\subseteq{X}.\) Então, \(d|_{M}M\times M:\rightarrow \mathbb{R}\)
define uma métrica, chamada métrica induzida em M. Isso faz de \((M, d|_M)\) um subespaço métrico. \(\square\)
\end{def*}
  Além da distância entre pontos, pode-se falar da distância entre um ponto e um subconjunto do espaço métrico
e da distância entre dois subconjuntos de um espaço métrico. Para isso, considere \((X, \rho )\) um espaço métrico, 
 \(x\in X\) e \(E, F\subseteq{X}.\) Definimos, então,
\begin{align*}
  &d(x, E)\coloneqq \inf\{\rho (x, e): e\in E\}\\
  &d(E, F)\coloneqq \inf\{d(e, F): e\in E\}.
\end{align*}
\textbf{Observação:} O nome ``distância'', aqui, não é sinônimo de métrico. De fato, há um exercício na lista que mostra que a distância entre conjuntos
\textbf{NÃO} é simétrica, ou seja, não define uma métrica.
\begin{prop*}
  Seja \((X, \rho )\) um espaço métrico e \(E\subseteq{X}.\) Então, 
    \[
      |d(x, E) - d(y, E)|\leq \rho (x, y) \quad \forall x, y\in X.
    \]
\end{prop*}
\begin{proof*}
  Note que, para todo e em E, 
    \[
      d(x, E)\leq \rho (x, e)\leq \rho (x, y) + \rho (y, E).
    \]
  Logo, para todo \(x, y\in X\), 
    \[
      d(x, E)\leq \rho (x, y) + d(y, E).
    \]
  Assim, temos 
    \[
      d(x, E) - d(y, E)\leq \rho (x, y).
    \]
  Analogamente, 
    \[
      d(y, E) - d(x, E)\leq \rho (x, y)
    \]
  Portanto, 
    \[
      |d(x, E) - d(y, E)|\leq \rho (x, y).\quad\text{\qedsymbol}
    \]
\end{proof*}
\begin{example}
  Considere \((\mathbb{R}, |\cdot |), A = (-1, 0]\) e x = -2. Por definição, 
    \[
      d(x, A) = \inf\{|-2-a|: a\in A\} = \inf\{|2+a|: a\in A\} = 1\quad [d(-2, 1+\varepsilon )= 1 +\varepsilon \forall \varepsilon >0]
    \]
  No entanto, \(d(-2, a) > 1\) para todo a em A.
\end{example}
\begin{crl*}
   Seja \((X, \rho )\) um espaço métrico. Vale a desigualdade 
     \[
       |\rho (x, z) - \rho (y, z)|\leq \rho (x, y).
     \]
 \end{crl*}
\begin{proof*}
  Seja \(E=\{z\}\). Pela proposição, o resultado já segue. \qedsymbol 
\end{proof*}
\subsection{Topologia de Espaços Métricos}
\begin{def*}
  Seja \((X, d)\) um espaço métrico. Dado x em X e \(r > 0\), o conjunto 
    \[
      B_{r}(x)\coloneqq \{y\in X: d(x, y) < r\}
    \]
  é chamado bola aberta de centro em x e raio r. O conjunto
    \[
      D_{r}(x)\coloneqq \{y\in X: d(x, y)\leq r\}
    \]
  é chamado bola fechada de centro em x e raio r. \(\square\)
\end{def*}
\begin{example}
  Considere \((\mathbb{R}^{2}, d_{p})\), em que \(d_{p}(x, y)=||x-y||_{p}, 1\leq p\leq \infty.\)
 \begin{align*}
   &d_{2}((x_{1}, y_{1}), (x_{2}, y_{2})) = [(x_{1}-x_{2})^{2}+(y_{1}-y_{2})^{2}]^{\frac{1}{2}} \Rightarrow B_{1}(0) = \{(x, y)\in \mathbb{R}^{2}: d_{2}((0, 0), (x, y)) < 1\}\\
   &d_{\infty}((x_{1}, y_{1}), (x_{2}, y_{2})) = \max\{|x_{1} - x_{2}|, |y_{1} - y_{2}|\} \Rightarrow B_{1}(0) = \{(x, y)\in \mathbb{R}^{2}: \max\{|x|, |y|\} < 1\}\\
   &d_{1}((x_{1}, y_{1}), (x_{2}, y_{2})) = |x_{1}-y_{1}| + |x_{2} - y_{2}| \Rightarrow B_{1}(0)=\{(x, y)\in \mathbb{R}^{2}: |x|+|y| < 1\}
 \end{align*}
\end{example}
\begin{def*}
  Seja \((X, d)\) um espaço métrico e x um ponto de X. Chamamos x de ponto isolado se existe \(r > 0\) tal que \(B_{r}(x) = \{x\}. \quad\square\)
\end{def*}
\begin{example}
  Seja \(S = [0, 1]\cup \{2\}\) munido da métrica usual, induzida da reta. Temos 
    \[
      B_{\frac{1}{2}}(2)=\{y\in S: |x-2| < \frac{1}{2}\} = \{2\}.
    \]
\end{example}
\begin{example}
  Seja \(X \neq\emptyset\) e d a métrica discreta. Então, 
 \begin{itemize}
   \item[-] Todo ponto é isolado;
   \item[-] \(D_{r}(x) = \{x\}\) se \(r > 1\);
   \item[-] \(B_{1}(x) = \{x\}\) e \(D_{1}(x) = X;\)
   \item[-] \(B_{r}(x) = X\) se \(r > 1\).
  De fato, \(D_{r}(x) = \{y\in X: d(x, y)\leq r\}.\) Por isso, se r = 1, \(D_{1}(x) = X.\) Pela mesma lógica, prova-se os outros itens.
 \end{itemize}
\end{example}
\begin{example}
\begin{itemize}
  \item[i)] Dado \((\mathbb{R}, d),\) d a métrica usual, nenhum ponto é isolado e \(B_{r}(x) = (x-r, x+r), x\in \mathbb{R}, r > 0.\)
  \item[ii)] Se \(M=[0, 2]\) com métrica induzida, \(B_{1}(0) = [0, 1).\)
  \item[iii)] Se \(M=\mathbb{Z}\) com a métrica induzida, então \(B_{1}(n) = \{n\}\)
  \item[iv)] Se \(M=\{\frac{1}{n}:n\in \mathbb{N}, n\neq0\}\cup\{0\}\), com a métrica induzida, então \(B_{1}(0)\neq\{0\}\) para todo \(r>0.\) 
Neste caso, se \(n, m\neq 0\), então existe \(r > 0\) tal que \(B_{r}\biggl(\frac{1}{n}\biggr)=\{\frac{1}{n}\}\) e \(B_{r}\biggl(\frac{1}{m}\biggr)=\{\frac{1}{m}\}.\) Porém,
para qualquer \(r> 0\), seja \(n_{0}\) tal que \(0 < \frac{1}{n_{0}} < \frac{r}{2}\) e \(\frac{1}{n_{0}}\in B_{r}(0).\) Portanto, \(B_{r}(0)\neq\{0\}\) para todo \(r> 0\).
\end{itemize}
\end{example}
\begin{example}
  Considere \((C([a, b]), ||\cdot ||_{\infty}\), com \(||f||_{\infty} = \sup\{|f(x)|: x\in[a, b]\}.\) Seja
 \(h\in C([a, b])\) e \(r > 0\). Temos 
   \[
     B_{r}(h) = \{f\in C([a, b]): ||f-h||_{\infty} < r\} = \{f\in C([a, b]):\max_{x\in[a, b]}\{|f(x)-h(x)|\} < r\}.
   \]
\end{example}
\begin{def*}
  Um subconjunto \(M \neq\emptyset\) de um espaço métrico (X, d) é limitado se 
    \[
      diam(X)\coloneqq \sup\{d(x, y):x, y\in X\} < \infty.
    \]
    Neste caso, diam(X) é chamado diâmetro de X. Caso contrário, diz-se que M é ilimitado e \(diam(X) = \infty.\quad\square\)
  \end{def*}
\begin{example}
  Seja \((X, d)\) métrico. Para todo x em X e \(r>0, B_{r}(x)\) é ilimitado e, além disso, \(diam(B_{r}(x))\leq 2r\), o que segue da relação
 \(d(y, z)\leq d(y, x) + d(x, z)\leq r + r = 2r.\) Além disso, se \((X, ||\cdot ||)\) é um espaço vetorial normado e a métrica d é induzida pela norma,
 então \(diam(B_{r}(x)) = 2r.\) Com efeito, seja \(s < 2r.\) Tome y em X com \(x\neq0.\) Definimos 
   \[
     v = \frac{t}{||y||}y,
   \]
   para algum t satisfazendo \(s < 2t < 2r.\) Neste caso, \(x-v, x+v\in B_{r}(x)\)  e 
     \[
       d(x+v, x-v) = 2||v|| = 2t > s,
     \]
     ou seja, a afirmação feita está garantida.
\end{example}
\newpage

\section{Aula 05 - 24/08/2023}
\subsection{Motivações}
\begin{itemize}
  \item Propriedades do diâmetro;
  \item Exemplo de limitado;
  \item Conjuntos abertos;
  \item Propriedade dos abertos.
\end{itemize}
\subsection{Propriedades do Diâmetro}
 \begin{example}
   A função 
     \[
       \rho_{p}(x,y) = \frac{||x-y||_{p}}{1+||x-y||_{p}},\quad \forall x, y\in \mathbb{R}^{n}
     \]
     é uma métrica em \(\mathbb{R}^{n}\). Para ver isso, comece definindo \(f(t) = \frac{t}{1+t}\) e exiba que essa função é
crescente, de forma que \(f(||x-y||_{p}\leq f(||x-z||_{p}+||z-y||_{p})\). Com essa métrica, afirmamos que o seguinte ocorre: 
  \[
    \mathbb{R}^{n} = B_{1}(0)\quad\&\quad diam(\mathbb{R}^{n})\leq 1.
  \]
  A inclusão \(B_{1}(0)\subseteq{\mathbb{R}^{n}}\) é automática. Por outro lado, seja \(y\in \mathbb{R}^{n}.\) Temos 
    \[
      \rho_{p}(y, 0) = \frac{||y||_{p}}{1 + ||y||_{p}} < 1. \Rightarrow \mathbb{R}^{n}\subseteq{B_{1}(0)}.
    \]
  Portanto, \(\mathbb{R}^{n}\) é limitado no espaço métrico \((\mathbb{R}^{n}, \rho_{p}),\) mas não é limitado em \((\mathbb{R}^{n}, d_{p})\).
 \end{example}
 Vale uma observação - apesar dessa diferença entre \(\rho_{p}\) e \(d_{p}\), veremos futuramente que as duas métricas induzem a mesma
estrutura de formato do espaço - em outras palavras, a mesma topologia.
\begin{prop*}
  Seja \((X, \rho )\) um espaço métrico.
 \begin{itemize}
   \item[1)] \(E\subseteq{X}\) é limitado se, e somente se, existe \(r>0\) tal que \(E\subseteq{B_{r}(x)}\) para todo \(x\in E\).
     \item[2)] Se \(E\subseteq{X}\) é limitado e não-vazio, então 
       \[
         diam(E) = \inf\{r > 0: E \subseteq{B_{r}(x)}, x \in E\}.
       \]
 \end{itemize}
\end{prop*}
\begin{proof*}
  \(1 \Rightarrow )\) A volta é simples, segue das propriedades do supremo. Por outro lado, se E é limitado, então \(diam(E) < \infty\). Seja
 \(r=2diam(E)\) e \(B_{r}(x)\supseteq{E}\) para todo x em E. De fato, se \(e\in E\), temos 
   \[
     d(e, x)\leq diam(E) < r.
   \]

   \(2 \Rightarrow )\) Seja \(A = \{r>0: E \subseteq{B_{r}(x)}, x\in E\}\). Mostremos que \(diam(E) = \inf{(A)}.\) Se \(r > diam(E)\) e x
é um ponto de E, então 
  \[
    d(x, y)\leq diam(E) < r\quad \forall y\in E,
  \]
  logo, \(E\subseteq{B_{r}(x)}\). Assim, o intervalo \((diam(E), \infty)\subseteq{A}.\) Deste modo, \(\inf{(A)}\leq diam(E).\)
  Por outro lado, se \(r < diam(E),\) então existem \(x, y\in E\) tais que 
    \[
      d(x, y) > r,\quad y\not\in B_{r}(x),\quad \&\quad r\not\in A.
    \]
    Consequentemente, \((0, diam(E))\cap A = \emptyset.\) Portanto, \(diam(E) = \inf{(A)}.\) \qedsymbol 
\end{proof*}
\begin{def*}
  Seja \((X, d)\) um espaço métrico. Um subconjunto \(E\subseteq{X}\) é dito aberto em \((X, d)\) se, para cada
x em E, existir \(r_{x} > 0\) tal que \(B_{r_{x}}(x)\subseteq{E}.\quad\square\)
\end{def*}
\end{document}
