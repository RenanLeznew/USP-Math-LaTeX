 \documentclass{article}
 \usepackage{amsmath}
 \usepackage{amsthm}
 \usepackage{amssymb}
 \usepackage{pgfplots}
 \usepackage{amsfonts}
 \usepackage[margin=2.5cm]{geometry}
 \usepackage{graphicx}
 \usepackage[export]{adjustbox}
 \usepackage{fancyhdr}
 \usepackage[portuguese]{babel}
 \usepackage{hyperref}
 \usepackage{lastpage}
 \usepackage{mathtools}

 \pagestyle{fancy}
 \fancyhf{}

 \pgfplotsset{compat = 1.18}

 \hypersetup{
     colorlinks,
     citecolor=black,
     filecolor=black,
     linkcolor=black,
     urlcolor=black
 }
 \newtheorem*{def*}{\underline{Defini\c c\~ao}}
 \newtheorem*{theorem*}{\underline{Teorema}}
 \newtheorem*{lemma*}{\underline{Lema}}
 \newtheorem*{prop*}{\underline{Proposi\c c\~ao}}
 \newtheorem{example}{\underline{Exemplo}}
 \newtheorem*{proof*}{\underline{Prova}}
 \newtheorem*{crl*}{\underline{Corol\'ario}}
 \renewcommand\qedsymbol{$\blacksquare$}
 \newcommand{\Lin}[1]{Lin_{\mathbb{K}}({#1})}

 \rfoot{P\'agina \thepage \hspace{1pt} de \pageref{LastPage}}

 \title{Notas de An\'alise}
 \author{Renan Wenzel}
 \date{\today}

 \begin{document}
 \begin{figure}[ht]
	\minipage{0.76\textwidth}
		\includegraphics[width=4cm]{../icmc.png}
		\hspace{7cm}
		\includegraphics[height=4.9cm,width=4cm]{../brasao_usp_cor.jpg}
	\endminipage	
\end{figure}

\begin{center}
	\vspace{1cm}
	\LARGE
	UNIVERSIDADE DE S\~AO PAULO

	\vspace{1.3cm}
	\LARGE
	INSTITUTO DE CI\^ENCIAS MATEM\'ATICAS E COMPUTACIONAIS - ICMC

	\vspace{1.7cm}
	\Large
	\textbf{Notas de Aula de An\'alise}

	\vspace{1.3cm}
	\large
	\textbf{Renan Wenzel - 11169472}

	\vspace{1.3cm}
	\large
	\textbf{Alexandre Nolasco de Carvalho - andcarva@icmc.usp.br}

	\vspace{1.3cm}
	\today
\end{center}

 \newpage

 \tableofcontents

 \newpage

\section{Aula 01 - 13/03/2023}
\subsection{Motiva\c c\~ao}
\begin{itemize}
  \item Relembrar sistemas b\'asicos da matem\'atica;
  \item Relembrar propriedades b\'asicas das principais estruturas ($\mathbb{N}, \mathbb{Z}, \mathbb{Q}$).
\end{itemize}

\subsection{Os N\'umeros Naturais}
  Os n\'umeros naturais s\~ao os que utilizamos para contar objetos, e s\~ao caracterizados pelos Axiomas de Peano:
 \begin{itemize}
   \item[1)] Todo n\'umero natural tem um \'unico sucessor; 
   \item[2)] N\'umeros naturais diferentes t\^em sucessores diferentes;
   \item[3)] Existe um \'unico n\'umero natural, zero (0), que n\~ao \'e sucessor de nenhum n\'umero natural.
   \item[4)] Seja $X \subseteq{\mathbb{N}}$ tal que $0\in{X}$ e, se n pertence a X, seu sucessor n+1 tamb\'em pertence 
a X. Ent\~ao, X = $\mathbb{N}.$ (Propriedade de Indu\c c\~ao).
 \end{itemize}

\begin{def*}
  Definimos a adi\c c\~ao por: $n + 0 = n, n\in \mathbb{N},\text{ e }n+(p+1) = (n+p)+1, p\in{\mathbb{N}}$. Al\'em disso,
a multiplica\c c\~ao \'e dada por: $n.0 = 0, n.(p+1) = n.p + n, n, p\in\mathbb{N}.$ Ou seja, sabendo somar ou multiplicar um n\'umero,
sabemos somar e multiplicar seu sucessor.
\end{def*}
  Com rela\c c\~ao ao quarto axioma, ele leva este nome porque um dos m\'etodos de demonstra\c c\~ao, conhecido como
prova por indu\c c\~ao. Nele, mostramos um caso base, o caso 0, e utilizamos a segunda parte para provar que, se um
resultado vale para o caso n, ele vale para n+1, portanto sendo verdadeiro para todos os naturais.

\begin{lemma*}
  Para todo n natural, 1 + n = n + 1.
\end{lemma*}
\begin{proof*}
  Note que o resultado \'e verdadeiro para n = 0. Suponha que o resultado seja v\'alido para n = k e mostremos que 
vale tamb\'em para n = k+1. Com efeito, segue pela propriedade de indu\c c\~ao e pela defini\c c\~ao de soma que
 $$
    1 + (k + 1) = (1 + k) + 1 =  (k + 1) + 1. 
 $$
 Segue que o resultado vale para todo n natural. \qedsymbol
\end{proof*}
  A seguir, mostramos a associatividade e a comutatividade, respectivamente, das opera\c c\~oes nos naturais.
\begin{lemma*}
  Para todo n, p, r naturais, (n + p) + r = n + (p + r).
\end{lemma*}
\begin{proof*}
  Note que o resultado \'e v\'alido trivialmente para r = 0 e r = 1. Suponha que o resultado seja v\'alido para
r = k e mostremos que vale tamb\'em para r = k + 1. Com efeito, pela hip\'otese de indu\c c\~ao e defini\c c\~ao de adi\c c\~ao,
  $$
    n + (p + (k + 1)) = n + ((p + k) + 1) = (n + (p + k)) + 1 = ((n + p) + k) + 1 = (n + p) + (k + 1).
  $$
  Segue o resultado por indu\c c\~ao. \qedsymbol
\end{proof*}
\begin{lemma*}
  Para todo n, p naturais, n + p = p + n. 
\end{lemma*}
\begin{proof*}
  Observe que j\'a mostramos o caso em que p = 1. Suponha que o resultado vale para p = k e vamos mostrar o caso
p = k + 1. De fato, pela hip\'otese de indu\c c\~ao e defini\c c\~ao de adi\c c\~ao, junto do lema de associatividade,
temos  
  $$
    n + (k + 1) = (n + k) + 1 = (k + n) + 1 = 1 + (k + n) = (1 + k) + n = (k + 1) + n.
  $$
  Por indu\c c\~ao, segue que isso vale para todo natural n. \qedsymbol
\end{proof*}

\begin{def*}
  Definimos uma ordem em $\mathbb{N}$ colocando que $m\leq{n}$ se existe p natural tal que $n = m + p. \square$
\end{def*}
  A rela\c c\~ao de ordem possui as seguintes propriedades:
 \begin{itemize}
   \item[i)] Reflexiva: Para todo n natural, $n\leq{n};$
   \item[ii)] Antissim\'etrica: Se $m\leq n$ e $n\leq m,$ ent\~ao $m = n;$
   \item[iii)] Transitiva: Se $m \leq n$ e $n \leq p$, ent\~ao $m\leq p;$
   \item[i] Dados m, n naturais, temos ou $m \leq n$, ou $n \leq m;$
   \item[v] Se $m \leq n$ e p \'e um natural, ent\~ao $n + p\leq n\text{ e } mp\leq np$
 \end{itemize}

 \subsection{N\'umeros Inteiros e Racionais}
  Usualmente, construimos os inteiros a partir dos naturais tomando os pares ordenados de n\'umeros naturais
com a seguinte identifica\c c\~ao (a, b) $\mathtt{\sim}$ (c, d) se a + d = b + c. Assim, podemos representar
  $$
  \mathbb{N} = \{(0, 0), (1, 0), (2, 0), (3, 0), \ldots\}, \quad -\mathbb{N}^* = \{\cdots, (0, 3), (0, 2), (0, 1)\}.
  $$
  Tomar o sucessor ser\'a somar 1 \`a primeira coordenada e, para os inteiros negativos, voltar a identificar (1, n) com (0, n-1).

  Os n\'umeros racionais s\~ao constru\'idos tomando o conjunto $\mathbb{Z}\times{\mathbb{Z}^*}$ e identificando os pares $(a, b)\mathtt{\sim}(c, d)$
para os quais ad = bc. Representamos um par (a, b) neste conjunto por $\displaystyle \frac{a}{b}.$ A soma e o produto em $\mathbb{Q}$
s\~ao dados, respectivamente, por: 
 \begin{align*}
   &\frac{a}{b} + \frac{c}{d} \coloneqq \frac{ad + bc}{bd} \\
   &\frac{a}{b}\cdot\frac{c}{d} \coloneqq \frac{ac}{bd}.
 \end{align*}
 Chamamos a adi\c c\~ao a opera\c c\~ao que a cada par $(x, y)\in \mathbb{Q}\times{\mathbb{Q}}$ associa sua soma $x+y\in \mathbb{Q}$
e chamamos multiplica\c c\~ao a opera\c c\~ao que a cada par $(x, y)\in \mathbb{Q}\times \mathbb{Q}$ associa seu produto $x.y\in \mathbb{Q}.$
A terna $(\mathbb{Q}, +, \cdot)$ satisfaz as propriedades de um corpo, i.e., 
 \begin{align*}
   &(A1) (x + y) + z = x + (y + z), \quad\forall x, y, z\in \mathbb{Q}\\
   &(A2) x + y = y + x, \quad\forall x, y\in \mathbb{Q}\\
   &(A3) \exists 0\in \mathbb{Q}: x + 0 = x, \quad\forall x\in \mathbb{Q}\\
   &(A4) \forall x\in \mathbb{Q}, \exists y\in \mathbb{Q} (y = -x): x + y = 0\\
   &(M1) (xy)z = x(yz), \quad\forall x, y, z\in \mathbb{Q}\\
   &(M2) xy = yx, \quad x, y\in \mathbb{Q}\\
   &(M3) \exists 1\in \mathbb{Q}: 1.x = x.1 = x, \quad\forall x\in \mathbb{Q}\\
   &(M4) \forall x\in \mathbb{Q}^*, \exists y = \frac{1}{x}\in \mathbb{Q}: x.y = 1\\
   &(D) x(y+z) = xy + xz,\quad\forall x, y, z\in \mathbb{Q}.
 \end{align*}
 \newpage

\section{Aula 02 - 15/03/2023}
\subsection{Motiva\c c\~oes}
\begin{itemize}
  \item Propriedades b\'asicas dos racionais;
  \item Constru\c c\~ao do corpo dos reais a partir dos racionais;
  \item Cortes de Dedekind.
\end{itemize}
\subsection{Propriedades de $\mathbb{Q}$ e sua Ordem}
  Com as 9 propriedades de corpo, conseguimos obter novas regras nos racionais, como a famosa lei do cancelamento:
 \begin{prop*}
   Em $\mathbb{Q},$ vale
   $$
    x + z = y + z \Rightarrow x = y
   $$
   e, se $z\neq{0}$,
   $$
    xz = yz \Rightarrow x = y 
   $$
 \end{prop*}
 \begin{proof*}
   \begin{align*}
   &x = x + 0 = x + (z + (-z)) = (x+z) + (-z) = (y + z) + (-z) = y + (z + (-z)) = y +0 = y\\
   &x = x.1 = x(z.\frac{1}{z}) = (xz)\frac{1}{z} = (yz)\frac{1}{z} = y (z \frac{1}{z}) = y.1 = y. \text{ \qedsymbol}
 \end{align*}
 \end{proof*}
 \begin{prop*}
  Os elementos neutros da adi\c c\~ao e multiplica\c c\~ao s\~ao \'unicos. Os elementos oposto e inverso tamb\'em o s\~ao.
 \end{prop*}
 \begin{prop*}
   Para todo x racional, x.0 = 0.
 \end{prop*}
 \begin{prop*}
   Para todo x racional, -x = (-1)x.
 \end{prop*}
 A maioria desses resultados acima seguem diretamente da lei do cancelamento. Suas demonstra\c c\~oes ficam como exerc\'icio.
\begin{def*}
  Diremos que 
  $$
    \frac{a}{b}\in \mathbb{Q} = \left\{\begin{array}{ll}
        \text{ n\~ao-negativo, } \quad ab\in \mathbb{N}\\
        \text{ positivo, } \quad ab\in \mathbb{N}, a\neq 0
      \end{array}\right.
  $$
  e diremos que 
  $$
    \frac{a}{b}\in \mathbb{Q} = \left\{\begin{array}{ll}
        \text{ n\~ao-positivo, } \quad \frac{a}{b} \text{ n\~ao for postivo}\\
        \text{ negativo, } \quad \frac{a}{b} \text{ n\~ao for n\~ao-negativo.}
      \end{array}\right.\square
  $$
\end{def*}
\begin{def*}
  Sejam x, y racionais. Diremos que x \'e menor e que y e escrevemos ``$x < y$'' se existir t racional positivo tal que 
  $$
    y = x + t.
  $$
  Neste mesmo caso, podemos dizer que y \'e maior que x, escrevendo ``$x > y$''. Em particular, temos $x > 0$ se x for positivo e 
$x < 0$ se x for negativo. 

  Ademais, se $x < y$ ou x = y, escrevemos ``$x \leq{y}$'' se existir racional t n\~ao-negativo tal que 
  $$
    y = x + t
  $$
  e, se $x > y$ ou x = y, escrevemos ``$x \geq{y}$'' caso exista racional t n\~ao-positivo com
  $$
    y = x + t. \square
  $$
\end{def*}
  A qu\'adrupla ($\mathbb{Q}, +, \cdot, \leq{}$) satisfaz as propriedades de um corpo ordenado, i.e., 
 \begin{align*}
   &(O1) x \leq{x}\forall x\in \mathbb{Q};\\
   &(O2) x\leq{y} \text{ e } y \leq{x}\Rightarrow x = y \forall x, y\in \mathbb{Q};\\
   &(O3) x \leq{y}, y \leq{z}\Rightarrow x \leq{z}\forall x, y, z\in \mathbb{Q};\\
   &(O4)\forall x, y \in \mathbb{Q}, x \leq{y} \text{ ou } y \leq{x};\\
   &(OA) x \leq{y}\Rightarrow x + z \leq{y + z};\\
   &(OM) x \leq{y} \text{ e } z \geq{0}\Rightarrow xz \leq{yz}. 
 \end{align*}
 \begin{prop*}
  Para quaisquer x, y, z, w no corpo ordenado dos racionais, valem 
 \begin{align*}
   &i.) x < y\Longleftrightarrow x + z < y + z \\
   &ii.) z > 0\Longleftrightarrow \frac{1}{z} > 0 \\
   &iii.) z > 0\Longleftrightarrow -z < 0 \\
   &iv.) z > 0\Rightarrow x < y\Longleftrightarrow xz < yz \\
   &v.) z < 0\Rightarrow x < y\Longleftrightarrow xz > yz \\
   &vi.) xz < yw\Longleftrightarrow \left\{\begin{array}{ll}
       0 \leq{x} < y \\
       0 \leq{z} < w
     \end{array}\right. \\
   &vii.) 0 < x < y\Longleftrightarrow 0 < \frac{1}{y} < \frac{1}{x}\\
   &viii.) x < y \text{ ou } x =y \text{ ou } x > y \\
   &ix.) xy = 0\Longleftrightarrow x = 0\text{ ou }y = 0. \\
   &x.) \left.\begin{array}{ll}
       x \leq{y} \\
       z \leq{w}
     \end{array}\right\}\Rightarrow x + z \leq{y + w}\\
   &xi.) \left.\begin{array}{ll}
       0 \leq{x} \leq{y}\\
       0 \leq{z} \leq{w}
   \end{array}\right\}\Rightarrow xz \leq{yw}.
 \end{align*}
 \end{prop*}
 \subsection{Incompletude de $\mathbb{Q}$}
  Os n\'umeros racionais podem ser representados por pontos em uma reta horizontal ordenada, chamada reta real. Se P for a representa\c c\~ao
de um n\'umero racional x, diremos que x \'e a abscissa de P. Note que nem todo ponto da reta real \'e racional. Para isso, considere
um quadrado de lado 1 e diagonal d. Pelo Teorema de Pit\'agoras, $d^{2} = 1^2 + 1^2 = 2.$ Agora, seja P a intersec\c c\~ao do eixo
x com a circunfer\^encia de centro em 0 e raio d. Mostremos que P \'e um ponto da reta com abscissa $x\not\in \mathbb{Q}.$
\begin{prop*}
  Seja a um inteiro. Ent\~ao, se a for \'impar, seu quadrado tamb\'em ser\'a. Al\'em disso, se a for par, seu quadrado tamb\'em \'e par.
\end{prop*}
\begin{prop*}
  A equa\c c\~ao $x^2 = 2$ n\~ao admite solu\c c\~ao racional.
\end{prop*}
  A ideia da prova \'e escrever um x na forma de fra\c c\~ao e chegar na contradi\c c\~ao de que tanto o numerador quanto o denominador
ser\~ao n\'umeros pares. Com isso, conclui-se que n\~ao existe racional irredut\'ivel com quadrado igual a 2, portanto n\~ao existe racional
satisfazendo a equa\c c\~ao. 

  Essa discuss\~ao mostra que existem v\~aos na ``reta'' dos racionais, requerindo a ado\c c\~ao de um novo corpo. Essa \'e a principal
motiva\c c\~ao por tr\'as dos n\'umeros reais, "preencher" os buracos deixados pelos racionais.
\begin{prop*}
  (Exerc\'icio.) Sejam $p_{1}, \ldots, p_{n}$ n\'umeros primos distintos. Ent\~ao, a equa\c c\~ao $x^{2} = p_{1}p_2\cdots p_{n}$ n\~ao
tem solu\c c\~ao racional.
\end{prop*}
  Vimos que os n\'umeros racionais com a sua adi\c c\~ao, multiplica\c c\~ao e rela\c c\~ao de ordem \'e um corpo ordenado. Nos interessamos,
tamb\'em, pelo corpo dos reais e dos racionais ($\mathbb{R}, \mathbb{C}$). De forma abstrata, um corpo \'e um conjunto n\~ao-vazio
 $\mathbb{F}$ em que est\~ao definidas duas opera\c c\~oes bin\'arias
 $$
    +:\mathbb{F}\times \mathbb{F}\rightarrow \mathbb{F}, \quad (x, y)\mapsto x + y
 $$
 e 
 $$
    \cdot: \mathbb{F}\times \mathbb{F}\rightarrow \mathbb{F}, \quad (x, y)\mapsto xy
 $$
 em que valem as oito propriedades vistas previamente para a defini\c c\~ao das opera\c c\~oes em $\mathbb{Q}$
 Se, ainda por cima, no corpo $\mathbb{F}$ est\'a definida uma ordem com propriedades an\'alogas \`as vistas para a qu\'adrupla
($\mathbb{Q}, +, \cdot, \leq{}$), diremos que ($\mathbb{F}, +, \cdot, \leq{}$) \'e um corpo ordenado.
\begin{def*}
  Diremos que um subconjunto A de um corpo $\mathbb{F}$ ordenado \'e limitado superiormente se existe um L neste corpo tal que $a \leq{L}$ para todo
a de A. 

  Definimos para um subconjunto limitado superiormente um n\'umero $\sup(A)\in \mathbb{F}$ como o menor limitante superior de
A, i.e., se $a \leq{\sup(A)}$ para todo a de A e se existe $f\in \mathbb{F}$ com $f < \sup(A),$ ent\~ao existe um a em A com $ f < a.$

  Por fim, diremos que um corpo ordenado \'e completo se todo subconjunto limitado superiormente possui supremo. $\square$
\end{def*}
  Nem todo subconjunto limitado superiormente de $\mathbb{Q}$ tem supremo, ou seja, $\mathbb{Q}$ n\~ao \'e completo.

\subsection{Os N\'umeros Reais ($\mathbb{R}$)}
  A ideia que iremos usar para construir o conjunto dos reais \'e que o conjunto dos n\'umeros reais preenche toda a reta real. Os Elementos
de $\mathbb{R}$ ser\~ao os subconjuntos de $\mathbb{Q}$ \`a esquerda de um ponto da reta real e ser\~ao chamados de cortes.
 \begin{def*}
   Um corte \'e um subconjunto $\alpha\subsetneq \mathbb{Q}$ com as seguintes propriedades:
  \begin{itemize}
    \item[i)] $\alpha\neq \emptyset$ e $\alpha \neq \mathbb{Q};$
    \item[ii)] Se $p\in \alpha$ e q \'e um racional com $q < p$, ent\~ao $q\in \alpha$ (todos os racionais \`a esquerda de um elemento
      de $\alpha$ est\~ao em $\alpha$);
    \item[iii)] Se $p\in \alpha$, existe um $r\in \alpha$ com $p < r$ ($\alpha$ n\~ao tem um maior elemento). $\square$
  \end{itemize}
 \end{def*}
  Essa ideia foi proposta inicialmente por Julius Wilhelm Richard Dedekind, um matem\'atico alem\~ao, em 1872, com o objetivo de encontrar uma explica\c c\~ao
e constru\c c\~ao elementar para os n\'umeros reais.
\begin{example}
  Se q \'e um racional, definimos $q^{*} = \{r\in \mathbb{Q}: r < q\}$.Ent\~ao, $q^{*}$ \'e um corte que chamamos de racional. Os 
cortes que n\~ao s\~ao desse tipo se chamam cortes irracionais.
\end{example}
\begin{example}
  $\sqrt{2} = \{q\in \mathbb{Q}: q^{2} < 2\}\cup \{q\in \mathbb{Q}: q < 0\}$ \'e um corte irracional.
\end{example}
  Observe que se $\alpha$ \'e um corte, p \'e um ponto dele e q n\~ao \'e, ent\~ao $p < q$. Al\'em disso, se r pertence a $\alpha$
e $r < s,$ ent\~ao s n\~ao pertence ao corte.
\begin{def*}
  Diremos que $\alpha < \beta$, em que $\alpha$ e $\beta$ s\~ao cortes, se $\alpha\subsetneq \beta.\square$
\end{def*}
\begin{prop*}
  Se $\alpha, \beta, \gamma$ s\~ao cortes,
 \begin{itemize}
   \item[i)] $\alpha < \beta$ e $\beta < \gamma$ implica que $\alpha < \gamma$;
   \item[ii)] Exatamente uma das seguintes rela\c c\~oes \'e v\'alida: $\alpha < \beta$ ou $\alpha = \beta$ ou $\beta < \alpha$
   \item[iii)] Todo subconjunto n\~ao-vazio e limitado superiormente de $\mathbb{R}$ tem supremo.
 \end{itemize}
\end{prop*}
\newpage

\section{Aula 03 - 17/03/2023}
\subsection{Motiva\c c\~oes}
\begin{itemize}
  \item Finalizar a constru\c c\~ao de $\mathbb{R}$ por cortes;
  \item Definir um corpo ordenado com base nos cortes;
\end{itemize}
\subsection{Cortes - Soma e Ordem}
  Coloquemos, para fins de conveni\^encia, $\mathbb{R}$ como a uni\~ao de todos os cortes.

  Vamos mostrar que os cortes racionais s\~ao, de fato, cortes. Considere, dado um racional q, $q^{*} = \{p\in \mathbb{Q}: p < q\}.$
Ele n\~ao pode completar todos os racionais, pois q + 1 n\~ao pertence a $q^{*}$. Al\'em disso, ele \'e n\~ao vazio, visto que
q-1 pertence a ele, mostrando a primeira propriedade dos cortes. 

  Ademais, se r pertence a $q^{*}$ e p \'e um racional menor que r, segue da transitividade da ordem que p \'e menor que
q, j\'a que r tamb\'em \'e. Assim, p pertence a $q^{*}$, mostrando a segunda propriedade dos cortes. 

  Por fim, dado um r em $q^{*},$ seja $s = \displaystyle \frac{r + q}{2}$. Ent\~ao, 
  $$
    r - \frac{r+q}{2} = \frac{r - q}{2} < 0,
  $$
tal que s \'e menor que r e, logo, pertence a $q^{*}$. Portanto, $q^{*}$ forma um corte.

  Daremos continuidade \`as atividades da aula anterior demonstrando a \'ultima proposi\c c\~ao vista.
\begin{prop*}
  Se $\alpha, \beta, \gamma$ s\~ao cortes,
 \begin{itemize}
   \item[i)] $\alpha < \beta$ e $\beta < \gamma$ implica que $\alpha < \gamma$;
   \item[ii)] Exatamente uma das seguintes rela\c c\~oes \'e v\'alida: $\alpha < \beta$ ou $\alpha = \beta$ ou $\beta < \alpha$
   \item[iii)] Todo subconjunto n\~ao-vazio e limitado superiormente de $\mathbb{R}$ tem supremo.
 \end{itemize}
\end{prop*}
  
 \begin{proof*}
   As duas primeiras partes seguem automaticamente da forma que definimos a ordem $\leq{}$ para os cortes. Resta mostrar a \'ultima.

   Vamos exibir o supremo explicitamente. Com efeito, seja $\mathcal{A}\subseteq{\mathbb{R}}$ um cole\c c\~ao de cortes
 limitada superiormente, i.e., existe um l em $\mathbb{R}$ tal que $\alpha \leq{l}$ para todo $\alpha$ em $\mathcal{A}.$
 Defina $\mathcal{S} = \bigcup_{\alpha\in \mathcal{A}}\alpha$. Mostremos que $\mathcal{S}$ \'e um corte. Com efeito,
 que $\mathcal{S}$ \'e n\~ao-vazio e diferente de $\mathbb{Q}$ \'e autom\'atico. Al\'em disso, dado q em $\mathcal{S}$ e $r < q,$
segue que $r\in \alpha_{0}$ para algum $\alpha_{0}$ em $\mathcal{A}.$

  Para ver que $\mathcal{S}$ \'e o supremo, suponha que $\mathcal{S}' < \mathcal{S}.$ Ent\~ao, existe r em $\mathcal{S}/\mathcal{S}'$.
Como r pertence a $\mathcal{S}$, r pertence a $\alpha_{0}$ para algum $\alpha_{0}\in \mathcal{A}.$ Logo, $\alpha_{0} > \mathcal{S}'.$ Portanto,
 $\mathcal{S}$ \'e o menor limitante superior de $\mathcal{A},$ ou seja, seu supremo. \qedsymbol
 \end{proof*}
 \begin{def*}
   Se $\alpha, \beta$ s\~ao cortes, definimos $\alpha + \beta$ como o conjunto de todos os racionais da forma $r + s$, com r em $\alpha$
 e s em $\beta$. Ademais, tome $0^* = \{s\in \mathbb{Q}: s < 0\}.\square$
 \end{def*}
  Vamos conferir a defini\c c\~ao, i.e., que $\alpha + \beta$ \'e um corte. Com efeito, $\alpha + \beta\neq\emptyset$, pois $\alpha\neq\emptyset$
e $\beta\neq\emptyset$. Al\'em disso, se p n\~ao pertence a $\alpha$ e q n\~ao pertence a $\beta$, mas r pertence a $\alpha$ e s a $\beta$,
ent\~ao $r + s < p + q$, tal que $p + q$ n\~ao pertence a $\alpha + \beta.$

  Al\'em disso, tome $r + s$ em $\alpha + \beta$ e $p < r + s$. Escreva $p = r' + s' = \underbrace{p - r}_{\in \beta} + \underbrace{r}_{\in \alpha}.$ Assim,
p pertence a $\alpha + \beta.$

  Por fim, tome $r + s$ em $\alpha + \beta$ e seja $r' > r$ (ambos em $\alpha$). Logo, $\underbrace{r' + s}_{\in \alpha + \beta} > r + s$. Portanto, 
$\alpha + \beta$ \'e um corte. 

  Fica de exerc\'icio mostrar que $0^*$ \'e um corte. Agora, mostremos os axiomas de corpo.

  A comutatividade e associatividade da adi\c c\~ao s\~ao triviais. Al\'em disso, dado r em $\alpha$ e s em $0^*,$
  $$
    r + s < r + 0 = r\Rightarrow r + s \in \alpha.
  $$
  Logo, $\alpha + 0^* \subseteq{\alpha}$. Por outro lado, dado r em $\alpha,$ existe $r'\text{ em } \alpha$ tal que $r' > r.$
  Assim, $r = \underbrace{r'}_{\alpha} + \underbrace{(r - r')}_{\in 0^*}$, pois $r - r' < 0.$ Portanto, $\alpha \subseteq{\alpha + 0^*}$ e
 $\alpha = \alpha + 0^*.$
 \begin{prop*}
   Dado um corte $\alpha$, existe um \'unico corte $\beta$ tal que $\alpha + \beta = 0^*$, em que 
   $$
    \beta = \{-p\in \mathbb{Q}: p - r\not\in \alpha \text{ para algum } r\in \mathbb{Q}, r > 0\}
   $$
   e \'e denotado por $-\alpha$. 
 \end{prop*}
\begin{proof*}
  Come\c camos mostrando que $\beta$ \'e um corte. Feito isso, vamos mostrar que $\beta + \alpha = 0^*.$

  Com efeito, dado -p em $\beta,$ segue que p n\~ao pertence a $\beta$. Caso $s = p + r$, -s pertence a $\beta$, tal que 
 $\beta$ \'e n\~ao-vazio. Ademais, se $p\in \alpha, -p\not\in \beta$, tal que $\beta$ \'e diferente de $\mathbb{Q}.$

  Al\'em disso, se $-q < -p$ e $-p\in \beta$, ent\~ao $-q \in \beta$. Por fim, se -p pertencer a $\beta$, $\displaystyle -p + \frac{r}{2}\in \beta$.
Portanto, $\beta$ \'e um corte. 

  Agora, vamos conferir o outro item. De fato, se r pertence a $\alpha$ e s a $-\alpha,$ ent\~ao $-s\not\in \alpha$ e $r < -s$, i.e.,
 $r + s < 0.$ Segue que $\alpha + (-\alpha) \subseteq{0^*}.$ Por outro lado, se $-2r\in 0^*$ com $r > 0,$ existe um inteiro n tal que
  $nr\in \alpha$ e $(n+1)r\not\in \alpha$. Escolha $p = -(n+2)r\in -\alpha$ e escreva $-2r = nr + p.$ Portanto, $0^*\subseteq{\alpha + (-\alpha)}$ e os
  conjuntos s\~ao iguais. \qedsymbol
\end{proof*}

\section{Aula 04 - 20/02/2023}
\subsection{Motiva\c c\~oes}
\begin{itemize}
  \item Definir multiplica\c c\~ao de cortes;
  \item Definir conceito de dist\^ancia entre n\'umeros de $\mathbb{R}$ 
\end{itemize}
\subsection{Cortes - Multiplica\c c\~ao}
\begin{def*}
  Se $\alpha, \beta$ s\~ao cortes,
  $$
    \alpha\beta = \left\{\begin{array}{ll}
      \alpha0^*, \quad\forall \alpha\in \mathbb{R}\\
      \{p\in\mathbb{Q}: \exists0 < r\in\alpha \text{ e }0 < s\in\alpha: p \leq rs\},\quad \alpha, \beta > 0^*\\
      (-\alpha)(-\beta), \quad \alpha, \beta < 0^*\\
      -[(-\alpha)\beta], \quad \alpha < 0^* e \beta > 0^*\\
      -[\alpha(-\beta)], \quad \alpha > 0^* e \beta < 0^*
      \end{array}\right.
  $$
  Definimos, tamb\'em, $1^*\{s\in\mathbb{Q}: s < 1\}$.
\end{def*}
\newpage

\subsection{$\mathbb{R}$ Como Corpo Ordenado Completo}
  Temos $\mathbb{Q}\subseteq{\mathbb{R}}$ e diremos que todo n\'umero que n\~ao \'e real \'e irracional. 
 \begin{theorem*}
   A qu\'adrupla $(\mathbb{R}, +, \cdot, \leq)$ satisfaz as condi\c c\~oes de corpo ordenado, de corpo e \'e completo.
 \end{theorem*}
\begin{def*}
  Seja $x\in \mathbb{R}.$ O m\'odulo, ou valor absoluto de x, \'e dado por
  $$
    |x| = \left\{\begin{array}{ll}
        x, \quad x \geq 0\\
        -x, x < 0
      \end{array}\right.
  $$
  Disto segue que $|x|\geq 0$ e $-|x|\leq x\leq |x|$ para todo x real.
\end{def*}
\begin{example}
  Mostre que $|x|^2 = x^2$, ou seja, o quadrado de um n\'umero real n\~ao muda quando se troca seu sinal.
\end{example}
\begin{example}
  A equa\c c\~ao $|x| = r$, com r maior que 0, tem como solu\c c\~oes apenas r e -r.
\end{example}
  Sejam P e Q dois pontos da reta real de abscissas x e y. Ent\~ao, a dist\^ancia de P a Q \'e definida por $|x-y|$. Assim,
 $|x-y|$ \'e a medida do segmento PQ. Em particular, como $|x|=|x-0|, |x|$ \'e a dist\^ancia de x a 0.
\begin{example}
  Seja r maior que 0. Ent\~ao, $|x| < r$ se, e somente se, $-r < x < r.$ Logo, o intervalo (-r, r) \'e o conjunto dos pontos reais
cuja dis\^ancia de 0 \'e menor que r.
\end{example}
\begin{example}
  Para quaisquer x, y reais, vale
  $$
    |xy| = |x||y|.
  $$
\end{example}
\begin{example}
  Para quaisquer x, y reais, temos
  $$
    |x+y| \leq |x| + |y|.
  $$
  Com efeito, somando $-|x|\leq{x}\leq{|x|}$ e $-|y|\leq{y}\leq{|y|},$ obtemos $-|x|-|y|\leq{x + y}\leq{|x|+|y|.}$\qedsymbol
\end{example}
\begin{def*}
  Um intervalo em $\mathbb{R}$ \'e um subconjunto de $\mathbb{R}$ que tem uma das seguintes formas:
 \begin{align*}
   &[a, b] = \{x\in \mathbb{R}: a\leq{x}\leq{b}\},\quad \text{ (Intervalo fechado.) }\\
   &(a, b) = \{x\in \mathbb{R}: a < x < b\},\quad \text{ (Intervalo aberto.) }\\
   &[a, b) = \{x\in \mathbb{R}: a \leq{x} < b\}\\
   &(a, b] = \{x\in \mathbb{R}: a < x \leq{b}\}\\
   &(-\infty, b] = \{x\in \mathbb{R}: x\leq{b}\}\\
   &(-\infty, b) = \{x\in \mathbb{R}: x > b\}\\
   &[a, +\infty) = \{x\in \mathbb{R}: x\geq{a}\}\\
   &(a, +\infty) = \{x\in \mathbb{R}: a < x\}\\
   &(-\infty, +\infty) = \mathbb{R}.
 \end{align*}
\end{def*}
\begin{def*}
  Um conjunto A de $\mathbb{R}$ \'e dito limitado se existir L positivo tal que $|x| \leq L$ para todo x em A.
\end{def*}
\begin{prop*}
  Um conjunto A de $\mathbb{R}$ \'e limitado se, e s\'o se, existir L positivo, tal que A est\'a contido em $[-L, L]$
\end{prop*}
\begin{example}
 \begin{itemize}
   \item[a)] $A = [0, 1]$ \'e limitado;
   \item[b)] $\mathbb{N}$ n\~ao \'e limitado;
   \item[c)] $B = \biggl\{\displaystyle \frac{2^n-1}{2^n}: n\in \mathbb{N}\biggr\}$ \'e limitado; 
   \item[d)] $C = \biggl\{\displaystyle \frac{2^n-1}{2^n}: n\in \mathbb{N}^{*}\biggr\}$ \'e limitado. 
 \end{itemize}
\end{example}
\begin{def*}
  Seja $A\subseteq{\mathbb{R}}$.
 \begin{itemize}
   \item A ser\'a dito limitado superiormente se existir um L real tal que $x\leq L$ para todo x de A. Diremos que L \'e o limitante superior de A.;
   \item A ser\'a dito limitado inferiormente se existir um L real tal que $x\geq L$ para todo x de A. Diremos que L \'e o limitante inferior de A.;
 \end{itemize}
 Caso ambos ocorram, diremos que A \'e limitado.
\end{def*}
\begin{def*}
  Seja A um subconjunto dos reais limitado superiormente e n\~ao-vazio. Diremos que $\overline{L}$ \'e o supremo de A se for um limitante superior
  e para qualquer outro limitante superior L de A, tivermos $\overline{L}\leq L$. Quando o supremo pertencer ao conjunto, chamaremos ele de m\'aximo.
\end{def*}
  Vimos que todo subconjunto n\~ao-vazio e limitado superiormente de $\mathbb{R}$ tem supremo.
\begin{def*}
  Seja A um subconjunto dos reais limitado inferiormente e n\~ao-vazio. Diremos que $\overline{l}$ \'e o \'infimo de A se for um limitante inferior
  e para qualquer outro limitante inferior l de A, tivermos $\overline{l}\geq l$. Quando o \'infimo pertencer ao conjunto, chamaremos ele de m\'inimo.
\end{def*}
\begin{prop*}
  Dado um subconjunto A dos reais n\~ao-vazio e limitado superiormente, $L = \sup{A}$ se, e somente se, 
 \begin{itemize}
   \item[a)] L for limitante superior de A;
   \item[b)] para todo $\epsilon > 0$, existe $a\in A$ tal que $a > L - \epsilon.$ 
 \end{itemize}
\end{prop*}
\begin{theorem*}
  O conjunto $A=\{nx: n\in\mathbb{N}\}$ ser\'a ilimitado para todo x n\~ao-nulo.
\end{theorem*}
\begin{proof*}
  Se $x > 0$, suponhamos, por absurdo, que A seja limitado e seja L seu supremo. Como $x > 0$, deve existir um natural m tal que
  $$
    L - x < mx \quad\text{ e } L = \sup{A} < (m+1)x.
  $$
  Mas isso \'e uma contradi\c c\~ao. 

  A prova para $x < 0$ \'e an\'aloga e ser\'a deixada como exerc\'icio. \qedsymbol 
\end{proof*}
 \begin{example}
  \begin{itemize}
    \item[a)] Considere $A = [0, 1).$ Ent\~ao, -2 e 0 s\~ao limitantes inferiores de A enquanto $1, \pi, 101$ s\~ao limitantes
  superiores de A.
    \item[b)] $\mathbb{N}$ n\~ao \'e limitado, mas \'e limitado inferiormente por 0, visto que $0\leq{x}$ para todo x natural.
    \item[c)] $B=\{x\in \mathbb{Q}: x\leq{\sqrt{2}}\}$ n\~ao \'e limitado, mas \'e limitado superiormente por L, em que $L\geq{2}.$ \qedsymbol
  \end{itemize}
 \end{example}
\begin{crl*}
  Para todo $\epsilon > 0$, existe um n natural tal que 
  $$
  \frac{1}{n} < \epsilon, \quad \frac{1}{n\sqrt{2}}<\epsilon, \quad 2^{-n} < \epsilon.
  $$
\end{crl*}
  J\'a sabemos, por constru\c c\~ao, que entre dois n\'umeros reais distintos existe um n\'umero racional. O mesmo vale para irracionais.
De fato, sejam a e b n\'umeros reais distintos. Se $a < b$ e $\epsilon = b - a > 0$, do corol\'ario, tome um natural n tal que
$\displaystyle \frac{1}{n\sqrt{2}} < \frac{1}{n} < \epsilon.$ Se a \'e racional, $r = \displaystyle a + \frac{1}{n\sqrt{2}}$ \'e irracional e
 $a < r < b.$ Por outro lado, se a \'e irracional, $r =\displaystyle a + \frac{1}{n}$ tamb\'em \'e, tal que $a < r < b.$ Portanto,
 dados dois n\'umeros reais quaisquer, existe um n\'umero irracional.
\begin{crl*}
  Qualquer intervalo aberto e n\~ao-vazio cont\'em infinitos n\'umeros racionais e infinitos irracionais.
\end{crl*}
\begin{crl*}
  Se $A = \biggr\{\displaystyle \frac{1}{n}: n\in \mathbb{N}^*\biggl\}$, ent\~ao $\inf A = 0.$
\end{crl*}
\begin{example}
 \begin{align*}
   &(a) \text{ Seja }A = (0, 1]. \text{ Ent\~ao, } \inf{A} = 0, \max{A} = 1; \\
   &(b) \sqrt{2} = \{r\in\mathbb{Q}: r \leq 0\}\cup \{r\in\mathbb{Q}: r^2 < 2\}\text{ \'e um corte.} 
   &(c) C = \{x\in\mathbb{Q}: x^2 < 2\} \Rightarrow \sqrt{2}=\sup{C}\text{ e }\inf{C} = -\sqrt{2}.
 \end{align*}
 Vamos analisar mais cautelosamente o item b e prov\'a-lo. De fato, se $0 < r\in \mathbb{Q}$ e $r^2 < 2,$ existe n natural tal que 
 $[2r + 1]\frac{1}{n} < 2 - r^2$ e $(r + \frac{1}{n})^2 < 2.$ As outras propriedades de cortes s\~ao triviais. 

 Olhando tamb\'em para o item C, como todos seus elementos s\~ao racionais saitsfazendo $x^2 < 2, \sqrt{2}$\'e um limitante superior de C.
 Agora, se $0 < L < \sqrt{2}$, existe um racional $r\in(L, \sqrt{2})$ e $L^2 < r^2 < 2.$ Logo, r pertence a C e L n\~ao \'e limitante superior para C,
 provando o resultado.
\end{example}
\begin{prop*}
  Se A \'e um subconjunto n\~ao-vazio e limitado inferiormente, ent\~ao $-A = \{-x: x\in A\}$ ser\'a limitado superiormente e 
  $\inf{A} = -\sup{(-A)}$. Analogamente, se for limitado superiormente, o conjunto -A ser\'a limitado inferiormente, e $\sup{A}=-\inf{(-A)}$
\end{prop*}
\begin{proof*}
  Se A for limitado inferiormente, $\inf{(A)} \leq{x}$ para todo x de A e, dado $\epsilon > 0$, deve existir a em A tal que
  $a < \inf{(A)} + \epsilon$, ou, trocando o sinal, $-\inf{(A)} \geq{-x}$ para todo -x de -A e, dado $\epsilon > 0,$ deve existir
  $b = -a$ em -A tal que $-a > -\inf{(A)} - \epsilon.$

  Com isso, segue que -A ser\'a limitado superiormente, e $\sup{(-A)} = -\inf{(A)}.$ A outra prova fica como exerc\'icio. \qedsymbol
\end{proof*}
\begin{crl*}
  Todo conjunto A n\~ao-vazio e limitado inferiormente de $\mathbb{R}$ tem \'infimo.
\end{crl*}
\begin{crl*}
  Todo conjunto A n\~ao-vazio e limitado de $\mathbb{R}$ tem \'infimo e supremo.
\end{crl*}
\begin{def*}
  Uma vizinhan\c ca de um n\'umero real a \'e qualquer intervalo aberto da reta contendo a.
\end{def*}
\begin{example}
  Se $\delta > 0, V_{\delta}(a)\coloneqq(a - \delta, a + \delta)$ \'e uma vizinhan\c ca de a que ser\'a chamada de $\delta-$vizinhan\c ca de a. 
\end{example}
\begin{def*}
  Sejam A um subconjunto de $\mathbb{R}$ e b um n\'umero real. Se, para todo $\delta > 0$, existir $a\in V_{\delta}(b)\cap{A}, a\neq b,$ 
ent\~ao b ser\'a dito um ponto de acumula\c c\~ao de A.
\end{def*}
\begin{example}
 \begin{itemize}
   \item[a)]O conjunto dos pontos de acumula\c c\~ao de (a, b) \'e [a, b];
    \item[b)]Seja $B = \mathbb{Z}.$ Ent\~ao, B n\~ao tem pontos de acumula\c c\~ao; 
      \item[c)] Subconjuntos finitos de $\mathbb{R}$ n\~ao t\^em pontos de acumula\c c\~ao;
        \item[d)] O conjunto dos pontos de acumula\c c\~ao de $\mathbb{Q}$ \'e $\mathbb{R}$.
 \end{itemize}
\end{example}
\begin{def*}
  Seja $B\subseteq{\mathbb{R}}$. Um ponto b de B ser\'a dito um ponto isolado de B, se existir $\delta > 0$ tal que $V_{\delta}(b)$
n\~ao cont\'em pontos de B distintos de b. $\square$
\end{def*}
\begin{example}
  Seja $B=\{1, \frac{1}{2}, \frac{1}{3}, \cdots\}$. Ent\~ao, o conjunto dos pontos de acumula\c c\~ao de B \'e $\{0\}$ e o conjunto dos pontos
isolados de B \'e o pr\'oprio conjunto B.
\end{example}
  Observe que existem conjuntos infinitos sem pontos de acumula\c c\~ao, tal como $\mathbb{Z}.$ Por outro lado, todo conjunto infinito e limitado possui
pelo menos um ponto de acumula\c c\~ao.
\begin{theorem*}
  Se A \'e um subconjunto infinito e limitado de $\mathbb{R},$ ent\~ao A possui pelo menos um ponto de acumula\c c\~ao.
\end{theorem*}
\begin{proof*}
  Se $A\subseteq{[-L, L]}$ e $[a_{n}, b_{n}], n\in \mathbb{N}$ s\~ao escolhidos tais que $[a_{n+1}, b_{n+1}]\subseteq{[a_{n}, b_{n}]}, b_{0} = -a_{0} = L,
  b_{n} - a_{n} = \frac{2L}{2^{n}}, n\in \mathbb{N}^*$ e $[a_{n}, b_{n}]$ cont\'em infinitos elementos de A. Seja $a = \sup{\{a_{n}: n\in \mathbb{N}\}}$.

  Note que $[a_{n}, b_{n}]\subseteq{a_{j}, b_{j}}, j\leq{n}$ e $[a_{j}, b_{j}]\subseteq{[a_{n}, b_{n}]}, j>n.$ Em qualquer um dos casos, $a_{n}\leq{b_{j}}$
para todo $j\in \mathbb{N}$. Logo, $a \leq{b_{j}}, j\in \mathbb{N}.$ Segue que $a_{n}\leq{a=\sup \{s_{n}: n\in \mathbb{N}\}}\leq{b_{n}}$ para todo
$n\in \mathbb{N}$ e $a\in\displaystyle\bigcap_{n\geq{1}}[a_{n}, b_{n}]$. Dado $\delta > 0,$ escolha $n\in \mathbb{N}$ tal que $\frac{2L}{2^{n}}<\delta.$ Segue
que $a\in[a_{n}, b_{n}]\subseteq{(a-\delta, a+\delta) = V_{\delta}(a)}$ e a \'e ponto de acumula\c c\~ao de A. \qedsymbol
\end{proof*}
\newpage

\section{Aula 05 - 22/03/2023}
\subsection{Motiva\c c\~oes}
\begin{itemize}
  \item Sequ\^encias de N\'umeros Reais;
  \item Converg\^encia de Sequ\^encias;
\end{itemize}
\subsection{Sequ\^encias de N\'umeros Reais}
\begin{def*}
  Uma sequ\^encia \'e uma fun\c c\~ao definida no conjunto dos n\'umeros reais que, para cada n natural, associa um n\'umero real $a_{n}$.
  \begin{align*}
    \mathbb{N}&=\{0, 1, 2, \cdots\}\\
              &f:\mathbb{N}\rightarrow \mathbb{R}\\
              &n\mapsto a_{n}.
  \end{align*}
  Denotamos a fun\c c\~ao por $\{a_{n}\}\square$
\end{def*}
 \begin{example}
   Sendo $a_{n}=\frac{f1}{n+1}$ para todo n natural, temos a sequ\^encia $\{1, \frac{1}{2}, \frac{1}{3}, \cdots\}$. 
 \end{example}
\begin{example}
  Sendo $a_{n} = 6$ para todo n natural, temos a sequ\^encia constante 
  $$
    \{6, 6, 6,\cdots\}.
  $$
\end{example}
\begin{example}
  Coloque $a_{2n+1} = 7, a_{2n}=4$ para todo n natural. Temos 
  $$
    \{4, 7, 4, 7, \cdots\}
  $$
\end{example}
  Consideremos as sequ\^encias
  $$
    \alpha_{n} = n, \quad \beta_{n} = (-1)^{n},\quad \text{ e } \gamma_{n} = \frac{1}{n}.
  $$
  Como fun\c c\~oes, elas podem ter os gr\'aficos tra\c cados, mas n\~ao s\~ao muito significativos, visto que consistem em
colet\^aneas de pontos discretos. Ademais, note que a sequ\^encia $(\alpha_{n})$ ``diverge'' para infinito, a sequ\^encia
$(\beta_{n})$ ``oscila'' e a sequ\^encia $(\gamma_{n})$ ``converge para 0''. Precisamente, 
 \begin{def*}
   A sequ\^encia $\{a_{n}\}$ \'e dita convergente com limite l se, para todo $\epsilon > 0$, existe um natural
N dependendo de $\epsilon (N = N(\epsilon)\in \mathbb{N})$ tal que $n > N$ implica em $|a_{n} - l|< \epsilon.$
  Ou seja, a partir de um certo N, os $a_{n}$ est\~ao no intervalo $(l-\epsilon, l+\epsilon)$ e, como $\epsilon$
\'e arbitr\'ario, os $a_{n}$ se juntam em torno de l. Disto, segue que a condi\c c\~ao exigida equivale a
  $$
  l - \epsilon < a_{n} < l + \epsilon, \quad n\geq{N}.
  $$
Denotamos esse fen\^omeno por $\displaystyle\lim_{n\to\infty}a_{n} = l$, ou $a_{n}\rightarrow l$, ou $a_{n}\overbracket[0pt]{\longrightarrow}^{n\to \infty}a.\square$.
\end{def*}
\begin{example}
  $\circ{}\frac{1}{n}\rightarrow0, n\rightarrow\infty$. De fato, dado $\epsilon > 0$, da propriedade arquimediana, segue que 
existe um N natural tal que $N\epsilon > 1.$ Logo, para todo $n\geq{N},$ temos 
  $$
  0 - \epsilon < \frac{1}{n}\leq{\frac{1}{N}} < 0 + \epsilon.
  $$

  $\circ \frac{n}{n+1}\rightarrow 1, n\rightarrow\infty$. Com efeito, dado $\epsilon > 0,$ queremos encontrar N natural n\~ao-nulo tal que
se n \'e maior que N, temos 
  $$
    \biggl|\frac{n}{n+1} - 1\biggr| < \epsilon.
  $$
  No entanto, $|\frac{n}{n+1}-1| = \frac{1}{n+1}$ e, da propriedade Archimediana, existe N em $\mathbb{N}^{\times}$ tal que
  $(N+1)\epsilon > 1$. Logo, se $n\geq{N},$
  $$
    1 - \epsilon < \frac{n}{n+1} < 1 + \epsilon.
  $$
\end{example}
\begin{def*}
  Uma sequ\^encia $\{a_{n}\}$ ser\'a divergente quando ela n\~ao for convergente.
 \begin{itemize}
    \item[I)] Sequ\^encia divergente para $+\infty:$ Este caso ocorre se dado $K > 0$, existe N natural tal que se $n > N,
a_{n} > K.$
    \item[II)] Sequ\^encia divergente para $-\infty:$ Acontece quando dado $K > 0$, existe N natural tal que se $n > N, 
a_{n} < -K.$
    \item[III)]Sequ\^encia oscilante: Por fim, ocorre quando a sequ\^encia diverge, mas nem para $+\infty$ e nem para $-\infty.\square$
 \end{itemize}
\end{def*}
  Note que, como sequ\^encias s\~ao fun\c c\~oes, podemos multiplic\'a-las por constante, somar, dividir e multiplicar por outras sequ\^encia. De fato,
 \begin{def*}
   Dadas sequ\^encias $\{a_{n}\}, \{b_{n}\}$ e um n\'umero real c, deifnimos 
  \begin{align*}
    &i) \{a_{n}\} + \{b_{n}\} = \{a_{n} + b_{n}\}\\
    &ii) c\{a_{n}\} = \{c \cdot a_{n}\}\\
    &iii) \{a_{n}\}\{b_{n}\} = \{a_{n}b_{n}\}\\
    &iv) \text{ Se }b_{n}\neq0\forall n\in \mathbb{N}, \frac{\{a_{n}\}}{\{b_{n}\}} = \biggl\{\frac{a_{n}}{b_{n}}\biggr\}\square
  \end{align*}
 \end{def*}
\begin{def*}
  Seja $\{a_{n}\}$ uma sequ\^encia de n\'umero reais. Diremos que $\{a_{n}\}$ \'e limitada se sua imagem for um subconjunto
limitado de $\mathbb{R}.\square$
\end{def*}
\begin{theorem*}
  Seja $\{a_{n}\}$ uma sequ\^encia de n\'umeros reais.
 \begin{itemize}
   \item[a)] $a_{n}\overbracket[0pt]{\longrightarrow}^{n\rightarrow\infty}a$ se, e somente, toda vizinhan\c ca de a cont\'em todo, exceto uma poss\'ivel quantidade
  finita de $a_{n}$'s.
   \item[b)] O limite \'e \'unico.
   \item[c)] Se $\{a_{n}\}$ \'e convergente, ent\~ao $\{a_{n}\}$ \'e limitada
   \item[d)] Se $a_{n}\overbracket[0pt]{\longrightarrow}^{n\rightarrow\infty}a$, exite N natural tal que $a_{n} > 0$ para todo $n\geq{N}.$
   \item[e)] Se $A\subseteq{\mathbb{R}}$ e a \'e um ponto de acumula\c c\~ao de A, ent\~ao existe uma sequ\^encia $\{a_{n}\}$ de elementos
    de A que converge para a.
 \end{itemize}
\end{theorem*}
\begin{proof*}
  O item a \'e trivial. Mostremos a unicidade do limite: Suponha que $a_{n}$ converge para a e para b, com a diferente de b. Ent\~ao,
dado $\epsilon > 0$, existem naturais $N_{1}, N_{2}$ tais que se $n\geq{N_{1}}, |a_{n}-a|<\epsilon$ e se $n\geq{N_{2}}, |a_{n} - b| < \epsilon.$
Tome $N = \min{N_{1}, N_{2}}$ e suponha que $n \geq{N}.$ Ent\~ao, temos 
 $$
  |b - a| \leq{|b - a_{n}| + |a_{n} - a|} = |b - a_{n}| + |a - a_{n}| < 2\epsilon.
 $$
 (\textit{P.S.}: pode ser boa pr\'atica tomar $\frac{\epsilon}{2}$ ao inv\'es de $\epsilon$, pois assim obtemos $|b-a|<\frac{2\epsilon}{2}=\epsilon.$)

 Como $\epsilon$ \'e abritr\'ario, podemos selecionar $\epsilon$ infinitamente pr\'oximo de 0. Portanto, b = a.

 Para o item c, suponha que $a_{n}$ converge para a, isto \'e, dado $\epsilon > 0, \epsilon = 1$ em particular, existe 
 $N\in \mathbb{N}$ tal que se $n \geq{N}, |a_{n} - a| < 1$. Logo, $a_{n}\in(a - 1, a + 1)$ para n maior que N suficientemente grande.
 Restam os N-1 primeiros elementos da sequ\^encia. Assim, tome $R = \max{\biggl\{|a_{1}|, \cdots, |a_{N-1}|, |a + 1|, |a - 1|\biggr\}}$. Deste modo,
 $a_{n}\in[-R, R]$ para todo n natural. 

 Com rela\c c\~ao ao item d, basta tomar $\epsilon = \frac{a}{2} > 0.$ 

 Por fim, quanto ao item e, suponha o que \'e dito no enunciado. Como a \'e ponto de acumula\c c\~ao, dado $\epsilon > 0,$ existe
 $a'\in{A}, a'\neq a$ tal que 
 $$
  a'\in V_{\epsilon}(a) = (a - \epsilon, a + \epsilon).
 $$
 Logo, tomadno $\epsilon = \frac{1}{n},$ podemos encontrar $a_{n}\in A, a_{n}\neq a$ tal que $a_{n}\in\biggl(a-\frac{1}{n}, a + \frac{1}{n}\biggr)$. A sequ\^encia
 $\{a_{n}\}$ converge para a. De fato, dado $\epsilon > 0$, tome N natural tal que $N\epsilon > 1.$ Assim, se $n\geq{N}, a_{n}\in(a-\frac{1}{n}, a+\frac{1}{n})\subseteq{a-\epsilon}, a+\epsilon)$.
 Portanto, $a_{n}\rightarrow a.$ \qedsymbol
\end{proof*}
 \begin{theorem*}
   Seja $a_{n}\overbracket[0pt]{\longrightarrow}^{n\to\infty}a, b_{n}\overbracket[0pt]{\rightarrow}^{n\to\infty}b$ e c um n\'umero real. Ent\~ao,
  \begin{align*}
    &a) a_{n} + b_{n}\overbracket[0pt]{\longrightarrow}^{n\to\infty} a + b.\\
    &b) ca_{n}\overbracket[0pt]{\longrightarrow}^{n\to\infty} ca\\
    &c) a_{n}b_{n}\overbracket[0pt]{\longrightarrow}^{n\to\infty} ab\\
    &d)\text{Se} b\neq0, b_{n}\neq0\forall n\in \mathbb{N}, \frac{a_{n}}{b_{n}}\overbracket[0pt]{\longrightarrow}^{n\to\infty}\frac{a}{b}.
  \end{align*}
 \end{theorem*}
\begin{proof*}
  Item c). Suponha $a_{n}\overbracket[0pt]{\longrightarrow}^{n\to \infty}a, b_{n}\overbracket[0pt]{\longrightarrow}^{n\to \infty}b$. Note que 
  $$
    |a_{n}b_{n} - ab| = a_{n}b_{n} - a_{n}b + a_{n}b - ab \leq{|a_{n}||b_{n}-b| + |b||a_{n}-a|}
  $$
  Como $\{a_{n}\}$ \'e convergente, ela \'e limitada pelo teorema anterior. Assim, existe $M > 0$ tal que $|a_{n}|\leq{M}$ para todo n natural, tal que
  Assim, 
  $$
    |a_{n}b_{n} - ab| \leq{|a_{n}||b_{n} - b| + |b||a_{n} - a|} \leq{M|b_{n} - b| + (|b| + 1)|a_{n} - a|}.
  $$
  Agora, dado $\epsilon > 0,$ existem naturais $N_{1}, N_{2}$ tais que 
 \begin{align*}
   &|a_{n}-a| < \frac{\epsilon}{2(|b|+1)},\quad\forall n\geq{N_{1}}\\
   &|b_{n}-b| < \frac{\epsilon}{2M},\quad\forall n\geq{N_{2}}.
 \end{align*}
 Logo, tomando $N = \max\{N_{1}, N_{2}\},$ se $n\geq{N},$
 $$
  |a_{n}b_{n}-ab| < \frac{\epsilon}{2} + \frac{\epsilon}{2} = \epsilon.
 $$
 Portanto, $a_{n}b_{n}\overbracket[0pt]{\longrightarrow}^{n\to \infty}ab.$ \qedsymbol
\end{proof*}
\begin{def*}
   Seja $\{a_{n}\}$ uma sequ\^encia. Diremos que $\{b_{n}\}$ \'e uma subsequ\^encia de $\{a_{n}\}$ se existir uma fun\c c\~ao
estritamente crescente $s:\mathbb{N}\rightarrow \mathbb{N}$ tal que $b_{k} = a_{s(k)}$ para todo k natural. $\square$
\end{def*}
\begin{def*}
  Seja $\{a_{n}\}$ uma sequ\^encia. Diremos que $\{a_{n}\}$ \'e de Cauchy se, dado $\epsilon > 0$, existe um natural
  $N = N(\epsilon)$ tal que $|a_{n}-a_{m}| < \epsilon$ para todo $n, m\geq{N}.\square$
\end{def*}
\begin{theorem*}
 \begin{itemize}
   \item[a)]Uma sequ\^encia \'e convergente se, e somente se, toda subsequ\^encia dela converge para o mesmo limite.
   \item[b)] Toda sequ\^encia convergente \'e de Cauchy;
   \item[c)] Toda sequ\^encia limitada tem subsequ\^encia convergente;
   \item[d)] Toda sequ\^encia de Cauchy \'e limitada;
   \item[e)] Toda sequ\^encia de Cauchy que tem subsequ\^encia convergente \'e convergente.
   \item[f)] Toda sequ\^encia de Cauchy \'e convergente;
   \item[g)] Toda sequ\^encia crescente e limitada \'e convergente;
   \item[h)] Toda sequ\^encia decrescente e limitada \'e convergente.
 \end{itemize}  
\end{theorem*}
\newpage

\section{Aula 06 - 24/03/2023}
\subsection{Motiva\c c\~oes}
 \begin{itemize}
   \item Provar o teorema da aula anterior;
   \item Exemplos.
 \end{itemize}
\subsection{Propriedades de Sequ\^encias}
  Recapitulemos o teorema da aula anterior:
\begin{theorem*}
 \begin{itemize}
   \item[a)]Uma sequ\^encia \'e convergente se, e somente se, toda subsequ\^encia dela converge para o mesmo limite.
   \item[b)] Toda sequ\^encia convergente \'e de Cauchy;
   \item[c)] Toda sequ\^encia limitada tem subsequ\^encia convergente;
   \item[d)] Toda sequ\^encia de Cauchy \'e limitada;
   \item[e)] Toda sequ\^encia de Cauchy que tem subsequ\^encia convergente \'e convergente.
   \item[f)] Toda sequ\^encia de Cauchy \'e convergente;
   \item[g)] Toda sequ\^encia crescente e limitada \'e convergente;
   \item[h)] Toda sequ\^encia decrescente e limitada \'e convergente.
   \end{itemize}
 \end{theorem*}
 \begin{proof*}
a.) $\Leftarrow)$ Se toda subsequ\^encia de $\{a_{n}\}$ converge, ent\~ao $\{a_{n}\}$ converge, pois ela \'e uma subsequ\^encia de si mesma (basta tomar $s:\mathbb{N}\rightarrow \mathbb{N}, s(n) = n.$)'

$\Rightarrow)$ Suponha que $a_{n}\overbracket[0pt]{\longrightarrow}^{n\to \infty}l$ e $\{b_{n}\}$ \'e uma subsequ\^encia de $\{a_{n}\}$, existe 
$s:\mathbb{N}\rightarrow \mathbb{N}$ estritamente crescente tal que $b_{k} = a_{s(k)}.$ Dado $\epsilon > 0$, seja N o natural tal que
 $|a_{n} - l| < \epsilon$ para todo $n\geq{N}.$ Note que $s(n)\geq{n},$ tal que se $n\geq{N},$ ent\~ao $s(n)\geq{N},$ de forma que 
 $|a_{s(n)} - l| < \epsilon$. Portanto, qualquer subsequ\^encia de $\{a_{n}\}$ \'e convergente.

 b.) Se $a_{n}\overbracket[0pt]{\longrightarrow}^{n\to \infty}l,$ ent\~ao dado $\epsilon > 0$, existe N natural tal que 
 $$
  |a_{n}-l|<\frac{\epsilon}{2},\quad\forall n\geq{N}.
 $$
 Logo, $|a_{n}-a_{m}| = |a_{n}-l + l-a_{m}| \leq{|a_{n}-l| + |l-a_{m}|} < \frac{\epsilon}{2} + \frac{\epsilon}{2} = \epsilon$ para todo $n, m \geq{N}.$

 c.) Suponha que $\{a_{n}\}$ \'e uma sequ\^encia limitada. Recorde que, do teorema de Bolzano-Weierstrass, todo conjunto inifinito e limitado
possui um ponto de acumula\c c\~ao. Segue que a imagem I da sequ\^encia \'e finita ou infinita.

No primeiro caso, se I \'e finito, um dos valores pertencentes a I \'e tal que $a_{n} = a$ para infinitos \'indices. 
Construiremos a sequ\^encia como segue - Coloque s(0) como o menor elemento do conjunto dos n's para os quais 
$a_{n} = a, i.e.,\{n\in \mathbb{N}: a_{n} = a\} = A. $ Al\'em disso, tome s(1) como  o menor elemento de A, com  exce\c c\~ao do
s(0). Repetindo esse processo, obtemos uma subsequ\^encia constante at\'e que se obtenha s(n) = a, ou seja, ela ser\'a convergente.

Agora, se I \'e infinito, segue de Bolzano-Weierstrass que I tem um ponto de acumula\c c\~ao, nomeie-o de a. Dado $\epsilon > 0, (a-\epsilon, a+\epsilon)$ 
tem infinitos elementos do conjunto I. Analogamente ao anterior, coloque N = s(0) como o menor elemento de $\{n\in \mathbb{N}: a_{n}\neq a, a_{n}\in(a-\epsilon, a+\epsilon)\}$ e coloque, tamb\'em,
$\epsilon_{1} = |a-a_{s(0)}|$. Em seguida, tome $s(1) = \{n\in \mathbb{N}: a_{n}\neq a, a_{n}\in(a-\frac{\epsilon}{2}, a+\frac{\epsilon}{2})\}$. Indutivamente,
$b = a_{s(n)}$ \'e convergente para a.

d.) Dado $\epsilon = 1,$ seja N um n\'umero natural tal que 
  $$
    |a_{n}-a_{m}| < 1,\quad\forall n\geq{N}.
  $$
  Considere $M = \{|a_{0}|, |a_{1}|, \cdots, |a_{N-1}|, |a_{N}+1|, |a_{N}-1|\}.$ Assim, $a_{n}\in[-M, M]$ para todo n natural.

e.) Seja $\{a_{n}\}$ de Cauchy e $\{a_{s(n)}\}$ convergente para l. Dado $\epsilon > 0$, existe um natural $N_{1}$ tal que 
  $$
    |a_{n}-a_{m}| < \frac{\epsilon}{2},\quad\forall n\geq{N_{1}}.
  $$
  Al\'em disso, existe $N_{2}$ natural tal que 
  $$
    |a_{s(n)} - l| < \frac{\epsilon}{2},\quad\forall s(n)\geq{N_{2}}.
  $$
  Seja $N=\max\{s(N_{2}), N_{1}\}$ e tome $n\geq{N}.$
  $$
    |a_{n}-l| = |a_{n} - a_{s(N_{2})} + a_{s(N_{2})} - l| \leq{|a_{n}-a_{s(N_{2})}| + |a_{s(N_{2})} - l|} < \frac{\epsilon}{2} + \frac{\epsilon}{2} = \epsilon.
  $$

f.) Segue os itens (e), (d) e (c), visto que toda subsequ\^encia de Cauchy ter\'a subsequ\^encia convergente pelos itens (d) e (c).

g.) Seja $\{a_{n}\}$ limitada e crescente, $l = \sup\{a_{n}:n\in \mathbb{N}\}.$ Ent\~ao, para todo $n \geq{N}$, em que N \'e tal que $a_{N}\in(l-\epsilon, l)$
 $$
  l-\epsilon < a_{N}\leq{a_{n}} \leq{l}.
 $$

h.) An\'aloga ao g.
\end{proof*}
 \begin{example}
   Mostre que 
 \begin{itemize}
   \item[i)] $\{a, a, a, \cdots\},a\in \mathbb{R}$ \'e convergente;
   \item[ii)] $\{0, 1, 0, 1\}$ n\~ao \'e convergente;
   \item[iii)] $\{n\}$ n\~ao \'e convergente.
 \end{itemize}
 \end{example}
\begin{example}
  Se a \'e um n\'umero real mais ou igual a zero, ent\~ao a sequ\^encia $\{a^{n}\}$ \'e convergente se $0\leq{a}\leq{1}$ e divergente
  se $a > 1$. Com efeito, se $a > 1, a = 1 + h, h > 0$. Ent\~ao, 
  $$
    a^{n} = (1+h)^{n} = \sum\limits_{k=0}^{n}\binom{n}{k}1^{n-k}h^{k} = 1 + nh + \cdots > 1 + nh.
  $$
  Mas, segue da Archimediana que $1 + nh$ sempre forma um conjunto ilimitado para n natural, ou seja, $a_{n}$ \'e ilimitada. Logo, a sequ\^encia
diverge.

  Por outro lado, suponha que a pertence a (0, 1). Ent\~ao, $a^{n+1} = a a^{n} < a^{n}$, ou seja, \'e uma sequ\^encia decrescente e limitada inferiormente.
Portanto $\{a_{n}\}$ \'e convergente.
\end{example}
\begin{example}
  Mostre que, se a \'e diferente de 1, 
  $$
    \sum\limits_{i=0}^{n}a^{i} = \frac{1-a^{n+1}}{1-a}
  $$
  e que a sequ\^encia $\biggl\{\frac{1-a^{n+1}}{1-a}\biggr\}$ \'e convergente se $0\leq{a}<1$ e divergente se $a > 1$.
\end{example}
\begin{example}
  Mostre que a sequ\^encia $\{a_{n}\}$, com $a_{n} = \displaystyle \sum\limits_{i=0}^{n}\frac{1}{i!}$ \'e convergente para todo n natural. (Crescente e limitada por 3.)
\end{example}
\begin{example}
  Mostre que as sequ\^encias $\biggl\{(1+\frac{1}{n}^{n})\biggr\}, \{n^{\frac{1}{n}}\}$ e $\{a^{\frac{1}{n}}\}$ com $a >0,$ s\~ao
convergentes.
 \begin{align*}
   &\circ (1+\frac{1}{n})^{n} = 1 + 1 + \frac{1}{2!}(1-\frac{1}{n}) + \cdots + \frac{1}{n!}(1-\frac{1}{n})(1-\frac{1}{n})(1-\frac{2}{n})\cdots(1-\frac{n-1}{n})\\
   &\circ n^{\frac{1}{n}} > (n+1)^{\frac{1}{n+1}}\Longleftrightarrow n^{n+1} > (n+1)^{n}\Longleftrightarrow n>(1+\frac{1}{n})^{n}\\
   &\circ x = a^{n} < 1\Rightarrow x < 1, x^{n} = a, x^{n+1} = a^{\frac{n+1}{n}},\text{ e } y^{n+1} = a \Rightarrow \biggl(\frac{x}{y}\biggr)^{n+1} = a^{\frac{1}{n}}.
 \end{align*}
\end{example}
\newpage

\section{Aula 07 - 27/03/2023}
\subsection{Motiva\c c\~oes}
 \begin{itemize}
   \item Exemplos de Sequ\^encias;
   \item Teorema da Compara\c c\~ao e do Sandu\'iche;
   \item Limites superior e inferior.
 \end{itemize}
\subsection{Exemplos de Sequ\^encias}
  Revisemos os exemplos da \'ultima aula, com um extra ao final.
  \begin{example}
   Mostre que 
 \begin{itemize}
   \item[i)] $\{a, a, a, \cdots\},a\in \mathbb{R}$ \'e convergente;
   \item[ii)] $\{0, 1, 0, 1\}$ n\~ao \'e convergente;
   \item[iii)] $\{n\}$ n\~ao \'e convergente.
 \end{itemize}
 \end{example}
\begin{example}
  Se a \'e um n\'umero real mais ou igual a zero, ent\~ao a sequ\^encia $\{a^{n}\}$ \'e convergente se $0\leq{a}\leq{1}$ e divergente
  se $a > 1$. Com efeito, se $a > 1, a = 1 + h, h > 0$. Ent\~ao, 
  $$
    a^{n} = (1+h)^{n} = \sum\limits_{k=0}^{n}\binom{n}{k}1^{n-k}h^{k} = 1 + nh + \cdots > 1 + nh.
  $$
  Mas, segue da Archimediana que $1 + nh$ sempre forma um conjunto ilimitado para n natural, ou seja, $a_{n}$ \'e ilimitada. Logo, a sequ\^encia
diverge.

  Por outro lado, suponha que a pertence a (0, 1). Ent\~ao, $a^{n+1} = a a^{n} < a^{n}$, ou seja, \'e uma sequ\^encia decrescente e limitada inferiormente.
Portanto $\{a_{n}\}$ \'e convergente.
\end{example}
\begin{example}
  Mostre que, se a \'e diferente de 1, 
  $$
    \sum\limits_{i=0}^{n}a^{i} = \frac{1-a^{n+1}}{1-a}
  $$
  e que a sequ\^encia $\biggl\{\frac{1-a^{n+1}}{1-a}\biggr\}$ \'e convergente se $0\leq{a}<1$ e divergente se $a > 1$.
\end{example}
\begin{example}
  Mostre que a sequ\^encia $\{a_{n}\}$, com $a_{n} = \displaystyle \sum\limits_{i=0}^{n}\frac{1}{i!}$ \'e convergente para todo n natural. (Crescente e limitada por 3.) 

  De fato, \'e claro que $\{a_{n}\} $ \'e crescente e que $\frac{1}{n!}\leq{\frac{1}{2^{n-1}}},$ para $n\geq{2}.$ Logo, 
  $$
  a_{n}\leq{1+\sum\limits_{k=0}^{n}\frac{1}{2^{k}}} = 
  1 + \frac{1-\frac{1}{2^{n+1}}}{1-\frac{1}{2}} < 3.
  $$
  Portanto, $\{a_{n}\}$ \'e convergente, e denotamos seu limite por e.
\end{example}
\begin{example}
  Mostre que as sequ\^encias $\biggl\{(1+\frac{1}{n}^{n})\biggr\}, \{n^{\frac{1}{n}}\}$ e $\{a^{\frac{1}{n}}\}$ com $a >0,$ s\~ao
convergentes.
 \begin{align*}
   &\circ (1+\frac{1}{n})^{n} = 1 + 1 + \frac{1}{2!}(1-\frac{1}{n}) + \cdots + \frac{1}{n!}(1-\frac{1}{n})(1-\frac{1}{n})(1-\frac{2}{n})\cdots(1-\frac{n-1}{n})\\
   &\circ n^{\frac{1}{n}} > (n+1)^{\frac{1}{n+1}}\Longleftrightarrow n^{n+1} > (n+1)^{n}\Longleftrightarrow n>(1+\frac{1}{n})^{n}\\
   &\circ x = a^{n} < 1\Rightarrow x < 1, x^{n} = a, x^{n+1} = a^{\frac{n+1}{n}},\text{ e } y^{n+1} = a \Rightarrow \biggl(\frac{x}{y}\biggr)^{n+1} = a^{\frac{1}{n}}.
 \end{align*}
   Ainda mais, uma delas t\^em como limite o n\'umero e definido no exemplo anterior. Para observar isso, considere o primeiro exemplo. Note que
  \begin{align*}
    b_{n} &= 1 + \binom{n}{1}n^{-1} + \binom{n}{2}n^{-2} + \cdots + \binom{n}{n-1}n^{-n+1} + \binom{n}{n}n^{-n} \\ 
          &= 1 + 1 + \frac{1}{2!}(1-\frac{1}{n}) + \cdots + \frac{1}{n!}(1-\frac{1}{n})(1-\frac{1}{n})(1-\frac{2}{n})\cdots(1-\frac{n-1}{n})\\
          &\leq{1 + 1 + \frac{1}{2!} + \cdots + \frac{1}{n!} = a_{n} < e}
 \end{align*}
 Como cada termo da soma que define $b_{n}$ \'e crescente, obtemos que $b_{n}$ \'e crescente, tal que ela converge com limite $l = \sup{\{b_{n}:n\in \mathbb{N}\}}$.
  
   Com rela\c c\~ao ao \'ultimo item, resta elaborar como ele converge para 1. Lembre-se que $a^{\frac{1}{n}}$ \'e o \'unico n\'umero real positivo
  x tal que $x^{n} = a.$ Logo, se $x = a^{n} $ e $y = a^{\frac{1}{n+1}}$, temos $x^{n+1} = y^{n+1}x$ e, deste modo,
  \begin{align*}
  & (a) 0 < a < 1 \Rightarrow x < 1 \text{ e } \biggl(\frac{x}{y}\biggr)^{n+1} = x < 1\text{ e, assim, } x < y.\\
  & (b) a > 1 \Rightarrow x > 1\text{ e }\biggl(\frac{x}{y}\biggr)^{n+1} = x > 1\text{ tal que } x > y.
  \end{align*}
  Logo, se $a < 1,\{a^{\frac{1}{n}}\} $ \'e crescente e limitada superiormente por 1, mostrando que ela \'e convergente. Al\'em disto,
se $a > 1,\{a^{\frac{1}{n}}\} $ \'e descrescente e limitada inferiormente por 1, tamb\'em sendo convergente. Por fim, segue de
$a^{\frac{1}{n(n+1)}} = \frac{a^{\frac{1}{n}}}{a^{\frac{1}{n+1}}}$. Portanto, do item (a) do teorema junto com a regra para quociente de
sequ\^encias, segue que l = 1 \'e o limite dela.
\end{example}\begin{example}
   Mostre que a sequ\^encia $\{c_{n}\}, c_{0} = 1, c_{n} = n^{\frac{1}{n}}, n\geq{1}$, \'e convergente. Com efeito, lembre-se que,
para $n\geq{3}, n > b_{n} = (1+\frac{1}{n})^{n}.$ Logo, para $n\geq{3}, n^{n+1}>(n+1)^{n}$ e, consequentemente, $n^{\frac{1}{n}} > (n+1)^{\frac{1}{n+1}}.$\
Disto segue de $\{n^{\frac{1}{n}}\} $ \'e limitada e, por (h), que $\{c_{n}\} $ \'e convergente com limite $l\geq{1}.$ Ainda mais,
 $(2n)^{\frac{1}{2n}}(2n)^{\frac{1}{2n}} = (2n)^{\frac{1}{n}} = 2^{\frac{1}{n}}n^{\frac{1}{n}}$ e, portanto, usamos (a) e o exemplo da \'ultima aula para
mostrar que $l^{2} = l = 1.$\qedsymbol
 \end{example}

\subsection{Teoremas da Compara\c c\~ao e do Sandu\'iche}
  Nota\c c\~ao: Se uma sequ\^encia tem limite 0, ela \'e chamada infinit\'esima.
\begin{theorem*}
  Se $\{a_{n}\}$ \'e limitada e $\{b_{n}\}$ \'e infinit\'esima, ent\~ao $\{a_{n}\cdot b_{n}\} $ \'e infinit\'esima.
\end{theorem*}
\begin{proof*}
  Como $\{a_{n}\}$ \'e limitada, seja $M > 0$ tal que $|a_{n}|\leq{M}$ para todo n natural. Como $\{b_{n}\} $ \'e infinit\'esima, dado $\epsilon > 0$,
seja N outro natural tal que $|b_{n}|<\frac{\epsilon}{M}$ para todo $n\geq{N}.$ Segue que
  $$
    |a_{n}b_{n}| \leq{M|b_{n}|} < M \frac{\epsilon}{M} = \epsilon,\quad \forall n\geq{N}.
  $$
  Portanto, $\{a_{n}b_{n}\}\overbracket[0pt]{\longrightarrow}^{n\to \infty}0$.\qedsymbol
\end{proof*}
 \begin{example}
   Mostre que $\biggl\{\frac{n+\cos{(n)}}{n+1}\biggr\}$ converge.
 \end{example}
 Os resultados a seguir s\~ao os dois mencionados previamente, o teorema da compara\c c\~ao e o do sandu\'iche, respectivamente.
\begin{theorem*}
  Se $a_{n}\overbracket[0pt]{\longrightarrow}^{n\to \infty}a, b_{n}\overbracket[0pt]{\longrightarrow}^{n\to \infty}b$ e existe N natural tal que, 
para todo $n\geq{N}, a_{n}\leq{b_{n}}$, ent\~ao $a\leq{b}.$
\end{theorem*}
 \begin{proof*}
   Dado $\epsilon > 0$, existe $N_{1}\leq{N}$ tal que, para todo $n\geq{N_{1}}$,
   $$
    a - \epsilon < a_{n} < a + \epsilon\quad\text{ e }\quad b-\epsilon < b_{n} < b + \epsilon.
   $$
   Logo, para todo $n\geq{N},$
   $$
    a-\epsilon < a_{n} \leq{b_{n}} < b + \epsilon.
   $$
   Desta forma, $a-b < \epsilon$ para todo $\epsilon > 0$ e, portanto, $a - b\leq{0}.$\qedsymbol
 \end{proof*}

\begin{theorem*}
  Se $a_{n}\overbracket[0pt]{\longrightarrow}^{n\to \infty}l, c_{n}\overbracket[0pt]{\longrightarrow}^{n\to \infty}l$ e existe um
N natural tal que, para todo $n\geq{N}, a_{n}\leq{b_{n}}\leq{c_{n}}$, ent\~ao $b_{n}\overbracket[0pt]{\longrightarrow}^{n\to \infty}l.$
\end{theorem*}
\begin{proof*}
  Dado $\epsilon>0$, existe $N_{1}\geq{N}$ tal que, para todo $n\geq{N_{1}},$
  $$
    l - \epsilon < a_{n} < l + \epsilon\quad\text{ e }\quad l-\epsilon < c_{n} < l+\epsilon.
  $$
  Logo, para todo $n\geq{N_{1}},$
  $$
    l - \epsilon < a_{n}\leq{b_{n}}\leq{c_{n}}<l + \epsilon.
  $$
  Disto segue que $|b_{n} - l|< \epsilon$ para todo $n\geq{N_{1}}$ e que, portanto, $\{b_{n}\} $ \'e convergente para l. \qedsymbol
\end{proof*}
\begin{example}
  Vamos mostrar que 
  $$
   e = \lim_{n\to\infty}\overbrace{(1 + \frac{1}{1!} + \frac{1}{2!} + \cdots + \frac{1}{n!})}^{a_{n}} = \lim_{n\to\infty}\overbrace{\biggl(1 + \frac{1}{n}\biggr)^{n}}^{b_{n}} = l.
  $$  
  De fato, como $a_{n} \geq{b_{n}}$ para todo n natural, segue do Teorema da Compara\c c\~ao que $e\geq{l}.$ Por outro lado, se 
 $n\geq{p}\geq{2},$
 $$
  b_{n} > 1 + 1 + \frac{1}{2!}(1-\frac{1}{n})+\cdots+\frac{1}{p!}(1-\frac{1}{n})(1-\frac{2}{n})\cdots(1-\frac{p-1}{n}).
 $$
 Agora, novamente pelo Teorema da Compara\c c\~ao, $l = \displaystyle \lim_{n\to\infty}b_{n}\geq{a_{p}}$ para todo natural p.
 Portanto, $l = \displaystyle \lim_{n\to\infty}b_{n}\geq{\sup\{a_{n}:n\in \mathbb{N}\}} = \lim_{n\to\infty}a_{n} = e.$ \qedsymbol
\end{example}
\begin{def*}
   Seja $\{a_{n}\} $ uma sequ\^encia. Um n\'umero real a \'e um valor de ader\^encia de $\{a_{n}\} $ se a sequ\^encia $\{a_{n}\}$ possui
uma subsequ\^encia convergente para a.$\quad\square$
\end{def*}
\begin{def*}
  Seja $\{a_{n}\} $  uma sequ\^encia limitada. Definimos o limite superior $\displaystyle\limsup_{n\to\infty}a_{n}(\text{ inferior}\liminf_{n\to\infty}a_{n})$ da 
sequ\^encia $\{a_{n}\} $ por 
  \begin{align*}
    &\limsup_{n\to\infty}a_{n} = \lim_{n\to\infty}\sup_{k\geq{n}}a_{k}\\
    &\liminf_{n\to\infty}a_{n} = \lim_{n\to\infty}\inf_{k\geq{n}}a_{k}\quad\square
  \end{align*}
\end{def*}
  Uma consequ\^encia direta do Teorema do Confronto que utiliza os conceitos acima nos permite dizer se uma sequ\^encia converge apenas
utilizando as ideias de limite superior e inferior:
 \begin{theorem*}
   Se a \'e um valor de ader\^encia da sequ\^encia $\{a_{n}\} $, ent\~ao
   $$
   \liminf_{n\to\infty}a_{n}\leq{a}\leq{\limsup_{n\to\infty}a_{n}}.
   $$
   Al\'em disso, uma sequ\^encia \'e convergente se, e somente se, $\liminf_{n\to\infty}a_{n} = \limsup_{n\to\infty}a_{n}.$
 \end{theorem*}

\end{document}

