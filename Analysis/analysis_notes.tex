 \documentclass{article}
 \usepackage{amsmath}
 \usepackage{amsthm}
 \usepackage{amssymb}
 \usepackage{pgfplots}
 \usepackage{amsfonts}
 \usepackage[margin=2.5cm]{geometry}
 \usepackage{graphicx}
 \usepackage[export]{adjustbox}
 \usepackage{fancyhdr}
 \usepackage[portuguese]{babel}
 \usepackage{hyperref}
 \usepackage{lastpage}
 \usepackage{mathtools}

 \pagestyle{fancy}
 \fancyhf{}

 \pgfplotsset{compat = 1.18}

 \hypersetup{
     colorlinks,
     citecolor=black,
     filecolor=black,
     linkcolor=black,
     urlcolor=black
 }
 \newtheorem*{def*}{\underline{Defini\c c\~ao}}
 \newtheorem*{theorem*}{\underline{Teorema}}
 \newtheorem*{lemma*}{\underline{Lema}}
 \newtheorem*{prop*}{\underline{Proposi\c c\~ao}}
 \newtheorem{example}{\underline{Exemplo}}
 \newtheorem*{proof*}{\underline{Prova}}
 \renewcommand\qedsymbol{$\blacksquare$}
 \newcommand{\Lin}[1]{Lin_{\mathbb{K}}({#1})}

 \rfoot{P\'agina \thepage \hspace{1pt} de \pageref{LastPage}}

 \title{Notas de An\'alise}
 \author{Renan Wenzel}
 \date{\today}

 \begin{document}
 \begin{figure}[ht]
	\minipage{0.76\textwidth}
		\includegraphics[width=4cm]{../icmc.png}
		\hspace{7cm}
		\includegraphics[height=4.9cm,width=4cm]{../brasao_usp_cor.jpg}
	\endminipage	
\end{figure}

\begin{center}
	\vspace{1cm}
	\LARGE
	UNIVERSIDADE DE S\~AO PAULO

	\vspace{1.3cm}
	\LARGE
	INSTITUTO DE CI\^ENCIAS MATEM\'ATICAS E COMPUTACIONAIS - ICMC

	\vspace{1.7cm}
	\Large
	\textbf{Notas de Aula de An\'alise}

	\vspace{1.3cm}
	\large
	\textbf{Renan Wenzel - 11169472}

	\vspace{1.3cm}
	\large
	\textbf{Alexandre Nolasco de Carvalho - andcarva@icmc.usp.br}

	\vspace{1.3cm}
	\today
\end{center}

 \newpage

 \tableofcontents

 \newpage

\section{Aula 01 - 13/03/2023}
\subsection{Motiva\c c\~ao}
\begin{itemize}
  \item Relembrar sistemas b\'asicos da matem\'atica;
  \item Relembrar propriedades b\'asicas das principais estruturas ($\mathbb{N}, \mathbb{Z}, \mathbb{Q}$).
\end{itemize}

\subsection{Os N\'umeros Naturais}
  Os n\'umeros naturais s\~ao os que utilizamos para contar objetos, e s\~ao caracterizados pelos Axiomas de Peano:
 \begin{itemize}
   \item[1)] Todo n\'umero natural tem um \'unico sucessor; 
   \item[2)] N\'umeros naturais diferentes t\^em sucessores diferentes;
   \item[3)] Existe um \'unico n\'umero natural, zero (0), que n\~ao \'e sucessor de nenhum n\'umero natural.
   \item[4)] Seja $X \subseteq{\mathbb{N}}$ tal que $0\in{X}$ e, se n pertence a X, seu sucessor n+1 tamb\'em pertence 
a X. Ent\~ao, X = $\mathbb{N}.$ (Propriedade de Indu\c c\~ao).
 \end{itemize}

\begin{def*}
  Definimos a adi\c c\~ao por: $n + 0 = n, n\in \mathbb{N},\text{ e }n+(p+1) = (n+p)+1, p\in{\mathbb{N}}$. Al\'em disso,
a multiplica\c c\~ao \'e dada por: $n.0 = 0, n.(p+1) = n.p + n, n, p\in\mathbb{N}.$ Ou seja, sabendo somar ou multiplicar um n\'umero,
sabemos somar e multiplicar seu sucessor.
\end{def*}
  Com rela\c c\~ao ao quarto axioma, ele leva este nome porque um dos m\'etodos de demonstra\c c\~ao, conhecido como
prova por indu\c c\~ao. Nele, mostramos um caso base, o caso 0, e utilizamos a segunda parte para provar que, se um
resultado vale para o caso n, ele vale para n+1, portanto sendo verdadeiro para todos os naturais.

\begin{lemma*}
  Para todo n natural, 1 + n = n + 1.
\end{lemma*}
\begin{proof*}
  Note que o resultado \'e verdadeiro para n = 0. Suponha que o resultado seja v\'alido para n = k e mostremos que 
vale tamb\'em para n = k+1. Com efeito, segue pela propriedade de indu\c c\~ao e pela defini\c c\~ao de soma que
 $$
    1 + (k + 1) = (1 + k) + 1 =  (k + 1) + 1. 
 $$
 Segue que o resultado vale para todo n natural. \qedsymbol
\end{proof*}
  A seguir, mostramos a associatividade e a comutatividade, respectivamente, das opera\c c\~oes nos naturais.
\begin{lemma*}
  Para todo n, p, r naturais, (n + p) + r = n + (p + r).
\end{lemma*}
\begin{proof*}
  Note que o resultado \'e v\'alido trivialmente para r = 0 e r = 1. Suponha que o resultado seja v\'alido para
r = k e mostremos que vale tamb\'em para r = k + 1. Com efeito, pela hip\'otese de indu\c c\~ao e defini\c c\~ao de adi\c c\~ao,
  $$
    n + (p + (k + 1)) = n + ((p + k) + 1) = (n + (p + k)) + 1 = ((n + p) + k) + 1 = (n + p) + (k + 1).
  $$
  Segue o resultado por indu\c c\~ao. \qedsymbol
\end{proof*}
\begin{lemma*}
  Para todo n, p naturais, n + p = p + n. 
\end{lemma*}
\begin{proof*}
  Observe que j\'a mostramos o caso em que p = 1. Suponha que o resultado vale para p = k e vamos mostrar o caso
p = k + 1. De fato, pela hip\'otese de indu\c c\~ao e defini\c c\~ao de adi\c c\~ao, junto do lema de associatividade,
temos  
  $$
    n + (k + 1) = (n + k) + 1 = (k + n) + 1 = 1 + (k + n) = (1 + k) + n = (k + 1) + n.
  $$
  Por indu\c c\~ao, segue que isso vale para todo natural n. \qedsymbol
\end{proof*}

\begin{def*}
  Definimos uma ordem em $\mathbb{N}$ colocando que $m\leq{n}$ se existe p natural tal que $n = m + p. \square$
\end{def*}
  A rela\c c\~ao de ordem possui as seguintes propriedades:
 \begin{itemize}
   \item[i)] Reflexiva: Para todo n natural, $n\leq{n};$
   \item[ii)] Antissim\'etrica: Se $m\leq n$ e $n\leq m,$ ent\~ao $m = n;$
   \item[iii)] Transitiva: Se $m \leq n$ e $n \leq p$, ent\~ao $m\leq p;$
   \item[iv)] Dados m, n naturais, temos ou $m \leq n$, ou $n \leq m;$
   \item[v)] Se $m \leq n$ e p \'e um natural, ent\~ao $m + p\leq n\text{ e } mp\leq np$
 \end{itemize}

 \subsection{N\'umeros Inteiros e Racionais}
  Usualmente, construimos os inteiros a partir dos naturais tomando os pares ordenados de n\'umeros naturais
com a seguinte identifica\c c\~ao (a, b) $\mathtt{\sim}$ (c, d) se a + d = b + c. Assim, podemos representar
  $$
  \mathbb{N} = \{(0, 0), (1, 0), (2, 0), (3, 0), \cdots\}, \quad -\mathbb{N}^* = \{\cdots, (0, 3), (0, 2), (0, 1)\}.
  $$
  Tomar o sucessor ser\'a somar 1 \`a primeira coordenada e, para os inteiros negativos, voltar a identificar (1, n) com (0, n-1).

  Os n\'umeros racionais s\~ao constru\'idos tomando o conjunto $\mathbb{Z}\times{\mathbb{Z}^*}$ e identificando os pares $(a, b)\mathtt{\sim}(c, d)$
para os quais ad = bc. Representamos um par (a, b) neste conjunto por $\displaystyle \frac{a}{b}.$ A soma e o produto em $\mathbb{Q}$
s\~ao dados, respectivamente, por: 
 \begin{align*}
   &\frac{a}{b} + \frac{c}{d} \coloneqq \frac{ad + bc}{bd} \\
   &\frac{a}{b}\cdot\frac{c}{d} \coloneqq \frac{ac}{bd}.
 \end{align*}
 Chamamos a adi\c c\~ao a opera\c c\~ao que a cada par $(x, y)\in \mathbb{Q}\times{\mathbb{Q}}$ associa sua soma $x+y\in \mathbb{Q}$
e chamamos multiplica\c c\~ao a opera\c c\~ao que a cada par $(x, y)\in \mathbb{Q}\times \mathbb{Q}$ associa seu produto $x.y\in \mathbb{Q}.$
A terna $(\mathbb{Q}, +, .)$ satisfaz as propriedades de um corpo, i.e., 
 \begin{align*}
   &(A1) (x + y) + z = x + (y + z), \quad\forall x, y, z\in \mathbb{Q}\\
   &(A2) x + y = y + x, \quad\forall x, y\in \mathbb{Q}\\
   &(A3) \exists 0\in \mathbb{Q}: x + 0 = x, \quad\forall x\in \mathbb{Q}\\
   &(A4) \forall x\in \mathbb{Q}, \exists y\in \mathbb{Q} (y = -x): x + y = 0\\
   &(M1) (xy)z = x(yz), \quad\forall x, y, z\in \mathbb{Q}\\
   &(M2) xy = yx, \quad x, y\in \mathbb{Q}\\
   &(M3) \exists 1\in \mathbb{Q}: 1.x = x.1 = x, \quad\forall x\in \mathbb{Q}\\
   &(M4) \forall x\in \mathbb{Q}, \exists y = \frac{1}{x}\in \mathbb{Q}: x.y = 1\\
   &(D) x(y+z) = xy + xz,\quad\forall x, y, z\in \mathbb{Q}.
 \end{align*}

\section{Aula 02 - 15/03/2023}
\subsection{Motiva\c c\~oes}
\begin{itemize}
  \item Propriedades b\'asicas dos racionais;
  \item Constru\c c\~ao do corpo dos reais a partir dos racionais;
  \item Cortes de Dedekind.
\end{itemize}
\subsection{Propriedades de $\mathbb{Q}$ e sua Ordem}
  Com as 9 propriedades de corpo, conseguimos obter novas regras nos racionais, como a famosa lei do cancelamento:
 \begin{prop*}
   Em $\mathbb{Q},$ vale
   $$
    x + z = y + z \Rightarrow x = y
   $$
   e, se $z\neq{0}$,
   $$
    xz = yz \Rightarrow x = y 
   $$
 \end{prop*}
 \begin{proof*}
   \begin{align*}
   &x = x + 0 = x + (z + (-z)) = (x+z) + (-z) = (y + z) + (-z) = y + (z + (-z)) = y +0 = y\\
   &x = x.1 = x(z.\frac{1}{z}) = (xz)\frac{1}{z} = (yz)\frac{1}{z} = y (z \frac{1}{z}) = y.1 = y. \text{ \qedsymbol}
 \end{align*}
 \end{proof*}
 \begin{prop*}
  Os elementos neutros da adi\c c\~ao e multiplica\c c\~ao s\~ao \'unicos. Os elementos oposto e inverso tamb\'em o s\~ao.
 \end{prop*}
 \begin{prop*}
   Para todo x racional, x.0 = 0.
 \end{prop*}
 \begin{prop*}
   Para todo x racional, -x = (-1)x.
 \end{prop*}
 A maioria desses resultados acima seguem diretamente da lei do cancelamento. Suas demonstra\c c\~oes ficam como exerc\'icio.
\begin{def*}
  Diremos que 
  $$
    \frac{a}{b}\in \mathbb{Q} = \left\{\begin{array}{ll}
        \text{ n\~ao-negativo, } \quad ab\in \mathbb{N}\\
        \text{ positivo, } \quad ab\in \mathbb{N}, a\neq 0
      \end{array}\right.
  $$
  e diremos que 
  $$
    \frac{a}{b}\in \mathbb{Q} = \left\{\begin{array}{ll}
        \text{ n\~ao-positivo, } \quad \frac{a}{b} \text{ n\~ao for postivo}\\
        \text{ negativo, } \quad \frac{a}{b} \text{ n\~ao for n\~ao-negativo.}
      \end{array}\right.\square
  $$
\end{def*}
\begin{def*}
  Sejam x, y racionais. Diremos que x \'e menor e que y e escrevemos ``$x < y$'' se existir t racional positivo tal que 
  $$
    y = x + t.
  $$
  Neste mesmo caso, podemos dizer que y \'e maior que x, escrevendo ``$x > y$''. Em particular, temos $x > 0$ se x for positivo e 
$x < 0$ se x for negativo. 

  Ademais, se $x < y$ ou x = y, escrevemos ``$x \leq{y}$'' se existir racional t n\~ao-negativo tal que 
  $$
    y = x + t
  $$
  e, se $x > y$ ou x = y, escrevemos ``$x \geq{y}$'' caso exista racional t n\~ao-positivo com
  $$
    y = x + t. \square
  $$
\end{def*}
  A qu\'adrupla ($\mathbb{Q}, +, ., \leq{}$) satisfaz as propriedades de um corpo ordenado, i.e., 
 \begin{align*}
   &(O1) x \leq{x}\forall x\in \mathbb{Q};\\
   &(O2) x\leq{y} \text{ e } y \leq{x}\Rightarrow x = y \forall x, y\in \mathbb{Q};\\
   &(O3) x \leq{y}, y \leq{z}\Rightarrow x \leq{z}\forall x, y, z\in \mathbb{Q};\\
   &(O4)\forall x, y \in \mathbb{Q}, x \leq{y} \text{ ou } y \leq{x};\\
   &(OA) x \leq{y}\Rightarrow x + z \leq{y + z};\\
   &(OM) x \leq{y} \text{ e } z \geq{0}\Rightarrow xz \leq{yz}. 
 \end{align*}
 \begin{prop*}
  Para quaisquer x, y, z, w no corpo ordenado dos racionais, valem 
 \begin{align*}
   &i.) x < y\Longleftrightarrow x + z < y + z \\
   &ii.) z > 0\Longleftrightarrow \frac{1}{z} > 0 \\
   &iii.) z > 0\Longleftrightarrow -z < 0 \\
   &iv.) z > 0\Rightarrow x < y\Longleftrightarrow xz < yz \\
   &v.) z < 0\Rightarrow x < y\Longleftrightarrow xz > yz \\
   &vi.) xz < yw\Longleftrightarrow \left\{\begin{array}{ll}
       0 \leq{x} < y \\
       0 \leq{z} < w
     \end{array}\right. \\
   &vii.) 0 < x < y\Longleftrightarrow 0 < \frac{1}{y} < \frac{1}{x}\\
   &viii.) x < y \text{ ou } x =y \text{ ou } x > y \\
   &ix.) xy = 0\Longleftrightarrow x = 0\text{ ou }y = 0. \\
   &x.) \left.\begin{array}{ll}
       x \leq{y} \\
       z \leq{w}
     \end{array}\right\}\Rightarrow x + z \leq{y + w}\\
   &xi.) \left.\begin{array}{ll}
       0 \leq{x} \leq{y}\\
       0 \leq{z} \leq{w}
   \end{array}\right\}\Rightarrow xz \leq{yw}.
 \end{align*}
 \end{prop*}
 \subsection{Incompletude de $\mathbb{Q}$}
  Os n\'umeros racionais podem ser representados por pontos em uma reta horizontal ordenada, chamada reta real. Se P for a representa\c c\~ao
de um n\'umero racional x, diremos que x \'e a abscissa de P. Note que nem todo ponto da reta real \'e racional. Para isso, considere
um quadrado de lado 1 e diagonal d. Pelo Teorema de Pit\'agoras, $d^{2} = 1^2 + 1^2 = 2.$ Agora, seja P a intersec\c c\~ao do eixo
x com a circunfer\^encia de centro em 0 e raio d. Mostremos que P \'e um ponto da reta com abscissa $x\not\in \mathbb{Q}.$
\begin{prop*}
  Seja a um inteiro. Ent\~ao, se a for \'impar, seu quadrado tamb\'em ser\'a. Al\'em disso, se a for par, seu quadrado tamb\'em \'e par.
\end{prop*}
\begin{prop*}
  A equa\c c\~ao $x^2 = 2$ n\~ao admite solu\c c\~ao racional.
\end{prop*}
  A ideia da prova \'e escrever um x na forma de fra\c c\~ao e chegar na contradi\c c\~ao de que tanto o numerador quanto o denominador
ser\~ao n\'umeros pares. Com isso, conclui-se que n\~ao existe racional irredut\'ivel com quadrado igual a 2, portanto n\~ao existe racional
satisfazendo a equa\c c\~ao. 

  Essa discuss\~ao mostra que existem v\~aos na ``reta'' dos racionais, requerindo a ado\c c\~ao de um novo corpo. Essa \'e a principal
motiva\c c\~ao por tr\'as dos n\'umeros reais, "preencher" os buracos deixados pelos racionais.
\begin{prop*}
  (Exerc\'icio.) Sejam $p_{1}, \cdots, p_{n}$ n\'umeros primos distintos. Ent\~ao, a equa\c c\~ao $x^{2} = p_{1}p_2\cdots p_{n}$ n\~ao
tem solu\c c\~ao racional.
\end{prop*}
  Vimos que os n\'umeros racionais com a sua adi\c c\~ao, multiplica\c c\~ao e rela\c c\~ao de ordem \'e um corpo ordenado. Nos interessamos,
tamb\'em, pelo corpo dos reais e dos racionais ($\mathbb{R}, \mathbb{C}$). De forma abstrata, um corpo \'e um conjunto n\~ao-vazio
 $\mathbb{F}$ em que est\~ao definidas duas opera\c c\~oes bin\'arias
 $$
    +:\mathbb{F}\times \mathbb{F}\rightarrow \mathbb{F}, \quad (x, y)\mapsto x + y
 $$
 e 
 $$
    .: \mathbb{F}\times \mathbb{F}\rightarrow \mathbb{F}, \quad (x, y)\mapsto xy
 $$
 em que valem as oito propriedades vistas previamente para a defini\c c\~ao das opera\c c\~oes em $\mathbb{Q}$
 Se, ainda por cima, no corpo $\mathbb{F}$ est\'a definida uma ordem com propriedades an\'alogas \`as vistas para a qu\'adrupla
($\mathbb{Q}, +, ., \leq{}$), diremos que ($\mathbb{F}, +, ., \leq{}$) \'e um corpo ordenado.
\begin{def*}
  Diremos que um subconjunto A de um corpo $\mathbb{F}$ ordenado \'e limitado superiormente se existe um L neste corpo tal que $a \leq{L}$ para todo
a de A. 

  Definimos para um subconjunto limitado superiormente um n\'umero $\sup(A)\in \mathbb{F}$ como o menor limitante superior de
A, i.e., se $a \leq{\sup(A)}$ para todo a de A e se existe $f\in \mathbb{F}$ com $f < \sup(A),$ ent\~ao existe um a em A com $ f < a.$

  Por fim, diremos que um corpo ordenado \'e completo se todo subconjunto limitado superiormente possui supremo. $\square$
\end{def*}
  Nem todo subconjunto limitado superiormente de $\mathbb{Q}$ tem supremo, ou seja, $\mathbb{Q}$ n\~ao \'e completo.

\subsection{Os N\'umeros Reais ($\mathbb{R}$)}
  A ideia que iremos usar para construir o conjunto dos reais \'e que o conjunto dos n\'umeros reais preenche toda a reta real. Os Elementos
de $\mathbb{R}$ ser\~ao os subconjuntos de $\mathbb{Q}$ \`a esquerda de um ponto da reta real e ser\~ao chamados de cortes.
 \begin{def*}
   Um corte \'e um subconjunto $\alpha\subsetneq \mathbb{Q}$ com as seguintes propriedades:
  \begin{itemize}
    \item[i)] $\alpha\neq \emptyset$ e $\alpha \neq \mathbb{Q};$
    \item[ii)] Se $p\in \alpha$ e q \'e um racional com $q < p$, ent\~ao $q\in \alpha$ (todos os racionais \`a esquerda de um elemento
      de $\alpha$ est\~ao em $\alpha$);
    \item[iii)] Se $p\in \alpha$, existe um $r\in \alpha$ com $p < r$ ($\alpha$ n\~ao tem um maior elemento). $\square$
  \end{itemize}
 \end{def*}
  Essa ideia foi proposta inicialmente por Julius Wilhelm Richard Dedekind, um matem\'atico alem\~ao, em 1872, com o objetivo de encontrar uma explica\c c\~ao
e constru\c c\~ao elementar para os n\'umeros reais.
\begin{example}
  Se q \'e um racional, definimos $q^{*} = \{r\in \mathbb{Q}: r < q\}$.Ent\~ao, $q^{*}$ \'e um corte que chamamos de racional. Os 
cortes que n\~ao s\~ao desse tipo se chamam cortes irracionais.
\end{example}
\begin{example}
  $\sqrt{2} = \{q\in \mathbb{Q}: q^{2} < 2\}\cup \{q\in \mathbb{Q}: q < 0\}$ \'e um corte irracional.
\end{example}
  Observe que se $\alpha$ \'e um corte, p \'e um ponto dele e q n\~ao \'e, ent\~ao $p < q$. Al\'em disso, se r pertence a $\alpha$
e $r < s,$ ent\~ao s n\~ao pertence ao corte.
\begin{def*}
  Diremos que $\alpha < \beta$, em que $\alpha$ e $\beta$ s\~ao cortes, se $\alpha\subsetneq \beta.$
\end{def*}
\begin{prop*}
  Se $\alpha, \beta, \gamma$ s\~ao cortes,
 \begin{itemize}
   \item[i)] $\alpha < \beta$ e $\beta < \gamma$ implica que $\alpha < \gamma$;
   \item[ii)] Exatamente uma das seguintes rela\c c\~oes \'e v\'alida: $\alpha < \beta$ ou $\alpha = \beta$ ou $\beta < \alpha$
   \item[iii)] Todo subconjunto n\~ao-vazio e limitado superiormente de $\mathbb{R}$ tem supremo.
 \end{itemize}
\end{prop*}
\end{document}
