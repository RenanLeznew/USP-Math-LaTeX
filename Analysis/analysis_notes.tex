   \documentclass{article}
 \usepackage{amsmath}
 \usepackage{amsthm}
 \usepackage{amssymb}
 \usepackage{pgfplots}
 \usepackage{amsfonts}
 \usepackage[margin=2.5cm]{geometry}
 \usepackage{graphicx}
 \usepackage[export]{adjustbox}
 \usepackage{fancyhdr}
 \usepackage[portuguese]{babel}
 \usepackage{hyperref}
 \usepackage{lastpage}
 \usepackage{mathtools}

 \pagestyle{fancy}
 \fancyhf{}

 \pgfplotsset{compat = 1.18}

 \hypersetup{
     colorlinks,
     citecolor=black,
     filecolor=black,
     linkcolor=black,
     urlcolor=black
 }
 \newtheorem*{def*}{\underline{Defini\c c\~ao}}
 \newtheorem*{theorem*}{\underline{Teorema}}
 \newtheorem*{lemma*}{\underline{Lema}}
 \newtheorem{example}{\underline{Exemplo}}
 \newtheorem*{proof*}{\underline{Prova}}
 \renewcommand\qedsymbol{$\blacksquare$}
 \newcommand{\Lin}[1]{Lin_{\mathbb{K}}({#1})}

 \rfoot{P\'agina \thepage \hspace{1pt} de \pageref{LastPage}}

 \title{Notas de An\'alise}
 \author{Renan Wenzel}
 \date{\today}

 \begin{document}
 \begin{figure}[ht]
	\minipage{0.76\textwidth}
		\includegraphics[width=4cm]{../icmc.png}
		\hspace{7cm}
		\includegraphics[height=4.9cm,width=4cm]{../brasao_usp_cor.jpg}
	\endminipage	
\end{figure}

\begin{center}
	\vspace{1cm}
	\LARGE
	UNIVERSIDADE DE S\~AO PAULO

	\vspace{1.3cm}
	\LARGE
	INSTITUTO DE CI\^ENCIAS MATEM\'ATICAS E COMPUTACIONAIS - ICMC

	\vspace{1.7cm}
	\Large
	\textbf{Notas de Aula de An\'alise}

	\vspace{1.3cm}
	\large
	\textbf{Renan Wenzel - 11169472}

	\vspace{1.3cm}
	\large
	\textbf{Alexandre Nolasco de Carvalho - andcarva@icmc.usp.br}

	\vspace{1.3cm}
	\today
\end{center}

 \newpage

 \tableofcontents

 \newpage

\section{Aula 01 - 13/03/2023}
\subsection{Motiva\c c\~ao}
\begin{itemize}
  \item Relembrar sistemas b\'asicos da matem\'atica;
  \item Relembrar propriedades b\'asicas das principais estruturas.
\end{itemize}

\subsection{Os N\'umeros Naturais}
  Os n\'umeros naturais s\~ao os que utilizamos para contar objetos, e s\~ao caracterizados pelos Axiomas de Peano:
 \begin{itemize}
   \item[1)] Todo n\'umero natural tem um \'unico sucessor; 
   \item[2)] N\'umeros naturais diferentes t\^em sucessores diferentes;
   \item[3)] Existe um \'unico n\'umero natural, zero (0), que n\~ao \'e sucessor de nenhum n\'umero natural.
   \item[4)] Seja $X \subseteq{\mathbb{N}}$ tal que $0\in{X}$ e, se n pertence a X, seu sucessor n+1 tamb\'em pertence 
a X. Ent\~ao, X = $\mathbb{N}.$ (Propriedade de Indu\c c\~ao).
 \end{itemize}

\begin{def*}
  Definimos a adi\c c\~ao por: $n + 0 = n, n\in \mathbb{N},\text{ e }n+(p+1) = (n+p)+1, p\in{\mathbb{N}}$. Al\'em disso,
a multiplica\c c\~ao \'e dada por: $n.0 = 0, n.(p+1) = n.p + n, n, p\in\mathbb{N}.$ Ou seja, sabendo somar ou multiplicar um n\'umero,
sabemos somar e multiplicar seu sucessor.
\end{def*}
  Com rela\c c\~ao ao quarto axioma, ele leva este nome porque um dos m\'etodos de demonstra\c c\~ao, conhecido como
prova por indu\c c\~ao. Nele, mostramos um caso base, o caso 0, e utilizamos a segunda parte para provar que, se um
resultado vale para o caso n, ele vale para n+1, portanto sendo verdadeiro para todos os naturais.

\begin{lemma*}
  Para todo n natural, 1 + n = n + 1.
\end{lemma*}
\begin{proof*}
  Note que o resultado \'e verdadeiro para n = 0. Suponha que o resultado seja v\'alido para n = k e mostremos que 
vale tamb\'em para n = k+1. Com efeito, segue pela propriedade de indu\c c\~ao e pela defini\c c\~ao de soma que
 $$
    1 + (k + 1) = (1 + k) + 1 =  (k + 1) + 1. 
 $$
 Segue que o resultado vale para todo n natural. \qedsymbol
\end{proof*}
  A seguir, mostramos a associatividade e a comutatividade, respectivamente, das opera\c c\~oes nos naturais.
\begin{lemma*}
  Para todo n, p, r naturais, (n + p) + r = n + (p + r).
\end{lemma*}
\begin{proof*}
  Note que o resultado \'e v\'alido trivialmente para r = 0 e r = 1. Suponha que o resultado seja v\'alido para
r = k e mostremos que vale tamb\'em para r = k + 1. Com efeito, pela hip\'otese de indu\c c\~ao e defini\c c\~ao de adi\c c\~ao,
  $$
    n + (p + (k + 1)) = n + ((p + k) + 1) = (n + (p + k)) + 1 = ((n + p) + k) + 1 = (n + p) + (k + 1).
  $$
  Segue o resultado por indu\c c\~ao. \qedsymbol
\end{proof*}
\begin{lemma*}
  Para todo n, p naturais, n + p = p + n. 
\end{lemma*}
\begin{proof*}
  Observe que j\'a mostramos o caso em que p = 1. Suponha que o resultado vale para p = k e vamos mostrar o caso
p = k + 1. De fato, pela hip\'otese de indu\c c\~ao e defini\c c\~ao de adi\c c\~ao, junto do lema de associatividade,
temos  
  $$
    n + (k + 1) = (n + k) + 1 = (k + n) + 1 = 1 + (k + n) = (1 + k) + n = (k + 1) + n.
  $$
  Por indu\c c\~ao, segue que isso vale para todo natural n. \qedsymbol
\end{proof*}

\begin{def*}
  Definimos uma ordem em $\mathbb{N}$ colocando que $m\leq{n}$ se existe p natural tal que n = m + p.
\end{def*}
  A rela\c c\~ao de ordem possui as seguintes propriedades:
 \begin{itemize}
   \item[i)] Reflexiva: Para todo n natural, $n\leq{n};$
   \item[ii)] Antissim\'etrica: Se $m\leq n$ e $n\leq m,$ ent\~ao $m = n;$
   \item[iii)] Transitiva: Se $m \leq n$ e $n \leq p$, ent\~ao $m\leq p;$
   \item[iv)] Dados m, n naturais, temos ou $m \leq n$, ou $n \leq m;$
   \item[v)] Se $m \leq n$ e p \'e um natural, ent\~ao $m + p\leq n\text{ e } mp\leq np$
 \end{itemize}

 \subsection{N\'umeros Inteiros e Racionais}
  Usualmente, construimos os inteiros a partir dos naturais tomando os pares ordenados de n\'umeros naturais
  com a seguinte identifica\c c\~ao (a, b) $\mathtt{\sim}$ (c, d) se a + d = b + c.
 \end{document}
