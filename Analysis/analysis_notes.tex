 \documentclass{article}
 \usepackage{amsmath}
 \usepackage{amsthm}
 \usepackage{amssymb}
 \usepackage{pgfplots}
 \usepackage{amsfonts}
 \usepackage[margin=2.5cm]{geometry}
 \usepackage{graphicx}
 \usepackage[export]{adjustbox}
 \usepackage{fancyhdr}
 \usepackage[portuguese]{babel}
 \usepackage{hyperref}
 \usepackage{lastpage}
 \usepackage{mathtools}

 \pagestyle{fancy}
 \fancyhf{}

 \pgfplotsset{compat = 1.18}

 \hypersetup{
     colorlinks,
     citecolor=black,
     filecolor=black,
     linkcolor=black,
     urlcolor=black
 }
 \newtheorem*{def*}{\underline{Defini\c c\~ao}}
 \newtheorem*{theorem*}{\underline{Teorema}}
 \newtheorem*{lemma*}{\underline{Lema}}
 \newtheorem*{prop*}{\underline{Proposi\c c\~ao}}
 \newtheorem{example}{\underline{Exemplo}}
 \newtheorem*{proof*}{\underline{Prova}}
 \newtheorem*{crl*}{\underline{Corol\'ario}}
 \renewcommand\qedsymbol{$\blacksquare$}
 \newcommand{\Lin}[1]{Lin_{\mathbb{K}}({#1})}

 \rfoot{P\'agina \thepage \hspace{1pt} de \pageref{LastPage}}

 \title{Notas de An\'alise}
 \author{Renan Wenzel}
 \date{\today}

 \begin{document}
 \begin{figure}[ht]
	\minipage{0.76\textwidth}
		\includegraphics[width=4cm]{../icmc.png}
		\hspace{7cm}
		\includegraphics[height=4.9cm,width=4cm]{../brasao_usp_cor.jpg}
	\endminipage	
\end{figure}

\begin{center}
	\vspace{1cm}
	\LARGE
	UNIVERSIDADE DE S\~AO PAULO

	\vspace{1.3cm}
	\LARGE
	INSTITUTO DE CI\^ENCIAS MATEM\'ATICAS E COMPUTACIONAIS - ICMC

	\vspace{1.7cm}
	\Large
	\textbf{Notas de Aula de An\'alise}

	\vspace{1.3cm}
	\large
	\textbf{Renan Wenzel - 11169472}

	\vspace{1.3cm}
	\large
	\textbf{Alexandre Nolasco de Carvalho - andcarva@icmc.usp.br}

	\vspace{1.3cm}
	\today
\end{center}

 \newpage

 \tableofcontents

 \newpage

\section{Aula 01 - 13/03/2023}
\subsection{Motiva\c c\~ao}
\begin{itemize}
  \item Relembrar sistemas b\'asicos da matem\'atica;
  \item Relembrar propriedades b\'asicas das principais estruturas ($\mathbb{N}, \mathbb{Z}, \mathbb{Q}$).
\end{itemize}

\subsection{Os N\'umeros Naturais}
  Os n\'umeros naturais s\~ao os que utilizamos para contar objetos, e s\~ao caracterizados pelos Axiomas de Peano:
 \begin{itemize}
   \item[1)] Todo n\'umero natural tem um \'unico sucessor; 
   \item[2)] N\'umeros naturais diferentes t\^em sucessores diferentes;
   \item[3)] Existe um \'unico n\'umero natural, zero (0), que n\~ao \'e sucessor de nenhum n\'umero natural.
   \item[4)] Seja $X \subseteq{\mathbb{N}}$ tal que $0\in{X}$ e, se n pertence a X, seu sucessor n+1 tamb\'em pertence 
a X. Ent\~ao, X = $\mathbb{N}.$ (Propriedade de Indu\c c\~ao).
 \end{itemize}

\begin{def*}
  Definimos a adi\c c\~ao por: $n + 0 = n, n\in \mathbb{N},\text{ e }n+(p+1) = (n+p)+1, p\in{\mathbb{N}}$. Al\'em disso,
a multiplica\c c\~ao \'e dada por: $n.0 = 0, n.(p+1) = n.p + n, n, p\in\mathbb{N}.$ Ou seja, sabendo somar ou multiplicar um n\'umero,
sabemos somar e multiplicar seu sucessor.
\end{def*}
  Com rela\c c\~ao ao quarto axioma, ele leva este nome porque um dos m\'etodos de demonstra\c c\~ao, conhecido como
prova por indu\c c\~ao. Nele, mostramos um caso base, o caso 0, e utilizamos a segunda parte para provar que, se um
resultado vale para o caso n, ele vale para n+1, portanto sendo verdadeiro para todos os naturais.

\begin{lemma*}
  Para todo n natural, 1 + n = n + 1.
\end{lemma*}
\begin{proof*}
  Note que o resultado \'e verdadeiro para n = 0. Suponha que o resultado seja v\'alido para n = k e mostremos que 
vale tamb\'em para n = k+1. Com efeito, segue pela propriedade de indu\c c\~ao e pela defini\c c\~ao de soma que
 $$
    1 + (k + 1) = (1 + k) + 1 =  (k + 1) + 1. 
 $$
 Segue que o resultado vale para todo n natural. \qedsymbol
\end{proof*}
  A seguir, mostramos a associatividade e a comutatividade, respectivamente, das opera\c c\~oes nos naturais.
\begin{lemma*}
  Para todo n, p, r naturais, (n + p) + r = n + (p + r).
\end{lemma*}
\begin{proof*}
  Note que o resultado \'e v\'alido trivialmente para r = 0 e r = 1. Suponha que o resultado seja v\'alido para
r = k e mostremos que vale tamb\'em para r = k + 1. Com efeito, pela hip\'otese de indu\c c\~ao e defini\c c\~ao de adi\c c\~ao,
  $$
    n + (p + (k + 1)) = n + ((p + k) + 1) = (n + (p + k)) + 1 = ((n + p) + k) + 1 = (n + p) + (k + 1).
  $$
  Segue o resultado por indu\c c\~ao. \qedsymbol
\end{proof*}
\begin{lemma*}
  Para todo n, p naturais, n + p = p + n. 
\end{lemma*}
\begin{proof*}
  Observe que j\'a mostramos o caso em que p = 1. Suponha que o resultado vale para p = k e vamos mostrar o caso
p = k + 1. De fato, pela hip\'otese de indu\c c\~ao e defini\c c\~ao de adi\c c\~ao, junto do lema de associatividade,
temos  
  $$
    n + (k + 1) = (n + k) + 1 = (k + n) + 1 = 1 + (k + n) = (1 + k) + n = (k + 1) + n.
  $$
  Por indu\c c\~ao, segue que isso vale para todo natural n. \qedsymbol
\end{proof*}

\begin{def*}
  Definimos uma ordem em $\mathbb{N}$ colocando que $m\leq{n}$ se existe p natural tal que $n = m + p. \square$
\end{def*}
  A rela\c c\~ao de ordem possui as seguintes propriedades:
 \begin{itemize}
   \item[i)] Reflexiva: Para todo n natural, $n\leq{n};$
   \item[ii)] Antissim\'etrica: Se $m\leq n$ e $n\leq m,$ ent\~ao $m = n;$
   \item[iii)] Transitiva: Se $m \leq n$ e $n \leq p$, ent\~ao $m\leq p;$
   \item[i] Dados m, n naturais, temos ou $m \leq n$, ou $n \leq m;$
   \item[v] Se $m \leq n$ e p \'e um natural, ent\~ao $n + p\leq n\text{ e } mp\leq np$
 \end{itemize}

 \subsection{N\'umeros Inteiros e Racionais}
  Usualmente, construimos os inteiros a partir dos naturais tomando os pares ordenados de n\'umeros naturais
com a seguinte identifica\c c\~ao (a, b) $\mathtt{\sim}$ (c, d) se a + d = b + c. Assim, podemos representar
  $$
  \mathbb{N} = \{(0, 0), (1, 0), (2, 0), (3, 0), \ldots\}, \quad -\mathbb{N}^* = \{\cdots, (0, 3), (0, 2), (0, 1)\}.
  $$
  Tomar o sucessor ser\'a somar 1 \`a primeira coordenada e, para os inteiros negativos, voltar a identificar (1, n) com (0, n-1).

  Os n\'umeros racionais s\~ao constru\'idos tomando o conjunto $\mathbb{Z}\times{\mathbb{Z}^*}$ e identificando os pares $(a, b)\mathtt{\sim}(c, d)$
para os quais ad = bc. Representamos um par (a, b) neste conjunto por $\displaystyle \frac{a}{b}.$ A soma e o produto em $\mathbb{Q}$
s\~ao dados, respectivamente, por: 
 \begin{align*}
   &\frac{a}{b} + \frac{c}{d} \coloneqq \frac{ad + bc}{bd} \\
   &\frac{a}{b}\cdot\frac{c}{d} \coloneqq \frac{ac}{bd}.
 \end{align*}
 Chamamos a adi\c c\~ao a opera\c c\~ao que a cada par $(x, y)\in \mathbb{Q}\times{\mathbb{Q}}$ associa sua soma $x+y\in \mathbb{Q}$
e chamamos multiplica\c c\~ao a opera\c c\~ao que a cada par $(x, y)\in \mathbb{Q}\times \mathbb{Q}$ associa seu produto $x.y\in \mathbb{Q}.$
A terna $(\mathbb{Q}, +, \cdot)$ satisfaz as propriedades de um corpo, i.e., 
 \begin{align*}
   &(A1) (x + y) + z = x + (y + z), \quad\forall x, y, z\in \mathbb{Q}\\
   &(A2) x + y = y + x, \quad\forall x, y\in \mathbb{Q}\\
   &(A3) \exists 0\in \mathbb{Q}: x + 0 = x, \quad\forall x\in \mathbb{Q}\\
   &(A4) \forall x\in \mathbb{Q}, \exists y\in \mathbb{Q} (y = -x): x + y = 0\\
   &(M1) (xy)z = x(yz), \quad\forall x, y, z\in \mathbb{Q}\\
   &(M2) xy = yx, \quad x, y\in \mathbb{Q}\\
   &(M3) \exists 1\in \mathbb{Q}: 1.x = x.1 = x, \quad\forall x\in \mathbb{Q}\\
   &(M4) \forall x\in \mathbb{Q}^*, \exists y = \frac{1}{x}\in \mathbb{Q}: x.y = 1\\
   &(D) x(y+z) = xy + xz,\quad\forall x, y, z\in \mathbb{Q}.
 \end{align*}
 \newpage

\section{Aula 02 - 15/03/2023}
\subsection{Motiva\c c\~oes}
\begin{itemize}
  \item Propriedades b\'asicas dos racionais;
  \item Constru\c c\~ao do corpo dos reais a partir dos racionais;
  \item Cortes de Dedekind.
\end{itemize}
\subsection{Propriedades de $\mathbb{Q}$ e sua Ordem}
  Com as 9 propriedades de corpo, conseguimos obter novas regras nos racionais, como a famosa lei do cancelamento:
 \begin{prop*}
   Em $\mathbb{Q},$ vale
   $$
    x + z = y + z \Rightarrow x = y
   $$
   e, se $z\neq{0}$,
   $$
    xz = yz \Rightarrow x = y 
   $$
 \end{prop*}
 \begin{proof*}
   \begin{align*}
   &x = x + 0 = x + (z + (-z)) = (x+z) + (-z) = (y + z) + (-z) = y + (z + (-z)) = y +0 = y\\
   &x = x.1 = x(z.\frac{1}{z}) = (xz)\frac{1}{z} = (yz)\frac{1}{z} = y (z \frac{1}{z}) = y.1 = y. \text{ \qedsymbol}
 \end{align*}
 \end{proof*}
 \begin{prop*}
  Os elementos neutros da adi\c c\~ao e multiplica\c c\~ao s\~ao \'unicos. Os elementos oposto e inverso tamb\'em o s\~ao.
 \end{prop*}
 \begin{prop*}
   Para todo x racional, x.0 = 0.
 \end{prop*}
 \begin{prop*}
   Para todo x racional, -x = (-1)x.
 \end{prop*}
 A maioria desses resultados acima seguem diretamente da lei do cancelamento. Suas demonstra\c c\~oes ficam como exerc\'icio.
\begin{def*}
  Diremos que 
  $$
    \frac{a}{b}\in \mathbb{Q} = \left\{\begin{array}{ll}
        \text{ n\~ao-negativo, } \quad ab\in \mathbb{N}\\
        \text{ positivo, } \quad ab\in \mathbb{N}, a\neq 0
      \end{array}\right.
  $$
  e diremos que 
  $$
    \frac{a}{b}\in \mathbb{Q} = \left\{\begin{array}{ll}
        \text{ n\~ao-positivo, } \quad \frac{a}{b} \text{ n\~ao for postivo}\\
        \text{ negativo, } \quad \frac{a}{b} \text{ n\~ao for n\~ao-negativo.}
      \end{array}\right.\square
  $$
\end{def*}
\begin{def*}
  Sejam x, y racionais. Diremos que x \'e menor e que y e escrevemos ``$x < y$'' se existir t racional positivo tal que 
  $$
    y = x + t.
  $$
  Neste mesmo caso, podemos dizer que y \'e maior que x, escrevendo ``$x > y$''. Em particular, temos $x > 0$ se x for positivo e 
$x < 0$ se x for negativo. 

  Ademais, se $x < y$ ou x = y, escrevemos ``$x \leq{y}$'' se existir racional t n\~ao-negativo tal que 
  $$
    y = x + t
  $$
  e, se $x > y$ ou x = y, escrevemos ``$x \geq{y}$'' caso exista racional t n\~ao-positivo com
  $$
    y = x + t. \square
  $$
\end{def*}
  A qu\'adrupla ($\mathbb{Q}, +, \cdot, \leq{}$) satisfaz as propriedades de um corpo ordenado, i.e., 
 \begin{align*}
   &(O1) x \leq{x}\forall x\in \mathbb{Q};\\
   &(O2) x\leq{y} \text{ e } y \leq{x}\Rightarrow x = y \forall x, y\in \mathbb{Q};\\
   &(O3) x \leq{y}, y \leq{z}\Rightarrow x \leq{z}\forall x, y, z\in \mathbb{Q};\\
   &(O4)\forall x, y \in \mathbb{Q}, x \leq{y} \text{ ou } y \leq{x};\\
   &(OA) x \leq{y}\Rightarrow x + z \leq{y + z};\\
   &(OM) x \leq{y} \text{ e } z \geq{0}\Rightarrow xz \leq{yz}. 
 \end{align*}
 \begin{prop*}
  Para quaisquer x, y, z, w no corpo ordenado dos racionais, valem 
 \begin{align*}
   &i.) x < y\Longleftrightarrow x + z < y + z \\
   &ii.) z > 0\Longleftrightarrow \frac{1}{z} > 0 \\
   &iii.) z > 0\Longleftrightarrow -z < 0 \\
   &iv.) z > 0\Rightarrow x < y\Longleftrightarrow xz < yz \\
   &v.) z < 0\Rightarrow x < y\Longleftrightarrow xz > yz \\
   &vi.) xz < yw\Longleftrightarrow \left\{\begin{array}{ll}
       0 \leq{x} < y \\
       0 \leq{z} < w
     \end{array}\right. \\
   &vii.) 0 < x < y\Longleftrightarrow 0 < \frac{1}{y} < \frac{1}{x}\\
   &viii.) x < y \text{ ou } x =y \text{ ou } x > y \\
   &ix.) xy = 0\Longleftrightarrow x = 0\text{ ou }y = 0. \\
   &x.) \left.\begin{array}{ll}
       x \leq{y} \\
       z \leq{w}
     \end{array}\right\}\Rightarrow x + z \leq{y + w}\\
   &xi.) \left.\begin{array}{ll}
       0 \leq{x} \leq{y}\\
       0 \leq{z} \leq{w}
   \end{array}\right\}\Rightarrow xz \leq{yw}.
 \end{align*}
 \end{prop*}
 \subsection{Incompletude de $\mathbb{Q}$}
  Os n\'umeros racionais podem ser representados por pontos em uma reta horizontal ordenada, chamada reta real. Se P for a representa\c c\~ao
de um n\'umero racional x, diremos que x \'e a abscissa de P. Note que nem todo ponto da reta real \'e racional. Para isso, considere
um quadrado de lado 1 e diagonal d. Pelo Teorema de Pit\'agoras, $d^{2} = 1^2 + 1^2 = 2.$ Agora, seja P a intersec\c c\~ao do eixo
x com a circunfer\^encia de centro em 0 e raio d. Mostremos que P \'e um ponto da reta com abscissa $x\not\in \mathbb{Q}.$
\begin{prop*}
  Seja a um inteiro. Ent\~ao, se a for \'impar, seu quadrado tamb\'em ser\'a. Al\'em disso, se a for par, seu quadrado tamb\'em \'e par.
\end{prop*}
\begin{prop*}
  A equa\c c\~ao $x^2 = 2$ n\~ao admite solu\c c\~ao racional.
\end{prop*}
  A ideia da prova \'e escrever um x na forma de fra\c c\~ao e chegar na contradi\c c\~ao de que tanto o numerador quanto o denominador
ser\~ao n\'umeros pares. Com isso, conclui-se que n\~ao existe racional irredut\'ivel com quadrado igual a 2, portanto n\~ao existe racional
satisfazendo a equa\c c\~ao. 

  Essa discuss\~ao mostra que existem v\~aos na ``reta'' dos racionais, requerindo a ado\c c\~ao de um novo corpo. Essa \'e a principal
motiva\c c\~ao por tr\'as dos n\'umeros reais, "preencher" os buracos deixados pelos racionais.
\begin{prop*}
  (Exerc\'icio.) Sejam $p_{1}, \ldots, p_{n}$ n\'umeros primos distintos. Ent\~ao, a equa\c c\~ao $x^{2} = p_{1}p_2\cdots p_{n}$ n\~ao
tem solu\c c\~ao racional.
\end{prop*}
  Vimos que os n\'umeros racionais com a sua adi\c c\~ao, multiplica\c c\~ao e rela\c c\~ao de ordem \'e um corpo ordenado. Nos interessamos,
tamb\'em, pelo corpo dos reais e dos racionais ($\mathbb{R}, \mathbb{C}$). De forma abstrata, um corpo \'e um conjunto n\~ao-vazio
 $\mathbb{F}$ em que est\~ao definidas duas opera\c c\~oes bin\'arias
 $$
    +:\mathbb{F}\times \mathbb{F}\rightarrow \mathbb{F}, \quad (x, y)\mapsto x + y
 $$
 e 
 $$
    \cdot: \mathbb{F}\times \mathbb{F}\rightarrow \mathbb{F}, \quad (x, y)\mapsto xy
 $$
 em que valem as oito propriedades vistas previamente para a defini\c c\~ao das opera\c c\~oes em $\mathbb{Q}$
 Se, ainda por cima, no corpo $\mathbb{F}$ est\'a definida uma ordem com propriedades an\'alogas \`as vistas para a qu\'adrupla
($\mathbb{Q}, +, \cdot, \leq{}$), diremos que ($\mathbb{F}, +, \cdot, \leq{}$) \'e um corpo ordenado.
\begin{def*}
  Diremos que um subconjunto A de um corpo $\mathbb{F}$ ordenado \'e limitado superiormente se existe um L neste corpo tal que $a \leq{L}$ para todo
a de A. 

  Definimos para um subconjunto limitado superiormente um n\'umero $\sup(A)\in \mathbb{F}$ como o menor limitante superior de
A, i.e., se $a \leq{\sup(A)}$ para todo a de A e se existe $f\in \mathbb{F}$ com $f < \sup(A),$ ent\~ao existe um a em A com $ f < a.$

  Por fim, diremos que um corpo ordenado \'e completo se todo subconjunto limitado superiormente possui supremo. $\square$
\end{def*}
  Nem todo subconjunto limitado superiormente de $\mathbb{Q}$ tem supremo, ou seja, $\mathbb{Q}$ n\~ao \'e completo.

\subsection{Os N\'umeros Reais ($\mathbb{R}$)}
  A ideia que iremos usar para construir o conjunto dos reais \'e que o conjunto dos n\'umeros reais preenche toda a reta real. Os Elementos
de $\mathbb{R}$ ser\~ao os subconjuntos de $\mathbb{Q}$ \`a esquerda de um ponto da reta real e ser\~ao chamados de cortes.
 \begin{def*}
   Um corte \'e um subconjunto $\alpha\subsetneq \mathbb{Q}$ com as seguintes propriedades:
  \begin{itemize}
    \item[i)] $\alpha\neq \emptyset$ e $\alpha \neq \mathbb{Q};$
    \item[ii)] Se $p\in \alpha$ e q \'e um racional com $q < p$, ent\~ao $q\in \alpha$ (todos os racionais \`a esquerda de um elemento
      de $\alpha$ est\~ao em $\alpha$);
    \item[iii)] Se $p\in \alpha$, existe um $r\in \alpha$ com $p < r$ ($\alpha$ n\~ao tem um maior elemento). $\square$
  \end{itemize}
 \end{def*}
  Essa ideia foi proposta inicialmente por Julius Wilhelm Richard Dedekind, um matem\'atico alem\~ao, em 1872, com o objetivo de encontrar uma explica\c c\~ao
e constru\c c\~ao elementar para os n\'umeros reais.
\begin{example}
  Se q \'e um racional, definimos $q^{*} = \{r\in \mathbb{Q}: r < q\}$.Ent\~ao, $q^{*}$ \'e um corte que chamamos de racional. Os 
cortes que n\~ao s\~ao desse tipo se chamam cortes irracionais.
\end{example}
\begin{example}
  $\sqrt{2} = \{q\in \mathbb{Q}: q^{2} < 2\}\cup \{q\in \mathbb{Q}: q < 0\}$ \'e um corte irracional.
\end{example}
  Observe que se $\alpha$ \'e um corte, p \'e um ponto dele e q n\~ao \'e, ent\~ao $p < q$. Al\'em disso, se r pertence a $\alpha$
e $r < s,$ ent\~ao s n\~ao pertence ao corte.
\begin{def*}
  Diremos que $\alpha < \beta$, em que $\alpha$ e $\beta$ s\~ao cortes, se $\alpha\subsetneq \beta.\square$
\end{def*}
\begin{prop*}
  Se $\alpha, \beta, \gamma$ s\~ao cortes,
 \begin{itemize}
   \item[i)] $\alpha < \beta$ e $\beta < \gamma$ implica que $\alpha < \gamma$;
   \item[ii)] Exatamente uma das seguintes rela\c c\~oes \'e v\'alida: $\alpha < \beta$ ou $\alpha = \beta$ ou $\beta < \alpha$
   \item[iii)] Todo subconjunto n\~ao-vazio e limitado superiormente de $\mathbb{R}$ tem supremo.
 \end{itemize}
\end{prop*}
\newpage

\section{Aula 03 - 17/03/2023}
\subsection{Motiva\c c\~oes}
\begin{itemize}
  \item Finalizar a constru\c c\~ao de $\mathbb{R}$ por cortes;
  \item Definir um corpo ordenado com base nos cortes;
\end{itemize}
\subsection{Cortes - Soma e Ordem}
  Coloquemos, para fins de conveni\^encia, $\mathbb{R}$ como a uni\~ao de todos os cortes.

  Vamos mostrar que os cortes racionais s\~ao, de fato, cortes. Considere, dado um racional q, $q^{*} = \{p\in \mathbb{Q}: p < q\}.$
Ele n\~ao pode completar todos os racionais, pois q + 1 n\~ao pertence a $q^{*}$. Al\'em disso, ele \'e n\~ao vazio, visto que
q-1 pertence a ele, mostrando a primeira propriedade dos cortes. 

  Ademais, se r pertence a $q^{*}$ e p \'e um racional menor que r, segue da transitividade da ordem que p \'e menor que
q, j\'a que r tamb\'em \'e. Assim, p pertence a $q^{*}$, mostrando a segunda propriedade dos cortes. 

  Por fim, dado um r em $q^{*},$ seja $s = \displaystyle \frac{r + q}{2}$. Ent\~ao, 
  $$
    r - \frac{r+q}{2} = \frac{r - q}{2} < 0,
  $$
tal que s \'e menor que r e, logo, pertence a $q^{*}$. Portanto, $q^{*}$ forma um corte.

  Daremos continuidade \`as atividades da aula anterior demonstrando a \'ultima proposi\c c\~ao vista.
\begin{prop*}
  Se $\alpha, \beta, \gamma$ s\~ao cortes,
 \begin{itemize}
   \item[i)] $\alpha < \beta$ e $\beta < \gamma$ implica que $\alpha < \gamma$;
   \item[ii)] Exatamente uma das seguintes rela\c c\~oes \'e v\'alida: $\alpha < \beta$ ou $\alpha = \beta$ ou $\beta < \alpha$
   \item[iii)] Todo subconjunto n\~ao-vazio e limitado superiormente de $\mathbb{R}$ tem supremo.
 \end{itemize}
\end{prop*}
  
 \begin{proof*}
   As duas primeiras partes seguem automaticamente da forma que definimos a ordem $\leq{}$ para os cortes. Resta mostrar a \'ultima.

   Vamos exibir o supremo explicitamente. Com efeito, seja $\mathcal{A}\subseteq{\mathbb{R}}$ um cole\c c\~ao de cortes
 limitada superiormente, i.e., existe um l em $\mathbb{R}$ tal que $\alpha \leq{l}$ para todo $\alpha$ em $\mathcal{A}.$
 Defina $\mathcal{S} = \bigcup_{\alpha\in \mathcal{A}}\alpha$. Mostremos que $\mathcal{S}$ \'e um corte. Com efeito,
 que $\mathcal{S}$ \'e n\~ao-vazio e diferente de $\mathbb{Q}$ \'e autom\'atico. Al\'em disso, dado q em $\mathcal{S}$ e $r < q,$
segue que $r\in \alpha_{0}$ para algum $\alpha_{0}$ em $\mathcal{A}.$

  Para ver que $\mathcal{S}$ \'e o supremo, suponha que $\mathcal{S}' < \mathcal{S}.$ Ent\~ao, existe r em $\mathcal{S}/\mathcal{S}'$.
Como r pertence a $\mathcal{S}$, r pertence a $\alpha_{0}$ para algum $\alpha_{0}\in \mathcal{A}.$ Logo, $\alpha_{0} > \mathcal{S}'.$ Portanto,
 $\mathcal{S}$ \'e o menor limitante superior de $\mathcal{A},$ ou seja, seu supremo. \qedsymbol
 \end{proof*}
 \begin{def*}
   Se $\alpha, \beta$ s\~ao cortes, definimos $\alpha + \beta$ como o conjunto de todos os racionais da forma $r + s$, com r em $\alpha$
 e s em $\beta$. Ademais, tome $0^* = \{s\in \mathbb{Q}: s < 0\}.\square$
 \end{def*}
  Vamos conferir a defini\c c\~ao, i.e., que $\alpha + \beta$ \'e um corte. Com efeito, $\alpha + \beta\neq\emptyset$, pois $\alpha\neq\emptyset$
e $\beta\neq\emptyset$. Al\'em disso, se p n\~ao pertence a $\alpha$ e q n\~ao pertence a $\beta$, mas r pertence a $\alpha$ e s a $\beta$,
ent\~ao $r + s < p + q$, tal que $p + q$ n\~ao pertence a $\alpha + \beta.$

  Al\'em disso, tome $r + s$ em $\alpha + \beta$ e $p < r + s$. Escreva $p = r' + s' = \underbrace{p - r}_{\in \beta} + \underbrace{r}_{\in \alpha}.$ Assim,
p pertence a $\alpha + \beta.$

  Por fim, tome $r + s$ em $\alpha + \beta$ e seja $r' > r$ (ambos em $\alpha$). Logo, $\underbrace{r' + s}_{\in \alpha + \beta} > r + s$. Portanto, 
$\alpha + \beta$ \'e um corte. 

  Fica de exerc\'icio mostrar que $0^*$ \'e um corte. Agora, mostremos os axiomas de corpo.

  A comutatividade e associatividade da adi\c c\~ao s\~ao triviais. Al\'em disso, dado r em $\alpha$ e s em $0^*,$
  $$
    r + s < r + 0 = r\Rightarrow r + s \in \alpha.
  $$
  Logo, $\alpha + 0^* \subseteq{\alpha}$. Por outro lado, dado r em $\alpha,$ existe $r'\text{ em } \alpha$ tal que $r' > r.$
  Assim, $r = \underbrace{r'}_{\alpha} + \underbrace{(r - r')}_{\in 0^*}$, pois $r - r' < 0.$ Portanto, $\alpha \subseteq{\alpha + 0^*}$ e
 $\alpha = \alpha + 0^*.$
 \begin{prop*}
   Dado um corte $\alpha$, existe um \'unico corte $\beta$ tal que $\alpha + \beta = 0^*$, em que 
   $$
    \beta = \{-p\in \mathbb{Q}: p - r\not\in \alpha \text{ para algum } r\in \mathbb{Q}, r > 0\}
   $$
   e \'e denotado por $-\alpha$. 
 \end{prop*}
\begin{proof*}
  Come\c camos mostrando que $\beta$ \'e um corte. Feito isso, vamos mostrar que $\beta + \alpha = 0^*.$

  Com efeito, dado -p em $\beta,$ segue que p n\~ao pertence a $\beta$. Caso $s = p + r$, -s pertence a $\beta$, tal que 
 $\beta$ \'e n\~ao-vazio. Ademais, se $p\in \alpha, -p\not\in \beta$, tal que $\beta$ \'e diferente de $\mathbb{Q}.$

  Al\'em disso, se $-q < -p$ e $-p\in \beta$, ent\~ao $-q \in \beta$. Por fim, se -p pertencer a $\beta$, $\displaystyle -p + \frac{r}{2}\in \beta$.
Portanto, $\beta$ \'e um corte. 

  Agora, vamos conferir o outro item. De fato, se r pertence a $\alpha$ e s a $-\alpha,$ ent\~ao $-s\not\in \alpha$ e $r < -s$, i.e.,
 $r + s < 0.$ Segue que $\alpha + (-\alpha) \subseteq{0^*}.$ Por outro lado, se $-2r\in 0^*$ com $r > 0,$ existe um inteiro n tal que
  $nr\in \alpha$ e $(n+1)r\not\in \alpha$. Escolha $p = -(n+2)r\in -\alpha$ e escreva $-2r = nr + p.$ Portanto, $0^*\subseteq{\alpha + (-\alpha)}$ e os
  conjuntos s\~ao iguais. \qedsymbol
\end{proof*}

\section{Aula 04 - 20/02/2023}
\subsection{Motiva\c c\~oes}
\begin{itemize}
  \item Definir multiplica\c c\~ao de cortes;
  \item Definir conceito de dist\^ancia entre n\'umeros de $\mathbb{R}$ 
\end{itemize}
\subsection{Cortes - Multiplica\c c\~ao}
\begin{def*}
  Se $\alpha, \beta$ s\~ao cortes,
  $$
    \alpha\beta = \left\{\begin{array}{ll}
      \alpha0^*, \quad\forall \alpha\in \mathbb{R}\\
      \{p\in\mathbb{Q}: \exists0 < r\in\alpha \text{ e }0 < s\in\alpha: p \leq rs\},\quad \alpha, \beta > 0^*\\
      (-\alpha)(-\beta), \quad \alpha, \beta < 0^*\\
      -[(-\alpha)\beta], \quad \alpha < 0^* e \beta > 0^*\\
      -[\alpha(-\beta)], \quad \alpha > 0^* e \beta < 0^*
      \end{array}\right.
  $$
  Definimos, tamb\'em, $1^*\{s\in\mathbb{Q}: s < 1\}$.
\end{def*}
\newpage

\subsection{$\mathbb{R}$ Como Corpo Ordenado Completo}
  Temos $\mathbb{Q}\subseteq{\mathbb{R}}$ e diremos que todo n\'umero que n\~ao \'e real \'e irracional. 
 \begin{theorem*}
   A qu\'adrupla $(\mathbb{R}, +, \cdot, \leq)$ satisfaz as condi\c c\~oes de corpo ordenado, de corpo e \'e completo.
 \end{theorem*}
\begin{def*}
  Seja $x\in \mathbb{R}.$ O m\'odulo, ou valor absoluto de x, \'e dado por
  $$
    |x| = \left\{\begin{array}{ll}
        x, \quad x \geq 0\\
        -x, x < 0
      \end{array}\right.
  $$
  Disto segue que $|x|\geq 0$ e $-|x|\leq x\leq |x|$ para todo x real.
\end{def*}
\begin{example}
  Mostre que $|x|^2 = x^2$, ou seja, o quadrado de um n\'umero real n\~ao muda quando se troca seu sinal.
\end{example}
\begin{example}
  A equa\c c\~ao $|x| = r$, com r maior que 0, tem como solu\c c\~oes apenas r e -r.
\end{example}
  Sejam P e Q dois pontos da reta real de abscissas x e y. Ent\~ao, a dist\^ancia de P a Q \'e definida por $|x-y|$. Assim,
 $|x-y|$ \'e a medida do segmento PQ. Em particular, como $|x|=|x-0|, |x|$ \'e a dist\^ancia de x a 0.
\begin{example}
  Seja r maior que 0. Ent\~ao, $|x| < r$ se, e somente se, $-r < x < r.$ Logo, o intervalo (-r, r) \'e o conjunto dos pontos reais
cuja dis\^ancia de 0 \'e menor que r.
\end{example}
\begin{example}
  Para quaisquer x, y reais, vale
  $$
    |xy| = |x||y|.
  $$
\end{example}
\begin{example}
  Para quaisquer x, y reais, temos
  $$
    |x+y| \leq |x| + |y|.
  $$
  Com efeito, somando $-|x|\leq{x}\leq{|x|}$ e $-|y|\leq{y}\leq{|y|},$ obtemos $-|x|-|y|\leq{x + y}\leq{|x|+|y|.}$\qedsymbol
\end{example}
\begin{def*}
  Um intervalo em $\mathbb{R}$ \'e um subconjunto de $\mathbb{R}$ que tem uma das seguintes formas:
 \begin{align*}
   &[a, b] = \{x\in \mathbb{R}: a\leq{x}\leq{b}\},\quad \text{ (Intervalo fechado.) }\\
   &(a, b) = \{x\in \mathbb{R}: a < x < b\},\quad \text{ (Intervalo aberto.) }\\
   &[a, b) = \{x\in \mathbb{R}: a \leq{x} < b\}\\
   &(a, b] = \{x\in \mathbb{R}: a < x \leq{b}\}\\
   &(-\infty, b] = \{x\in \mathbb{R}: x\leq{b}\}\\
   &(-\infty, b) = \{x\in \mathbb{R}: x > b\}\\
   &[a, +\infty) = \{x\in \mathbb{R}: x\geq{a}\}\\
   &(a, +\infty) = \{x\in \mathbb{R}: a < x\}\\
   &(-\infty, +\infty) = \mathbb{R}.
 \end{align*}
\end{def*}
\begin{def*}
  Um conjunto A de $\mathbb{R}$ \'e dito limitado se existir L positivo tal que $|x| \leq L$ para todo x em A.
\end{def*}
\begin{prop*}
  Um conjunto A de $\mathbb{R}$ \'e limitado se, e s\'o se, existir L positivo, tal que A est\'a contido em $[-L, L]$
\end{prop*}
\begin{example}
 \begin{itemize}
   \item[a)] $A = [0, 1]$ \'e limitado;
   \item[b)] $\mathbb{N}$ n\~ao \'e limitado;
   \item[c)] $B = \biggl\{\displaystyle \frac{2^n-1}{2^n}: n\in \mathbb{N}\biggr\}$ \'e limitado; 
   \item[d)] $C = \biggl\{\displaystyle \frac{2^n-1}{2^n}: n\in \mathbb{N}^{*}\biggr\}$ \'e limitado. 
 \end{itemize}
\end{example}
\begin{def*}
  Seja $A\subseteq{\mathbb{R}}$.
 \begin{itemize}
   \item A ser\'a dito limitado superiormente se existir um L real tal que $x\leq L$ para todo x de A. Diremos que L \'e o limitante superior de A.;
   \item A ser\'a dito limitado inferiormente se existir um L real tal que $x\geq L$ para todo x de A. Diremos que L \'e o limitante inferior de A.;
 \end{itemize}
 Caso ambos ocorram, diremos que A \'e limitado.
\end{def*}
\begin{def*}
  Seja A um subconjunto dos reais limitado superiormente e n\~ao-vazio. Diremos que $\overline{L}$ \'e o supremo de A se for um limitante superior
  e para qualquer outro limitante superior L de A, tivermos $\overline{L}\leq L$. Quando o supremo pertencer ao conjunto, chamaremos ele de m\'aximo.
\end{def*}
  Vimos que todo subconjunto n\~ao-vazio e limitado superiormente de $\mathbb{R}$ tem supremo.
\begin{def*}
  Seja A um subconjunto dos reais limitado inferiormente e n\~ao-vazio. Diremos que $\overline{l}$ \'e o \'infimo de A se for um limitante inferior
  e para qualquer outro limitante inferior l de A, tivermos $\overline{l}\geq l$. Quando o \'infimo pertencer ao conjunto, chamaremos ele de m\'inimo.
\end{def*}
\begin{prop*}
  Dado um subconjunto A dos reais n\~ao-vazio e limitado superiormente, $L = \sup{A}$ se, e somente se, 
 \begin{itemize}
   \item[a)] L for limitante superior de A;
   \item[b)] para todo $\epsilon > 0$, existe $a\in A$ tal que $a > L - \epsilon.$ 
 \end{itemize}
\end{prop*}
\begin{theorem*}
  O conjunto $A=\{nx: n\in\mathbb{N}\}$ ser\'a ilimitado para todo x n\~ao-nulo.
\end{theorem*}
\begin{proof*}
  Se $x > 0$, suponhamos, por absurdo, que A seja limitado e seja L seu supremo. Como $x > 0$, deve existir um natural m tal que
  $$
    L - x < mx \quad\text{ e } L = \sup{A} < (m+1)x.
  $$
  Mas isso \'e uma contradi\c c\~ao. 

  A prova para $x < 0$ \'e an\'aloga e ser\'a deixada como exerc\'icio. \qedsymbol 
\end{proof*}
 \begin{example}
  \begin{itemize}
    \item[a)] Considere $A = [0, 1).$ Ent\~ao, -2 e 0 s\~ao limitantes inferiores de A enquanto $1, \pi, 101$ s\~ao limitantes
  superiores de A.
    \item[b)] $\mathbb{N}$ n\~ao \'e limitado, mas \'e limitado inferiormente por 0, visto que $0\leq{x}$ para todo x natural.
    \item[c)] $B=\{x\in \mathbb{Q}: x\leq{\sqrt{2}}\}$ n\~ao \'e limitado, mas \'e limitado superiormente por L, em que $L\geq{2}.$ \qedsymbol
  \end{itemize}
 \end{example}
\begin{crl*}
  Para todo $\epsilon > 0$, existe um n natural tal que 
  $$
  \frac{1}{n} < \epsilon, \quad \frac{1}{n\sqrt{2}}<\epsilon, \quad 2^{-n} < \epsilon.
  $$
\end{crl*}
  J\'a sabemos, por constru\c c\~ao, que entre dois n\'umeros reais distintos existe um n\'umero racional. O mesmo vale para irracionais.
De fato, sejam a e b n\'umeros reais distintos. Se $a < b$ e $\epsilon = b - a > 0$, do corol\'ario, tome um natural n tal que
$\displaystyle \frac{1}{n\sqrt{2}} < \frac{1}{n} < \epsilon.$ Se a \'e racional, $r = \displaystyle a + \frac{1}{n\sqrt{2}}$ \'e irracional e
 $a < r < b.$ Por outro lado, se a \'e irracional, $r =\displaystyle a + \frac{1}{n}$ tamb\'em \'e, tal que $a < r < b.$ Portanto,
 dados dois n\'umeros reais quaisquer, existe um n\'umero irracional.
\begin{crl*}
  Qualquer intervalo aberto e n\~ao-vazio cont\'em infinitos n\'umeros racionais e infinitos irracionais.
\end{crl*}
\begin{crl*}
  Se $A = \biggr\{\displaystyle \frac{1}{n}: n\in \mathbb{N}^*\biggl\}$, ent\~ao $\inf A = 0.$
\end{crl*}
\begin{example}
 \begin{align*}
   &(a) \text{ Seja }A = (0, 1]. \text{ Ent\~ao, } \inf{A} = 0, \max{A} = 1; \\
   &(b) \sqrt{2} = \{r\in\mathbb{Q}: r \leq 0\}\cup \{r\in\mathbb{Q}: r^2 < 2\}\text{ \'e um corte.} 
   &(c) C = \{x\in\mathbb{Q}: x^2 < 2\} \Rightarrow \sqrt{2}=\sup{C}\text{ e }\inf{C} = -\sqrt{2}.
 \end{align*}
 Vamos analisar mais cautelosamente o item b e prov\'a-lo. De fato, se $0 < r\in \mathbb{Q}$ e $r^2 < 2,$ existe n natural tal que 
 $[2r + 1]\frac{1}{n} < 2 - r^2$ e $(r + \frac{1}{n})^2 < 2.$ As outras propriedades de cortes s\~ao triviais. 

 Olhando tamb\'em para o item C, como todos seus elementos s\~ao racionais saitsfazendo $x^2 < 2, \sqrt{2}$\'e um limitante superior de C.
 Agora, se $0 < L < \sqrt{2}$, existe um racional $r\in(L, \sqrt{2})$ e $L^2 < r^2 < 2.$ Logo, r pertence a C e L n\~ao \'e limitante superior para C,
 provando o resultado.
\end{example}
\begin{prop*}
  Se A \'e um subconjunto n\~ao-vazio e limitado inferiormente, ent\~ao $-A = \{-x: x\in A\}$ ser\'a limitado superiormente e 
  $\inf{A} = -\sup{(-A)}$. Analogamente, se for limitado superiormente, o conjunto -A ser\'a limitado inferiormente, e $\sup{A}=-\inf{(-A)}$
\end{prop*}
\begin{proof*}
  Se A for limitado inferiormente, $\inf{(A)} \leq{x}$ para todo x de A e, dado $\epsilon > 0$, deve existir a em A tal que
  $a < \inf{(A)} + \epsilon$, ou, trocando o sinal, $-\inf{(A)} \geq{-x}$ para todo -x de -A e, dado $\epsilon > 0,$ deve existir
  $b = -a$ em -A tal que $-a > -\inf{(A)} - \epsilon.$

  Com isso, segue que -A ser\'a limitado superiormente, e $\sup{(-A)} = -\inf{(A)}.$ A outra prova fica como exerc\'icio. \qedsymbol
\end{proof*}
\begin{crl*}
  Todo conjunto A n\~ao-vazio e limitado inferiormente de $\mathbb{R}$ tem \'infimo.
\end{crl*}
\begin{crl*}
  Todo conjunto A n\~ao-vazio e limitado de $\mathbb{R}$ tem \'infimo e supremo.
\end{crl*}
\begin{def*}
  Uma vizinhan\c ca de um n\'umero real a \'e qualquer intervalo aberto da reta contendo a.
\end{def*}
\begin{example}
  Se $\delta > 0, V_{\delta}(a)\coloneqq(a - \delta, a + \delta)$ \'e uma vizinhan\c ca de a que ser\'a chamada de $\delta-$vizinhan\c ca de a. 
\end{example}
\begin{def*}
  Sejam A um subconjunto de $\mathbb{R}$ e b um n\'umero real. Se, para todo $\delta > 0$, existir $a\in V_{\delta}(b)\cap{A}, a\neq b,$ 
ent\~ao b ser\'a dito um ponto de acumula\c c\~ao de A.
\end{def*}
\begin{example}
 \begin{itemize}
   \item[a)]O conjunto dos pontos de acumula\c c\~ao de (a, b) \'e [a, b];
    \item[b)]Seja $B = \mathbb{Z}.$ Ent\~ao, B n\~ao tem pontos de acumula\c c\~ao; 
      \item[c)] Subconjuntos finitos de $\mathbb{R}$ n\~ao t\^em pontos de acumula\c c\~ao;
        \item[d)] O conjunto dos pontos de acumula\c c\~ao de $\mathbb{Q}$ \'e $\mathbb{R}$.
 \end{itemize}
\end{example}
\begin{def*}
  Seja $B\subseteq{\mathbb{R}}$. Um ponto b de B ser\'a dito um ponto isolado de B, se existir $\delta > 0$ tal que $V_{\delta}(b)$
n\~ao cont\'em pontos de B distintos de b. $\square$
\end{def*}
\begin{example}
  Seja $B=\{1, \frac{1}{2}, \frac{1}{3}, \cdots\}$. Ent\~ao, o conjunto dos pontos de acumula\c c\~ao de B \'e $\{0\}$ e o conjunto dos pontos
isolados de B \'e o pr\'oprio conjunto B.
\end{example}
  Observe que existem conjuntos infinitos sem pontos de acumula\c c\~ao, tal como $\mathbb{Z}.$ Por outro lado, todo conjunto infinito e limitado possui
pelo menos um ponto de acumula\c c\~ao.
\begin{theorem*}
  Se A \'e um subconjunto infinito e limitado de $\mathbb{R},$ ent\~ao A possui pelo menos um ponto de acumula\c c\~ao.
\end{theorem*}
\begin{proof*}
  Se $A\subseteq{[-L, L]}$ e $[a_{n}, b_{n}], n\in \mathbb{N}$ s\~ao escolhidos tais que $[a_{n+1}, b_{n+1}]\subseteq{[a_{n}, b_{n}]}, b_{0} = -a_{0} = L,
  b_{n} - a_{n} = \frac{2L}{2^{n}}, n\in \mathbb{N}^*$ e $[a_{n}, b_{n}]$ cont\'em infinitos elementos de A. Seja $a = \sup{\{a_{n}: n\in \mathbb{N}\}}$.

  Note que $[a_{n}, b_{n}]\subseteq{a_{j}, b_{j}}, j\leq{n}$ e $[a_{j}, b_{j}]\subseteq{[a_{n}, b_{n}]}, j>n.$ Em qualquer um dos casos, $a_{n}\leq{b_{j}}$
para todo $j\in \mathbb{N}$. Logo, $a \leq{b_{j}}, j\in \mathbb{N}.$ Segue que $a_{n}\leq{a=\sup \{s_{n}: n\in \mathbb{N}\}}\leq{b_{n}}$ para todo
$n\in \mathbb{N}$ e $a\in\displaystyle\bigcap_{n\geq{1}}[a_{n}, b_{n}]$. Dado $\delta > 0,$ escolha $n\in \mathbb{N}$ tal que $\frac{2L}{2^{n}}<\delta.$ Segue
que $a\in[a_{n}, b_{n}]\subseteq{(a-\delta, a+\delta) = V_{\delta}(a)}$ e a \'e ponto de acumula\c c\~ao de A. \qedsymbol
\end{proof*}
\newpage

\section{Aula 05 - 22/03/2023}
\subsection{Motiva\c c\~oes}
\begin{itemize}
  \item Sequ\^encias de N\'umeros Reais;
  \item Converg\^encia de Sequ\^encias;
\end{itemize}
\subsection{Sequ\^encias de N\'umeros Reais}
\begin{def*}
  Uma sequ\^encia \'e uma fun\c c\~ao definida no conjunto dos n\'umeros reais que, para cada n natural, associa um n\'umero real $a_{n}$.
  \begin{align*}
    \mathbb{N}&=\{0, 1, 2, \cdots\}\\
              &f:\mathbb{N}\rightarrow \mathbb{R}\\
              &n\mapsto a_{n}.
  \end{align*}
  Denotamos a fun\c c\~ao por $\{a_{n}\}\square$
\end{def*}
 \begin{example}
   Sendo $a_{n}=\frac{f1}{n+1}$ para todo n natural, temos a sequ\^encia $\{1, \frac{1}{2}, \frac{1}{3}, \cdots\}$. 
 \end{example}
\begin{example}
  Sendo $a_{n} = 6$ para todo n natural, temos a sequ\^encia constante 
  $$
    \{6, 6, 6,\cdots\}.
  $$
\end{example}
\begin{example}
  Coloque $a_{2n+1} = 7, a_{2n}=4$ para todo n natural. Temos 
  $$
    \{4, 7, 4, 7, \cdots\}
  $$
\end{example}
  Consideremos as sequ\^encias
  $$
    \alpha_{n} = n, \quad \beta_{n} = (-1)^{n},\quad \text{ e } \gamma_{n} = \frac{1}{n}.
  $$
  Como fun\c c\~oes, elas podem ter os gr\'aficos tra\c cados, mas n\~ao s\~ao muito significativos, visto que consistem em
colet\^aneas de pontos discretos. Ademais, note que a sequ\^encia $(\alpha_{n})$ ``diverge'' para infinito, a sequ\^encia
$(\beta_{n})$ ``oscila'' e a sequ\^encia $(\gamma_{n})$ ``converge para 0''. Precisamente, 
 \begin{def*}
   A sequ\^encia $\{a_{n}\}$ \'e dita convergente com limite l se, para todo $\epsilon > 0$, existe um natural
N dependendo de $\epsilon (N = N(\epsilon)\in \mathbb{N})$ tal que $n > N$ implica em $|a_{n} - l|< \epsilon.$
  Ou seja, a partir de um certo N, os $a_{n}$ est\~ao no intervalo $(l-\epsilon, l+\epsilon)$ e, como $\epsilon$
\'e arbitr\'ario, os $a_{n}$ se juntam em torno de l. Disto, segue que a condi\c c\~ao exigida equivale a
  $$
  l - \epsilon < a_{n} < l + \epsilon, \quad n\geq{N}.
  $$
Denotamos esse fen\^omeno por $\displaystyle\lim_{n\to\infty}a_{n} = l$, ou $a_{n}\rightarrow l$, ou $a_{n}\overbracket[0pt]{\longrightarrow}^{n\to \infty}a.\square$.
\end{def*}
\begin{example}
  $\circ{}\frac{1}{n}\rightarrow0, n\rightarrow\infty$. De fato, dado $\epsilon > 0$, da propriedade arquimediana, segue que 
existe um N natural tal que $N\epsilon > 1.$ Logo, para todo $n\geq{N},$ temos 
  $$
  0 - \epsilon < \frac{1}{n}\leq{\frac{1}{N}} < 0 + \epsilon.
  $$

  $\circ \frac{n}{n+1}\rightarrow 1, n\rightarrow\infty$. Com efeito, dado $\epsilon > 0,$ queremos encontrar N natural n\~ao-nulo tal que
se n \'e maior que N, temos 
  $$
    \biggl|\frac{n}{n+1} - 1\biggr| < \epsilon.
  $$
  No entanto, $|\frac{n}{n+1}-1| = \frac{1}{n+1}$ e, da propriedade Archimediana, existe N em $\mathbb{N}^{\times}$ tal que
  $(N+1)\epsilon > 1$. Logo, se $n\geq{N},$
  $$
    1 - \epsilon < \frac{n}{n+1} < 1 + \epsilon.
  $$
\end{example}
\begin{def*}
  Uma sequ\^encia $\{a_{n}\}$ ser\'a divergente quando ela n\~ao for convergente.
 \begin{itemize}
    \item[I)] Sequ\^encia divergente para $+\infty:$ Este caso ocorre se dado $K > 0$, existe N natural tal que se $n > N,
a_{n} > K.$
    \item[II)] Sequ\^encia divergente para $-\infty:$ Acontece quando dado $K > 0$, existe N natural tal que se $n > N, 
a_{n} < -K.$
    \item[III)]Sequ\^encia oscilante: Por fim, ocorre quando a sequ\^encia diverge, mas nem para $+\infty$ e nem para $-\infty.\square$
 \end{itemize}
\end{def*}
  Note que, como sequ\^encias s\~ao fun\c c\~oes, podemos multiplic\'a-las por constante, somar, dividir e multiplicar por outras sequ\^encia. De fato,
 \begin{def*}
   Dadas sequ\^encias $\{a_{n}\}, \{b_{n}\}$ e um n\'umero real c, deifnimos 
  \begin{align*}
    &i) \{a_{n}\} + \{b_{n}\} = \{a_{n} + b_{n}\}\\
    &ii) c\{a_{n}\} = \{c \cdot a_{n}\}\\
    &iii) \{a_{n}\}\{b_{n}\} = \{a_{n}b_{n}\}\\
    &iv) \text{ Se }b_{n}\neq0\forall n\in \mathbb{N}, \frac{\{a_{n}\}}{\{b_{n}\}} = \biggl\{\frac{a_{n}}{b_{n}}\biggr\}\square
  \end{align*}
 \end{def*}
\begin{def*}
  Seja $\{a_{n}\}$ uma sequ\^encia de n\'umero reais. Diremos que $\{a_{n}\}$ \'e limitada se sua imagem for um subconjunto
limitado de $\mathbb{R}.\square$
\end{def*}
\begin{theorem*}
  Seja $\{a_{n}\}$ uma sequ\^encia de n\'umeros reais.
 \begin{itemize}
   \item[a)] $a_{n}\overbracket[0pt]{\longrightarrow}^{n\rightarrow\infty}a$ se, e somente, toda vizinhan\c ca de a cont\'em todo, exceto uma poss\'ivel quantidade
  finita de $a_{n}$'s.
   \item[b)] O limite \'e \'unico.
   \item[c)] Se $\{a_{n}\}$ \'e convergente, ent\~ao $\{a_{n}\}$ \'e limitada
   \item[d)] Se $a_{n}\overbracket[0pt]{\longrightarrow}^{n\rightarrow\infty}a$, exite N natural tal que $a_{n} > 0$ para todo $n\geq{N}.$
   \item[e)] Se $A\subseteq{\mathbb{R}}$ e a \'e um ponto de acumula\c c\~ao de A, ent\~ao existe uma sequ\^encia $\{a_{n}\}$ de elementos
    de A que converge para a.
 \end{itemize}
\end{theorem*}
\begin{proof*}
  O item a \'e trivial. Mostremos a unicidade do limite: Suponha que $a_{n}$ converge para a e para b, com a diferente de b. Ent\~ao,
dado $\epsilon > 0$, existem naturais $N_{1}, N_{2}$ tais que se $n\geq{N_{1}}, |a_{n}-a|<\epsilon$ e se $n\geq{N_{2}}, |a_{n} - b| < \epsilon.$
Tome $N = \min{N_{1}, N_{2}}$ e suponha que $n \geq{N}.$ Ent\~ao, temos 
 $$
  |b - a| \leq{|b - a_{n}| + |a_{n} - a|} = |b - a_{n}| + |a - a_{n}| < 2\epsilon.
 $$
 (\textit{P.S.}: pode ser boa pr\'atica tomar $\frac{\epsilon}{2}$ ao inv\'es de $\epsilon$, pois assim obtemos $|b-a|<\frac{2\epsilon}{2}=\epsilon.$)

 Como $\epsilon$ \'e abritr\'ario, podemos selecionar $\epsilon$ infinitamente pr\'oximo de 0. Portanto, b = a.

 Para o item c, suponha que $a_{n}$ converge para a, isto \'e, dado $\epsilon > 0, \epsilon = 1$ em particular, existe 
 $N\in \mathbb{N}$ tal que se $n \geq{N}, |a_{n} - a| < 1$. Logo, $a_{n}\in(a - 1, a + 1)$ para n maior que N suficientemente grande.
 Restam os N-1 primeiros elementos da sequ\^encia. Assim, tome $R = \max{\biggl\{|a_{1}|, \cdots, |a_{N-1}|, |a + 1|, |a - 1|\biggr\}}$. Deste modo,
 $a_{n}\in[-R, R]$ para todo n natural. 

 Com rela\c c\~ao ao item d, basta tomar $\epsilon = \frac{a}{2} > 0.$ 

 Por fim, quanto ao item e, suponha o que \'e dito no enunciado. Como a \'e ponto de acumula\c c\~ao, dado $\epsilon > 0,$ existe
 $a'\in{A}, a'\neq a$ tal que 
 $$
  a'\in V_{\epsilon}(a) = (a - \epsilon, a + \epsilon).
 $$
 Logo, tomadno $\epsilon = \frac{1}{n},$ podemos encontrar $a_{n}\in A, a_{n}\neq a$ tal que $a_{n}\in\biggl(a-\frac{1}{n}, a + \frac{1}{n}\biggr)$. A sequ\^encia
 $\{a_{n}\}$ converge para a. De fato, dado $\epsilon > 0$, tome N natural tal que $N\epsilon > 1.$ Assim, se $n\geq{N}, a_{n}\in(a-\frac{1}{n}, a+\frac{1}{n})\subseteq{a-\epsilon}, a+\epsilon)$.
 Portanto, $a_{n}\rightarrow a.$ \qedsymbol
\end{proof*}
 \begin{theorem*}
   Seja $a_{n}\overbracket[0pt]{\longrightarrow}^{n\to\infty}a, b_{n}\overbracket[0pt]{\rightarrow}^{n\to\infty}b$ e c um n\'umero real. Ent\~ao,
  \begin{align*}
    &a) a_{n} + b_{n}\overbracket[0pt]{\longrightarrow}^{n\to\infty} a + b.\\
    &b) ca_{n}\overbracket[0pt]{\longrightarrow}^{n\to\infty} ca\\
    &c) a_{n}b_{n}\overbracket[0pt]{\longrightarrow}^{n\to\infty} ab\\
    &d)\text{Se} b\neq0, b_{n}\neq0\forall n\in \mathbb{N}, \frac{a_{n}}{b_{n}}\overbracket[0pt]{\longrightarrow}^{n\to\infty}\frac{a}{b}.
  \end{align*}
 \end{theorem*}
\begin{proof*}
  Item c). Suponha $a_{n}\overbracket[0pt]{\longrightarrow}^{n\to \infty}a, b_{n}\overbracket[0pt]{\longrightarrow}^{n\to \infty}b$. Note que 
  $$
    |a_{n}b_{n} - ab| = a_{n}b_{n} - a_{n}b + a_{n}b - ab \leq{|a_{n}||b_{n}-b| + |b||a_{n}-a|}
  $$
  Como $\{a_{n}\}$ \'e convergente, ela \'e limitada pelo teorema anterior. Assim, existe $M > 0$ tal que $|a_{n}|\leq{M}$ para todo n natural, tal que
  Assim, 
  $$
    |a_{n}b_{n} - ab| \leq{|a_{n}||b_{n} - b| + |b||a_{n} - a|} \leq{M|b_{n} - b| + (|b| + 1)|a_{n} - a|}.
  $$
  Agora, dado $\epsilon > 0,$ existem naturais $N_{1}, N_{2}$ tais que 
 \begin{align*}
   &|a_{n}-a| < \frac{\epsilon}{2(|b|+1)},\quad\forall n\geq{N_{1}}\\
   &|b_{n}-b| < \frac{\epsilon}{2M},\quad\forall n\geq{N_{2}}.
 \end{align*}
 Logo, tomando $N = \max\{N_{1}, N_{2}\},$ se $n\geq{N},$
 $$
  |a_{n}b_{n}-ab| < \frac{\epsilon}{2} + \frac{\epsilon}{2} = \epsilon.
 $$
 Portanto, $a_{n}b_{n}\overbracket[0pt]{\longrightarrow}^{n\to \infty}ab.$ \qedsymbol
\end{proof*}
\begin{def*}
   Seja $\{a_{n}\}$ uma sequ\^encia. Diremos que $\{b_{n}\}$ \'e uma subsequ\^encia de $\{a_{n}\}$ se existir uma fun\c c\~ao
estritamente crescente $s:\mathbb{N}\rightarrow \mathbb{N}$ tal que $b_{k} = a_{s(k)}$ para todo k natural. $\square$
\end{def*}
\begin{def*}
  Seja $\{a_{n}\}$ uma sequ\^encia. Diremos que $\{a_{n}\}$ \'e de Cauchy se, dado $\epsilon > 0$, existe um natural
  $N = N(\epsilon)$ tal que $|a_{n}-a_{m}| < \epsilon$ para todo $n, m\geq{N}.\square$
\end{def*}
\begin{theorem*}
 \begin{itemize}
   \item[a)]Uma sequ\^encia \'e convergente se, e somente se, toda subsequ\^encia dela converge para o mesmo limite.
   \item[b)] Toda sequ\^encia convergente \'e de Cauchy;
   \item[c)] Toda sequ\^encia limitada tem subsequ\^encia convergente;
   \item[d)] Toda sequ\^encia de Cauchy \'e limitada;
   \item[e)] Toda sequ\^encia de Cauchy que tem subsequ\^encia convergente \'e convergente.
   \item[f)] Toda sequ\^encia de Cauchy \'e convergente;
   \item[g)] Toda sequ\^encia crescente e limitada \'e convergente;
   \item[h)] Toda sequ\^encia decrescente e limitada \'e convergente.
 \end{itemize}  
\end{theorem*}
\newpage

\section{Aula 06 - 24/03/2023}
\subsection{Motiva\c c\~oes}
 \begin{itemize}
   \item Provar o teorema da aula anterior;
   \item Exemplos.
 \end{itemize}
\subsection{Propriedades de Sequ\^encias}
  Recapitulemos o teorema da aula anterior:
\begin{theorem*}
 \begin{itemize}
   \item[a)]Uma sequ\^encia \'e convergente se, e somente se, toda subsequ\^encia dela converge para o mesmo limite.
   \item[b)] Toda sequ\^encia convergente \'e de Cauchy;
   \item[c)] Toda sequ\^encia limitada tem subsequ\^encia convergente;
   \item[d)] Toda sequ\^encia de Cauchy \'e limitada;
   \item[e)] Toda sequ\^encia de Cauchy que tem subsequ\^encia convergente \'e convergente.
   \item[f)] Toda sequ\^encia de Cauchy \'e convergente;
   \item[g)] Toda sequ\^encia crescente e limitada \'e convergente;
   \item[h)] Toda sequ\^encia decrescente e limitada \'e convergente.
   \end{itemize}
 \end{theorem*}
 \begin{proof*}
a.) $\Leftarrow)$ Se toda subsequ\^encia de $\{a_{n}\}$ converge, ent\~ao $\{a_{n}\}$ converge, pois ela \'e uma subsequ\^encia de si mesma (basta tomar $s:\mathbb{N}\rightarrow \mathbb{N}, s(n) = n.$)'

$\Rightarrow)$ Suponha que $a_{n}\overbracket[0pt]{\longrightarrow}^{n\to \infty}l$ e $\{b_{n}\}$ \'e uma subsequ\^encia de $\{a_{n}\}$, existe 
$s:\mathbb{N}\rightarrow \mathbb{N}$ estritamente crescente tal que $b_{k} = a_{s(k)}.$ Dado $\epsilon > 0$, seja N o natural tal que
 $|a_{n} - l| < \epsilon$ para todo $n\geq{N}.$ Note que $s(n)\geq{n},$ tal que se $n\geq{N},$ ent\~ao $s(n)\geq{N},$ de forma que 
 $|a_{s(n)} - l| < \epsilon$. Portanto, qualquer subsequ\^encia de $\{a_{n}\}$ \'e convergente.

 b.) Se $a_{n}\overbracket[0pt]{\longrightarrow}^{n\to \infty}l,$ ent\~ao dado $\epsilon > 0$, existe N natural tal que 
 $$
  |a_{n}-l|<\frac{\epsilon}{2},\quad\forall n\geq{N}.
 $$
 Logo, $|a_{n}-a_{m}| = |a_{n}-l + l-a_{m}| \leq{|a_{n}-l| + |l-a_{m}|} < \frac{\epsilon}{2} + \frac{\epsilon}{2} = \epsilon$ para todo $n, m \geq{N}.$

 c.) Suponha que $\{a_{n}\}$ \'e uma sequ\^encia limitada. Recorde que, do teorema de Bolzano-Weierstrass, todo conjunto inifinito e limitado
possui um ponto de acumula\c c\~ao. Segue que a imagem I da sequ\^encia \'e finita ou infinita.

No primeiro caso, se I \'e finito, um dos valores pertencentes a I \'e tal que $a_{n} = a$ para infinitos \'indices. 
Construiremos a sequ\^encia como segue - Coloque s(0) como o menor elemento do conjunto dos n's para os quais 
$a_{n} = a, i.e.,\{n\in \mathbb{N}: a_{n} = a\} = A. $ Al\'em disso, tome s(1) como  o menor elemento de A, com  exce\c c\~ao do
s(0). Repetindo esse processo, obtemos uma subsequ\^encia constante at\'e que se obtenha s(n) = a, ou seja, ela ser\'a convergente.

Agora, se I \'e infinito, segue de Bolzano-Weierstrass que I tem um ponto de acumula\c c\~ao, nomeie-o de a. Dado $\epsilon > 0, (a-\epsilon, a+\epsilon)$ 
tem infinitos elementos do conjunto I. Analogamente ao anterior, coloque N = s(0) como o menor elemento de $\{n\in \mathbb{N}: a_{n}\neq a, a_{n}\in(a-\epsilon, a+\epsilon)\}$ e coloque, tamb\'em,
$\epsilon_{1} = |a-a_{s(0)}|$. Em seguida, tome $s(1) = \{n\in \mathbb{N}: a_{n}\neq a, a_{n}\in(a-\frac{\epsilon}{2}, a+\frac{\epsilon}{2})\}$. Indutivamente,
$b = a_{s(n)}$ \'e convergente para a.

d.) Dado $\epsilon = 1,$ seja N um n\'umero natural tal que 
  $$
    |a_{n}-a_{m}| < 1,\quad\forall n\geq{N}.
  $$
  Considere $M = \{|a_{0}|, |a_{1}|, \cdots, |a_{N-1}|, |a_{N}+1|, |a_{N}-1|\}.$ Assim, $a_{n}\in[-M, M]$ para todo n natural.

e.) Seja $\{a_{n}\}$ de Cauchy e $\{a_{s(n)}\}$ convergente para l. Dado $\epsilon > 0$, existe um natural $N_{1}$ tal que 
  $$
    |a_{n}-a_{m}| < \frac{\epsilon}{2},\quad\forall n\geq{N_{1}}.
  $$
  Al\'em disso, existe $N_{2}$ natural tal que 
  $$
    |a_{s(n)} - l| < \frac{\epsilon}{2},\quad\forall s(n)\geq{N_{2}}.
  $$
  Seja $N=\max\{s(N_{2}), N_{1}\}$ e tome $n\geq{N}.$
  $$
    |a_{n}-l| = |a_{n} - a_{s(N_{2})} + a_{s(N_{2})} - l| \leq{|a_{n}-a_{s(N_{2})}| + |a_{s(N_{2})} - l|} < \frac{\epsilon}{2} + \frac{\epsilon}{2} = \epsilon.
  $$

f.) Segue os itens (e), (d) e (c), visto que toda subsequ\^encia de Cauchy ter\'a subsequ\^encia convergente pelos itens (d) e (c).

g.) Seja $\{a_{n}\}$ limitada e crescente, $l = \sup\{a_{n}:n\in \mathbb{N}\}.$ Ent\~ao, para todo $n \geq{N}$, em que N \'e tal que $a_{N}\in(l-\epsilon, l)$
 $$
  l-\epsilon < a_{N}\leq{a_{n}} \leq{l}.
 $$

h.) An\'aloga ao g.
\end{proof*}
 \begin{example}
   Mostre que 
 \begin{itemize}
   \item[i)] $\{a, a, a, \cdots\},a\in \mathbb{R}$ \'e convergente;
   \item[ii)] $\{0, 1, 0, 1\}$ n\~ao \'e convergente;
   \item[iii)] $\{n\}$ n\~ao \'e convergente.
 \end{itemize}
 \end{example}
\begin{example}
  Se a \'e um n\'umero real mais ou igual a zero, ent\~ao a sequ\^encia $\{a^{n}\}$ \'e convergente se $0\leq{a}\leq{1}$ e divergente
  se $a > 1$. Com efeito, se $a > 1, a = 1 + h, h > 0$. Ent\~ao, 
  $$
    a^{n} = (1+h)^{n} = \sum\limits_{k=0}^{n}\binom{n}{k}1^{n-k}h^{k} = 1 + nh + \cdots > 1 + nh.
  $$
  Mas, segue da Archimediana que $1 + nh$ sempre forma um conjunto ilimitado para n natural, ou seja, $a_{n}$ \'e ilimitada. Logo, a sequ\^encia
diverge.

  Por outro lado, suponha que a pertence a (0, 1). Ent\~ao, $a^{n+1} = a a^{n} < a^{n}$, ou seja, \'e uma sequ\^encia decrescente e limitada inferiormente.
Portanto $\{a_{n}\}$ \'e convergente.
\end{example}
\begin{example}
  Mostre que, se a \'e diferente de 1, 
  $$
    \sum\limits_{i=0}^{n}a^{i} = \frac{1-a^{n+1}}{1-a}
  $$
  e que a sequ\^encia $\biggl\{\frac{1-a^{n+1}}{1-a}\biggr\}$ \'e convergente se $0\leq{a}<1$ e divergente se $a > 1$.
\end{example}
\begin{example}
  Mostre que a sequ\^encia $\{a_{n}\}$, com $a_{n} = \displaystyle \sum\limits_{i=0}^{n}\frac{1}{i!}$ \'e convergente para todo n natural. (Crescente e limitada por 3.)
\end{example}
\begin{example}
  Mostre que as sequ\^encias $\biggl\{(1+\frac{1}{n}^{n})\biggr\}, \{n^{\frac{1}{n}}\}$ e $\{a^{\frac{1}{n}}\}$ com $a >0,$ s\~ao
convergentes.
 \begin{align*}
   &\circ (1+\frac{1}{n})^{n} = 1 + 1 + \frac{1}{2!}(1-\frac{1}{n}) + \cdots + \frac{1}{n!}(1-\frac{1}{n})(1-\frac{1}{n})(1-\frac{2}{n})\cdots(1-\frac{n-1}{n})\\
   &\circ n^{\frac{1}{n}} > (n+1)^{\frac{1}{n+1}}\Longleftrightarrow n^{n+1} > (n+1)^{n}\Longleftrightarrow n>(1+\frac{1}{n})^{n}\\
   &\circ x = a^{n} < 1\Rightarrow x < 1, x^{n} = a, x^{n+1} = a^{\frac{n+1}{n}},\text{ e } y^{n+1} = a \Rightarrow \biggl(\frac{x}{y}\biggr)^{n+1} = a^{\frac{1}{n}}.
 \end{align*}
\end{example}
\newpage

\section{Aula 07 - 27/03/2023}
\subsection{Motiva\c c\~oes}
 \begin{itemize}
   \item Exemplos de Sequ\^encias;
   \item Teorema da Compara\c c\~ao e do Sandu\'iche;
   \item Limites superior e inferior.
 \end{itemize}
\subsection{Exemplos de Sequ\^encias}
  Revisemos os exemplos da \'ultima aula, com um extra ao final.
  \begin{example}
   Mostre que 
 \begin{itemize}
   \item[i)] $\{a, a, a, \cdots\},a\in \mathbb{R}$ \'e convergente;
   \item[ii)] $\{0, 1, 0, 1\}$ n\~ao \'e convergente;
   \item[iii)] $\{n\}$ n\~ao \'e convergente.
 \end{itemize}
 \end{example}
\begin{example}
  Se a \'e um n\'umero real mais ou igual a zero, ent\~ao a sequ\^encia $\{a^{n}\}$ \'e convergente se $0\leq{a}\leq{1}$ e divergente
  se $a > 1$. Com efeito, se $a > 1, a = 1 + h, h > 0$. Ent\~ao, 
  $$
    a^{n} = (1+h)^{n} = \sum\limits_{k=0}^{n}\binom{n}{k}1^{n-k}h^{k} = 1 + nh + \cdots > 1 + nh.
  $$
  Mas, segue da Archimediana que $1 + nh$ sempre forma um conjunto ilimitado para n natural, ou seja, $a_{n}$ \'e ilimitada. Logo, a sequ\^encia
diverge.

  Por outro lado, suponha que a pertence a (0, 1). Ent\~ao, $a^{n+1} = a a^{n} < a^{n}$, ou seja, \'e uma sequ\^encia decrescente e limitada inferiormente.
Portanto $\{a_{n}\}$ \'e convergente.
\end{example}
\begin{example}
  Mostre que, se a \'e diferente de 1, 
  $$
    \sum\limits_{i=0}^{n}a^{i} = \frac{1-a^{n+1}}{1-a}
  $$
  e que a sequ\^encia $\biggl\{\frac{1-a^{n+1}}{1-a}\biggr\}$ \'e convergente se $0\leq{a}<1$ e divergente se $a > 1$.
\end{example}
\begin{example}
  Mostre que a sequ\^encia $\{a_{n}\}$, com $a_{n} = \displaystyle \sum\limits_{i=0}^{n}\frac{1}{i!}$ \'e convergente para todo n natural. (Crescente e limitada por 3.) 

  De fato, \'e claro que $\{a_{n}\} $ \'e crescente e que $\frac{1}{n!}\leq{\frac{1}{2^{n-1}}},$ para $n\geq{2}.$ Logo, 
  $$
  a_{n}\leq{1+\sum\limits_{k=0}^{n}\frac{1}{2^{k}}} = 
  1 + \frac{1-\frac{1}{2^{n+1}}}{1-\frac{1}{2}} < 3.
  $$
  Portanto, $\{a_{n}\}$ \'e convergente, e denotamos seu limite por e.
\end{example}
\begin{example}
  Mostre que as sequ\^encias $\biggl\{(1+\frac{1}{n}^{n})\biggr\}, \{n^{\frac{1}{n}}\}$ e $\{a^{\frac{1}{n}}\}$ com $a >0,$ s\~ao
convergentes.
 \begin{align*}
   &\circ (1+\frac{1}{n})^{n} = 1 + 1 + \frac{1}{2!}(1-\frac{1}{n}) + \cdots + \frac{1}{n!}(1-\frac{1}{n})(1-\frac{1}{n})(1-\frac{2}{n})\cdots(1-\frac{n-1}{n})\\
   &\circ n^{\frac{1}{n}} > (n+1)^{\frac{1}{n+1}}\Longleftrightarrow n^{n+1} > (n+1)^{n}\Longleftrightarrow n>(1+\frac{1}{n})^{n}\\
   &\circ x = a^{n} < 1\Rightarrow x < 1, x^{n} = a, x^{n+1} = a^{\frac{n+1}{n}},\text{ e } y^{n+1} = a \Rightarrow \biggl(\frac{x}{y}\biggr)^{n+1} = a^{\frac{1}{n}}.
 \end{align*}
   Ainda mais, uma delas t\^em como limite o n\'umero e definido no exemplo anterior. Para observar isso, considere o primeiro exemplo. Note que
  \begin{align*}
    b_{n} &= 1 + \binom{n}{1}n^{-1} + \binom{n}{2}n^{-2} + \cdots + \binom{n}{n-1}n^{-n+1} + \binom{n}{n}n^{-n} \\ 
          &= 1 + 1 + \frac{1}{2!}(1-\frac{1}{n}) + \cdots + \frac{1}{n!}(1-\frac{1}{n})(1-\frac{1}{n})(1-\frac{2}{n})\cdots(1-\frac{n-1}{n})\\
          &\leq{1 + 1 + \frac{1}{2!} + \cdots + \frac{1}{n!} = a_{n} < e}
 \end{align*}
 Como cada termo da soma que define $b_{n}$ \'e crescente, obtemos que $b_{n}$ \'e crescente, tal que ela converge com limite $l = \sup{\{b_{n}:n\in \mathbb{N}\}}$.
  
   Com rela\c c\~ao ao \'ultimo item, resta elaborar como ele converge para 1. Lembre-se que $a^{\frac{1}{n}}$ \'e o \'unico n\'umero real positivo
  x tal que $x^{n} = a.$ Logo, se $x = a^{n} $ e $y = a^{\frac{1}{n+1}}$, temos $x^{n+1} = y^{n+1}x$ e, deste modo,
  \begin{align*}
  & (a) 0 < a < 1 \Rightarrow x < 1 \text{ e } \biggl(\frac{x}{y}\biggr)^{n+1} = x < 1\text{ e, assim, } x < y.\\
  & (b) a > 1 \Rightarrow x > 1\text{ e }\biggl(\frac{x}{y}\biggr)^{n+1} = x > 1\text{ tal que } x > y.
  \end{align*}
  Logo, se $a < 1,\{a^{\frac{1}{n}}\} $ \'e crescente e limitada superiormente por 1, mostrando que ela \'e convergente. Al\'em disto,
se $a > 1,\{a^{\frac{1}{n}}\} $ \'e descrescente e limitada inferiormente por 1, tamb\'em sendo convergente. Por fim, segue de
$a^{\frac{1}{n(n+1)}} = \frac{a^{\frac{1}{n}}}{a^{\frac{1}{n+1}}}$. Portanto, do item (a) do teorema junto com a regra para quociente de
sequ\^encias, segue que l = 1 \'e o limite dela.
\end{example}\begin{example}
   Mostre que a sequ\^encia $\{c_{n}\}, c_{0} = 1, c_{n} = n^{\frac{1}{n}}, n\geq{1}$, \'e convergente. Com efeito, lembre-se que,
para $n\geq{3}, n > b_{n} = (1+\frac{1}{n})^{n}.$ Logo, para $n\geq{3}, n^{n+1}>(n+1)^{n}$ e, consequentemente, $n^{\frac{1}{n}} > (n+1)^{\frac{1}{n+1}}.$\
Disto segue de $\{n^{\frac{1}{n}}\} $ \'e limitada e, por (h), que $\{c_{n}\} $ \'e convergente com limite $l\geq{1}.$ Ainda mais,
 $(2n)^{\frac{1}{2n}}(2n)^{\frac{1}{2n}} = (2n)^{\frac{1}{n}} = 2^{\frac{1}{n}}n^{\frac{1}{n}}$ e, portanto, usamos (a) e o exemplo da \'ultima aula para
mostrar que $l^{2} = l = 1.$\qedsymbol
 \end{example}

\subsection{Teoremas da Compara\c c\~ao e do Sandu\'iche}
  Nota\c c\~ao: Se uma sequ\^encia tem limite 0, ela \'e chamada infinit\'esima.
\begin{theorem*}
  Se $\{a_{n}\}$ \'e limitada e $\{b_{n}\}$ \'e infinit\'esima, ent\~ao $\{a_{n}\cdot b_{n}\} $ \'e infinit\'esima.
\end{theorem*}
\begin{proof*}
  Como $\{a_{n}\}$ \'e limitada, seja $M > 0$ tal que $|a_{n}|\leq{M}$ para todo n natural. Como $\{b_{n}\} $ \'e infinit\'esima, dado $\epsilon > 0$,
seja N outro natural tal que $|b_{n}|<\frac{\epsilon}{M}$ para todo $n\geq{N}.$ Segue que
  $$
    |a_{n}b_{n}| \leq{M|b_{n}|} < M \frac{\epsilon}{M} = \epsilon,\quad \forall n\geq{N}.
  $$
  Portanto, $\{a_{n}b_{n}\}\overbracket[0pt]{\longrightarrow}^{n\to \infty}0$.\qedsymbol
\end{proof*}
 \begin{example}
   Mostre que $\biggl\{\frac{n+\cos{(n)}}{n+1}\biggr\}$ converge.
 \end{example}
 Os resultados a seguir s\~ao os dois mencionados previamente, o teorema da compara\c c\~ao e o do sandu\'iche, respectivamente.
\begin{theorem*}
  Se $a_{n}\overbracket[0pt]{\longrightarrow}^{n\to \infty}a, b_{n}\overbracket[0pt]{\longrightarrow}^{n\to \infty}b$ e existe N natural tal que, 
para todo $n\geq{N}, a_{n}\leq{b_{n}}$, ent\~ao $a\leq{b}.$
\end{theorem*}
 \begin{proof*}
   Dado $\epsilon > 0$, existe $N_{1}\leq{N}$ tal que, para todo $n\geq{N_{1}}$,
   $$
    a - \epsilon < a_{n} < a + \epsilon\quad\text{ e }\quad b-\epsilon < b_{n} < b + \epsilon.
   $$
   Logo, para todo $n\geq{N},$
   $$
    a-\epsilon < a_{n} \leq{b_{n}} < b + \epsilon.
   $$
   Desta forma, $a-b < \epsilon$ para todo $\epsilon > 0$ e, portanto, $a - b\leq{0}.$\qedsymbol
 \end{proof*}

\begin{theorem*}
  Se $a_{n}\overbracket[0pt]{\longrightarrow}^{n\to \infty}l, c_{n}\overbracket[0pt]{\longrightarrow}^{n\to \infty}l$ e existe um
N natural tal que, para todo $n\geq{N}, a_{n}\leq{b_{n}}\leq{c_{n}}$, ent\~ao $b_{n}\overbracket[0pt]{\longrightarrow}^{n\to \infty}l.$
\end{theorem*}
\begin{proof*}
  Dado $\epsilon>0$, existe $N_{1}\geq{N}$ tal que, para todo $n\geq{N_{1}},$
  $$
    l - \epsilon < a_{n} < l + \epsilon\quad\text{ e }\quad l-\epsilon < c_{n} < l+\epsilon.
  $$
  Logo, para todo $n\geq{N_{1}},$
  $$
    l - \epsilon < a_{n}\leq{b_{n}}\leq{c_{n}}<l + \epsilon.
  $$
  Disto segue que $|b_{n} - l|< \epsilon$ para todo $n\geq{N_{1}}$ e que, portanto, $\{b_{n}\} $ \'e convergente para l. \qedsymbol
\end{proof*}
\begin{example}
  Vamos mostrar que 
  $$
   e = \lim_{n\to\infty}\overbrace{(1 + \frac{1}{1!} + \frac{1}{2!} + \cdots + \frac{1}{n!})}^{a_{n}} = \lim_{n\to\infty}\overbrace{\biggl(1 + \frac{1}{n}\biggr)^{n}}^{b_{n}} = l.
  $$  
  De fato, como $a_{n} \geq{b_{n}}$ para todo n natural, segue do Teorema da Compara\c c\~ao que $e\geq{l}.$ Por outro lado, se 
 $n\geq{p}\geq{2},$
 $$
  b_{n} > 1 + 1 + \frac{1}{2!}(1-\frac{1}{n})+\cdots+\frac{1}{p!}(1-\frac{1}{n})(1-\frac{2}{n})\cdots(1-\frac{p-1}{n}).
 $$
 Agora, novamente pelo Teorema da Compara\c c\~ao, $l = \displaystyle \lim_{n\to\infty}b_{n}\geq{a_{p}}$ para todo natural p.
 Portanto, $l = \displaystyle \lim_{n\to\infty}b_{n}\geq{\sup\{a_{n}:n\in \mathbb{N}\}} = \lim_{n\to\infty}a_{n} = e.$ \qedsymbol
\end{example}
\begin{def*}
   Seja $\{a_{n}\} $ uma sequ\^encia. Um n\'umero real a \'e um valor de ader\^encia de $\{a_{n}\} $ se a sequ\^encia $\{a_{n}\}$ possui
uma subsequ\^encia convergente para a.$\quad\square$
\end{def*}
\begin{def*}
  Seja $\{a_{n}\} $  uma sequ\^encia limitada. Definimos o limite superior $\displaystyle\limsup_{n\to\infty}a_{n}(\text{ inferior}\liminf_{n\to\infty}a_{n})$ da 
sequ\^encia $\{a_{n}\} $ por 
  \begin{align*}
    &\limsup_{n\to\infty}a_{n} = \lim_{n\to\infty}\sup_{k\geq{n}}a_{k}\\
    &\liminf_{n\to\infty}a_{n} = \lim_{n\to\infty}\inf_{k\geq{n}}a_{k}\quad\square
  \end{align*}
\end{def*}
  Uma consequ\^encia direta do Teorema do Confronto que utiliza os conceitos acima nos permite dizer se uma sequ\^encia converge apenas
utilizando as ideias de limite superior e inferior:
 \begin{theorem*}
   Se a \'e um valor de ader\^encia da sequ\^encia $\{a_{n}\} $, ent\~ao
   $$
   \liminf_{n\to\infty}a_{n}\leq{a}\leq{\limsup_{n\to\infty}a_{n}}.
   $$
   Al\'em disso, uma sequ\^encia \'e convergente se, e somente se, $\liminf_{n\to\infty}a_{n} = \limsup_{n\to\infty}a_{n}.$
 \end{theorem*}
\newpage

\section{Aula 08 - 29/03/2023}
\begin{itemize}
  \item Rela\c c\~ao entre limite superior (e inferior), limites normais e valores de ader\^encia;
  \item Aproxima\c c\~oes sucessivas de valores.
\end{itemize}
\subsection{Limite Superior e Inferior}
\begin{def*}
   Seja $\{a_{n}\} $ uma sequ\^encia. Um n\'umero real a \'e um valor de ader\^encia de $\{a_{n}\} $ se a sequ\^encia $\{a_{n}\}$ possui
uma subsequ\^encia convergente para a.$\quad\square$
\end{def*}
\begin{def*}
  Seja $\{a_{n}\} $  uma sequ\^encia limitada. Definimos o limite superior $\displaystyle\limsup_{n\to\infty}a_{n}(\text{ inferior }\liminf_{n\to\infty}a_{n})$ da 
sequ\^encia $\{a_{n}\} $ por 
  \begin{align*}
    &\limsup_{n\to\infty}a_{n} = \lim_{n\to\infty}\sup_{k\geq{n}}a_{k} = \inf_{n\in \mathbb{N}}\sup_{k\geq{n}}a_{k}\\
    &\liminf_{n\to\infty}a_{n} = \lim_{n\to\infty}\inf_{k\geq{n}}a_{k} = \sup_{n\in \mathbb{N}}\inf_{k\geq{n}}a_{k}\quad\square
  \end{align*}
\end{def*}
  Uma consequ\^encia direta do Teorema do Confronto que utiliza os conceitos acima nos permite dizer se uma sequ\^encia converge apenas
utilizando as ideias de limite superior e inferior:
 \begin{theorem*}
   Se $\{a_{n}\} $ \'e uma sequ\^encia limitada, ent\~ao $a = \liminf_{n\to\infty}a_{n}$ e $b = \limsup_{n\to\infty}a_{n}$
   s\~ao valores de ader\^encia de $\{a_{n}\} $
 \end{theorem*}
\begin{proof*}
  A prova se baseia em verificar que, dada uma vizinhan\c ca $V_{a}$ de a, temos $a_{n}\in V_{a}$ para infinitos \'indices n. 
Dado $\epsilon > 0$, existe N natural tal que, colocando $a =\displaystyle \liminf_{n\to\infty} = \lim_{n\to\infty}\inf_{k\geq{n}}a_{k} = \lim_{n\to\infty}i_{n},$
  $$
    a - \epsilon < i_{n} < a + \epsilon \quad \forall n \geq{N}.
  $$
  Assim, existe um $a_{\overline{k}}, \overline{k}\geq{N}$ em $(a-\epsilon, a+\epsilon).$ Assim, existe $\overline{n} > \overline{k}$ tal que 
    $$
      a - \epsilon < i_{\overline{n}} < a + \epsilon.
    $$
    Repetindo o racioc\'icio, existe $a_{\overline{\overline{k}}}, \overline{\overline{k}} \geq{\overline{n}} > k$ em $(a - \epsilon, a + \epsilon).$
  Dando continuidade a este racioc\'inio ad infinitum, o teorema est\'a provado. \qedsymbol
\end{proof*}
 \begin{theorem*}
   Se a \'e um valor de ader\^encia da sequ\^encia $\{a_{n}\} $, ent\~ao
   $$
     \liminf_{n\to\infty}a_{n}\leq{a}\leq{\limsup_{n\to\infty}a_{n}}.
   $$
   Al\'em disso, uma sequ\^encia \'e convergente se, e somente se, $\liminf_{n\to\infty}a_{n} = \limsup_{n\to\infty}a_{n}.$
 \end{theorem*}
\begin{proof*}
  Defina $i_{n} = \inf_{k\geq{n}}a_{k}$. Segue que $i_{s(n)}\leq{a_{s(n)}}\overbracket[0pt]{\longrightarrow}^{\to }a$, pois o conjunto
  $\{a_{k}: k\geq{s(n)}\}$ cont\'em $a_{s(n)}$. Logo, como $i_{s(n)}$ converge para $\liminf_{n\to\infty}a_{n}$, segue do Teorema da compara\c c\~ao que 
    $$
      \liminf_{n\to\infty}a_{n} \leq{\lim_{n\to\infty}a_{s(n)} = a}.
    $$
    Analogamente, como $a_{s(n)}\leq{\sup_{k\geq{s(n)}}{(a_{k})}} = s_{s(n)}$ e $\sup_{k\geq{s(n)}}{(a_{k})}\overbracket[0pt]{\longrightarrow}^{n\to\infty}a_{n}$, 
  pelo teorema da compara\c c\~ao, chegamos novamente em 
    $$
      \lim_{n\to\infty}a_{s(n)} = a \leq{\limsup_{n\to\infty}a_{n}}.
    $$
    Portanto, juntando ambos, segue o resultado. \qedsymbol
\end{proof*}
  A seguir, mostraremos um m\'etodo para aproximar n\'umeros por meio de sequ\^encias de Cauchy.
\begin{theorem*}
  Se $\kappa\in{[0, 1)},\{a_{n}\} $ \'e uma sequ\^encia tal que, para todo $n\in \mathbb{N}, |a_{n+2}-a_{n+1}|\leq{\lambda|a_{n+1}-a_{n}|}$, 
  ent\~ao $\{a_{n}\} $ \'e de Cauchy.
\end{theorem*}
\begin{proof*}
  Se $m > n$ s\~ao naturais, m = n + p para algum natural n\~ao-nulo p. Assim, como 
    $$
      |a_{n+p} - a_{n}| = |a_{n+p} - a_{n+p-1} + a_{n+p-1}+\cdots + a_{n+1} - a_{n}|,
    $$
  segue da desigualdade triangular que
 \begin{align*}
   |a_{n+p}-a_{n}| &\leq{\kappa|a_{n+p-1}-a_{n+p-2}|+\kappa|a_{n+p-2}-a_{n+p-3}|+\cdots+\kappa|a_{n}-a_{n-1}|}\\ 
                   &\leq{\kappa^{n+p-1}|a_{n+p} - a_{n+p-1}|}+\cdots+{\kappa^{n}|a_{p}-a_{p-1}|}\\
                   &={\kappa^{n}\biggl[\kappa^{p-1}+\cdots+1\biggr]}|a_{1}-a_{0}| < \frac{\kappa^{n}}{1-\kappa}|a_{1}-a_{0}|.
 \end{align*}
 Dado $\epsilon > 0,$ escolha N natural tal que $\displaystyle \frac{\kappa^{n}}{1-\kappa}|a_{1}-a_{0}| < \epsilon.$ Assim, segue que,
 se $m, n\geq{N}, |a_{m}-a_{n}|<\epsilon$ e $\{a_{n}\} $ \'e de Cauchy. \qedsymbol
\end{proof*}
 \begin{example}
   Seja $a > 0$ e $\{a_{n}\}$ a sequ\^encia definida por $a_{0} = c > 0$ e $a_{n+1}=\displaystyle \frac{1}{2}\biggl(a_{n} + \frac{a}{a_{n}}\biggr)$. Mostre que 
   $\{a_{n}\}$ \'e convergente com limite $\sqrt{a}$. 

   Com efeito, observe que 
     $$
       a_{n+2} - a_{n+1} = \frac{1}{2}(a_{n+1}-a_{n}) + \frac{a}{2}\biggl(\frac{1}{a_{n+1}}-\frac{1}{a_{n}}\biggr) = \biggl(\frac{1}{2} - \frac{a}{2a_{n}a_{n+1}}\biggr)(a_{n+1}-a_{n})
     $$
     e note que, para todo $t > 0, \frac{1}{2}(t + \frac{a}{t}) > \sqrt{\frac{a}{2}}.$ Logo, $a_{n} > \sqrt{\frac{a}{2}}$ para todo n maior ou igual que 1.
Disto segue que $2a_{n}a_{n+1} > a$ e que 
  $$
    \biggl|\frac{1}{2} - \frac{a}{2a_{n}a_{n+1}}\biggr| < \frac{1}{2}.
  $$
  Portanto, segue do m\'etodo das aproxima\c c\~oes sucessivas que $\{a_{n}\}$ \'e convergente e seu limite l deve satisfazer 
 $l = \frac{1}{2}(l + \frac{a}{l})$, ou seja, $l^{2} = a.$ \qedsymbol
 \end{example}
\newpage

\section{Aula 09 - 31/03/2023}
\subsection{Motiva\c c\~oes}
  \begin{itemize}
    \item Limites Infinitos;
    \item Sequ\^encias divergentes.
  \end{itemize}

\subsection{Pontos de Ader\^encia e Limites Superiores/Inferiores.}
\begin{def*}
   Seja $\{a_{n}\} $ uma sequ\^encia. Um n\'umero real a \'e um valor de ader\^encia de $\{a_{n}\} $ se a sequ\^encia $\{a_{n}\}$ possui
uma subsequ\^encia convergente para a.$\quad\square$
\end{def*}
\begin{def*}
  Seja $\{a_{n}\} $  uma sequ\^encia limitada. Definimos o limite superior $\displaystyle\limsup_{n\to\infty}a_{n}(\text{ inferior }\liminf_{n\to\infty}a_{n})$ da 
sequ\^encia $\{a_{n}\} $ por 
  \begin{align*}
    &\limsup_{n\to\infty}a_{n} = \lim_{n\to\infty}\sup_{k\geq{n}}a_{k} = \inf_{n\in \mathbb{N}}\sup_{k\geq{n}}a_{k}\\
    &\liminf_{n\to\infty}a_{n} = \lim_{n\to\infty}\inf_{k\geq{n}}a_{k} = \sup_{n\in \mathbb{N}}\inf_{k\geq{n}}a_{k}\quad\square
  \end{align*}
\end{def*}
  Uma consequ\^encia direta do Teorema do Confronto que utiliza os conceitos acima nos permite dizer se uma sequ\^encia converge apenas
utilizando as ideias de limite superior e inferior:
 \begin{theorem*}
   Se $\{a_{n}\} $ \'e uma sequ\^encia limitada, ent\~ao $a = \liminf_{n\to\infty}a_{n}$ e $b = \limsup_{n\to\infty}a_{n}$
   s\~ao valores de ader\^encia de $\{a_{n}\} $
 \end{theorem*}
\begin{proof*}
  A prova se baseia em verificar que, dada uma vizinhan\c ca $V_{a}$ de a, temos $a_{n}\in V_{a}$ para infinitos \'indices n. 
Dado $\epsilon > 0$, existe N natural tal que, colocando $a =\displaystyle \liminf_{n\to\infty} = \lim_{n\to\infty}\inf_{k\geq{n}}a_{k} = \lim_{n\to\infty}i_{n},$
  $$
    a - \epsilon < i_{n} < a + \epsilon \quad \forall n \geq{N}.
  $$
  Assim, existe um $a_{\overline{k}}, \overline{k}\geq{N}$ em $(a-\epsilon, a+\epsilon).$ Assim, existe $\overline{n} > \overline{k}$ tal que 
    $$
      a - \epsilon < i_{\overline{n}} < a + \epsilon.
    $$
    Repetindo o racioc\'icio, existe $a_{\overline{\overline{k}}}, \overline{\overline{k}} \geq{\overline{n}} > k$ em $(a - \epsilon, a + \epsilon).$
  Dando continuidade a este racioc\'inio ad infinitum, segue que $V_{a}$ cont\'em $a_{n}$ para infinitos ind\'ices n, tal que o teorema est\'a provado. \qedsymbol
\end{proof*}
\begin{crl*}
  (Exerc\'icio) Nas condi\c c\~oes do teorema, a \'e um valor de ader\^encia de $\{a_{n}\}$ e $b = \limsup_{n\to\infty}a_{n}$ tamb\'em \'e um valor de
  ader\^encia de $\{a_{n}\}$
\end{crl*}
 \begin{theorem*}
   Se a \'e um valor de ader\^encia da sequ\^encia $\{a_{n}\} $, ent\~ao
   $$
     \liminf_{n\to\infty}a_{n}\leq{a}\leq{\limsup_{n\to\infty}a_{n}}.
   $$
   Al\'em disso, uma sequ\^encia \'e convergente se, e somente se, $\liminf_{n\to\infty}a_{n} = \limsup_{n\to\infty}a_{n}.$
 \end{theorem*}
\begin{proof*}
  Defina $i_{n} = \inf_{k\geq{n}}a_{k}$. Segue que $i_{s(n)}\leq{a_{s(n)}}\overbracket[0pt]{\longrightarrow}^{\to }a$, pois o conjunto
  $\{a_{k}: k\geq{s(n)}\}$ cont\'em $a_{s(n)}$. Logo, como $i_{s(n)}$ converge para $\liminf_{n\to\infty}a_{n}$, segue do Teorema da compara\c c\~ao que 
    $$
      \liminf_{n\to\infty}a_{n} \leq{\lim_{n\to\infty}a_{s(n)} = a}.
    $$
    Analogamente, como $a_{s(n)}\leq{\sup_{k\geq{s(n)}}{(a_{k})}} = s_{s(n)}$ e $\sup_{k\geq{s(n)}}{(a_{k})}\overbracket[0pt]{\longrightarrow}^{n\to\infty}a_{n}$, 
  pelo teorema da compara\c c\~ao, chegamos novamente em 
    $$
      \lim_{n\to\infty}a_{s(n)} = a \leq{\limsup_{n\to\infty}a_{n}}.
    $$
    Portanto, juntando ambos, segue o resultado. \qedsymbol
\end{proof*}

\subsection{Sequ\^encias Divergente para $\pm\infty$.}
 Recorde-se que 
\begin{def*}
  Diremos que uma sequ\^encia $\{a_{n}\}$ diverge para $+\infty(-\infty)$ se, dado $M >0,$ existe um natural N tal que $a_{n}\geq{M}(a_{n}\leq{-M})$
para todo $n\geq{N}.$ Escreveremos $\lim_{n\to\infty}a_{n} = +\infty(-\infty),$ ou $a_{n}\overbracket[0pt]{\longrightarrow}^{n\to \infty}+\infty(-\infty)\square.$
\end{def*}
 \begin{def*}
   Diremos que a sequ\^encia $\{a_{n}\}$ \'e eventualmente positiva (negativa) se existe um natural N tal que $a_{n} > 0 (a_{n} < 0)$
para todo $n\geq{N}.\square$
 \end{def*}
  Vejamos a seguir algumas das propriedades dessas sequ\^encias.
 \begin{theorem*}
  \begin{itemize}
    \item[a)] Se $a_{n}\overbracket[0pt]{\longrightarrow}^{n\to \infty}\infty$ e $\{b_{n}\}$ \'e limitada inferiormente, ent\~ao
  $a_{n}+b_{n}\overbracket[0pt]{\longrightarrow}^{n\to \infty}\infty.$
    \item[b)] Se $a_{n}\overbracket[0pt]{\longrightarrow}^{n\to \infty}\infty$ e $\liminf_{n\to\infty}b_{n} > 0,$ ent\~ao
  $\lim_{n\to\infty}a_{n}b_{m} = +\infty$
  \item[c)] Seja $\{a_{n}\}$ uma sequ\^encia com $a_{n}\neq0$ para todo n natural. $\{a_{n}\}$ \'e eventualmente negativa e
  $a_{n}\overbracket[0pt]{\longrightarrow}^{n\to \infty}0$ se, e somente se, $\displaystyle \frac{1}{a_{n}}\overbracket[0pt]{\longrightarrow}^{n\to \infty}\infty.$
  \item[d)] Sejam $\{a_{n}\},\{b_{n}\} $ sequ\^encias eventualmente positivas, $b_{n}\neq 0$ para todo n natural.
    \item[d.1)]Se $\liminf_{n\to\infty}a_{n} > 0$ e $b_{n}\overbracket[0pt]{\longrightarrow}^{n\to \infty}0$, ent\~ao $\displaystyle\lim_{n\to\infty}\frac{a_{n}}{b_{n}} = +\infty.$
    \item[d.2)] Se $\{a_{n}\}$ \'e limitada e $b_{n}\overbracket[0pt]{\longrightarrow}^{n\to \infty}\infty$, ent\~ao $\displaystyle \frac{a_{n}}{b_{n}}\overbracket[0pt]{\longrightarrow}^{n\to \infty}0.$
  \end{itemize}
 \end{theorem*}
\begin{proof*}
  (a) $\Rightarrow)$ Como $\{b_{n}\}$ \'e limitada inferiormente, existe um n\'umero real $l > 0$ tal que $b_{n} \geq{ -l}$ para todo 
n natural. Como $a_{n}\overbracket[0pt]{\longrightarrow}^{n\to \infty}\infty,$ dado M positivo, existe N natural tal que 
 $a_{n}\geq{M}+l$ para todo $n\geq{N}.$ Logo, 
   $$
     a_{n} + b_{n}\geq{M+l-l} = M,\quad \forall n\geq{N}.
   $$
   Portanto, $a_{n} + b_{n}\overbracket[0pt]{\longrightarrow}^{n\to \infty}\infty.$

  (b) $\Rightarrow)$ Como $a_{n}\overbracket[0pt]{\longrightarrow}^{n\to \infty}\infty$ e $\liminf_{n\to\infty}b_{n} = r > 0,$
  existe $N_{1}$ natural tal que $\liminf_{k\geq{n}}a_{n} \geq{\frac{r}{2}}$ para todo $n\geq{N_{1}.}$ Como $a_{n}\overbracket[0pt]{\longrightarrow}^{n\to \infty}\infty,$
  dado M positivo, existe $N_{2}$ natural tal que $a_{n} > \frac{2M}{r}$ para todo $n\geq{N_{2}}.$ Disto segue que, para $n\geq{N = \max\{N_{1}, N_{2}\}}$, 
    $$
      a_{n}b_{n}\geq{\frac{2M}{r}\frac{r}{2}} = M,\quad \forall n\geq{N}.
    $$
  donde segue o que quer\'iamos.

  (c) $\Rightarrow)$ Se $\{a_{n}\}$ \'e infint\'esima e eventualmente positiva, dado M positivo, seja N natural tal que 
 $0 < a_{n} < \frac{1}{M}$ para quaisquer $n\geq{N}.$ Logo, $\frac{1}{a_{n}} > M$ para todos os $n\geq{N},$ mostrando que 
 $\frac{1}{a_{n}}\overbracket[0pt]{\longrightarrow}^{n\to \infty}\infty.$

 Reciprocamente, se $\frac{1}{a_{n}}\overbracket[0pt]{\longrightarrow}^{n\to \infty}\infty,$ dado $\epsilon > 0,$ seja N natural tal que
 $\frac{1}{a_{n}} > \frac{1}{\epsilon}$ para todo $n\geq{N}.$ Desta forma, $0 < a_{n} < \epsilon $ para todo $n\geq{N},$ provando o 
 resultado.

 (d.1) $\Rightarrow)$ De fato, se $\liminf_{n\to\infty}a_{n} = r > 0,$ existe $N_{1}$ natural tal que $a_{n}\geq{\frac{r}{2}}$ para todo $n\geq{N_{1}.}$
Dado M positivo, seja $N_{2}$ outro natural tal que $0 < a_{n} < \frac{r}{2M}$ para todo $n\geq{N_{2}}.$ Logo, $\frac{a_{n}}{b_{n}}>\frac{r}{2}\frac{2M}{r} = M$
para todo $n\geq{N} = \max\{N_{1}, N_{2}\}$, tal que $\frac{a_{n}}{b_{n}}\overbracket[0pt]{\longrightarrow}^{n\to \infty}\infty$.

  (d.2) $\Rightarrow)$ Seja $L > 0 $ tal que $|a_{n}|\leq{L}$ para todo n natural. Como $b_{n}\overbracket[0pt]{\longrightarrow}^{n\to \infty}\infty,$
  dado $\epsilon > 0,$ existe N natural tal que $b_{n} > \frac{L}{\epsilon}(\frac{1}{b_{n}} < \frac{\epsilon}{L})$ para todo $n\geq{N}.$ Logo,
  $$
    \biggl|\frac{a_{n}}{b_{n}} - 0\biggr| < L \cdot \frac{\epsilon}{L} = \epsilon,\quad \forall n\geq{N},
  $$
  mostrando que $\frac{a_{n}}{b_{n}}\overbracket[0pt]{\longrightarrow}^{n\to \infty}0.$\qedsymbol
\end{proof*}
  \'E importante notar que, se $a_{n}\overbracket[0pt]{\longrightarrow}^{n\to \infty}\infty$ e $b_{n}\overbracket[0pt]{\longrightarrow}^{n\to \infty}-\infty$, nada
  podemos afirmar de $\lim_{n\to\infty}(a_{n}+b_{n}).$ Neste caso, tudo pode ocorrer! $\{a_{n} + b_{n}\} $ pode convergir para qualquer n\'umero real,
  divergir para $+\infty, -\infty$ ou pode oscilar. Vamos ilustrar a situa\c c\~ao.
 \begin{example}
   Se $a_{n} = \sqrt{n+1}, b_{n} = -\sqrt{n},$ para todo n natural, \'e f\'acil de ver que $a_{n}\overbracket[0pt]{\longrightarrow}^{n\to \infty}\infty$ e $b_{n}\overbracket[0pt]{\longrightarrow}^{n\to \infty}-\infty$.
   Para ver o que ocorre com a sequ\^encia $\{a_{n}+b_{n}\} $, observe que 
     $$
     \sqrt{n+1} - \sqrt{n} = \frac{(\sqrt{n+1}-\sqrt{n})(\sqrt{n+1}+\sqrt{n})}{\sqrt{n+1}+\sqrt{n}} = \frac{1}{\sqrt{n+1}+\sqrt{n}}.
     $$ 
     Segue de (d.2) que $\{a_{n}+b_{n}\} $ \'e infinit\'esima.
 \end{example}
\begin{example}
  Se $a > 1,$ ent\~ao a sequ\^encia $\{a_{n}\}$ dada por $a_{n} = \frac{a^{n}}{n}$ diverge para $+\infty.$ De fato, basta ver que
a = 1 + h, com h positivo, e escrever 
  $$
    \frac{a^{n}}{n} = \frac{(1+h)^{n}}{n} = \frac{1}{n} + h + (n-1)\frac{h^{2}}{2!} + s_{n}.
  $$
  O resultado segue aplicando (a).
\end{example}
 \begin{example}
   Se $a > 1$, ent\~ao a sequ\^encia $\{a_{n}\}$ com $a_{n} = \frac{n!}{a^{n}}$ diverge para $+\infty.$ Com efeito, basta escolher
 $n_{0}$ tal que $\frac{n_{0}}{a} > 2$ e escrever, para $n\geq{n_{0}}, a_{n} = \frac{n_{0}!}{a^{n_{0}}}\frac{n!}{n_{0}!}\frac{1}{a^{n-n_{0}}}.$
 Se $r = \frac{n_{0}!}{a^{n_{0}}}\frac{n!}{n_{0}!},$ temos 
   $$
   a_{n} = r \frac{n(n-1)\cdots(n_{0}+1)}{a^{n-n_{0}}} = r2^{n-n_{0}} + s_{n} = r(n+1 - n_{0}) + \tilde{s}_{n}
   $$
   Novamente, o resultado segue aplicando (a).
 \end{example}
  Por fim, algumas outras propriedades que n\~ao foram inclusas no teorema s\~ao deixadas como exerc\'icio ao leitor.
 \begin{itemize}
  \item[a)] Se $a_{n}\overbracket[0pt]{\longrightarrow}^{n\to \infty}-\infty$ e $\{b_{n}\}$ \'e limitada superiormente, ent\~ao
  $a_{n}+b_{n}\overbracket[0pt]{\longrightarrow}^{n\to \infty}-\infty.$
  \item[b)] Se $a_{n}\overbracket[0pt]{\longrightarrow}^{n\to \infty}-\infty$ e $\liminf_{n\to\infty}b_{n} > 0,$ ent\~ao $\lim_{n\to\infty}a_{n}b_{m} = -\infty.$
  \item[c)] Seja $\{a_{n}\}$ uma sequ\^encia com $a_{n}\neq 0$ para todo n natural. $\{a_{n}\}$ \'e eventualmente negativa e $a_{n}\overbracket[0pt]{\longrightarrow}^{n\to \infty}0$
    se, e somente se, $\frac{1}{a_{n}}\overbracket[0pt]{\longrightarrow}^{n\to \infty}-\infty.$
  \item[d)] Sejam $\{a_{n}\},\{b_{n}\}$ duas sequ\^encias eventualmente negativas com $b_{n}\neq0$ para todo n natural.
    \item[d.1)] Se $\liminf_{n\to\infty}a_{n} < 0$ e $b_{n}\overbracket[0pt]{\longrightarrow}^{n\to \infty}0$, ent\~ao $\lim_{n\to\infty}\frac{a_{n}}{b_{n}} = +\infty.$
    \item[d.2)] Se $\{a_{n}\}$ \'e limitada e $b_{n}\overbracket[0pt]{\longrightarrow}^{n\to \infty}-\infty$, ent\~ao $\frac{a_{n}}{b_{n}}\overbracket[0pt]{\longrightarrow}^{n\to \infty}0.$
    \item[e)] No item (d), analise a situa\c c\~ao em que $\{a_{n}\}$ \'e eventualmente positiva e $\{b_{n}\}$ \'e eventualmente negativa.
 \end{itemize}
\newpage

\section{Aula 10 - 10/04/2023}
\subsection{Motiva\c c\~oes}
 \begin{itemize}
 \item S\'eries;
 \item S\'eries de n\'umeros positivos;
 \item Testes da ra\'iz e da raz\~ao.
 \end{itemize}

\subsection{S\'eries de N\'umeros Reais}
  S\'eries s\~ao tipos particulares de sequ\^encias que possuem suas pr\'oprias propriedades extras. Vamos come\c car com um exemplo, chamado s\'erie harm\^onica (nome derivado da m\'edia harm\^onica).
 \begin{example}
   Considere $s_{n} = 1 + \frac{1}{2} + \cdots + \frac{1}{n}.$ Provemos que $\{s_{n}\}\overbracket[0pt]{\longrightarrow}^{n\to \infty}\infty.$
   Com efeito, como $\{s_{n}\} $ \'e crescente, basta mostrar que ela \'e ilimitada: 
  \begin{align*}
    &s_{1} = 1\\
    &s_{2} = 1 + \frac{1}{2} = 3\frac{1}{2}\\
    &s_{3} = 1 + \frac{1}{2} + \frac{1}{3}\\
    &s_{4} = 1 + \frac{1}{2} + \frac{1}{3} + \frac{1}{4} > 1 + \frac{1}{2} + \frac{1}{2} = 4\frac{1}{2}\\
    &s_{8} = 1 + \frac{1}{2} + \underbrace{\frac{1}{3} + \frac{1}{4}}_{> \frac{1}{2}} + \underbrace{\frac{1}{5} + \frac{1}{6} + \frac{1}{7} + \frac{1}{8}}_{4\frac{1}{8}} > 1 + \frac{1}{2} + \frac{1}{2} + \frac{1}{2} = 5\frac{1}{2}\\
    &s_{16} = s_{8} + \frac{1}{9} +\cdots + \frac{1}{16} > 5\frac{1}{2} + \frac{1}{2} = 6\frac{1}{2}.\\
    \vdots
  \end{align*}
  Analisando o padr\~ao de repeti\c c\~ao desta s\'erie, chegamos em 
    $$
      s_{2^{n}} > (n+2)\frac{1}{2} \Rightarrow s_{2^{n+1}} > s_{2^n} + \frac{1}{2^{n}+1} + \cdots + \frac{1}{2^{n}+ 2^{n}} > (n+2)\frac{1}{2} + \frac{2^{n}}{2^{n+1}} = ((n+1)+2)\frac{1}{2}.
    $$
    Logo, por indu\c c\~ao, $s_{2^{k}} > (k+2)\frac{1}{2}$ para todo k natural, mostrando que $\{s_{n}\} $ n\~ao \'e limitada superiormente.
  Portanto, $\lim_{n\to\infty}s_{n} = \infty.$ \qedsymbol
 \end{example}
 Consideremos a sequ\^encia $\{a_{n}\}$. A partir da sequ\^encia $\{a_{n}\} $, vamos construir a sequ\^encia das somas parciais da seguinte forma: 
\begin{align*}
  &s_{1} = a_{1}\\
  &s_{2} = a_{1} + a_{2}\\
  &s_{3} = a_{1} + a_{2} + a_{3}\\
  &\vdots\\
  &s_{n} = a_{1} + a_{2} + a_{3} + \cdots + a_{n}\\
  &\cdots.
\end{align*}
 \begin{def*}
  A sequ\^encia $\{s_{n}\}$ das somas paricais \'e chamada s\'erie associada a $\{a_{n}\}.$ Cada $s_{n}$ \'e chamado soma parcial da s\'erie, e cada
 $a_{n}$ leva o nome de termo da s\'erie. As nota\c c\~oes s\~ao: 
   $$
   \sum\limits_{n\geq{1}}^{}a_{n},\quad \sum\limits_{}^{}a_{n},\quad \sum\limits_{n=1}^{\infty}a_{n},\quad\text{ ou } a_{1}+a_{2}+\cdots+a_{n}+\cdots.
   $$
 \end{def*}
 Observe que, \`as vezes, consideraremos s\'eries que comecem em algum termo $n_{0}$ ao inv\'es do termo 1. Neste caso, escreveremos 
 $\sum\limits_{n=n_{0}}^{\infty}a_{n}.$
\begin{example}
 \begin{align*}
   &\{a_{n}\} = \{(-1)^{n+1}\}.\text{ A sequ\^encia das somas parciais ser\'a:}\\
   &s_{1} = a_{1} = 1\\
   &s_{2} = a_{1} + a_{2} = 1 - 1 = 0\\
   &\vdots\\
   &s_{n} = 0\text{ ou } 1.
 \end{align*} 
\end{example}
\begin{example}
  Construimos a s\'erie $\{a_{n}\} = \biggl\{\frac{1}{n}\biggr\}$ no exemplo anterior.
\end{example}
\begin{example}
  $\{a_{n}\} = \biggl\{\frac{6}{10^{n}}\biggr\}$ \'e uma s\'erie que, quando constru\'ida, converge para $\frac{2}{3}.$
\end{example}
\begin{def*}
  Diremos que uma s\'erie \'e convergente se a sequ\^encia das somas parciais \'e convergente. Caso contr\'ario, ser\'a dita divergente.
  Se a sequ\^encia $\{s_{n}\} $ \'e convergente para S, dizemos que a s\'erie $\sum\limits_{1}^{\infty}a_{n}$ \'e convergente com soma S. $\square$
\end{def*}
  A soma, multiplica\c c\~ao, etc. De s\'eries \'e definida como no caso das sequ\^encias. Denotaremos s\'eries convergentes por 
    $$
      \sum\limits_{n=1}^{\infty}a_{n} = S = \lim_{n\to\infty}s_{n} = \lim_{n\to\infty}\biggl(\sum\limits_{k=1}^{n}a_{k}\biggr)
    $$
   \begin{example}
     A s\'erie telesc\'opica \'e dada por 
       $$
         \sum\limits_{n=1}^{\infty}\frac{1}{n(n+1)} = 1.
       $$
     Observe que 
     \begin{align*}
       s_{n} &= \frac{1}{1.2} + \frac{1}{2.3} + \cdots + \frac{1}{n(n+1)}\\
             &= (1-\frac{1}{2}) + (\frac{1}{2}-\frac{1}{3}) + \cdots + (\frac{1}{n} - \frac{1}{n+1})\\
             &= 1 - \frac{1}{n+1}.
      \end{align*}
    Portanto, $\lim_{n\to\infty}s_{n} = \lim_{n\to\infty}(1 - \frac{1}{n+1}) = 1 - 0 = 1.$ \qedsymbol
   \end{example}
  \begin{example}
   \begin{itemize}
     \item[i)]$\sum\limits_{1}^{\infty}(-1)^{n}$ \'e divergente, visto que, para termos \'impares da soma parcial, ela vale 0, mas para termos pares,
    ela vale 1.
     \item[ii)]$\sum\limits_{1}^{\infty}2^{n}$ diverge, pois a soma parcial dos termos dela n\~ao \'e limitada.
     \item[iii)]$\sum\limits_{n=1}^{\infty}\frac{1}{n}$ diverge, e recebe o nome de s\'erie harm\^onica. Aqui, $s_{n} = 1 + \frac{1}{2}+\cdots+\frac{1}{n}.$
   \end{itemize}
  \end{example}
  Algumas s\'eries s\~ao importantes, pois podem ser utilizadas para obter informa\c c\~oes sobre outras atrav\'es da compara\c c\~ao, tais como a 
  s\'erie telesc\'opica e a harm\^onica. Outra importante \'e a s\'erie geom\'etrica:
 \begin{example}
   Definimos ela como $\sum\limits_{n\geq{1}}^{}ar^{n-1} = a + ar + ar^{2} + \cdots$. Afirmamos que ela \'e convergente se, e s\'o se,
  $|r| < 1$, convergindo para $\frac{a}{1-r}$. Assim, 
    $$
      a + ar + ar^{2} + \cdots + ar^{n} + \cdots = \frac{a}{1-r}, \quad |r| < 1.
    $$
    Com efeito, se r = 1, ent\~ao $s_{n} = a + a + a + \cdots + a = na,$ que tende a infinito ou menos infinito, dependendo do sinal de a, mostrando sua diverg\^encia.
Agora, se r for diferente de 1, temos: 
  $$
    s_{n} = a + ar + ar^{2} + \cdots + ar^{n-1},\quad rs_{n} = ar + ar^{2} + ar^{3} + \cdots ar^{n}.
  $$
  Subtraindo membro a membro, obtemos 
    $$
      s_{n}(1-r) = a - ar^{n} = a(1-r^{n}).
    $$
    Portanto, $s_{n} = a \frac{1-r^{n}}{1-r} = \frac{a}{1-r} - \frac{a}{1-r}r^{n}.$ Se $|r|<1$, j\'a vimos anteriormente que
  $r^{n}\overbracket[0pt]{\longrightarrow}^{n\to \infty}0,$ tal que 
    $$
      \lim_{n\to\infty}s_{n} = \lim_{n\to\infty}\biggl(\frac{a}{1-r} - \frac{a}{1-r}r^{n}\biggr) = \frac{a}{1-r}.
    $$
    Caso $|r| > 1,$ ou r = -1, vimos anteriormente que $r^{n}$ \'e divergente, provando o que quer\'iamos. \qedsymbol
 \end{example}
\begin{theorem*}
  Se $\sum\limits_{}^{}a_{n}$ \'e uma s\'erie convergente, ent\~ao $\lim_{n\to\infty}a_{n} = 0.$ A rec\'iproca \'e falsa.
\end{theorem*}
\begin{proof*}
  Note que, se $\{s_{n}\}$ a sequ\^encia das somas parciais dos an's \'e convergente, temos 
    $$
      a_{n} = s_{n} - s_{n-1}\overbracket[0pt]{\longrightarrow}^{n\to \infty} s-s= 0,
    $$
    provando o resultado. Um contraexemplo da volta \'e a s\'erie harm\^onica. \qedsymbol
\end{proof*}
 \begin{example}
   Exerc\'icio: $\sum\limits_{n=1}^{\infty}a_{n}$ \'e uma s\'erie convergente se, e somente se, $\sum\limits_{n=n_{0}}^{\infty}a_{n}$ \'e convergente.
 \end{example}
\begin{theorem*}
  Se $a_{n}\geq{0}$ para todo n natural, $\sum\limits_{}^{}a_{n}$ \'e uma s\'erie convergente se, e somente se, a sequ\^encia das somas paricais \'e limitada.
\end{theorem*}
\begin{theorem*}
  Sejam $\sum\limits_{}^{}a_{n}, \sum\limits_{}^{}b_{n}$ s\'eries de termos positivos. Se existem c positivo e $n_{0}$ natural tais que
 $a_{n} \leq{cb_{n}}$ para todo $n\geq{n_{0}},$ ent\~ao 
\begin{itemize}
  \item[i)] Se $\sum\limits_{}^{}b_{n}$ \'e convergente, ent\~ao $\sum\limits_{}^{}a_{n}$ \'e convergente.
    \item[ii)] Se $\sum\limits_{}^{}a_{n}$ \'e divergente, ent\~ao $\sum\limits_{}^{}b_{n}$ \'e divergente. 
\end{itemize}
\end{theorem*}
\begin{example}
  Se $r > 1, \sum\limits_{}^{}\frac{1}{n^{r}}$ \'e convergente. De fato, note que 
  \begin{align*}
    s_{2^{n}-1} &< 1 + \biggl(\frac{1}{2^{r}}+\frac{1}{3^{r}}\biggr) + \biggl(\frac{1}{4^{r}} + \frac{1}{5^{r}} + \frac{1}{6^{r}}+\frac{1}{7^{r}}\biggr) + \cdots\\
                &+\Biggl(\frac{1}{2^{n}-2^{n-1}}+\cdots+\frac{1}{(2^{n}-2)^{r}} + \frac{1}{(2^{n}-1)^{r}}\Biggr) \\
                &< 1 + \frac{2}{2^{r}} + \frac{4}{4^{r}} + \cdots + \frac{2^{n-1}}{2^{(n-1)r}} \\
                &= 1 + \frac{1}{2^{r-1}} + \frac{1}{4^{r-1}} + \cdots + \frac{1}{2^{(n-1)(r-1)}}\leq{\frac{1}{1-2^{-r+1}}}.
  \end{align*}
\end{example}
Como uma sequ\^encia convergente \'e equivalente a uma de Cauchy, vale o seguinte resultado:
\begin{theorem*}
  A s\'erie $\sum\limits_{}^{}a_{n}$ \'e convergente se, e somente se, dado $\epsilon > 0$, existe um natural N tal que 
    $$
      |s_{n+p} - s_{n}| = |a_{n} + \cdots + a_{n+p}| < \epsilon
    $$
    para todo $n\geq{N}$ e todo natural p.
\end{theorem*}
 \begin{def*}
   Uma s\'erie $\sum\limits_{}^{}a_{n}$ \'e absolutamente convergente quando $\sum\limits_{}^{}|a_{n}|$ \'e convergente.$\quad\square.$
 \end{def*}
 Segue do crit\'erio de Cauchy que 
\begin{theorem*}
  Se $\sum\limits_{}^{}a_{n}$ \'e absolutamente convergente, ent\~ao $\sum\limits_{}^{}a_{n}$ \'e convergente. A volta n\~ao vale.
\end{theorem*}
  Com rela\c c\~ao a volta, considere a s\'erie $\sum\limits_{}^{}\frac{(-1)^{n+1}}{n}$ n\~ao \'e absolutamente convergente, mas converge. 
  De fato, note que 
    $$
      s_{2} = 1 - \frac{1}{2}, s_{4} = (1-\frac{1}{2}) + (\frac{1}{3}-\frac{1}{4}), s_{6} = (1-\frac{1}{2}) + (\frac{1}{3}-\frac{1}{4}) + (\frac{1}{5}-\frac{1}{6}), \cdots
    $$
    Assim, $s_{2}<s_{4}<\cdots<s_{2n}<\cdots$ e $\{s_{2n}\}$ \'e crescente e limitada por 1, tal que ela converge. Por outro lado, 
      $$
        s_{1} = 1, s_{3} = 1 + (-\frac{1}{2} + \frac{1}{3}), s_{5} = 1 + (-\frac{1}{2}+\frac{1}{3}) + (-\frac{1}{4}+\frac{1}{5}), \cdots
      $$ 
      Deste forma, $s_{1} > s_{3} > s_{5} > \cdots > s_{2(n-1)} > \cdots$, ou seja, $\{s_{2(n-1)}\}$ \'e decrescente limitada, portanto convergente.

      Por outro lado, $s_{2n+1}-s_{n} = \frac{1}{2n+1}\overbracket[0pt]{\longrightarrow}^{n\to \infty}0,$ mostrando a converg\^encia da s\'erie.
 \begin{example}
   Exerc\'icio: Seja $\{a_{n}\}$ uma sequ\^encia infinit\'esima de termos n\~ao negativos que \'e decrescente. Mostre que $\sum\limits_{}^{}(-1)^{n}a_{n}$
  \'e convergente.
 \end{example}
\begin{example}
  Exerc\'icios: Seja $\sum\limits_{}^{}b_{n}$ uma s\'erie convergente de termos n\~ao-negativos. Se existem k positivo e $n_{0}$
natural tais que $|a_{n}|\leq{kb_{n}}$ para todo $n\geq{n_{0}},$ ent\~ao a s\'erie $\sum\limits_{}^{}a_{n}$ \'e absolutamente convergente.

  Se existem $c\in(0, 1), k > 0$ e $n_{0}$ natural tais que $|a_{n}|\leq{kc^{n}}$ para todo $n\geq{n_{0}},$ ent\~ao a s\'erie 
 $\sum\limits_{}^{}a_{n}$ \'e absolutamente convergente.
\end{example}

\subsection{Testes da Raz\~ao e da Ra\'iz}
 \begin{theorem*}
   Se $\{a_{n}\}$ \'e uma sequ\^encia limitada e $\limsup_{n\to\infty}|a_{n}|^{\frac{1}{n}} = c < 1$, ent\~ao $\sum\limits_{}^{}a_{n}$
  \'e absolutamente convergente.
 \end{theorem*}
\begin{proof*}
  Existe N natural tal que $\sup_{k\geq{n}}|a_{k}|^{\frac{1}{k}} < r = \frac{c+1}{2} < 1$ para todo natural n. Logo, $|a_{n}|<r^{n}$ para
todo $n\geq{N}.$ Segue do Teorema da Compara\c c\~ao que $\sum\limits_{}^{}|a_{n}|$ \'e convergente, ou seja, $\sum\limits_{}^{}a_{n}$ \'e absolutamente convergente. \qedsymbol
\end{proof*}
\begin{example}
  Se p \'e natural, ent\~ao $\sum\limits_{}^{}n^{p}a^{n}$ \'e convergente para $|a|< 1$ e divergente para $|a_{n}| \geq{1}.$ De fato,
basta ver que $\limsup_{n\to\infty}|n^{p}a^{n}|^{\frac{1}{n}} = |a| < 1$ e aplicar o teste da ra\'iz. Para ver que a s\'erie \'e divergente 
quando o m\'odulo de a \'e maior ou igual a 1, basta notar que a sequ\^encia dos termos da s\'erie n\~ao converge para zero neste caso. \qedsymbol
\end{example}
\newpage

\section{Aula 11 - 12/04/2023}
\subsection{Motiva\c c\~oes}
\begin{itemize}
  \item Teste da Raz\~ao;
  \item Teoremas de Dirichlet, Abel e Leibniz;
  \item S\'eries Rearranjadas.
\end{itemize}
\subsection{Teste da Ra\'iz e da Raz\~ao.}
  Comecemos revisitando o Teste da Ra\'iz.
\begin{theorem*}
   Se $\{a_{n}\}$ \'e uma sequ\^encia limitada e $\limsup_{n\to\infty}|a_{n}|^{\frac{1}{n}} = c < 1$, ent\~ao $\sum\limits_{}^{}a_{n}$
  \'e absolutamente convergente.
 \end{theorem*}
\begin{proof*}
  Existe N natural tal que $\sup_{k\geq{n}}|a_{k}|^{\frac{1}{k}} < r = \frac{c+1}{2} < 1$ para todo natural n. Logo, $|a_{n}|<r^{n}$ para
todo $n\geq{N}.$ Segue do Teorema da Compara\c c\~ao que $\sum\limits_{}^{}|a_{n}|$ \'e convergente, ou seja, $\sum\limits_{}^{}a_{n}$ \'e absolutamente convergente. \qedsymbol
\end{proof*}
\begin{example}
  Se p \'e natural, ent\~ao $\sum\limits_{}^{}n^{p}a^{n}$ \'e convergente para $|a|< 1$ e divergente para $|a_{n}| \geq{1}.$ De fato,
basta ver que $\limsup_{n\to\infty}|n^{p}a^{n}|^{\frac{1}{n}} = |a| < 1$ e aplicar o teste da ra\'iz. Para ver que a s\'erie \'e divergente 
quando o m\'odulo de a \'e maior ou igual a 1, basta notar que a sequ\^encia dos termos da s\'erie n\~ao converge para zero neste caso. \qedsymbol
\end{example}
  Como esperado, o resultado seguinte \'e o que conhecemos como Teste da Raz\~ao, e costuma ser mais simples de aplicar na maioria dos casos.
\begin{theorem*}
  Se $\sum\limits_{}^{}b_{n}$ \'e uma s\'erie convergente de termos positivos e $\sum\limits_{}^{}a_{n}$ \'e uma s\'erie de termos
n\~ao-nulos tais que existe $n_{0}\in \mathbb{N}$ satisfazendo 
  $$
    \frac{|a_{n+1}|}{|a_{n}|} \leq{\frac{b_{n+1}}{b_{n}}},\quad \forall n\geq{n_{0}},
  $$
  ent\~ao $\sum\limits_{}^{}a_{n}$ converge absolutamente. Em particular, se $\limsup_{n\to\infty}\frac{|a_{n+1}|}{|a_{n}|} = c < 1,$
tamb\'em vale que $\sum\limits_{}^{}a_{n}$ \'e absolutamente convergente.
\end{theorem*}
\begin{proof*}
  De fato, segue que 
    $$
      \frac{|a_{n_{0}+1}|}{|a_{n_{0}}|} \leq{\frac{b_{n_{0}+1}}{b_{n_{0}}}}, \frac{|a_{n_{0}+2}|}{|a_{n_{0}+1}|} \leq{ \frac{b_{n_{0}+2}}{b_{n_{0}+1}}}, \cdots
    $$
    Logo, $\frac{|a_{n_{0}+p}|}{|a_{n_{0}}|}\leq{\frac{b_{n_{0}+p}}{b_{n_{0}}}}$ e o resultado segue usando o Teorema da Compara\c c\~ao. Por
    fim, o caso particular segue tomando $b_{n} = c^{n}$. \qedsymbol
\end{proof*}
\begin{example}
  Provemos que $\sum\limits_{}^{}\frac{n!}{n^{n}}a^{n}$ \'e convergente para $|a|<e.$ Com efeito, note que, para a n\~ao-nulo, 
    $$
      \frac{\biggl|\frac{(n+1)!}{(n+1)^{(n+1)}}a^{(n+1)}\biggr|}{|\frac{n!}{n^{n}}a^{n}|} = \frac{1}{(1+\frac{1}{n})^{n}}|a|\overbracket[0pt]{\longrightarrow}^{n\to \infty}\frac{|a|}{e}.
    $$ 
    O resultado agora segue do Teste da Raz\~ao.
\end{example}
\begin{example}
  Considere a s\'erie $\sum\limits_{}^{}a_{n}$ com $a_{2n}=2a^{n-1}, a_{2n-1} = a^{2(n-1)}.$ Vamos aplicar o crit\'erio da ra\'iz
e o crit\'erio da raz\~ao para esta s\'erie. Caso n = 2k, temos 
  $$
    \frac{|a_{2k+1}|}{|a_{2k}|} = \frac{|a^{2k}|}{2|a^{2k-1}|} = \frac{|a|}{2}.
  $$
  Agora, se n = 2k-1, 
    $$
      \frac{|a_{2k}|}{|a_{2k-1}|} = \frac{2|a^{2k-1}|}{|a^{2(k-1)}|} = 2|a|.
    $$
    Segue que $\limsup_{n\to\infty}\frac{|a_{n+1}|}{|a_{n}|} = 2|a|.$ Por outro lado, $\sqrt[n]{a_{n}}\overbracket[0pt]{\longrightarrow}^{n\to \infty}|a|.$
Desta forma, o teste da ra\'iz nos d\'a converg\^encia para $|a|<1$, enquanto que o teste da raz\~ao nos d\'a converg\^encia apenas
para $|a|<\frac{1}{2}.$
\end{example}
O exemplo anterior indica uma diferen\c ca no n\'ivel de efic\'acia dos testes. De fato, o Teste da Ra\'iz \'e mais eficiente do que 
o teste da raz\~ao, como segue 
\begin{theorem*}
  Seja $\{a_{n}\}$ uma sequ\^encia limitada de n\'umeros reais n\~ao-nulos. Ent\~ao, 
    $$
      \liminf_{n\to\infty}\frac{|a_{n+1}|}{|a_{n}|}\leq{\liminf_{n\to\infty}}|a_{n}|^{\frac{1}{n}}\leq{\limsup_{n\to\infty}|a_{n}|^{\frac{1}{n}}}\leq{\limsup_{n\to\infty}\frac{|a_{n+1}|}{|a_{n}|}}.
    $$
\end{theorem*}
\begin{proof*}
  Mostremos primeiramente que $\limsup_{n\to\infty}|a_{n}|^{\frac{1}{n}}\leq{\limsup_{n\to\infty}\frac{|a_{n+1}|}{|a_{n}|}}.$
Se n\~ao, seja $c>0$ com $\limsup_{n\to\infty}\frac{|a_{n+1}|}{|a_{n}|}<c<\limsup_{n\to\infty}|a_{n}|^{\frac{1}{n}}.$ Logo,
existe um natural N tal que 
  $$
    \frac{|a_{n+1}|}{|a_{n}|}<c,\quad \forall n\geq{N}.
  $$
  Disto segue que $|a_{N+p}|<|a_{N}|c^{-N}c^{N+p}$ para todo p natural n\~ao-nulo. Sendo assim, $c < \limsup_{n\to\infty}|a_{n}|^{\frac{1}{n}}\leq{c}$,
uma contradi\c c\~ao. 

  Para ver que $\liminf_{n\to\infty}\frac{|a_{n+1}|}{|a_{n}|}\leq{\liminf_{n\to\infty}|a_{n}|^{\frac{1}{n}}},$ procedemos de modo similar.
Comece supondo que existe um $c > 0$ tal que $\liminf_{n\to\infty}\frac{|a_{n+1}|}{|a_{n}|}> c > \liminf_{n\to\infty}|a_{n}|^{\frac{1}{n}}.$
Logo, existe N natural tal que $\frac{|a_{n+1}|}{|a_{n}|}>c$ para todo $n\geq{N},$ e $|a_{N+p}|>|a_{N}|c^{-N}c^{N+p}$ para todo
 $p\in \mathbb{N}^{*}.$ Portanto, temos uma contradi\c c\~ao, pois $c > \liminf_{n\to\infty}|a_{n}|^{\frac{1}{n}} \geq{c}.$ \qedsymbol
\end{proof*}
\begin{crl*}
  Seja $\{a_{n}\}$ uma sequ\^encia limitada de n\'umeros reais n\~ao-nulos. Se existe o limite $\lim_{\to}\frac{|a_{n+1}|}{|a_{n}|}$,
ent\~ao o limite $\lim_{\to}|a_{n}|^{\frac{1}{n}}$ tamb\'em existe e ambos t\^em o mesmo valor.
\end{crl*}
\begin{example}
  Vamos mostrar que 
    $$
      \lim_{\to}\frac{n}{(n!)^{\frac{1}{n}}}=e.
    $$
  De fato, seja $a_{n}=\frac{n^{n}}{n!}$ e note que $(a_{n})^{\frac{1}{n}} = \frac{n}{(n!)^{\frac{1}{n}}}.$ Note tamb\'em que 
    $$
      \frac{a_{n+1}}{a_{n}} = \frac{\frac{(n+1)^{n+1}}{(n+1)!}}{\frac{n^{n}}{n!}} = \frac{(n+1)^{n}}{n^{n}} = (1+\frac{1}{n})^{n}\overbracket[0pt]{\longrightarrow}^{n\to \infty}e.
    $$
\end{example}
  A seguir, apresentaremos tr\^es resultados, o Teorema de Dirichlet sobre o produto de s\'eries, o de Abel sobre o produto de
uma s\'erie por uma sequ\^encia n\~ao-crescente, e o de Leibniz.
\begin{theorem*}
  Seja $\sum\limits_{}^{}a_{n}$ uma s\'erie e $s_{n} = a_{1} + \cdots + a_{n},$ sendo n um natural, as suas somas parciais.
  Se $\{s_{n}\}$ \'e limitada e $\{b_{n}\}$ \'e uma sequ\^encia de n\'umeros reais positivos que \'e n\~ao-crescente e infinit\'esima,
ent\~ao $\sum\limits_{}^{}a_{n}b_{n}$ \'e convergente.
\end{theorem*}
\begin{proof*}
  Segue por indu\c c\~ao que, se $s_{n} = a_{1}+\cdots+a_{n},$
  \begin{align*}
    \sum\limits_{k=1}^{n}a_{k}b_{k} &= a_{1}b_{1} + a_{2}b_{2}+\cdots + a_{n}b_{n}\\
                                    &= s_{1}(b_{1}-b_{2})+s_{2}(b_{2}-b_{3})+\cdots+s_{n-1}(b_{n-1}-b_{n})+s_{n}b_{n}\\
                                    &=\sum\limits_{k=1}^{n-1}s_{k}(b_{k}-b_{k+1}) + s_{n}b_{n}.
  \end{align*}
  Seja $M=\sup{\{|s_{n}|:n\in \mathbb{N}\}}.$ Como $\sum\limits_{}^{}(b_{n}-b_{n+1})$ \'e uma s\'erie convergente de n\'umero reais n\~ao-negativos
e $s_{n}b_{n}\overbracket[0pt]{\longrightarrow}^{n\to \infty}0,$ temos 
  $$
    |s_{k}(b_{k}-b_{k+1})| \leq{M(b_{k}-b_{k+1})},
  $$
  segue que $\sum\limits_{n=1}^{\infty}s_{n}(b_{n}-b_{n+1})$ \'e convergente e que $\sum\limits_{}^{}a_{n}b_{n}$ \'e convergente. \qedsymbol
\end{proof*}
\begin{example}
  Para cada n\'umero real x que n\~ao \'e m\'ultiplo inteiro de $2\pi$, as s\'eries $\sum\limits_{}^{}\frac{\cos{(nx)}}{n}$ e $\sum\limits_{}^{}\frac{\sin{(nx)}}{n}$
  s\~ao convergentes. De fato, vamos utilizar um resultado dos n\'umeros complexos, que afirma que 
    $$
    e^{i\theta} = \cos{(\theta)} + i\sin{(\theta)},\quad i\text{ \'e tal que} i^{2} = -1.
    $$
    Sendo assim, para ver que $\{\sum\limits_{k=1}^{n}\cos{(nx)}\}(\text{ ou }\{\sum\limits_{k=1}^{n}\sin{(nx)}\}$ \'e limitada
  utilizamos que 
    $$
      1 + e^{ix} + e^{i2x} + \cdots + e^{inx} = \frac{1 - e^{i(n+1)x}}{1-e^{ix}},
    $$
    e tomamos parte real e a parte imagin\'aria. O resultado agora segue do Teorema de Dirichlet. \qedsymbol
\end{example}
\begin{theorem*}
  Seja $\sum\limits_{}^{}a_{n}$ uma s\'erie convergente e $\{b_{n}\}$ uma sequ\^encia n\~ao crescente de n\'umeros positivos
(n\~ao necessariamente infinit\'esima). Ent\~ao, a s\'erie $\sum\limits_{}^{}a_{n}b_{n}$ \'e convergente.
\end{theorem*}
\begin{proof*}
  Seja $c = \lim_{n\to\infty}b_{n} = \inf_{n\in \mathbb{N}}b_{n}$ e $s_{n} = a_{1} + \cdots + a_{n}.$ Note que 
    $$
      \sum\limits_{k=1}^{n}a_{k}b_{k} = \sum\limits_{k=1}^{n}a_{k}(b_{k}-c) + c \sum\limits_{k=1}^{n}a_{k}.
    $$
  Agora, pelo Teorema de Dirichlet, $\sum\limits_{n=1}^{\infty}a_{n}(b_{n}-c)$ \'e convergente com soma s. Portanto, 
    $$
    \sum\limits_{}^{}a_{n}b_{n} = s + c \sum\limits_{}^{}a_{n}.\quad\text{\qedsymbol}
    $$
\end{proof*}
\begin{theorem*}
  Seja $\{b_{n}\}$ uma sequ\^encia n\~ao crescente e infinit\'esima. Ent\~ao, a s\'erie $\sum\limits_{}^{}(-1)^{n}b_{n}$ \'e convergente.
\end{theorem*}
\begin{proof*}
  Observe que, no caso em que $a_{n} = (-1)^{n}$ e $s_{n} = a_{1} + \cdots + a_{n},$ ent\~ao $\{s_{n}\}$ \'e limitada. Com isso,
  utilizando o Teorema de Dirichlet, $\sum\limits_{}^{}(-1)^{n}b_{n}$ \'e, portanto, convergente. \qedsymbol
\end{proof*}
\newpage

\section{Aula 12 - 14/04/2023}
\subsection{Motiva\c c\~oes}
\begin{itemize}
  \item Entender s\'eries de pot\^encias;
  \item Estudar s\'eries rearranjadas.
\end{itemize}

\subsection{Testes da Raz\~ao e Ra\'iz Modificados.}
  Come\c camos, antes, com outros exemplos com rela\c c\~ao \`a \'ultima aula. O primeiro \'e 
 \begin{theorem*}
   Seja $\{a_{n}\}$ uma sequ\^encia n\~ao-crescente de n\'umeros reais n\~ao-negativos. A S\'erie $\sum\limits_{}^{}a_{n}$ converge se, e somente se,
  a s\'erie $\sum\limits_{}^{}2^{k}a_{2^{k}}$ \'e convergente.
 \end{theorem*}
 \begin{proof*}
   Sejam $\{s_{n}\}, \{s_{n}^{*}\}$ as sequ\^encias das somas parciais de $\sum\limits_{}^{}a_{n}$ e $\sum\limits_{}^{}2^{k}a_{2^{k}}.$
   Ent\~ao, para todo n natural, 
   \begin{align*}
     s_{n} = a_{1} + a_{2} + \cdots + a_{n} \leq{}s_{2^{n}-1} &= a_{1} + (a_2 + a_{3}) + (a_{4}+a_{5}+a_{6}+a_{7})+\cdots+(a_{2^{n-1}}+\cdots+a_{2^{n}-1})
                                                              &\leq{s_{n-1}^{*}}\leq{s_{n}^{*}.}
   \end{align*}
   Logo, se $\{s_{n}^{*}\}$ \'e limitada, segue que $\{s_{n}\}$ \'e limitada.

   Agora, note que, para todo n natural, 
  \begin{align*}
    s_{2^{n}}=a_{1} + a_{2} + \cdots + a_{2^{n}} &\geq{} \frac{a_{1}}{2} + a_{2} + (a_{3}+a_{4}) + (a_{5}+a_{6}+a_{7}+a_{8})+\cdots+(a_{2^{n-1}+1}+\cdots+a_{2^{n}})\\
                                                 &\geq{}\frac{a_{1}}{2}+a_{2}+2a_{4} + 4a_{8}+\cdots+2^{n-1}a_{2^{n}}=\frac{1}{2}s_{n}^{*}.
  \end{align*}
  Portanto, se $\{s_{n}\}$ \'e limitada, segue que $\{s_{n}^{*}\}$ \'e limitada. \qedsymbol
 \end{proof*}
 \begin{example}
   Do resultado anterior, a s\'erie $\sum\limits_{}^{}\frac{1}{n^{p}}$ \'e convergente se, e s\'o se, a s\'erie $\sum\limits_{}^{}\frac{2^{n}}{2^{np}} = \sum\limits_{}^{}2^{(1-p)n}$ \'e
se, somente se, $p>1.$
 \end{example}
 \begin{example}
   A s\'erie $\sum\limits_{n=2}^{\infty}\frac{1}{n(\log{(n)})}^{p}$ \'e convergente se, e somente se, $p>1.$ Com efeito, segue do resultado anterior que 
     $$
     \sum\limits_{n=1}^{\infty}\frac{2^{n}}{2^{n}(\log{(2^{n})})^{p}} = \sum\limits_{n=1}^{\infty}\frac{1}{\log{2}n^{p}}
     $$
     \'e convergente se, e somente se, $p>1$.
 \end{example}
 \begin{example}
   Vamos provar que e n\~ao \'e racional. De fato, seja $s_{n} = 1 + 1 + \frac{1}{2!} + \frac{1}{3!} + \cdots + \frac{1}{n!}.$ Assim, 
  \begin{align*}
    0 < e - s_{n} &= \frac{1}{(n+1)!}+\frac{1}{(n+2)!} + \frac{1}{(n+3)!} + \cdots \\
                  &< \frac{1}{(n+1)!}[1+\frac{1}{n+1}+\frac{1}{(n+1)^{2}}+ \frac{1}{(n+1)^{3}}+\cdots]
                  &= \frac{1}{(n+1)!}\frac{1}{1-\frac{1}{n+1}} = \frac{1}{nn!}.
  \end{align*}
  Agora, suponha que existem inteiros postivios p e q tais que $e = \frac{p}{q}$. Temos 
    $$
      0 < q!(e-s_{q})<\frac{1}{q}\leq{1}.
    $$
  Por hip\'otese, $q!e$ \'e um inteiro e, como $q!s_{q}$ tamb\'em \'e inteiro, segue que $q!(e-s_{q})$ \'e inteiro em $(0,1)$ e temos
uma contradi\c c\~ao.
 \end{example}
 O exemplo a seguir ilustra um caso em que o teste da raz\~ao n\~ao se aplica, mas o teste da ra\'iz indica converg\^encia.
 \begin{example}
   Considere a s\'erie $\{a_{n}\}$ com $a_{2n} = \frac{1}{2^{n}}$ e $a_{2n-1}= \frac{1}{3^{n}}.$ Note que 
     $$
     \frac{a_{2n}}{a_{2n-1}} = \biggl(\frac{3}{2}\biggr)^{n}\overbracket[0pt]{\longrightarrow}^{n\to \infty}+\infty\quad\text{e}\quad \frac{a_{2n+1}}{a_{2n}} = \frac{1}{3}\biggl(\frac{2}{3}\biggr)^{n}\overbracket[0pt]{\longrightarrow}^{n\to \infty}0.
     $$
  No entanto, fazendo o teste da ra\'iz, 
    $$
    \sqrt[2n]{a_{2n}} = \frac{1}{\sqrt{2}}\overbracket[0pt]{\longrightarrow}^{n\to \infty}\frac{1}{\sqrt[]{2}}\quad\text{e}\quad \sqrt[2n-1]{a_{2n-1}}=\biggl(\frac{1}{3}\biggr)^{\frac{n}{2n-1}}\overbracket[0pt]{\longrightarrow}^{n\to \infty}\frac{1}{\sqrt[]{3}}.
    $$
  Portanto, $\limsup_{n\to\infty}\frac{a_{n+1}}{a_{n}} = +\infty$ e $\limsup_{n\to\infty}\sqrt[n]{a_{n}} = \frac{1}{\sqrt[]{2}}$. Em outras palavras,
o teste da ra\'iz se indica converg\^encia, mas o teste da raz\~ao \'e inconclusivo. \qedsymbol
 \end{example}
  Agora, vamos tornar o teste da ra\'iz mais completo.
  \begin{theorem*}
   Se $\{a_{n}\}$ \'e uma sequ\^encia limitada e $\limsup_{n\to\infty}|a_{n}|^{\frac{1}{n}} = c < 1$, ent\~ao $\sum\limits_{}^{}a_{n}$
  \'e absolutamente convergente. Se $\limsup_{n\to\infty}|a_{n}|^{\frac{1}{n}} = c > 1$, ent\~ao $\sum\limits_{}^{}a_{n}$ \'e divergente. 
  Se c = 1, nada podemos concluir.
 \end{theorem*}
\begin{proof*}
  Existe N natural tal que $\sup_{k\geq{n}}|a_{k}|^{\frac{1}{k}} < r = \frac{c+1}{2} < 1$ para todo natural n. Logo, $|a_{n}|<r^{n}$ para
todo $n\geq{N}.$ Segue do Teorema da Compara\c c\~ao que $\sum\limits_{}^{}|a_{n}|$ \'e convergente, ou seja, $\sum\limits_{}^{}a_{n}$ \'e absolutamente convergente. \qedsymbol
  
  Para mostrar que, se $\limsup_{n\to\infty}|a_{n}|^{\frac{1}{n}} = c > 1$, a s\'erie diverge, considere a fun\c c\~ao $\varphi:\mathbb{N}\rightarrow \mathbb{N}$
  estritamente crescente tal que $\{|a_{\varphi(n)}|^{\frac{1}{\varphi(n)}}\}$ converge para $c>1,$ tal que $\{|a_{\varphi(n)}|\}$
  n\~ao converge para zero. Para ver que nada pode ser dito quando c = 1, tome as s\'eries $\sum\limits_{}^{}\frac{1}{n}, \sum\limits_{}^{}\frac{1}{n^{2}}.$
\end{proof*}
\begin{theorem*}
  Se $\sum\limits_{}^{}a_{n}$ \'e uma s\'erie de termos n\~ao-nulos e $\limsup_{n\to\infty}\frac{|a_{n+1}|}{|a_{n}|} = c < 1,$ ent\~ao
 $\sum\limits_{}^{}a_{n}$ \'e absolutamente convergente. Por\'em, se $\frac{|a_{n+1}|}{|a_{n}|}\geq{1}$ para todo $n\geq{n_{0}}, n_{0}$ algum natural,
 ent\~ao a s\'erie diverge.
\end{theorem*}

\subsection{S\'eries de Pot\^encias.}
\begin{def*}
  Dada uma sequ\^encia $\{a_{n}\}$ de n\'umeros reais, a s\'erie 
    $$
      \sum\limits_{n=0}^{\infty}a_{n}x^{n}
    $$
  \'e chamada de S\'erie de Pot\^encia. Os n\'umeros $a_{n}$ s\~ao chamados de coeficientes da s\'eries e x \'e um n\'umero real.
\end{def*}
  Dependendo da escolha de x, a s\'erie pode convergir ou divergir, como indica o reusltado a seguir.
 \begin{theorem*}
   Ddada a s\'erie de pot\^encias $\sum\limits_{n=0}^{\infty}a_{n}x^{n},$ seja $\alpha = \limsup_{n\to\infty}\sqrt[n]{|a_{n}|}$ e defina 
  \begin{align*}
    & R = \frac{1}{\alpha}\text{ se } 0 < \alpha < \infty\\
    & R = 0 \text{ se } \alpha = \infty\\
    & R = \infty \text{ se } \alpha = 0.
  \end{align*}
 \end{theorem*}
 Ent\~ao, $\sum\limits_{n-0}^{\infty}a_{n}x^{n}$ converge se $|x|<R$, diverge se $|x|>R$ e nada podemos afirmar de $|x|=R.$
\begin{proof*}
  Basta notar que $\limsup_{n\to\infty}\sqrt[n]{|a_{n}x^{n}|} = |x|\limsup_{n\to\infty}\sqrt[n]{|a_{n}|} = |x|\alpha$ e aplicar
o teste da ra\'iz. \qedsymbol
\end{proof*}
 Vamos analisar a converg\^encia das s\'eries de pot\^encias.
\begin{example}
 \begin{itemize}
   \item $\sum\limits_{}^{}n^{n}x^{n}, R = 0. $
   \item $\sum\limits_{}^{}\frac{n^{n}}{n!}x^{n}, R = e^{-1}.$ Com efeito, segue de 
     $$
        \sqrt[n]{\frac{n^{n}}{n!}} = \frac{n}{\sqrt[]{n!}}\overbracket[0pt]{\longrightarrow}^{n\to \infty}e.
     $$
    \item $\sum\limits_{}^{}\frac{x^{n}}{n!}, R = \infty.$
    \item $\sum\limits_{}^{}x^{n}, R=1.$
    \item $\sum\limits_{}^{}\frac{x^{n}}{n^{p}}, p > 0, R = 1.$
 \end{itemize}
\end{example}

\subsection{S\'eries Rearranjadas}
  Seja $\sum\limits_{}^{}a_{n}$ uma s\'erie. Defina as sequ\^encias 
 \begin{itemize}
   \item[+)] $\{a_{n}^{+}\}$ com $a_{n}^{+} = a_{n}$ se $a_{n} > 0$ e $a_{n}^{+}=0$ se $a_{n}\leq{0}.$
   \item[-)] $\{a_{n}^{-}\}$ com $a_{n}^{-} = -a_{n}$ se $a_{n} < 0$ e $a_{n}^{-}=0$ se $a_{n}\geq{0}.$
 \end{itemize}
 As sequ\^encias $\{a_{n}^{+}\}$ e $\{a_{n}^{-}\}$ ser\~ao chamadas de parte positiva e parte negativa de $\{a_{n}\}$. Sendo assim,
 $|a_{n}|=a_{n}^{+}+a_{n}^{-}, a_{n} = a_{n}^{+}-a_{n}^{-}$ e $|a_{n}| = a_{n} + 2a_{n}^{-}.$

 Note que, se $\sum\limits_{}^{}a_{n}$ converge absolutamente, ent\~ao $\sum\limits_{}^{}a_{n}^{+}$ e $\sum\limits_{}^{}a_{n}^{-}$
 s\~ao convergentes. A rec\'iproca tamb\'em vale.

 Al\'em disso, se $\sum\limits_{}^{}a_{n}$ \'e convergente, mas n\~ao absolutamente, segue que ambas $\sum\limits_{}^{}a_{n}^{+}$ e $\sum\limits_{}^{}a_{n}^{-}$ divergem.
\begin{def*}
  Seja $\{a_{n}\}$ a sequ\^encia dos termos da s\'erie $\sum\limits_{}^{}a_{n}, \xi:\mathbb{N}\rightarrow \mathbb{N}$
uma bije\c c\~ao e $b_{n} = a_{\xi(n)}.$ A s\'erie $\sum\limits_{}^{}b_{n}$ \'e chamada uma s\'erie rearranjada de $\sum\limits_{}^{}a_{n}.$
\end{def*}
\begin{example}
  Considere a s\'erie $\sum\limits_{n=1}^{\infty}\frac{(-1)^{n+1}}{n}$. Mostraremos que esta s\'erie converge. Se s \'e a sua soma, temos 
 \begin{align*}
   &s = \sum\limits_{n=1}^{\infty}\frac{(-1)^{n+1}}{n} = 1 - \frac{1}{2} + \frac{1}{3} - \frac{1}{4} + \frac{1}{5} - \frac{1}{6} + \frac{1}{7} -\cdots\\
   &\frac{1}{2}s = \sum\limits_{n=1}^{\infty}\frac{(-1)^{n+1}}{2n} = 0 + \frac{1}{2} + 0 - \frac{1}{4} + 0 + \frac{1}{6} + 0 - \frac{1}{8} + 0 + \cdots\\
   &\frac{3}{2}s = \sum\limits_{n=1}^{\infty}\frac{(-1)^{n+1}}{n} = 1 + 0 + \frac{1}{3} - \frac{1}{2} + \frac{1}{5} + 0 + \frac{1}{7}-\frac{1}{4}+\frac{1}{9}+0+\cdots
 \end{align*}
 Logo, uma s\'erie rearranjada pode ter soma distinta da s\'erie original. 
\end{example}
\begin{theorem*}
  Toda s\'erie rearranjada de uma s\'erie absolutamente convergente \'e convergente com mesma soma.
\end{theorem*}
\begin{proof*}
  Se $a_{n}\geq{}0$, ent\~ao para todo n natural, $\xi:\mathbb{N}\rightarrow \mathbb{N}$ \'e uma bije\c c\~ao e $b_{n} = a_{\xi(n)}$.
  Dado n natural, seja $m_{n} = \max{\{\xi(1), \cdots, \xi(n)\}}$, ent\~ao 
    $$
      \sum\limits_{k=1}^{n}b_{k} \leq{\sum\limits_{k=1}^{m_{n}}a_{k}\leq{\sum\limits_{n=1}^{\infty}a_{n}}}
    $$
  e $\sum\limits_{}^{}b_{n}$ \'e convergente com $\sum\limits_{n=1}^{\infty}b_{n}\leq{\sum\limits_{n=1}^{\infty}a_{n}}$. Por outro lado,
  dado um natural m, seja $n_{m}=\max{\{\xi^{-1}(1), \cdots, \xi^{-1}(m)\}}$. Sendo assim, 
    $$
      \sum\limits_{k=1}^{m}a_{k}\leq{\sum\limits_{k=1}^{n_{m}}b_{k}\leq{\sum\limits_{n=1}^{\infty}b_{k}}}
    $$
  e $\sum\limits_{n=1}^{\infty}a_{n}\leq{\sum\limits_{n=1}^{\infty}b_{n}.}$ Para o caso geral, note que 
    $$
      \sum\limits_{n=1}^{\infty}a_{n} = \sum\limits_{n=1}^{\infty}a_{n}^{+} - \sum\limits_{n=1}^{\infty}a_{n}^{-}
    $$
  de forma que 
    $$
      \sum\limits_{n=1}^{\infty}a_{n} = \sum\limits_{n=1}^{\infty}a_{n}^{+} - \sum\limits_{n=1}^{\infty}a_{n}^{-} = \sum\limits_{n=1}^{\infty}b_{n}^{+}-\sum\limits_{n=1}^{\infty}b_{n}^{-} = \sum\limits_{n=1}^{\infty}b_{n}.
    $$
  Portanto, a s\'erie rearranjada converge para a soma da s\'erie original. \qedsymbol
\end{proof*}
\begin{theorem*}
  Se $\sum\limits_{}^{}a_{n}$ \'e convergente e n\~ao \'e absolutamente convergente, ent\~ao  
 \begin{itemize}
   \item[i)] Dado c real, existe bije\c c\~ao $\xi^{c}:\mathbb{N}\rightarrow \mathbb{N}$ tal que $\sum\limits_{n=1}^{\infty}a_{\xi^{c}(n)} = c$
     \item[ii)] Existem bije\c c\~oes $\xi_{+}$ e $\xi_{-}$ tais que $\sum_{n=1}^{\infty}a_{\xi_{+}(n)}$ diverge para $+\infty$
  e $\sum\limits_{n=1}^{\infty}a_{\xi_{-}(n)}$ diverge para $-\infty.$
 \end{itemize}
\end{theorem*}
\begin{proof*}
  Seja $\{p_{n}\}$ a sequ\^encia dos termos positivos de $\{a_{n}\}$ na ordem em que eles aparecem e $\{q_{n}\}$ a sequ\^encia dos termos 
  n\~ao positivos de $\{a_{n}\}$ na ordem em que eles aparecem. Sabemos que $\sum\limits_{}^{}p_{n}$ e $\sum\limits_{}^{}q_{n}$ divergem.
Dado c real, seja $n_{1}$ o primeiro inteiro tal que 
  $$
    \sum\limits_{n=1}^{n_{1}}p_{n} > c.
  $$
  Em seguida escolha $n_{2}$ o menor inteiro tal que 
    $$
      \sum\limits_{n=1}^{n_{1}}p_{n} + \sum\limits_{n=1}^{n_{2}}q_{n} < c
    $$
  e prossiga com este processo. Desta forma, para todo $k > 1,$ 
    $$
      0 < \sum\limits_{n=1}^{n_{1}}p_{n} - \sum\limits_{n=1}^{n_{2}}q_{n} + \cdots - \sum\limits_{n=1}^{n_{2k-2}}q_{n} + \sum\limits_{n=1}^{n_{2k-1}}p_{n} - c \leq{p_{n_{2k-1}}}
    $$
    e 
      $$
        q_{n_{2k}}\leq{\sum\limits_{n-1}^{n_{1}}pn} - \sum\limits_{n=1}^{n_{2}}q_{n} + \cdots + \sum\limits_{n=1}^{n_{2k-1}}p_{n} - \sum\limits_{n=1}^{n_{2k}}q_{n} - c < 0.
      $$
    Agora, como $q_{n}\overbracket[0pt]{\longrightarrow}^{n\to \infty}0$ e $p_{n}\overbracket[0pt]{\longrightarrow}^{n\to \infty}0,$ o resultado segue.
    Por fim, um processo an\'alogo prova a segunta parte do resultado. \qedsymbol
\end{proof*}
\newpage

\section{Aula 13 - 17/04/2023}
\subsection{Motiva\c c\~oes}
\begin{itemize}
  \item Limites e Continuidade de Fun\c c\~oes;
  \item Limites laterais.
\end{itemize}
\subsection{Intui\c c\~ao e Exemplos Iniciais}
Entender continuidade e limites é fundamental no cálculo. Aqui estão alguns exemplos que podem ajudar a esclarecer esses conceitos:

\begin{enumerate}
    \item Continuidade: \\
    Uma função é contínua em um ponto se o seu valor nesse ponto é igual ao limite da função quando se aproxima desse ponto. Em outras palavras, não há quebras ou saltos no gráfico da função. Um exemplo clássico de uma função contínua é a função quadrática:
    
    \begin{equation*}
        f(x) = x^2
    \end{equation*}
    
    Essa função é contínua para todos os números reais, pois não há quebras ou saltos no gráfico.

    \begin{center}
    \begin{tikzpicture}
    \begin{axis}[
        xlabel={$x$},
        ylabel={$f(x)=x^2$},
        domain=-3:3,
        samples=100,
        axis lines=middle
    ]
    \addplot[blue, thick] {x^2};
    \end{axis}
    \end{tikzpicture}
    \end{center}

    \item Limites: \\
    Um limite é o valor que uma função se aproxima quando a entrada (valor x) se aproxima de um determinado valor. O conceito de limites nos ajuda a entender o comportamento das funções próximas aos pontos onde elas não são definidas ou não são contínuas.
    
    Exemplo 1: Limite simples \\
    Considere a função:
    
    \begin{equation*}
        g(x) = 3x + 2
    \end{equation*}
    
    Encontre o limite quando x se aproxima de 1:
    
    \begin{equation*}
        \lim_{x \to 1} (3x + 2)
    \end{equation*}
    
    Como essa é uma função linear e contínua em todos os lugares, o limite é o mesmo que o valor da função no ponto:
    
    \begin{align*}
        g(1) &= 3(1) + 2 \\
        &= 5
    \end{align*}
    
    Portanto, $\lim_{x \to 1} (3x + 2) = 5$.

    \begin{center}
    \begin{tikzpicture}
    \begin{axis}[
        xlabel={$x$},
        ylabel={$g(x)=3x+2$},
        domain=-2:3,
        samples=100,
        axis lines=middle
    ]
    \addplot[red, thick] {3*x+2};
    \end{axis}
    \end{tikzpicture}
    \end{center}
    
    Exemplo 2: Limite de uma função com descontinuidade \\
    Considere a função:
    
    \begin{equation*}
        h(x) = \frac{x^2 - 1}{x - 1}
    \end{equation*}
    
    Essa função não é definida para x = 1, mas ainda podemos encontrar o limite quando x se aproxima de 1:
    
    \begin{equation*}
        \lim_{x \to 1} \frac{x^2 - 1}{x - 1}
    \end{equation*}
    
    Observe que $(x^2 - 1)$ pode ser fatorado como $(x + 1)(x - 1)$. Então, obtemos:
    
    \begin{equation*}
        \lim_{x \to 1} \frac{(x + 1)(x - 1)}{x - 1}
    \end{equation*}
    
    Agora, podemos cancelar os termos $(x - 1)$:
    
    \begin{equation*}
        \lim_{x \to 1} (x + 1)
    \end{equation*}
    
    Neste ponto, podemos substituir x = 1:

    \begin{align*}
        1 + 1 &= 2
    \end{align*}
    
    Portanto, $\lim_{x \to 1} (h(x)) = 2$, embora a função em si não seja definida para x = 1.

    \begin{center}
    \begin{tikzpicture}
    \begin{axis}[
        xlabel={$x$},
        ylabel={$h(x)=\frac{x^2 - 1}{x - 1}$},
        domain=-3:3,
        ymin=-5, ymax=5,
        samples=100,
        axis lines=middle,
        restrict y to domain=-5:5
    ]
    \addplot[green, thick] {((x^2 - 1)/(x - 1))};
    \addplot[only marks, mark=o, mark options={scale=1.5}] coordinates {(1,2)};
    \end{axis}
    \end{tikzpicture}
    \end{center}
\end{enumerate}

Esses exemplos demonstram como a continuidade e os limites nos ajudam a analisar e entender o comportamento das funções, mesmo em pontos onde elas podem não ser definidas. 
A seguir, formalizaremos essas ideias e provaremos algumas propriedades.

\subsection{Limites de Fun\c c\~oes}
 \begin{def*}
   Seja D um subconjunto de $\mathbb{R}, f:D\rightarrow \mathbb{R}$ uma fun\c c\~ao e p um ponto de acumula\c c\~ao de D. Diremos que
  o limite de f(x) quando x tende a p \'e L se, dado $\varepsilon > 0,$ existe um $\delta > 0$ tal que 
    $$
    x\in D\text{ e } 0 < |x-p| < \delta, \Rightarrow |f(x) - L | < \varepsilon.
    $$
    Em outras palavras, dado $\varepsilon > 0$, existe $\delta = \delta(\varepsilon, p) > 0$ tal que 
      $$
        f(D\cap(p-\delta, p+\delta)/a)\subseteq{(L-\varepsilon, L+\varepsilon).} \square
      $$
 \end{def*}
 Note que, se n\~ao existe um n\'umero real L tal que $\lim_{x\to p}= L,$ ent\~ao diremos que o limite n\~oa existe. Al\'em disso,
o ponto p n\~ao precisa ser um ponto do dom\'inio D e, mesmo que perten\c ca, o valor de f em p n\~ao \'e importante para a defini\c c\~ao.
Apenas os valores de f em pontos arbitrariamente pr\'oximos a p s\~ao importantes para a defini\c c\~ao.
\begin{theorem*}
  Seja $f:D\rightarrow \mathbb{R}$ uma fun\c c\~ao e p um ponto de acumula\c c\~ao de D. O limite de f(x) quando x tende a p,
  caso existe, \'e \'unico, e ser\'a dneotado por 
    $$
    \lim_{x\to p}f(x) = L.
    $$
\end{theorem*}
\begin{proof*}
  De fato, se L e L' s\~ao limites de f(x) quando x tende a p, dado $\varepsilon > 0$, existe $\delta > 0$ tal que 
    $$
      x\in D, 0 < |x-p| < \delta \Rightarrow |f(x)- L| < \varepsilon, |f(x) - L'|< \varepsilon.
    $$
    Logo, dado $\varepsilon > 0$, com a escolha de $\delta$ acima e x em D satisfazendo $0 <|x-p|<\delta$, temos 
      $$
        |L - L'| = |L -f(x) + f(x) - L'| \leq{} |L - f(x)| + |f(x) - L'| < 2\varepsilon.
      $$
    Portanto, como $\varepsilon$ \'e arbitr\'ario, L = L'. \qedsymbol
\end{proof*}
  Quando nos referimos a uma fun\c c\~ao, fica impl\'icito que ela tem um dom\'inio especificado. Dada a fun\c c\~ao
  $f:D\rightarrow \mathbb{R}$ e $D'\subseteq{D},$ denotaremos por $f_{|_{D'}}:D'\rightarrow \mathbb{R}$ a fun\c c\~ao
  definida por $f_{|_{D'}}(x) = f(x)$ para x em D'. Segue dessa defini\c c\~ao que 
 \begin{theorem*}
   Seja D um subconjunto dos reais, $f:D\rightarrow \mathbb{R}$ uma fun\c c\~ao, D' um subconjunto de D e p um ponto de aucmula\c c\~ao de D'.
  Se $\lim_{x\to p}f(x) = L,$ ent\~ao $\lim_{x\to p}f_{|_{D'}}(x) = L.$
 \end{theorem*}
\begin{theorem*}
  Seja $f:D\rightarrow \mathbb{R}$ uma fun\c c\~ao, D' e D'' subconjuntos de D e p um n\'umero real que \'e ponto de acumula\c c\~ao
de D' e de D''. 
 \begin{itemize}
   \item[i)] Se um dos limites $\lim_{x\to p}f_{|_{D'}}(x)$ ou $\lim_{x\to p}f_{|_{D''}}(x)$ n\~ao existe, ou ambos existem,
     mas s\~ao diferentes, ent\~ao o limite $\lim_{x\to p}f(x)$ n\~ao existe.
   \item[ii)] Se $(D'\cup D'')/\{p\} = D/\{p\}$, o limite $\lim_{x\to p}f(x)$ existe se, e somente se, $\lim_{x\to p}f_{|_{D'}}(x)$ e
$\lim_{x\to p}f_{|_{D''}}(x)$ existem e t\^em o mesmo valor.
 \end{itemize}
\end{theorem*}
\begin{proof*}
  A prova da primeira parte segue direto. Para a segunda, existe $\lim_{x\to p }f(x) = L$ se, e somente se, dado $\varepsilon > 0$,
existe $\delta > 0$ tal que 
  $$
    x\in D, 0 < |x-p| < \delta \Rightarrow |f(x)-L| < \varepsilon,
  $$
  o que equivale a dizer que, dado $\varepsilon > 0$, existe $\delta > 0$ tal que 
    $$
      x\in D', 0 < |x-p| < \delta \Rightarrow |f(x) - L|< \varepsilon \Rightarrow |f(x)_{|_{D'}}|<\varepsilon
    $$
  e
  $$
      x\in D'', 0 < |x-p| < \delta \Rightarrow |f(x) - L|< \varepsilon \Rightarrow |f(x)_{|_{D''}}|<\varepsilon
  $$
  Portanto, $\lim_{x\to p}f_{|_{D''}}(x) = \lim_{x\to p}f_{|_{D'}}(x)$. \qedsymbol
\end{proof*}
\begin{def*}
  Se D \'e um subconjunto de $\mathbb{R}$, diremos que p real \'e um ponto de acumula\c c\~ao \`a direita de D se \'e um ponto
  de acumula\c c\~ao de $D_{p}^{+} = D\cap{(p, \infty)}$. Seja $f:D\rightarrow \mathbb{R}$ uma fun\c c\~ao e p um ponto de
  acumula\c c\~ao \`a direita de D. O limite de f(x) quando x tende a p pela direita \'e 
    $$
      \lim_{x\to p^{+}}f(x)\coloneqq \lim_{x\to p}f_{|_{D_{p}^{+}}}(x).
    $$
  Define-se analogamente limite de f(x) quando x tende a pela esquerda e ponto de acumula\c c\~ao \`a esquerda.
\end{def*}
\begin{crl*}
  Seja $f:D\rightarrow \mathbb{R}$ uma fun\c c\~ao e p um ponto de acumula\c c\~ao \`a direita e \`a esquerda de D. Ent\~ao, 
    $$
      \lim_{x\to p}f(x)
    $$
    existe se, e somente se, os limites laterais existem e s\~ao iguais.
\end{crl*}
\begin{theorem*}
  Seja D subconjunto real e $f:D\rightarrow \mathbb{R}$ uma fun\c c\~ao. Tome p como um ponto de acumula\c c\~ao de D. Se
  existe $\lim_{x\to p}f(x) - L$, ent\~ao f \'e limitada em uma vizinha\c ca de p, i.e., existem $M>0$ e $\delta > 0$ tais que 
  se x pertence a D e $0 <|x-p| < \delta,$ ent\~ao $|f(x)|< M$
\end{theorem*}
\begin{proof*}
  Existem $\delta > 0$ tal que se x pertence a D e $0<|x-p|<\delta,$ ent\~ao $|f(x)-L|<1.$ Logo, 
    $$
      |f(x)| \leq{|f(x)-L| + |L|} \leq{1 + |L| = M},
    $$
\end{proof*}
\begin{theorem*}
  Seja $D\subseteq{\mathbb{R}}, f, g, h:D\rightarrow \mathbb{R}$ fun\c c\~oes e p um ponto de acumula\c c\~ao de D. Se
  para todo x em D diferente de p, $f(x)\leq{g(x)}\leq{h(x)}$ e $\lim_{x\to p}f(x) = \lim_{x\to p }h(x) = L,$ ent\~ao $\lim_{x\to p}g(x) = L.$
\end{theorem*}
\begin{proof*}
  Com efeito, dado $\varepsilon > 0$, existe $\delta > 0$ tal que se x pertence a D e $0 < |x-p| < \delta,$ ent\~ao
  $|f(x)-L|<\varepsilon, |h(x)-L|<\varepsilon.$ Logo, 
    $$
      L - \varepsilon < f(x) \leq{g(x)} \leq{h(x)} < L +\varepsilon, \quad \forall x\in D, 0<|x-p|<\delta.
    $$
  Segue que $L-\varepsilon < g(x) < L+\varepsilon,$ para todo x em D que satisfa\c ca a condi\c c\~ao. Em outras palavras, 
    $$
    |g(x)-L| < \varepsilon, \quad \forall x\in D, 0<|x-p|<\delta.\text{ \qedsymbol}
    $$
\end{proof*}
\begin{theorem*}
  Seja D subconjunto de $\mathbb{R}, f:D\rightarrow \mathbb{R}$ e p um ponto de acumula\c c\~ao de D. Se $\lim_{x\to p}f(x) = L > 0,$
  ent\~ao existe $\delta > 0$ tal que $f(x) > 0$ para todo x em D com $0<|x-p|<\delta.$
\end{theorem*}
\begin{proof*}
  Dado $\varepsilon = \frac{L}{2}$ tal que 
  $$
    -\frac{L}{2} < f(x) - L < \frac{L}{2}
  $$
  para todo x em D, $0 < |x-p|<\delta.$ Logo, $0 <\frac{L}{2}<f(x)$ para todo x em D com $0 <|x-p|<\delta.$ \qedsymbol
\end{proof*}
\begin{theorem*}
  Seja $D\subseteq{\mathbb{R}}, f, g:D\rightarrow \mathbb{R}<$ uma fun\c c\~ao e p um ponto de acumula\c c\~ao de D. Se existe $\delta > 0$
  tal que $f(x)\leq{g(x)}$ para todo x em D com $0 <|x-p|<\delta$ e existe $\lim_{x\to p}f(x) = L_{g}$, ent\~ao $L_{f}\leq{L_{g}}.$
\end{theorem*}
\begin{proof*}
  De fato, dado $\varepsilon > 0$, existe $\delta > 0$ tal que x pertence a D e satisfaz a propriedade, ent\~ao 
    $$
      L_{f} - \frac{\varepsilon}{2} \leq{f(x)}\leq{g(x)}\leq{L_{g}+\frac{\varepsilon}{2}.}
    $$
    Portanto, $L_{f}-L_{g}\leq{\varepsilon}$ e, como $\varepsilon > 0$ \'e abritr\'ario, o resultado segue. \qedsymbol
\end{proof*}
\begin{theorem*}
  Seja D subconjunto real, $f:D\rightarrow \mathbb{R}$ e p ponto de acumula\c c\~ao. O limite $\lim_{x\to p}f(x)=L$ se, e s\'o se,
  $\lim_{n\to\infty}f(x_{n})$ existe para toda sequ\^encia $\{x_{n}\}$ em $D/\{p\}$ que converge para p.
\end{theorem*}
\begin{proof*}
  Se $\lim_{x\to p }f(x) = L$ e $\{x_{n}\}$ \'e uma sequ\^encia em $D/\{p\}$ com $x_{n}\overbracket[0pt]{\longrightarrow}^{n\to \infty}p$, dado
  $\varepsilon > 0$, podemos encontrar $\delta > 0$ tal que $|f(x)-L|<\varepsilon$ para quaisquer $x\in D, 0<|x-p|<\delta.$

  Seja N natural tal que $|x_{n}-p|<\delta$ para todo $n\geq{N}.$ Logo, $|f(x_{n})-L| < \varepsilon$ para todo $n\geq{N}.$
  Portanto, $\lim_{n\to\infty}f(x_{n})=L.$ 

  Por outro lado, note que, se $\lim_{n\to\infty}f(x_{n})$ existe para toda sequ\^encia $\{x_{n}\}$ em $D/\{p\}$ que converge
  para p, todas as sequ\^encias $\{f(x_{n})\}$ t\^em o mesmo limite, visto que se elas n\~ao tivessem, construir\'iamos uma sequ\^encia
  $\{x_{n}'\}$ em $D/\{p\}$ que converge para p, mas $\{f(x_{n}')\}$ n\~ao convergiria.

  Agora, se $\lim_{x\to p}f(x)$ n\~ao \'e L, existe $\varepsilon > 0$ tal que para todo n natural n\~ao nulo, $x_{n}\in D, 0 <|x_{n}-p|<\frac{1}{n}$ tal 
  que $|f(x_{n})-L|\geq{\varepsilon}$. Logo, $\lim_{n\to\infty}f(x_{n})$ n\~ao \'e L.\qedsymbol
\end{proof*}
\begin{theorem*}
  Seja $D\subseteq{\mathbb{R}}, f, g:D\rightarrow \mathbb{R}$ fun\c c\~oes e p um ponto de acumula\c c\~ao de D e $\lambda\in \mathbb{R}.$
 \begin{itemize}
   \item[i)] Se existem M e $\delta$ positivos tais que $|f(x)|\leq{M}$ para todo x em D, $0 <|x-p|<\delta$ e $\lim_{x\to p}g(x)=0,$
     ent\~ao $\lim_{x\to p}(f \cdot g)(x) = 0.$
     \item[ii)] Se $\lim_{x\to p}f(x) = L_{f}$ e $\lim_{x\to p}g(x) = L_{g},$ ent\~ao $\lim_{x\to p}(f + \lambda g)(x) = L_{f} + \lambda L_{g}.$
     e $\lim_{x\to p}(f \cdot g)(x) = L_{f}L_{g},$. Al\'em disso, se $L_{g}\neq0,$ ent\~ao $\lim_{x\to p}\frac{f}{g}(x) = \frac{L_{f}}{L_{g}}.$
 \end{itemize}
\end{theorem*}
\begin{theorem*}
  Seja $D\subseteq{\mathbb{R}}, f, g:D\rightarrow \mathbb{R}$ fun\c c\~oes e p um ponto de acumula\c c\~ao de D. O limite
  $\lim_{x\to p}f(x)$ existe se, e somente se, f \'e de Cauchy em p, i.e., dado $\varepsilon > 0$, existe $\delta > 0$ tal que
  para x, y em D, $0<|x-p|<\delta$ e $0<|y-p|<\delta$ implica em $|f(x)-f(y)|<\varepsilon.$
\end{theorem*}
\begin{proof*}
  \'E claro que se $\lim_{x\to p}f(x) = L$, ent\~ao f \'e de Cauchy em p. De fato, dado $\varepsilon > 0$, existe $\delta > 0$ 
  tal que se x \'e um elemento de D para o qual $0<|x-p|<\delta,$ ent\~ao $|f(x)-L|<\frac{\varepsilon}{2}.$ Assim, dados
  x, y em D satisfazendo $0<|x-p|<\delta, 0<|y-p|<\delta,$ ent\~ao 
    $$
      |f(x)-f(y)| = |f(x)-L+L-f(y)|\leq{|f(x)-L| + |f(y)-L|}< \frac{\varepsilon}{2}+\frac{\varepsilon}{2} = \varepsilon.
    $$

  Reciprocamente, se f \'e de Cauchy em p e $\{x_{n}\}$ \'e uma sequ\^encia em $D/\{p\}$ que converge para p, 
  $\{f(x_{n})\}$ \'e de Cauchy e portanto convergente. Detalhando, dado $\varepsilon > 0$, existe $\delta > 0$ tal que 
  se x, y s\~ao elementos de D, $0 < |x-p| <\delta$ e $0 < |x-p| <\delta$ implicar em $|f(x)-f(y)|<\varepsilon$,
seja $\{x_{n}\}$ uma sequ\^encia em $D/\{p\}, x_{n}\overbracket[0pt]{\longrightarrow}^{n\to \infty}p$. Dado $\varepsilon > 0$
tome $\delta$ da defini\c c\~ao de ``f \'e de Cauchy em p'' e N tal que $0<|x_{n}-p|<\delta$ para todo $n\geq{N}.$ Ent\~ao, 
  $$
  |f(x_{n})-f(x_{m})| < \varepsilon\quad \forall n, m\geq{N}.
  $$
  Portanto, $\{f(x_{n})\}$ converge. \qedsymbol
\end{proof*}
\begin{def*}
  Seja D um subconjunto ilimitado superiormente de $\mathbb{R}$ e $f:D\rightarrow \mathbb{R}$ uma fun\c c\~ao. Diremos que 
  o limite de f(x) quando x tende para infinito \'e L em $\mathbb{R}$ se, dado $\varepsilon > 0$, existe $M = M(\varepsilon) > 0$
  tal que 
    $$
      x\in D, x > M \Rightarrow |f(x)-L|<\varepsilon.
    $$
  Escreveremos 
    $$
      \lim_{x\to\infty}f(x) = L.
    $$
  Analogamente, quando D \'e ilimitado inferiormente, definimos 
    $$
      \lim_{x\to-\infty}f(x) = L.\square
    $$
\end{def*}
  O limite da sequ\^encia \'e um caso particular de limite infinito, especificamente o caso $D = \mathbb{N}.$
 \begin{def*}
   Seja D um subconjunto de $\mathbb{R}$ e $f:D\rightarrow \mathbb{R}$ uma fun\c c\~ao. Se p \'e um ponto de acumula\c c\~ao de D,
   diremos que f diverge para $+\infty$ quando x tende para p se, dado M positivo, existe $\varepsilon = \varepsilon(M) > 0$
   tal que 
     $$
       x\in D, 0<|x-p|<\varepsilon \Rightarrow f(x) > M.
     $$
     Escreveremos $\lim_{x\to p}f(x) = +\infty$. Analogamente, define-se $\lim_{x\to p}f(x) =-\infty$. 

     Se D \'e ilimitado superiormente(inferiormente), definimos tamb\'em 
       $$
         \lim_{x\to+\infty}f(x) = \pm\infty\quad(\lim_{x\to-\infty}f(x)=\pm\infty.). \square
       $$
 \end{def*}
\begin{def*}
  Seja D um subconjunto real e $f:D\rightarrow \mathbb{R}$. Se p \'e um ponto de acumula\c c\~ao de D, suponha que existe $\delta_{0} > 0$
  tal que 
    $$
    \sup{\{f(x):x\in D, 0<|x-p|<\delta_{0}\}} < \infty.
    $$
    Ent\~ao, existe (ou diverge para $-\infty$) o limite 
      $$
      \limsup_{x\to p}f(x)\coloneqq \limsup_{\delta\to0}\{f(x):x\in D, 0<|x-p|<\delta\}.
      $$
      Escrevemos $\limsup_{x\to p}f(x) = +\infty$ quando f n\~ao \'e limitada superiormente em nenhuma vizinhan\c ca de p.

      De maneira an\'aloga, se 
        $$
        \inf{\{f(x):x\in D, 0<|x-p|<\delta\}} > -\infty,
        $$
      definimos (podendo ser $+\infty$) 
        $$
        \liminf_{x\to p}f(x)\coloneqq \liminf_{\delta\to0}\{f(x):x\in D, 0<|x-p|<\delta\}.
        $$
      Escrevemos $\liminf_{x\to p}f(x) = -\infty$ quando f n\~ao for limitada inferiormente em uma vizinhan\c ca de p. $\square$
\end{def*}
\newpage

\section{Aula 14 - 19/04/2023}
\subsection{Motiva\c c\~oes}
\begin{itemize}
  \item Valor de ader\^encia - valores para os quais alguma sequ\^encia converge sob a imagem de uma fun\c c\~ao
  \item Continuidade;
  \item Resultados sobre Continuidade.
\end{itemize}
\subsection{Limites Superior e Inferior}
\begin{def*}
  Dizemos que um n\'umero real y \'e um valor de ader\^encia de f no ponto p se existe uma sequ\^encia $\{x_{n}\}$
em $D/\{p\}, x_{n}\overbracket[0pt]{\longrightarrow}^{n\to \infty}p$ e $\lim_{n\to\infty}f(x_{n}) = y.\square$
\end{def*}
\begin{theorem*}
  Seja $D\subseteq{\mathbb{R}}, f:D\rightarrow \mathbb{R}$ uma fun\c c\~ao e p um ponto de acumula\c c\~ao de D.
 \begin{itemize}
   \item[1)] Se l \'e um valor de ader\^encia de f em p, ent\~ao $\liminf_{x\to p}f(x) \leq{l}\leq{\limsup_{x\to p}f(x)}$
    \item[2)] Se f \'e limitada em uma em uma vizinhan\c ca de p, ent\~ao $\limsup_{x\to p}f(x)$ e
    $\liminf_{x\to p}f(x)$ s\~ao valores de ader\^encia de f.
    \item[3)] $\lim_{x\to p}f(x)$ existe se, e somente se, f \'e limitada em uma vizinhan\c ca de p e o conjunto dos
    valores de ader\^encia de f em p \'e unit\'ario.
    \item[4)] Se f \'e limitada em uma vizinhan\c ca de p, dado $\varepsilon > 0$, existe $\delta > 0$ tal que 
      $\liminf_{x\to p}f(x) - \varepsilon < f(x) < \limsup_{x\to p} + \varepsilon$ para todo x em D com $0<|x-p|<\delta.$ 
 \end{itemize}
\end{theorem*}
Antes de prov\'a-lo, observe que, se $L_{\delta} = \sup\{f(x):x\in D, 0 < |x-p| < \delta\},$ ent\~ao 
 $$
    L = \limsup_{x\to p}f(x) = \lim_{\delta\to 0^{+}}L_{\delta}.
 $$
 Assim, dado $\varepsilon > 0$, existe $\delta_{\varepsilon}>0$ tal que $0 < \delta < \delta_{\varepsilon}$ implica 
   $$
     |L_{\delta}-L|< \varepsilon.
   $$
  Isso funciona para provar que um valor \'e o lim sup de uma fun\c c\~ao.
\begin{proof*}
  Prova de 1): Se $\liminf_{x\to p}f(x) = l$ e $\limsup_{x\to p}f(x) = L,$ dado $\varepsilon > 0,$ existe $\delta_{\varepsilon}>0$
  tal que 
    $$
    l - \varepsilon < \inf{\{f(x):x\in D, 0<|x-p|<\delta\}} < l + \varepsilon
    $$
  e 
    $$
      L - \varepsilon < \sup{\{f(x):x\in D, 0<|x-p|<\delta\}} < L + \varepsilon.
    $$
  Para todo $0<\delta<\delta_{\varepsilon}.$ Escolha $\delta_{0} < \delta_{\varepsilon}$. Se r \'e um valor de ader\^encia
  de f em p, existe $x_{n}\in D/\{p\}, x_{n}\overbracket[0pt]{\longrightarrow}^{n\to \infty}p,$ com $f(x_{n})\overbracket[0pt]{\longrightarrow}^{n\to \infty}l$
  Seja N natural tal que $|x_{n}-p| < \delta_{0}$ para todo $n\geq{N}.$ Logo, para todo $n\geq{N},$
 \begin{align*}
   l - \varepsilon &< \inf\{f(x):x\in D, 0 < |x-p| < \delta_{0}\}\\
   &\leq{}f(x_{n})\leq{}\sup\{f(x):x\in D, 0 <|x-p|<\delta_{0}\} < \varepsilon
 \end{align*}
 Segue que $l-\varepsilon\leq{r}\leq{L+\varepsilon}$ para todo $\varepsilon > 0$. Portanto, $l\leq{r}\leq{L}.$

 Prova de 2): Note que, para algum $\delta_{0} > 0$, temos 
   $$
   -\infty < \inf{\{f(x):x\in D, 0 < |x-p|< \delta_{0}\}}\leq{}\sup\{f(x):x\in D, 0 < |x-p| <\delta_{0}\} < \infty.
   $$
   Como 
  \begin{itemize}
    \item $(0, \delta_{0})\ni\delta\mapsto\inf\{f(x):x\in D, 0<|x-p|<\delta\}$ \'e n\~ao-decrescente e 
    \item $(0, \delta_{0})\ni\delta\mapsto\sup\{f(x):x\in D, 0<|x-p|<\delta\}$ \'e n\~ao-crescente,
  \end{itemize}
  existem os limites 
    $$
    \lim_{\delta\to 0^{+}}\inf\{f(x):x\in D, 0 < |x-p| < \delta\} = l, \quad\lim_{\delta\to 0^{+}}\sup\{f(x):x\in D, 0 < |x-p| < \delta\} = L
    $$
    Como p \'e um ponto de acumula\c c\~ao de D, seja $\{x_{n}'\}$ e $\{x_{n}^{L}\}$ sequ\^encias em D tais que $0<\max\{|x_{n}'-p|, |x_{n}^{L}-p|\}<\frac{\delta_{0}}{n}$
    e 
      $$
      \inf{\{f(x):x\in D, 0 < |x-p| < \frac{\delta_{0}}{n}\}}\leq{f(x_{n}')}\leq{}\inf{\{f(x):x \in D, 0 <|x-p|<\frac{\delta_{0}}{n}\}} + \frac{1}{n}
      $$
    e
    $$
      \sup{\{f(x):x\in D, 0 < |x-p| < \frac{\delta_{0}}{n}\}}-\frac{\delta_{0}}{n}\leq{f(x_{n}^{L})}\leq{}\sup{\{f(x):x \in D, 0 <|x-p|<\frac{\delta_{0}}{n}\}} + \frac{1}{n}.
    $$
    O resultado agora segue tomando o limite nas express\~oes acima. 

  Prova de 3): Se o limite existe, f \'e limitada em uma vizinhan\c ca de p e todos os valores de ader\^encia coincidem e,
em particular, o $\limsup_{x\to p}f(x) = \liminf_{x\to p}f(x).$ Por outro lado, se f \'e limitada em uma vizinhan\c ca de um ponto p,
e o conjunto dos valores de ader\^encia \'e unit\'ario, $\liminf_{x\to p}f(x)=\limsup_{x\to p}f(x)$ e todos os valores de ader\^encia
coincidem. Portanto, o limite existe 
  
  Prova de 4): Se $\liminf_{x\to p}f(x) = l$ e $\limsup_{x\to p}f(x) = L,$ dado $\varepsilon > 0,$ existe $\delta_{\varepsilon}>0$
  tal que 
    $$
    l - \varepsilon < \inf{\{f(x):x\in D, 0<|x-p|<\delta\}} < l + \varepsilon
    $$
  e 
    $$
      L - \varepsilon < \sup{\{f(x):x\in D, 0<|x-p|<\delta\}} < L + \varepsilon.
    $$
  Para todo $0<\delta<\delta_{\varepsilon}.$  Logo, para todo $\delta<\delta_{\varepsilon}$ e x em D,$0<|x-p|<\delta$
 \begin{align*}
   l - \varepsilon &< \inf\{f(x):x\in D, 0 < |x-p| < \delta_{0}\}\\
   &\leq{}f(x_{n})\leq{}\sup\{f(x):x\in D, 0 <|x-p|<\delta_{0}\} < \varepsilon
 \end{align*}
 Segue que $l-\varepsilon\leq{r}\leq{L+\varepsilon}$ para todo $\varepsilon > 0$. Portanto, $l\leq{r}\leq{L}.$
\end{proof*}

\subsection{Fun\c c\~oes Cont\'inuas}
 \begin{def*}
   Seja $f:D_{f}\rightarrow \mathbb{R}$ uma fun\c c\~ao e p um ponto de $D_{f}.$ Diremos que $f(x)$ \'e cont\'inua em p
   se, dado $\varepsilon > 0$, existe um $\delta > 0$ tal que 
     $$
       x\in D_{f}, |x-p|<\delta \Rightarrow |f(x)-f(p)|<\varepsilon. 
     $$
    Se isto ocorre para todos os pontos p em $D_{f},$ diremos apenas que f \'e cont\'inua. $\square$
 \end{def*}
 Note que, se p \'e um ponto de $D_{f}$ que \'e de acumula\c c\~ao, ent\~ao f \'e cont\'inua em p se, e somente se,
 $\lim_{x\to p}f(x)=f(p)$ e, se p \'e um ponto isolado de $D_{f},$ ent\~ao f \'e cont\'inua em p.
 \begin{example}
   As seguintes fun\c c\~oes s\~ao cont\'inuas em x=p para todo p real:
  \begin{itemize}
    \item[i)]$f(x) = k$
    \item[ii)]$f(x) = x$
    \item[iii)]$f(x) = x + 1$
    \item[iv)]$f(x) = x^{2}$
  \end{itemize}
 \end{example}
 No entanto, a fun\c c\~ao
 $$
 f(x) = \left\{\begin{array}{ll}
                \frac{x^{2}-1}{x-1},\quad x\neq 1\\
                0,\quad x=1
              \end{array}\right.
 $$
 n\~ao \'e cont\'inua em x=1, pois $\lim_{x\to 1}f(x) = 2\neq 0 = f(1).$
 As fun\c c\~oes cont\'inuas t\^em a seguinte propriedade:
\begin{theorem*}
  Sejam $f_{i}:d_{i}\rightarrow \mathbb{R},i=1,2$ fun\c c\~oes. Suponha que p seja um ponto de acumula\c c\~ao de 
  $D_{f_{1}}\cap{D_{f_{2}}}$ e que $\lim_{x\to p}f_{i}(x) = f_{i}(p), i=1, 2.$ Ent\~ao,
\begin{align*}
 &1) \lim_{x\to p}(f_{1}+f_{2})(x) = \lim_{x\to p}f_{1}(x) + \lim_{x\to p}f_{2}(x) = f_{1}(p)+f_{2}(p)\\
 &2) \lim_{x\to p}kf_{1}(x) = kf_{1}(p)\\
 &3) \lim_{x\to p}f_{1}(x)f_{2}(x) = \lim_{x\to p}f_{1}(x)\lim_{x\to p}f_{2}(x) = f_{1}(p)f_{2}(p)\\
 &4) \text{Se } f_{2}(x)\neq0, \lim_{x\to p}\frac{f_{1}(x)}{f_{2}(x)} = \frac{f_{1}(p)}{f_{2}(p)}.
\end{align*}
\end{theorem*}
\begin{theorem*}
  Se p \'e um ponto de $D_{f}\cap D_{g}$ e $f(x)\leq{g(x)}$ sempre que $x\in{(D_{f}\cap D_{g})}$ e os limites de f e
  quando x tendem a p existem, ent\~ao 
    $$
      \lim_{x\to p}f(x) = f(p)\leq{g(p)} = \lim_{x\to p}g(x).
    $$
\end{theorem*}
Vale tamb\'em a an\'aloga do Teorema do Confronto, mas para fun\c c\~oes cont\'inuas. O resultado a seguir precisava
de continuidade para ser provado:
\begin{theorem*}
  Sejam $f:D_{f}\rightarrow \mathbb{R}, g:D_{g}\rightarrow \mathbb{R}$ fun\c c\~oes tais que a imagem de g est\'a contida
  no dom\'inio de f e $L\in D_{f}.$ Se p \'e um ponto de acumula\c c\~ao de $D_{g}, \lim_{x\to p}g(x) = L$ e f \'e cont\'inua
  em L, ent\~ao 
    $$
      \lim_{x\to p}f(g(x)) = f\biggl(\lim_{x\to p}g(x)\biggr) = f(L).
    $$
\end{theorem*}
\begin{proof*}
  Como f \'e cont\'inua em L, dado $\varepsilon > 0$, exsite $\delta_{f}>0$ tal que 
    $$
      y\in D_{f},\quad |y-L|<\delta_{f} \Rightarrow |f(y)-f(L)|<\varepsilon.
    $$
    Como $\lim_{x\to p}g(x) = L,$ dado $\delta_{f}>0$, existe $\delta_{g}>0$ tal que 
      $$
        x\in D_{g},\quad 0 <|x-p|<\delta_{g} \Rightarrow |g(x)-L|<\delta_{f}.
      $$
    Deta forma, como $Im(g)\subseteq{D_{f}}, D_{f\circ{g}}=D_{g}$ e 
      $$
        x\in D_{g}=D_{f\circ{g}}, \quad 0<|x-p|<\delta_{g}\Rightarrow|g(x)-L|<\delta_{f}\Rightarrow|f(g(x))-L|<\varepsilon.
      $$
      Portanto, $\lim_{x\to p}f(g(x))=f(L).$ \qedsymbol
\end{proof*}
  Em suma, a soma, a multiplica\c c\~ao, a divis\~ao e a composta de fun\c c\~oes cont\'inuas \'e uma fun\c c\~ao cont\'inua. Fun\c c\~oes
  racionais e trigonom\'etricas s\~ao cont\'inuas.
  
\subsection{Resultados Avan\c cados de Continuidade - Parte 1.}
Come\c camos apresentando o resultado conhecido como Teorema da Conserva\c c\~ao do Sinal
 \begin{theorem*}
   Seja $f:D_{f}\rightarrow \mathbb{R}$ uma fun\c c\~ao cont\'inua e $\overline{x}\in D_{f}$ tal que $f(\overline{x})>0.$ 
   Ent\~ao, existe $\delta>0$ tal que $f(x)>0$ para todo $x\in D_{f}$ e $x\in(\overline{x}-\delta, \overline{x}+\delta).$
 \end{theorem*}
 \begin{proof*}
   Como f \'e cont\'inua em $\overline{x},$ dado $\varepsilon = f(\overline{x}) >0$, exsite $\delta > 0$ tal que 
     $$
       x\in D_{f}, x\in(\overline{x}-\delta, \overline{x}+\delta) \Rightarrow f(x)\in (f(\overline{x})-\varepsilon, f(\overline{x})+\varepsilon)) =
  (0, 2f(\overline{x})).
     $$
     Isto prova o resultado. \qedsymbol
 \end{proof*}
 O pr\'oximo \'e o Teorema do Anulamento.
 \begin{theorem*}
   Se $f:[a,b]\rightarrow \mathbb{R}$ \'e cont\'inua e $f(a)<0<f(b),$ ent\~ao existe $\overline{x}\in(a, b)$ tal que $f(\overline{x}) =0.$
 \end{theorem*}
\begin{proof*}
  Faremos apenas o caso $f(a)<0<g(b).$ Seja 
    $$
    A = \{x\in[a,b]:f(s)>0 \forall s\in[x, b].\}
    $$
    Note que $A\neq\emptyset$ e $A\subseteq{[a, b]}.$ Seja $z =\inf{A}$. Pelo Teorema da Conserva\c c\~ao de Sinal,
    $z\in(a, b)$ e $z\not\in A.$ Destarte, $f(z)\leq{0}.$ Por outro lado, do Teorema da Compara\c c\~ao, $f(z)=
    \lim_{x\to z^{+}}f(x)\geq{0}$, pois como $x > z,$ x \'e um elemento de A, o que torna $f(x)>0$. Portanto, f(z)=0.
\end{proof*}
 A seguir, veremos o Teorema do Valor Intermedi\'ario.
\begin{theorem*}
  Seja $f:[a, b]\rightarrow \mathbb{R}$ uma fun\c c\~ao cont\'inua e tal que  $f(a)<f(b)$. Se $f(a)<k<f(b),$
  ent\~ao existe $\overline{x}\in(a, b)$ tal que $f(\overline{x})=k.$
\end{theorem*}
\begin{proof*}
  Considere a fun\c c\~ao $g(x)=f(x)-k.$ Ent\~ao, $g:[a, b]\rightarrow \mathbb{R}$ \'e cont\'inua, $g(a)<0, g(b)>0$
  e do Teorema do Anulamento, existe $\overline{x}\in[a, b]$ tal que $g(\overline{x})=0.$ Portanto, $f(\overline{x})=k.$\qedsymbol
\end{proof*}
Todos eles possuem vers\~oes para trocas de sinais.
\end{document}
